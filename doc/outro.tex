\section{Conclusions and future work}

One oddity of the model is that
$\sem{r \GETS x\SEMI y \GETS r\SEMI\SKIP}$ includes:
\[\begin{tikzpicture}[node distance=1em]
  \event{rx0}{\DR{x}{0}}{}
  \nonevent{wy1}{\DW{y}{1}}{right=of rx0}
  \po{rx0}{wy1}
\end{tikzpicture}\]
where the write action guessed its value incorrectly, and therefore has
precondition $0=1$.   This form of speculative
execution does not appear to be used in practice. In order to disallow it,
one could change the semantics of $\SKIP$ to introduce a tick 
action denoting successful completion of the thread and only consider
executions in which the precondition of every tick action is satisfiable.  We
leave the elaboration of this idea as future work.

\todo{Comment on the following:}
\begin{itemize}
\item coherence = per location total order on $\not<$

\item Validation of write removal requires some tricks to ensure that thread
  does not rf its own write

\item Definition of rf can use $\not<$ in first clause, rather than $\leq$.
  We chose the stronger definition because it makes some of the examples
  simpler: in particular, the example motivating rf needs a volatile in the
  other thread if you don't have rf-implies-hb. Old text: The notion
  rf-pomset is sufficient to capture hardware models and release/acquire
  access in C++, where reads-from implies happens-before \cite{alglave}.  To
  model C++ relaxed access, it would be necessary to use a more general
  notion of rf-pomset, where $(\bEv,\aLoc,\aEv) \in \RF$ does not necessarily
  imply $\bEv < \aEv$, instead requiring that $(\mathord< \cup \mathord\RF)$
  be acyclic.

\item The design space for transactions is very rich
  \cite{DBLP:journals/pacmpl/DongolJR18}.  We have only presented one option.
  we pun between abort and false commit
  
\item causality test cases 1, 6, 8, 9, 18, 20  (12?) require that the logic make
  assertions about the domain of variables

\end{itemize}
