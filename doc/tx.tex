\subsection{Transactions}
\label{sec:transactions}

We present a model of transactional memory~\cite{Larus:2007:TM:1207012} that is sufficient to capture
\textsc{Prime+Abort} attacks~\cite{DBLP:conf/uss/DisselkoenKPT17}.  We make
several simplifying assumptions: transactions are serializable, strongly
isolated, and only abort due to cache conflicts.

The action $(\DB{\aVal})$ %\in\Acq$
represents the begin of a transaction with
id $\aVal$, and $(\DC{\aVal})$ %\in \Rel$
represents the corresponding commit.
We model a language in which transactions have explicit identifiers (which we
elide in examples) and abort handlers (which we elide when they are empty):
\[\begin{array}{l}
    \sem{\BEGINVAL\SEMI \eCmd\SEMI \RECOVERYVAL \bCmd \ENDREC} \\\quad
    = (\DB{\aVal}) \prefix \bigl(\sem{\eCmd} \cup \bigl((\FALSE \guard \sem{\eCmd}) \parallel \sem{\bCmd}\bigr)\bigr)
    \\[\jot]
    \sem{\COMMITVAL\SEMI \bCmd} \\\quad
    = (\DC{\aVal}) \prefix \sem{\bCmd}
  \end{array}\]
The semantics of a transaction has two cases: a committed case
(executing only the transaction body) and an aborted case (executing both the body and the
recovery code, where the body is marked unsatisfiable). For example, two executions of
\begin{math}
  (\BEGIN\SEMI \aLoc\GETS1\SEMI \aLoc\GETS2\SEMI \COMMIT\SEMI \RECOVERY \bLoc\GETS1\ENDREC)
\end{math}
are:
\[\begin{array}{c}\begin{tikzpicture}[node distance=1em]
  \event{b0}{\DB{}}{}
  \event{wx0}{\DW{x}{1}}{right=of b0}
  \event{wx1}{\DW{x}{2}}{right=of wx0}
  \event{c0}{\DC{}}{right=of wx1}
  \po{b0}{wx0}
  \po[out=30,in=150]{b0}{wx1}
  \po[out=30,in=150]{wx0}{c0}
  \po{wx1}{c0}
  \wk{wx0}{wx1}
\end{tikzpicture}
\\
\begin{tikzpicture}[node distance=1em]
  \event{b0}{\DB{}}{}
  \nonevent{wx0}{\DW{x}{1}}{right=of b0}
  \nonevent{wx1}{\DW{x}{2}}{right=of wx0}
  \nonevent{c0}{\DC{}}{right=of wx1}
  \event{wy1}{\DW{y}{1}}{right=of c0}
  \po{b0}{wx0}
  \po[out=30,in=150]{b0}{wx1}
  \po[out=30,in=160]{b0}{wy1}
  \po[out=30,in=150]{wx0}{c0}
  \po{wx1}{c0}
  \wk{wx0}{wx1}
\end{tikzpicture}\end{array}\]
At top level, we require that pomsets be \emph{serializable}, as defined below.
\begin{definition}
  We say that event $\comEv$ \emph{matches} $\begEv$ if
  $\labelling(\comEv)=(\DC{\aVal})$ and
  $\labelling(\begEv)=(\DB{\aVal})$. %, for some $\aVal$.
  % We say that a begin event \emph{aborts} if every matching commit is
  % unsatisfiable.
  We say that begin event $\begEv$ \emph{begins} $\aEv$ if
  $\begEv\le\aEv$ and there is no intervening matching commit; in this case
  $\aEv$ \emph{belongs to} $\begEv$.
  % event $\comEv$ such that $\begEv\le\comEv\le\aEv$
  We say that commit event $\comEv$ \emph{commits} $\aEv$ if $\aEv\le\comEv$
  and there is no intervening matching begin.
  % event $\begEv$ such that $\aEv\le\begEv\le\comEv$.
\end{definition}
\begin{definition}
  A pomset is \emph{serializable} if:
  \begin{enumerate}
  \item\label{tx:1} no two begins have the same id,
  \item\label{tx:2} every commit follows the matching begin,
  \item\label{tx:3} $\le$ totally orders tautological begins and commits,
  \item\label{tx:4} if $\begEv$ begins $\aEv$, but not $\bEv$, and $\bEv\le\aEv$ then $\bEv\le\begEv$,
  \item\label{tx:5} if $\comEv$ ends $\aEv$, but not $\bEv$, and $\aEv\le\bEv$ then $\comEv\le\bEv$,
  % \item\label{tx:4} if $\begEv$ begins $\aEv$, but not $\bEv$, then
  %   $\bEv\le\aEv$ implies $\bEv\le\begEv$ and $\aEv\ltN\bEv$ implies $\begEv\ltN\bEv$
  % \item\label{tx:5} if $\comEv$ ends $\aEv$, but not $\bEv$, then
  %   $\aEv\le\bEv$ implies $\comEv\le\bEv$ and $\bEv\ltN\aEv$ implies $\bEv\ltN\comEv$,
  \item\label{tx:6} if $\aEv$ and $\bEv$ belong to $\begEv$ and read the same
    location, then both read the same value, and
    % note that read events are optional, so we can assume they come from
    % outside the transaction.
  %
  % \item\label{tx:6} if $\begEv$ begins $\aEv$ then some matching $\comEv$ both implies and ends $\aEv$,
  % \item\label{tx:6} if $\begEv$ begins $\aEv$ then some matching $\comEv$
  %   ends $\aEv$ such that both $\aEv$ implies $\comEv$ and $\comEv$ implies $\aEv$,
  \item\label{tx:7} if $\aEv$ belongs to $\begEv$, then $\aEv$ implies some
    matching $\comEv$ that ends $\aEv$.
  \end{enumerate}
\end{definition}
%In discussion, we identify transactions by their unique begin event.
%A transaction that does not abort is \emph{successful}.
%
Conditions \ref{tx:1}-\ref{tx:5} ensure serializability of committed
transactions.  Conditions \ref{tx:4}-\ref{tx:6} also ensure strong isolation
for non-transactional events
\cite{DBLP:journals/pacmpl/DongolJR18}. Condition \ref{tx:7} ensures that all
events in aborted transactions are unsatisfiable.
%
For example Conditions \ref{tx:5} and \ref{tx:7} rule out
executions (which violate strong isolation and atomicity):
\[\begin{array}{c}\begin{tikzpicture}[node distance=1em]
  \event{b0}{\DB{}}{}
  \event{wx0}{\DW{x}{0}}{right=of b0}
  \event{wx1}{\DW{x}{1}}{right=of wx0}
  \event{c0}{\DC{}}{right=of wx1}
  \event{rx0}{\DR{x}{0}}{right=of c0}
  \po{b0}{wx0}
  \po[out=30,in=150]{b0}{wx1}
  \po[out=30,in=150]{wx0}{c0}
  \po{wx1}{c0}
  \wk{wx0}{wx1}
  \rf[out=30,in=150]{wx0}{rx0}
\end{tikzpicture}
\\
\begin{tikzpicture}[node distance=1em]
  \event{b0}{\DB{}}{}
  \event{wx0}{\DW{x}{0}}{right=of b0}
  \nonevent{wx1}{\DW{x}{1}}{right=of wx0}
  \nonevent{c0}{\DC{}}{right=of wx1}
  \event{wy1}{\DW{y}{1}}{right=of c0}
  \po{b0}{wx0}
  \po[out=30,in=150]{b0}{wx1}
  \po[out=30,in=160]{b0}{wy1}
  \po[out=30,in=150]{wx0}{c0}
  \po{wx1}{c0}
  \wk{wx0}{wx1}
\end{tikzpicture}\end{array}\]

In order to model \textsc{Prime+Abort}, we need a mechanism for modeling
\emph{why} a transaction aborts, as this can be used as a back channel.
We model a simple form of concurrent transaction, which aborts when it
encounters a memory conflict---this is similar to
the treatment of $\TOUCHED$ in \S\ref{sec:spectre}.

\begin{definition}
  \label{def:abort}
  A commit event $\comEv$ matching $\begEv$ \emph{aborts due to memory conflict}
  if there is some $\aEv$ ended by $\comEv$, and some tautologous $\begEv\gtN\bEv\gtN\comEv$ that does not
  belong to $\begEv$ such that $\aEv$ and $\bEv$ touch the same location.
\end{definition}

The attack requires an honest agent whose %cache-set
access pattern depends upon a secret.
% If $a[0]$ and $a[1]$ belong to separate cache sets, then
Such an honest agent is:
\[
  a[\SEC]\,\GETS\,1
\]
% \ignore{
% \begin{verbatim}
%   a[SECRET] := 1
% \end{verbatim}
% }
%The attack relies on discovery of some $y$ which belongs to the cache-set of $a[1]$.
Then the attacker program
\[
\BEGIN\SEMI a[1]\GETS0\SEMI r\GETS\COMMIT\SEMI \RECOVERY x\GETS1\ENDREC
\]
% \ignore{
% \begin{verbatim}
%   begin; y:=0; commit; onabort; x:=1;
% \end{verbatim}
% }
can write $1$ to $x$ if the \texttt{SECRET} is $1$, in which case the
following execution is possible.
\[\begin{tikzpicture}[node distance=1em,baselinecenter]
  \event{wa1}{\DW{a[1]}{1}}{}
  \event{b}{\DB{}}{right=2.5em of wa1}
  \nonevent{e}{\DW{a[1]}{0}}{right=of b}
  \nonevent{c}{\DC{}}{right=of e}
  \event{wx1}{\DW{x}{1}}{right=of c}
  \po{b}{e}
  \po{e}{c}
  \po[out=30,in=155]{b}{wx1}
  \wk{b}{wa1}
  \wk[out=25,in=155]{wa1}{c}
\end{tikzpicture}\]
If the attacker knows that commits only abort due to memory conflicts,
then this attack is an information flow, since the memory conflict only happens
when the \texttt{SECRET} is $1$.

The attacker code here must have write access to the high security variable
$a$.  Such a ``write up'' is allowed by secrecy analyses such as the
Smith-Volpano type system \cite{Smith:1998:SIF:268946.268975}, which
is meant to guarantee noninterference.

If we require that the attacker and honest agent access disjoint locations in
memory, then we must include a bit of microarchitecture to model the attack.
Suppose that the set of locations $\Loc$ is partitioned into \emph{cache
  sets} and update Definition~\ref{def:abort} so that the commit event
{aborts due to memory conflict} if $\aEv$ and $\bEv$ touch locations \emph{in
  the same cache set}.

\textsc{Prime+Abort} exploits an honest agent whose cache-set
access pattern depends upon a secret.
If $a[0]$ and $a[1]$ belong to separate cache sets, then
such an honest agent is, as before:
\[
  a[\SEC]\,\GETS\,1
\]
The attack relies on discovery of some $y$ which belongs to the cache-set of
$a[1]$.
Then the attack can be written as:
\[
  \BEGIN\SEMI y\GETS0\SEMI r\GETS\COMMIT\SEMI \RECOVERY x\GETS1\ENDREC
\]
As before, if the attacker knows that commits only abort due to memory
conflicts, then there is an information flow, since the memory conflict
only happens when the \texttt{SECRET} is $1$.

This style of attack can be thwarted by requiring that the honest agent and
attack code access disjoint cache sets.  This approach is pursued in \cite{dawg}.

Another defense is to require a speculation barrier at the beginning of each
transaction.  This would have the effect, however, of undermining any
optimistic execution strategy for transactions: the transaction would only be
able to begin when it is known that its commit will succeed.

% \ignore{
% \begin{verbatim}
%   begin; y:=0; commit; onabort; x:=1;
% \end{verbatim}
% }

%   The definition handles simple examples:
% \begin{itemize}
% \item Single threaded example: $\DB_1 \DC_1 \DB_2 \DC_2$.  Because
%   $\DC_2$ is a release, we know that $\DC_1<\DC_2$.  Because
%   $\DB_1$ is an acquire, we know that $\DB_1<\DB_2$ By lifting, either of
%   these is sufficient to require that $\DC_1<\DB_2$.
% \[\begin{tikzpicture}[node distance=1em,baselinecenter]
%   \event{b1}{\DB_1}{}
%   \event{c1}{\DC_1}{right=of b1}
%   \event{b2}{\DB_2}{below right=of b1}
%   \event{c2}{\DC_2}{right=of b2}
%   \po{b1}{c1}
%   \po{b1}{b2}
%   \po{b2}{c2}
%   \po{c1}{c2}
% \end{tikzpicture}
% \;\text{implies}\;
% \begin{tikzpicture}[node distance=1em,baselinecenter]
%   \event{b1}{\DB_1}{}
%   \event{c1}{\DC_1}{right=of b1}
%   \event{b2}{\DB_2}{below right=of b1}
%   \event{c2}{\DC_2}{right=of b2}
%   \po{b1}{c1}
%   \po{c1}{b2}
%   \po{b2}{c2}
% \end{tikzpicture}\]
% \item Abort example:
% \[\begin{tikzpicture}[node distance=1em,baselinecenter]
%   \event{b1}{\DB}{}
%   \event{wx1}{\DW{x}{1}}{right=of b1}
%   \event{c1}{\DC_1}{above right=of wx1}
%   \nonevent{c2}{\DC_2}{below right=of wx1}
%   \event{rx1}{\DR{x}{1}}{right=2.5 em of wx1}
%   \po{b1}{wx1}
%   \po{wx1}{c1}
%   \po{wx1}{c2}
%   \rf{wx1}{rx1}
% \end{tikzpicture}
% \;\text{implies}\;
% \begin{tikzpicture}[node distance=1em,baselinecenter]
%   \event{b1}{\DB}{}
%   \event{wx1}{\DW{x}{1}}{right=of b1}
%   \event{c1}{\DC_1}{above right=of wx1}
%   \nonevent{c2}{\DC_2}{below right=of wx1}
%   \event{rx1}{\DR{x}{1}}{right=2.5 em of wx1}
%   \po{b1}{wx1}
%   \po{wx1}{c1}
%   \po{wx1}{c2}
%   \rf{wx1}{rx1}
%   \po{c1}{rx1}
% \end{tikzpicture}\]

% % \item Clause \eqref{xrf} stops transaction from reading two different values
% %   for the same variable from transactions (it is possible with no
% %   transactional writes).
% % \item Clause \eqref{xrf} also stops transactional IRIW.
% \end{itemize}
% % \begin{definition}
% %   An rf-pomset is transaction-closed if the $\DB$ and $\DC$ actions with
% %   satisfiable preconditions are totally ordered by $<$.
% % \end{definition}

% Let ``$\END\SEMI \bCmd$'' be syntax sugar for
% ``$\IF\COMMIT\vec\aLoc\THEN\bCmd \ELSE \bCmd$'', where $\vec\aLoc$ are the
% free variables of $\bCmd$.

% The semantics of
% \begin{alltt}
%   x:=1; begin; x:=2; end; y:=x;
% \end{alltt}
% includes
% \[\begin{tikzpicture}[node distance=1em]
%   \event{wx1}{\DW{x}{1}}{}
%   \event{b}{\DB}{right=of wx1}
%   \event{wx2}{\DW{x}{2}}{right=of b}
%   \event{c}{\DC}{right=of wx2}
%   \event{wy2}{\DW{y}{2}}{right=of c}
%   \nonevent{wy1}{\DW{y}{1}}{below=of wy2}
%   \po{b}{wx2}
%   \po[bend right]{b}{wy1}
%   \po[bend left]{b}{wy2}
%   \po{wx2}{c}
%   \po[bend left]{wx1}{c}
%   %\po{rz0}{wy2}
% \end{tikzpicture}\]
% and
% \[\begin{tikzpicture}[node distance=1em]
%   \event{wx1}{\DW{x}{1}}{}
%   \event{b}{\DB}{right=of wx1}
%   \nonevent{wx2}{\DW{x}{2}}{right=of b}
%   \nonevent{c}{\DC}{right=of wx2}
%   \nonevent{wy2}{\DW{y}{2}}{right=of c}
%   \event{wy1}{\DW{y}{1}}{below=of wy2}
%   \po{b}{wx2}
%   \po[bend right]{b}{wy1}
%   \po[bend left]{b}{wy2}
%   \po{wx2}{c}
%   \po[bend left]{wx1}{c}
%   %\po{rz0}{wy2}
% \end{tikzpicture}\]

% Publication example:
% \begin{alltt}
%   var x; var f; x:=0; f:=0;
%      x:=1; (begin; f:=1; end;) || (begin; r:=f; end; s:=x;)
% \end{alltt}

% Note: we could also include a transaction factory, and close the factory.
% \begin{alltt}
%   TransactionFactory T; var x; var f; x:=0; f:=0; fence;
%      x:=1; (begin T; f:=1; f:=2; end T;) || (begin T; r:=f; end T; s:=x;)
% \end{alltt}

% Before defining atomicity, we provide some auxiliary notation.
%
% We say that $\aEv$ is a \emph{begin event} if
% $\labelling(\aEv)=(\aForm\mid\DB)$ and a \emph{commit event} if
% $\labelling(\aEv)=(\aForm\mid\DC)$.
%
% We write $\aForm_\aEv$ for the formula and $\aAct_\aEv$ for the
% action of $\aEv$; that is, when $\labelling(\aEv)=(\aForm_\aEv\mid\aAct_\aEv)$.
%
% We say that $\aEv$ is \emph{compatible with} $\bEv$ when
% $\aForm_\aEv\land\aForm_\bEv$ is satisfiable.
%
% \begin{definition}
%   A pomset is \emph{atomic} when for any $\aEv$ that belongs to $(\begEv,\vec\comEv)$:
%   \begin{enumerate}
%   \item\label{xcommitform} $\aForm_{\aEv}$ implies $\textstyle\bigvee_i\aForm_{\comEv_i}$,
%   \item\label{xliftb} if $\bEv<\aEv$ then $\bEv<\begEv$,
%   \item\label{xliftc} if $\aEv<\bEv$ and $\bEv$ is compatible with
%     $\comEv_i$ then $\comEv_i<\bEv$,
%   \item\label{xtotal} if $\aEv'\neq\aEv$ belongs to $(\begEv',\vec\comEv')$ but not
%     $(\begEv,\dontcare)$ and
%     \begin{itemize}
%     \item $\comEv'_j$ is compatible with $\aEv$, and
%     \item $\comEv_i$ is compatible with $\aEv'$
%     \end{itemize}
%     then either
%     $\comEv'_j<\begEv$ or
%     $\comEv_i<\begEv'$,
%   % \item\label{xrf} if $\aEv'\neq\aEv$ belongs to $(\begEv,\dontcare)$ but not
%   %   $(\begEv,\dontcare)$ and
%   %   \begin{itemize}
%   %   \item $\aEv$ reads from $\bEv'$ that belongs to $(\begEv',\vec\comEv')$,
%   %   \item $\aEv'$ reads from $\bEv''\neq\beV'$ that belongs to $(\begEv'',\vec\comEv'')$,
%   %   \item $\comEv''_j$ is compatible with $\bEv'$, and
%   %   \item $\comEv'_i$ is compatible with $\bEv''$
%   %   \end{itemize}
%   %   then either
%   %   $\comEv''_j<\begEv'$ or
%   %   $\comEv'_i<\begEv''$ .
%   \item\label{xcommitvars} if $\aEv$ writes $\aLoc$ and $\comEv_i$ writes
%     $\vec\aLoc$ then $\aLoc=\aLoc_i$, for some $i$, and
%   \item\label{xreadunique} if $\aEv$ reads $\aLoc$ and $\aEv'\neq\aEv$ reads
%     $\aLoc$, belongs to $(\begEv,\dontcare)$ and is compatible with $\aEv$
%     then $\aAct_{\aEv}=\aAct_{\aEv'}$.
%   \end{enumerate}
% \end{definition}
% Clause \eqref{xcommitform} requires that the precondition on $\aEv$ is false on an
% aborted transaction.
% The \emph{lifting clauses}, \eqref{xliftb} and \eqref{xliftc}, require order
% come in or out of $\aEv$ is lifted to the corresponding begin or commit event.
% % Clause \eqref{xrf} requires that whenever a transaction reads from two other
% % transactions, the other transactions must be ordered.
% Clause \eqref{xtotal} requires that transactions be totally ordered.
% Clause \eqref{xcommitvars} requires that all writes be committed.
% Clause \eqref{xreadunique} requires that multiple reads of a location in a
% single transaction must see the same value.
%
% The definition of atomicity guarantees strong isolation.  For weak isolation,
% clauses \eqref{xcommitvars} and \eqref{xreadunique} are unnecessary,
% \eqref{xliftb} only applies when $\bEv$ is a commit, and \eqref{xliftc} only
% applies when $\bEv$ is a begin.

% Local Variables:
% TeX-master: "x"
% End:
