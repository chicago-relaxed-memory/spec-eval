% \documentclass[acmsmall,review,anonymous]{acmart}
\documentclass[acmsmall]{acmart}
\settopmatter{printfolios=true,printacmref=false}

%% \acmJournal{PACMPL}
%% \acmVolume{1}
%% \acmNumber{CONF} % CONF = POPL or ICFP or OOPSLA
%% \acmArticle{1}
%% \acmYear{2018}
%% \acmMonth{1}
%% \acmDOI{} % \acmDOI{10.1145/nnnnnnn.nnnnnnn}
%% \startPage{1}

%\setcopyright{none}
\setcopyright{rightsretained}
\bibliographystyle{ACM-Reference-Format}

\usepackage{macros}
\newcommand{\ignore}[1]{}
\newcommand{\todo}[1]{{\color{red}\textbf{\{#1\}}}}

\begin{document}

\title{A model of speculative evaluation}

\author{Craig Disselkoen}
\affiliation{\institution{University of California San Diego}}
\affiliation{\institution{Mozilla Research Internship}}
\email{cdisselk@cs.ucsd.edu}

\author{Radha Jagadeesan}
\affiliation{\institution{DePaul University}}
\email{rjagadeesan@cs.depaul.edu}

\author{Alan Jeffrey}
\affiliation{\institution{Mozilla Research}}
\email{ajeffrey@mozilla.com}

\author{James Riely}
\affiliation{\institution{DePaul University}}
\email{jriely@cs.depaul.edu}

\begin{abstract}
  This paper studies information flow caused by speculation mechanisms
  in hardware and software.  The Spectre attack shows that there are
  practical information flow attacks which use an interaction of
  dynamic security checks, speculative evaluation and cache timing.
  Previous formal models of program execution have not been designed
  to model speculative evalutation, and so do not capture attacks such
  as Spectre. In this paper, we propose a model based on pomsets which
  is designed to model speculative evaluation. The model provides a
  compositional semantics for a simple shared-memory concurrent
  language, which captures features such as data and control
  dependencies, relaxed memory and transactions. We provide models for
  existing information flow attacks based on speculative evaluation
  and transactions, and new information flow attacks on compiler
  optimizations. The new attacks are experimentally validated against
  gcc and clang.  A simple temporal logic provides reasoning principals,
  including composition and proofs using invariants.
\end{abstract}

\maketitle

\section{Introduction}

This paper studies information flow caused by speculation mechanisms
in hardware and software.

Information flow provides a formal
foundation for end-to-end security.  Informally, a program is secure
if there is no observable dependency of low-security outputs on high-security inputs.
The precise formalization of this intuitive idea has been the topic of
extensive research \cite{Sabelfeld:2006:LIS:2312191.2314769}, encompassing a variety of language
features such as non-determinism~\cite{Wittbold1990InformationFI},
concurrency~\cite{Smith:1998:SIF:268946.268975}, reactivity~\cite{O'Neill:2006:ISI:1155442.1155677}, and
probability~\cite{Gray:1992:TMF:2699806.2699811}. The static and dynamic enforcement
of these definitions in general purpose languages~\cite{myers-popl99} has % also
% been studied extensively and has
influenced language design and implementation.

A key parameter in defining information flow is the \emph{observational power} of the attacker model. Whereas the classical
input-output behavior is often an adequate foundation,
it has long been known~\cite{Lampson:1973:NCP:362375.362389,Biswas:2017:STC:3058791.3023872} that side-channels that leak
information arise from other observables such as execution time and
power consumption.
Recently, the Spectre family of attacks~\cite{DBLP:journals/corr/abs-1801-01203} has
shown that branch prediction, in conjunction with cache-timing side-channels,
allows adversaries to bypass dynamic security checks.

\citet{Chien:2018} argues that Spectre-like attacks ``extend the functional
specification of the architecture to include its detailed performance'' and
thus ``making strong assurances of application security on a computing system
requires detailed performance information.''
This approach has been pursued in the information flow literature, by
enriching language semantics with observables such as execution time and  power consumption
\cite{Zhang:2012:LCM:2345156.2254078,hyperflow}.

In this paper, we adopt the opposite approach, attempting to understand
Spectre-like attacks as \emph{abstractly} as possible and thus to reveal the
``essence'' of Spectre.  We develop a novel model of \emph{speculative
  evaluation} and show that it sufficient to both capture known attacks and
predict new attacks.  Our model is defined at the \emph{language} level,
rather than the hardware level; thus, we do not model micro-architectural
details such as caches or timing, as in
\cite{Zhang:2012:LCM:2345156.2254078,hyperflow}.
% We try to give as
% simple a model as possible, while still capturing shared-memory concurrency
% and speculation.


There are several sources of speculative evaluation in modern computer
systems, intended to improve performance without effecting the observable
behavior of the program: Failed speculations are meant to be
undetectable. Yet, Spectre-like attacks show that failed speculations are not
always undetectable.  Our model provides a unifying mechanism to understand
these sources of speculation.  Because failed speculations are part of the
model, it is easily enriched with operators to detect operations that occur
in failed speculations.
% so that failed speculation does not affect the input-output behavior
% of the program, but may affect other observable behavior, opening an opportunity
% for side-channels:
\begin{itemize}
\item Relaxed memory models
  \cite{SparcV9,Manson:2005:JMM:1047659.1040336,Boehm:2008:FCC:1375581.1375591,DBLP:conf/popl/ZhaoNMZ12}
  allow speculative execution to varying degrees. Relaxed execution
  is known affect the validity of information flow analyses
  \cite{6957104,Vaughan:2012:SIF}.  More troubling, relaxed memory model
  allow for the observation of control and data dependencies. This creates an
  opportunity for information flows caused by optimizing compilers, whose
  behavior is driven by dependency analysis.  Our basic model captures this
  dependency analysis precisely.
  %% For example,
  %% $(\IF(\aReg)\THEN \aLoc\GETS1 \ELSE \aLoc\GETS1 \FI)$ can be optimized to
  %% $(\aLoc\GETS1)$, whereas  %% $(\IF(\aReg)\THEN \aLoc\GETS1 \ELSE \aLoc\GETS2 \FI)$ cannot be so
  %% optimized.  
\item Pipelined micro-architectures use \emph{branch prediction}
  to speculatively execute the result of
  a conditional jump or indirect jump instruction.
  Spectre \cite{DBLP:journals/corr/abs-1801-01203} exploits
  cache timing to detect the operations performed in a mispredicted branch
  before being flushed from the pipeline.  We capture Spectre by enriching
  our language with a single operation that allows one to test whether a
  location has been touched, even in a failed speculation.
  %% This means,
  %% for example, that a single execution of
  %% $(\IF(\aExp)\THEN \aCmd \ELSE \bCmd \FI)$ may depend on both $\aCmd$ and
  %% $\bCmd$.  This differs from the standard semantics of the conditional, in
  %% which executions of $\aCmd$ and $\bCmd$ are disjoint.
\item Some microprocessors support transactional
  memory~\cite{ChongSW18}, where aborted transactions are meant to be
  unobservable.  \textsc{Prime+Abort}
  \cite{DBLP:conf/uss/DisselkoenKPT17} uses cache timing to detect the
  operations performed in aborted transaction.  We capture \textsc{Prime+Abort} by enriching
  our language with transactions that abort when a location used by the
  transaction is touched outside the transaction, even in a failed speculation.
\end{itemize}

%% This line of research was initiated by~\citet{Zhang:2012:LCM:2345156.2254078}.  
%% Whereas they explore static annotations to address side channels in the context of hardware description languages, we explore a model of programs
%% that captures enough detail to reveal and analyze the presence of side
%% channels revealed by speculative execution.  
%

Our model is based on \emph{partially ordered multisets}~\cite{GISCHER1988199,Plotkin:1997:TSP:266557.266600}
(``pomsets''), whose labels are given by read and write actions. These can be
visualized as a graph where the edges indicate dependencies, for example
$(\aReg\GETS\aLoc\SEMI \bLoc\GETS1\SEMI \cLoc\GETS\aReg+1)$
has an execution modeled by the pomset:
\[\begin{tikzpicture}[node distance=1em]
  \event{rx1}{\DR{\aLoc}{1}}{}
  \event{wy1}{\DW{\bLoc}{1}}{right=of rx1}
  \event{wz2}{\DW{\cLoc}{2}}{right=of wy1}
  \po[out=25,in=155]{rx1}{wz2}
\end{tikzpicture}\]
The edge from $(\DR{\aLoc}{1})$ to $(\DW{\cLoc}{2})$ indicates a
data dependency. Since there is no dependency between
$(\DW{\bLoc}{1})$ and $(\DW{\cLoc}{2})$, the write actions may
take place in either order.  Such reorderings may arise in
hardware (for example, caching) or in the compiler (for example,
instruction reordering).

The novel aspect of the model is that events have
\emph{preconditions} which may be false. These are used in giving the
semantics of conditionals and transactions, modeling failed branch
prediction and aborted transactions. For example the program
$(\IF(\aLoc)\THEN \bLoc\GETS1\SEMI\cLoc\GETS1 \ELSE \bLoc\GETS2\SEMI\cLoc\GETS1\FI)$
has an execution:
\[\begin{tikzpicture}[node distance=1em]
  \event{rx1}{\DR{\aLoc}{1}}{}
  \event{wy1}{\DW{\bLoc}{1}}{right=of rx1}
  \nonevent{wy2}{\DW{\bLoc}{2}}{below=of wy1}
  \event{wz1}{\DW{\cLoc}{1}}{right=of wy1}
  \po{rx1}{wy1}
  \po{rx1}{wy2}
\end{tikzpicture}\]
The edges from $(\DR{\aLoc}{1})$ to $(\DW{\bLoc}{1})$ and
$(\DW{\bLoc}{2})$ indicate control dependencies. The presence of
a crossed out $(\DW{\bLoc}{2})$ indicates an event with an unsatisfiable precondition,
modeling an unsuccessful speculation.
Since the $(\DW{\cLoc}{1})$ action is performed on both branches of the conditional,
there is no control dependency from $(\DR{\aLoc}{1})$.  Indeed, from an information-flow perspective,
this refined treatment of dependencies in conditionals identifies a novel distinguishing feature of our model, namely that the traditional conditional is a self-composition operator in the sense of~\cite{Barthe:2004:SIF:1009380.1009669}.  

There do exist models of programs which include speculation, notably
the Java Memory Model~\cite{Manson:2005:JMM:1047659.1040336}, and the
generative~\cite{Jagadeesan:2010:GOS:2175486.2175503} and
promising~\cite{DBLP:conf/popl/KangHLVD17} operational semantics for
relaxed memory.  In all of these models a valid execution is defined
with reference to other possible executions of the program. These
models are not, however, designed for modeling Spectre-style attacks
on speculation. For example all of these models will consider the
straight-line code:
\[
  r\GETS x\SEMI s\GETS\SEC \SEMI
  a[r]\GETS 1
\]
to be the same as the conditional code:
\[\begin{array}{ll}
  r\GETS x\SEMI s\GETS\SEC \SEMI \\[\jot]
  \IF(r\EQ s) \THEN a[s]\GETS 1 \ELSE a[r]\GETS 1 \FI
\end{array}\]
and indeed an optimizing compiler might choose to rewrite
either of these programs to be the other.

An attacker can mount a Spectre-style attack on the
conditional code, for example by setting $x$ to be~$0$,
flushing the cache,
executing the program, then using timing effects to
determine if $a[1]$ is in the cache. If it is, then $\SEC$
must have been~$1$. This attack is not possible against
the straight-line code, and so any model trying to
capture Spectre must distinguish them.

Most definitions of non-interference will say that in both
programs, there is no observable dependency of the low-security
outputs ($a$) on the high-security inputs ($\SEC$) and so both programs
are safe.
  The only existing models of
non-interference which capture this information flow are ones such
as~\cite{Zhang:2012:LCM:2345156.2254078} which model
micro-architectural features such as caching and timing.

In our model, the straight-line and conditional programs are not equated, since the conditional code has the execution:
\[\begin{tikzpicture}[node distance=1em]
  \event{rx0}{\DR{\aLoc}{0}}{}
  \event{rs1}{\DR{\SEC}{1}}{below=of rx0}
  \event{wa01}{\DW{a[0]}{1}}{right=3em of rx0}
  \nonevent{wa11}{\DW{a[1]}{1}}{below=of wa01}
  \po{rx0}{wa01}
  \po{rs1}{wa01}
  \po{rx0}{wa11}
  \po{rs1}{wa11}
\end{tikzpicture}\]
which is not matched in the straight-line code.

Static analyses such as the Smith-Volpano type
system~\cite{Smith:1998:SIF:268946.268975} will reject the conditional
program, due to $a[s]\GETS 1$, in which a low-security assignment depends on
a high-security variable.  We show how to circumvent such analyses in
\S\ref{sec:spectre}.


The model in this paper leads to new attacks on optimizing
compilers~(\S\ref{sec:info-flow-attack} and~\S\ref{sec:dse}) which
were discovered as a consequence of building the model. A natural
question is whether these attacks are an artifact of the model, or if
they can be mounted in practice? We mounted the attacks on gcc and
clang, where they succeeded in leaking a $\SEC$ as long as the secret
was a constant known at compile time. By itself this is not too
worrying, since secrets are not normally static constants. If the same
attacks could be mounted against Just-In-Time~(JIT) compilers, this
is potentially significant, as secrets are often known at JIT-compile
time. Fortunately, our attempts to mount the attacks against SpiderMonkey,
V8 and HotSpot did not succeed.

The novel contributions of this paper are:
\begin{itemize}

\item a compositional model of program execution that includes speculation
  (\S\ref{sec:model} and \S\ref{sec:semantics}),

\item examples showing how the model can be applied,
  including existing information flow attacks on
  hardware and transactional memory, and new attacks on optimizing compilers (\S\ref{sec:examples}), and

\item experimental evidence about how practical it is to mount
  the new class of attacks (\S\ref{sec:experiments}).

\end{itemize}
Readers who wish to focus on the impact of the model can skip to \S\ref{sec:examples}
on first reading, referring to prior sections as needed.

\section{Model}

\subsection{Preliminaries}

We assume:
\begin{itemize}
\item a set of \emph{memory locations} $\Loc$, ranged over by
  $\aLoc$ and $\bLoc$,
\item a set of \emph{registers} $\Reg$, ranged over by
  $\aReg$ and $\bReg$,
\item a set of \emph{values} $\Val$, ranged over by
  $\aVal$ and $\bVal$,
\item a set of \emph{expressions} $\Exp$, ranged over by
  $\aExp$ and $\bExp$,
\item a set of \emph{logical formulae} $\Formulae$, ranged over by
  $\aForm$ and $\bForm$, and
\item a set of \emph{actions} $\Act$, ranged over by $\aAct$ and $\bAct$,
\end{itemize}
such that:
\begin{itemize}
\item values include at least the constants $0$ and $1$,
\item expressions include at least registers and values,
\item expressions are closed under substitution, written $\aExp[\bExp/\aReg]$,
\item formulae include at least $\TRUE$, $\FALSE$, and equalities of the form $(\aExp=\bExp)$ and $(\aLoc=\bExp)$,
\item formulae are closed under negation, conjunction, disjunction,
\item formulae are closed under substitution, written $\aForm[\bExp/\aLV]$, and
\item actions include at least \emph{reads} of the form $(\DR{\aLoc}{\aVal})$
  and \emph{writes} of the form $(\DW{\aLoc}{\aVal})$.
\end{itemize}
Let the set of \emph{lvalues} be $\LVal=(\Loc\cup\Reg)$, ranged over by $\aLV$ and $\bLV$.

\subsection{Pomsets}

\begin{definition}
  A \emph{pomset} $(\Event, {\le}, \labelling)$ with alphabet $\Alphabet$
  is a partial order $(\Event, {\le})$ together with a function
  $\labelling: \Event \fun \Alphabet$.
\end{definition}
Going forward, we fix the alphabet $\Alphabet=(\Formulae\times\Act)$.
We will write $(\aForm \mid \aAct)$ for the pair $(\aForm,\aAct)$,
$\aAct$ for $(\TRUE,\aAct)$ and $\NEVER\aAct$ for $(\FALSE,\aAct)$.

We visualize a pomset as a graph where the nodes are drawn from
$\Event$, each node $\aEv$ is labelled with $\labelling(\aEv)$,
and an edge $\bEv \rightarrow \aEv$ corresponds to an ordering
$\bEv\le\aEv$. For example:

\[\begin{tikzpicture}[node distance=1em]
  \event{x1}{\DR{\aLoc}{1}}{}
  \nonevent{y0}{\DW{\bLoc}{0}}{below left=of x1}
  \event{y1}{\DW{\bLoc}{1}}{below right=of x1}
  \po{x1}{y0}
  \po{x1}{y1}
\end{tikzpicture}\]
is a visualization of the pomset where:
\[
  0 \le 1 \quad
  0 \le 2 \quad
  \labelling(0) = (\TRUE, \DR{\aLoc}{1}) \quad
  \labelling(1) = (\FALSE, \DW{\bLoc}{0}) \quad
  \labelling(2) = (\TRUE, \DW{\bLoc}{1}) \quad
\]
As we shall see, this is a possible execution of the
program:
\begin{verbatim}
    r := x; if (r) { y := 1; } else { y := 0; }
\end{verbatim}

\begin{definition}
  An \emph{rf-pomset} is a pomset together with a
  $\RF \subseteq \Event\times\Event$ such that for any $(\bEv,\aEv) \in \RF$: 
  \begin{itemize}
  \item $\bEv < \aEv$,
  \item $\labelling(\bEv) = (\aForm \mid \DW{\aLoc}{\aVal})$
    and $\labelling(\aEv) = (\bForm \mid \DR{\aLoc}{\aVal})$, and
  \item there is no $\bEv < \cEv < \aEv$ such that
    $\labelling(\cEv) = (\cForm \mid \DW{\aLoc}{\bVal})$.
  \end{itemize}
\end{definition}
We visualize rf-pomsets by drawing a dashed edge between edges in $\RF$,
for example:
\[\begin{tikzpicture}[node distance=1em]
  \event{wx1}{\DW{\aLoc}{1}}{}
  \event{x1}{\DR{\aLoc}{1}}{right=5em of wx1}
  \nonevent{y0}{\DW{\bLoc}{0}}{below left=of x1}
  \event{y1}{\DW{\bLoc}{1}}{below right=of x1}
  \rf{wx1}{x1}
  \po{x1}{y0}
  \po{x1}{y1}
\end{tikzpicture}\]
As we shall see, this is a possible execution of the
program:
\begin{verbatim}
    x := 1;  ||  r := x; if (r) { y := 1; } else { y := 0; }
\end{verbatim}
\begin{definition}
  An rf-pomset is $\aLoc$-closed if
  for $\aEv\in\Event$ with $\labelling(\aEv)=(\aForm \mid \aAct)$:
  \begin{itemize}
  \item $\aForm$ is independent of $\aLoc$, and
  \item if $\aAct$ is an $\aLoc$ read action, then there is a $\bEv$ with $(\bEv,\aEv) \in \RF$.
  \end{itemize}
\end{definition}

\subsection{Sets of pomsets}

Let $\aPSS_1 \sqcup \aPSS_2$ be the set $\aPSS_0$ where $\aPS_0 \in \aPSS_0$
whenever there are $\aPS_1 \in \aPSS_1$ and  $\aPS_2 \in \aPSS_2$ such that:
\begin{itemize}
\item $\Event_0 = \Event_1 \cup \Event_2$,
\item $\RF_0 = \RF_1 \cup \RF_2$,
\item if $\aEv \le_1 \bEv$ or $\aEv \le_2 \bEv$ then $\aEv \le_0 \bEv$,
\item if $\labelling_0(\aEv) = (\aForm_0 \mid \aAct)$ then either:
  \begin{itemize}
  \item $\labelling_1(\aEv) = (\aForm_1 \mid \aAct)$ and $\labelling_2(\aEv) = (\aForm_2 \mid \aAct)$
    and $\aForm_0$ implies $\aForm_1 \lor \aForm_2$,
  \item $\labelling_1(\aEv) = (\aForm_1 \mid \aAct)$ and $\aEv \not\in \Event_2$
    and $\aForm_0$ implies $\aForm_1$, or
  \item $\labelling_2(\aEv) = (\aForm_2 \mid \aAct)$ and $\aEv \not\in \Event_1$
    and $\aForm_0$ implies $\aForm_2$.
  \end{itemize}
\end{itemize}
Let $\aPSS_1 \parallel \aPSS_2$ be defined the same as $\aPSS_1 \sqcup \aPSS_2$ except that:
\begin{itemize}
\item $\RF_0 \supseteq \RF_1 \cup \RF_2$, and
 for any $\bEv, \aEv \in \Event_i$, if $(\bEv,\aEv)\in\RF_0$ then $(\bEv,\aEv)\in\RF_i$.
\end{itemize}
Let $(\phi \mid \DW\aLoc\aVal) \prefix \aPSS$ be the set $\aPSS'$ where $\aPS'\in\aPSS'$ whenever
there is $\aPS\in\aPSS$ such that:
\begin{itemize}
\item $\Event' = \Event \cup \{0\}$,
\item $\RF' = \RF$,
\item if $\bEv \le \aEv$ then $\bEv \le' \aEv$,
\item $\labelling'(0) = (\aForm, \DW\aLoc\aVal)$, where $\bForm$ implies $\aForm$, and
\item if $\labelling(\aEv) = (\bForm \mid \aAct)$ then $\labelling'(\aEv) = (\bForm \mid \aAct)$.
\end{itemize}
Let $(\phi \mid \DR\aLoc\aVal) \prefix \aPSS$ be the set $\aPSS'$ where $\aPS'\in\aPSS'$ whenever
there is $\aPS\in\aPSS$ such that:
\begin{itemize}
\item $\Event' = \Event \cup \{0\}$,
\item $\RF' = \RF$,
\item if $\bEv \le \aEv$ then $\bEv \le' \aEv$,
\item $\labelling'(0) = (\bForm, \DR\aLoc\aVal)$, where $\bForm$ implies $\aForm$, and
\item if $\labelling(\aEv) = (\bForm \mid \aAct)$ then
  $\labelling'(\aEv) = (\bForm' \mid \aAct)$,
  $\bForm'$ implies $\bForm[\aVal/\aLoc]$, and
  $0 \le' \aEv$ or $\bForm'$ implies $\bForm$.
\end{itemize}
Let $\aPSS[\aExp/\aLV]$ be the set $\aPSS'$ where $\aPS'\in\aPSS'$ whenever
there is $\aPS\in\aPSS$ such that:
\begin{itemize}
\item $\Event' = \Event$,
\item $\RF' = \RF$,
\item if $\bEv \le \aEv$ then $\bEv \le' \aEv$, and
\item if $\labelling(\aEv) = (\bForm \mid \aAct)$ then $\labelling'(\aEv) = (\bForm[\aExp/\aLV] \mid \aAct)$.
\end{itemize}
Let $(\aForm \mid \aPSS)$ be the subset of $\aPSS$ such that $\aPS\in\aPSS$ whenever:
\begin{itemize}
\item if $\labelling(\aEv) = (\bForm \mid \aAct)$ then $\aForm$ implies $\bForm$.
\end{itemize}
Let $(\nu\aLoc\st\aPSS)$ be the subset of $\aPSS$ such that $\aPS\in\aPSS$ whenever
$\aPS$ is $\aLoc$-closed.

\subsection{Semantics of programs}

Define:
\begin{eqnarray*}
  \sem{\SKIP} & = & \{ \emptyset \} \\
  \sem{\aLoc\GETS\aExp\SEMI C} & = & \bigcup_\aVal\; (\aExp=\aVal \mid \DW\aLoc\aVal) \prefix \sem{C}[\aExp/\aLoc] \\
  \sem{\aReg\GETS\aLoc\SEMI C} & = & \bigcup_\aVal\; (\TRUE \mid \DR\aLoc\aVal) \prefix \sem{C}[\aLoc/\aReg] \\
  \sem{\IF \aExp \THEN C \ELSE D} & = & (\aExp \neq 0 \mid \sem{C}) \sqcup (\aExp=0 \mid \sem{D}) \\
  \sem{C \PAR D} & = & \sem{C} \parallel \sem{D} \\
  \sem{\VAR\aLoc\SEMI C} & = & \nu \aLoc \st \sem{C}
\end{eqnarray*}

\section{Examples}
\label{sec:examples}

\subsection{Sequential memory accesses}
\label{sec:sequential-memory}

In the semantics of memory, there are two very different ways memory
can be accessed: sequentially or concurrently. These are modelled
differently, since hardware and compilers give very different
guarantees about their behaviour.
In this section, we discuss the sequential semantics, and leave
the concurrent semantics to \S\ref{sec:concurrent-memory}.

Consider the program $(\aLoc\GETS0\SEMI \bLoc\GETS\aLoc+1)$.  One execution of
this program is where the write to $y$ uses the sequential value of
$x$, which is $0$:
\[\begin{tikzpicture}[node distance=1em]
  \event{wx0}{\DW{x}{0}}{}
  \event{wy1}{\DW{y}{1}}{right=of wx0}
\end{tikzpicture}\]
To see how this execution is modelled, we first
expand out the syntax sugar to get the program
$(\aLoc\GETS0\SEMI \aReg\GETS\aLoc\SEMI \bLoc\GETS\aReg+1\SEMI\SKIP)$
Now $\sem{\SKIP}$ is just $\{\emptyset\}$, and
$\sem{y \GETS r+1\SEMI \SKIP}$ includes:
\[
   (r+1=1 \mid (\DW y1) \prefix \sem{\SKIP}[1/y])
\]
which contains the pomset:
\[\begin{tikzpicture}[node distance=1em]
  \event{wy1}{r+1=1 \mid \DW{y}{1}}{}
\end{tikzpicture}\]
expressing that this program can write $1$ to $y$,
as long as the precondition $(r+1=1)$ is satisfied.
Now $\sem{r \GETS x\SEMI y \GETS r+1\SEMI \SKIP}$
has two cases, the sequential case
(which does not introduce a read action)
and the concurrent case (which does).
For the moment, we are interested in the sequential case, which is:
\[
   \sem{y \GETS r+1\SEMI \SKIP}[x/r]
\]
which contains the pomset:
\[\begin{tikzpicture}[node distance=1em]
  \event{wy1}{x+1=1 \mid \DW{y}{1}}{}
\end{tikzpicture}\]
In this pomset, the precondition is $(x+1=1)$, which specifies a property
of the thread-local value of $x$.
Finally $\sem{x \GETS 0\SEMI r \GETS x\SEMI y \GETS r+1\SEMI \SKIP}$ includes:
\[
   (0=0 \mid (\DW x0) \prefix \sem{r \GETS x\SEMI y \GETS r+1\SEMI \SKIP}[0/x])
\]
which contains the pomset:
\[\begin{tikzpicture}[node distance=1em]
  \event{wx0}{0=0 \mid \DW{x}{0}}{}
  \event{wy1}{0+1=1 \mid \DW{y}{1}}{right=of wx0}
\end{tikzpicture}\]
all of whose preconditions are tautologies, so this has the expected behaviour:
\[\begin{tikzpicture}[node distance=1em]
  \event{wx0}{\DW{x}{0}}{}
  \event{wy1}{\DW{y}{1}}{right=of wx0}
\end{tikzpicture}\]
Note that there is no
requirement of order between $(\DW x0)$ and $(\DW y1)$.

This example demonstrates how preconditions
capture the sequential semantics of memory.
In an execution containing an event with label
$(\aForm \mid \aAct)$, one way the precondition $\aForm$
can be discharged is by an assignment $\aLoc\GETS\aExp$,
which performs a substitution $[\aExp/\aLoc]$.
This is a variant of the Hoare semantics for
assignment, where if $\aCmd$ has preconditon $\aForm$
then $\aLoc\GETS\aExp\SEMI\aCmd$ has precondition
$\aForm[\aExp/\aLoc]$.

\subsection{Concurrent memory accesses}
\label{sec:concurrent-memory}

We now turn to the case of concurrent accesses to memory.
Consider a concurrent version of the program from \S\ref{sec:sequential-memory}:
$(\aLoc\GETS1 \PAR \bLoc\GETS\aLoc+1)$.
One execution of this program is where the write to $y$
performs a concurrent read of $x$:
\[\begin{tikzpicture}[node distance=1em]
  \event{wx1}{\DW{x}{1}}{}
  \event{rx1}{\DR{x}{1}}{right=2.5em of wx1}
  \event{wy2}{\DW{y}{2}}{right=of rx1}
  \rf{wx1}{rx1}
  \po{rx1}{wy2}
\end{tikzpicture}\]
To see how this execution is modelled, we first
expand out the syntax sugar to get the program
$(\aLoc\GETS1\SEMI\SKIP \PAR \aReg\GETS\aLoc\SEMI \bLoc\GETS\aReg+1\SEMI\SKIP)$.
As before, $\sem{y \GETS r+1\SEMI \SKIP}$ includes:
\[
   (r+1=2 \mid (\DW y2) \prefix \sem{\SKIP}[2/y])
\]
which contains the pomset:
\[\begin{tikzpicture}[node distance=1em]
  \event{wy2}{r+1=2 \mid \DW{y}{2}}{}
\end{tikzpicture}\]
As before, $\sem{r \GETS x\SEMI y \GETS r+1\SEMI \SKIP}$ has two cases.
We are now interested in the concurrent case, which includes:
\[
   (\DR x1) \prefix \sem{y \GETS r+1\SEMI \SKIP}[x/r]
\]
which contains the pomset:
\[\begin{tikzpicture}[node distance=1em]
  \event{rx1}{\DR{x}{1}}{}
  \event{wy2}{\DW{y}{2}}{right=of rx1}
  \po{rx1}{wy2}
\end{tikzpicture}\]
Note that $(\DR x1)$ reads $1$ from $x$, and while
$(x+1=2)[1/x]$ is a tautlogy,
$(x+1=2)$ is not,
and so there is a dependency
$(\DR x1) < (\DW y2)$.

Now, $\sem{x \GETS 1\SEMI \SKIP}$ includes the pomset:
\[\begin{tikzpicture}[node distance=1em]
  \event{wx1}{\DW{x}{1}}{}
\end{tikzpicture}\]
and so $\sem{x \GETS 1\SEMI \SKIP \PAR r \GETS x\SEMI y \GETS r+1\SEMI \SKIP}$ includes:
\[\begin{tikzpicture}[node distance=1em]
  \event{wx1}{\DW{x}{1}}{}
  \event{rx1}{\DR{x}{1}}{right=2.5em of wx1}
  \event{wy2}{\DW{y}{2}}{right=of rx1}
  \rf{wx1}{rx1}
  \po{rx1}{wy2}
\end{tikzpicture}\]
as expected, including a reads-from dependency
$(\DW x1) < (\DR x1)$.

This example demonstrates how read and write events
capture the concurrent semantics of memory.
In an execution containing an event with label
$(\DR \aLoc\aVal)$, if the execution is
$\aLoc$-closed, then there must be an event
it reads from, for example one labelled
$(\DW \aLoc\aVal)$.

\subsection{Independent writes}

Consider an example with two independent writes
$(\aLoc\GETS1\SEMI \bLoc\GETS2)$.
This has semantics including:
\[
  (\DW x1) \prefix
    (\DW y2) \prefix
      \{\emptyset\}
\]
One of the executions this contains is:
\[\begin{tikzpicture}[node distance=1em]
  \event{wx1}{\DW{x}{1}}{}
  \event{wy2}{\DW{y}{2}}{right=of wx1}
  \po{wx1}{wy2}
\end{tikzpicture}\]
but it also contains:
\[\begin{tikzpicture}[node distance=1em]
  \event{wx1}{\DW{x}{1}}{}
  \event{wy2}{\DW{y}{2}}{right=of wx1}
\end{tikzpicture}\]
and:
\[\begin{tikzpicture}[node distance=1em]
  \event{wx1}{\DW{x}{1}}{}
  \event{wy2}{\DW{y}{2}}{right=of wx1}
  \po{wy2}{wx1}
\end{tikzpicture}\]
since there is no requirement that
$(\DW{x}{1}) < (\DW{y}{2})$.

Thus, the semantics of $(\aLoc\GETS1\SEMI \bLoc\GETS2)$
is the same as the semantics of $(\bLoc\GETS2\SEMI \aLoc\GETS1)$.

\subsection{Independent reads and writes}

Whereas write prefixing introduces weak dependencies on events which write
to the same location, read prefixing introduces strong dependencies on
preconditions which depend on the location being read. For example
in \S\ref{sec:concurrent-memory} we saw that the program
$(\bLoc\GETS\aLoc+1)$ includes the pomset:
\[\begin{tikzpicture}[node distance=1em]
  \event{rx1}{\DR{x}{1}}{}
  \event{wy2}{\DW{y}{2}}{right=of rx1}
  \po{rx1}{wy2}
\end{tikzpicture}\]
but since $(x+1=2)$ depends on $x$,
we have the requirement that $(\DR{x}{1}) \le (\DW{y}{2})$.

This is in contrast to the program 
$(\aReg\GETS\aLoc\SEMI \bLoc\GETS\aReg+2-\aReg)$.
Since $(x+2-x=2)$ is independent of $x$
(at least for integer arithmetic)
this contains:
\[\begin{tikzpicture}[node distance=1em]
  \event{rx1}{\DR{x}{1}}{}
  \event{wy2}{\DW{y}{2}}{right=of rx1}
\end{tikzpicture}\]
and so the semantics of $(\aReg\GETS\aLoc\SEMI \bLoc\GETS\aReg+2-\aReg)$
is the same as the semantics of $(\bLoc\GETS2\SEMI \aReg\GETS\aLoc)$.

Note this this example shows that we are not just dealing
with a syntactic notion of dependency, which is common
in hardware models of memory. In syntactic dependency,
since $r$ occurs free in $(y\GETS r+2-r)$, there would be
a dependency between $(r\GETS x)$ and $(y\GETS r+2-r)$.
In contrast, this model is based on logical implication,
which can be interpreted semantically.

\subsection{Control dependencies}
\label{sec:control-dep}

Conditionals introduce control dependencies, for example consider the program:
\[
  \aReg\GETS\cLoc\SEMI
  \IF(\aReg)\THEN \aLoc\GETS1 \ELSE \bLoc\GETS2 \FI
\]
This includes executions in which the false branch is taken:
\[\begin{tikzpicture}[node distance=1em]
  \event{rz0}{\DR{z}{0}}{}
  \nonevent{wx1}{\DW{x}{1}}{right=of rz0}
  \event{wy2}{\DW{y}{2}}{right=of wx1}
  \po{rz0}{wx1}
  \po[out=30,in=150]{rz0}{wy2}
\end{tikzpicture}\]
and ones where the true branch is taken:
\[\begin{tikzpicture}[node distance=1em]
  \event{rz1}{\DR{z}{1}}{}
  \event{wx1}{\DW{x}{1}}{right=of rz1}
  \nonevent{wy2}{\DW{y}{2}}{right=of wx1}
  \po{rz1}{wx1}
  \po[out=30,in=150]{rz1}{wy2}
\end{tikzpicture}\]
In both cases, we record the actions in the branch that was
not taken. This is a novel feature of this model, and is
intended to capture speculative evaluation. In \S\ref{sec:spectre}
we will show how this model captures Spectre-like information
flow attacks, once the attacker is provided with the ability to
observe such speculations.

To see how these executions are modelled, consider the semantics of
$\sem{x\GETS 1\SEMI\SKIP}$, which contains any pomset of the form:
\[\begin{tikzpicture}[node distance=1em]
  \event{wx1}{\aForm \mid \DW{x}{1}}{}
\end{tikzpicture}\]
in particular it contains:
\[\begin{tikzpicture}[node distance=1em]
  \event{wx1}{r\neq0 \mid \DW{x}{1}}{}
\end{tikzpicture}\]
Similarly $\sem{y\GETS 2\SEMI\SKIP}$ contains:
\[\begin{tikzpicture}[node distance=1em]
  \event{wy2}{r=0 \mid \DW{y}{2}}{}
\end{tikzpicture}\]
and so $\sem{\IF(r)\THEN x\GETS 1\SEMI\SKIP \ELSE y\GETS 2\SEMI\SKIP \FI}$
contains:
\[\begin{tikzpicture}[node distance=1em]
  \event{wx1}{r\neq0 \mid \DW{x}{1}}{}
  \event{wy2}{r=0 \mid \DW{y}{2}}{right=of wx1}
\end{tikzpicture}\]
Now, the semantics of concurrent read performs substitutions, for example:
\[\begin{tikzpicture}[node distance=1em]
  \event{rz0}{\DR{z}{0}}{}
  \event{wx1}{0\neq0 \mid \DW{x}{1}}{right=of rz0}
  \event{wy2}{0=0 \mid \DW{y}{2}}{right=of wx1}
  \po{rz0}{wx1}
  \po[out=25,in=155]{rz0}{wy2}
\end{tikzpicture}\]
which gives the required pomset:
\[\begin{tikzpicture}[node distance=1em]
  \event{rz0}{\DR{z}{0}}{}
  \nonevent{wx1}{\DW{x}{1}}{right=of rz0}
  \event{wy2}{\DW{y}{2}}{right=of wx1}
  \po{rz0}{wx1}
  \po[out=30,in=150]{rz0}{wy2}
\end{tikzpicture}\]
Note that the precondition $r=0$ is dependent on $r$,
and so there is a dependency $(\DR z0) < (\DW y2)$,
modelling the control dependency introduced by the conditional.

\subsection{Control independencies}

In most models of control dependencies, the dependency relation
is syntactic, based on whether the action occurs inside syntactically
inside a conditional. In contrast, the notion in this model is
semantic: if an action can occur on both sides of a conditional,
there is no control dependency. Consider a variant of the example
from \S\ref{sec:control-dep}:
\[
  \aReg\GETS\cLoc\SEMI
  \IF(\aReg)\THEN \aLoc\GETS1 \ELSE \aLoc\GETS1 \FI
\]
This has the expected execution in which the control
dependencies exist:
\[\begin{tikzpicture}[node distance=1em]
  \event{rz0}{\DR{z}{0}}{}
  \nonevent{nwx1}{\DW{x}{1}}{right=of rz0}
  \event{wx1}{\DW{x}{1}}{right=of nwx1}
  \po{rz0}{nwx1}
  \po[out=30,in=150]{rz0}{wx1}
\end{tikzpicture}\]
but it also has an execution in which the two writes
of $1$ to $x$ are merged, resulting in no dependency:
\[\begin{tikzpicture}[node distance=1em]
  \event{rz0}{\DR{z}{0}}{}
  \event{wx1}{\DW{x}{1}}{right=of rz0}
\end{tikzpicture}\]
To see how this arises,
consider the definition of $\sem{\IF(r)\THEN x\GETS1\SEMI\SKIP \ELSE x\GETS1\SEMI\SKIP \FI}$:
\[
   \aPSS_1 \parallel \aPSS_2 \quad\mbox{where}\quad
   \aPSS_1 = (r\neq 0 \mid \sem{x\GETS1\SEMI\SKIP})  \quad\mbox{and}\quad
   \aPSS_2 = (r=0 \mid \sem{x\GETS1\SEMI\SKIP})
\]
Now, one pomset in $\aPSS_1$ is:
\[\begin{tikzpicture}[node distance=1em]
  \event{wx1}{r\neq0 \mid \DW{x}{1}}{}
\end{tikzpicture}\]
that is $\aPS_1$ where:
\[
  \Event_1 = \{\aEv\} \quad
  \labelling_1(\aEv) = (r\neq 0, \DW x1)
\]
and similarly, one pomset in $\aPSS_2$ is:
\[\begin{tikzpicture}[node distance=1em]
  \event{wx1}{r=0 \mid \DW{x}{1}}{}
\end{tikzpicture}\]
that is $\aPS_2$ where:
\[
  \Event_2 = \{\aEv\} \quad
  \labelling_2(\aEv) = (r= 0, \DW x1)
\]
Crucuially, in the definition of $\aPSS_1 \parallel \aPSS_2$
there is \emph{no} requirement that $\Event_1$ and $\Event_2$ are disjoint,
and in this case they overlap at $\aEv$. As a result, one pomset in
$\aPSS_1\sqcup\aPSS_2$ is $\aPS_0$ where:
\[
  \Event_0 = \{\aEv\} \quad
  \labelling_0(\aEv) = (r\neq0 \lor r=0, \DW x1)
\]
that is:
\[\begin{tikzpicture}[node distance=1em]
  \event{wx1}{\DW{x}{1}}{}
\end{tikzpicture}\]
Note that this pomset has no precondition dependent on $r$,
since $(r\neq0 \lor r=0)$ does not depend on $r$, which is why
we end up with an execution without a control dependency:
\[\begin{tikzpicture}[node distance=1em]
  \event{rz0}{\DR{z}{0}}{}
  \event{wx1}{\DW{x}{1}}{right=of rz0}
\end{tikzpicture}\]
This semantics captures compiler optimizations which may, for example
merge code executed on both branches of a conditional, or hoist
constant assignments out of loops.

We can now see the counterintuitive behavior of conditionals
in the presence of control dependencies.
There are programs such as
\(
  \IF(\cLoc)\THEN \aLoc\GETS1 \ELSE \aLoc\GETS1 \FI
\)
executions in which  $(\DW x1)$ is independent of $(\DR z1)$:
\[\begin{tikzpicture}[node distance=1em]
  \event{rz1}{\DR{z}{1}}{}
  \event{wx1}{\DW{x}{1}}{right=of rz1}
\end{tikzpicture}\]
while programs such as
\(
  \IF(\cLoc)\THEN \aLoc\GETS1 \ELSE \bLoc\GETS2 \FI
\)
only have executions in which $(\DW x1)$ is dependent on $(\DR z1)$:
\[\begin{tikzpicture}[node distance=1em]
  \event{rz1}{\DR{z}{1}}{}
  \event{wx1}{\DW{x}{1}}{right=of rz1}
  \nonevent{wy2}{\DW{y}{2}}{right=of wx1}
  \po{rz1}{wx1}
  \po[out=30,in=150]{rz1}{wy2}
\end{tikzpicture}\]
so these programs have different dependency relations, depending
on conditional branches that were not taken. In \S\ref{sec:info-flow-attack}
we shall see that this has security implications, since relaxed
memory can observe dependency. The attack is similar to Spectre, so
we shall take a detour to see how Spectre can be modeled in this
setting.

\subsection{Spectre}
\label{sec:spectre}

We give a simplified model of Spectre attacks, ignoring the details of
timing.  In this model, we extend programs with the ability to tell
whether a memory location has been touched (in practice this is
implemented using timing attacks on the cache). For example,
we can model Spectre by:
\[\begin{array}{l}
  \VAR a\SEMI \IF(\CANREAD(\SEC))\THEN a[\SEC]\GETS1
  \brELIF(\TOUCHED a[0])\THEN x\GETS0
  \brELIF(\TOUCHED a[1])\THEN x\GETS1 \FI
\end{array}\]
This is a low-security program, which is attempting to discover the
value of a high-security variable $\SEC$. The low-security program
is allowed to attempt to escalate its privileges by checking that it is
allowed to read a high-security variable:
\[
  \IF(\CANREAD(\SEC))\THEN \cdots\mbox{code allowed to read $\SEC$}\cdots
  \ELSE \cdots \FI
\]
In this case, $\CANREAD(\SEC)$ is false, so the fallback code
is executed. Unfortunately, the escalated code is speculatively
evaluated, which allows information to leak by testing for which
memory locations have been touched.

We model the $\TOUCHED$ test by introducing a new action
$(\DT{\aLoc})$ and defining:
\begin{eqnarray*}
  \sem{\IF (\TOUCHED\aLoc) \THEN \aCmd \ELSE \bCmd \FI} & = & ((\DT\aLoc) \prefix \sem{\aCmd}) \cup \sem{\bCmd}
\end{eqnarray*}
The additional requirement we need to add for $\aLoc$-closure is:
\begin{itemize}
\item if $\labelling(\aEv)=(\aForm \mid \DT{\aLoc})$
  then there is $\bEv<\aEv$
  where $\bEv$ reads or writes $\aLoc$.
\end{itemize}
Note that there is no requirement that $\bEv$ be satisfiable,
and indeed Spectre has the execution:
\[\begin{tikzpicture}[node distance=1em]
  \nonevent{rs}{\DR{\SEC}{1}}{}
  \nonevent{wa}{\DW{a[1]}{1}}{right=of rs}
  \event{ta}{\DT{a[1]}}{right=of wa}
  \event{wx}{\DW{x}{1}}{right=of ta}
  \po{rs}{wa}
  \po{wa}{ta}
  \po{ta}{wx}
\end{tikzpicture}\]
Putting this in parallel with a high-security write to \verb|SECRET| gives:
\[\begin{tikzpicture}[node distance=1em]
  \event{ws}{\DW{\SEC}{1}}{}
  \nonevent{rs}{\DR{\SEC}{1}}{right=2.5em of ws}
  \nonevent{wa}{\DW{a[1]}{1}}{right=of rs}
  \event{ta}{\DT{a[1]}}{right=of wa}
  \event{wx}{\DW{x}{1}}{right=of ta}
  \rf{ws}{rs}
  \po{rs}{wa}
  \po{wa}{ta}
  \po{ta}{wx}
\end{tikzpicture}\]
but due the requirement of $a$-closure we do \emph{not} have:
\[\begin{tikzpicture}[node distance=1em]
  \event{ws}{\DW{\SEC}{0}}{}
  \nonevent{rs}{\DR{\SEC}{0}}{right=2.5em of ws}
  \nonevent{wa}{\DW{a[0]}{1}}{right=of rs}
  \event{ta}{\DT{a[1]}}{right=of wa}
  \event{wx}{\DW{x}{1}}{right=of ta}
  \rf{ws}{rs}
  \po{rs}{wa}
  \po{wa}{ta}
  \po{ta}{wx}
\end{tikzpicture}\]
Thus, the attacker has managed to leak the value of a high-security
location to a low-security one: if $(\DW x1)$ is observed, the \verb|SECRET|
must have been 1.

This shows how our model of speculation can express
(very abstract, untimed) Spectre attacks.

\subsection{Relaxed memory}
\label{sec:relaxed-memory}

In \S\ref{sec:info-flow-attack} we present an information flow attack
on relaxed memory, similar to Spectre in that it relies on speculative
evaluation. Unlike Spectre it does not depend on timing attacks,
but instead is based on the sensitivity of relaxed memory to data
dependencies. For this reason, we present a simple model of relaxed
memory, which is strong enough to capture this attack.
The model includes concurrent memory accesses, which can introduce concurrent
reads-from. 
Since we are allowing events to be partially ordered, this gives a simple
model of relaxed memory, for example an independent read independent write
(IRIW) example is:
\[
  x\GETS0\SEMI x\GETS x+1
  \PAR
  y\GETS0\SEMI y\GETS y+1
  \PAR
  \IF(x)\THEN r\GETS y \FI
  \PAR
  \IF(y)\THEN s\GETS x \FI
\]
which includes the execution:
\[\begin{tikzpicture}[node distance=1em]
  \event{wx0}{\DW{x}{0}}{}
  \event{wx1}{\DW{x}{1}}{right=of wx0}
  \event{wy0}{\DW{y}{0}}{right=2.5em of wx1}
  \event{wy1}{\DW{y}{1}}{right=of wy0}
  \event{ry1}{\DR{y}{1}}{right=2.5 em of wy1}
  \event{rx0}{\DR{x}{0}}{right=of ry1}
  \event{rx1}{\DR{x}{1}}{right=2.5 em of rx0}
  \event{ry0}{\DR{y}{0}}{right=of rx1}
  \rf[out=20,in=160]{wx1}{rx1}
  \rf[out=20,in=160]{wy0}{ry0}
  \rf[out=20,in=160]{wy1}{ry1}
  \rf[out=20,in=160]{wx0}{rx0}
  \wk{wx0}{wx1}
  \wk{wy0}{wy1}
  \po{rx1}{ry0}
  \po{ry1}{rx0}
\end{tikzpicture}\]
This model does not introduce thin-air reads (TAR),
for example the TAR pit is
\((
  x\GETS y \PAR y \GETS x
)\)
but an attempt to produce a value from thin air fails,
for the usual reason of producing a cycle in $\le$, as shown on the left below:
\begin{align*}
\begin{tikzpicture}[node distance=1em]
  \event{ry42}{\DR{y}{42}}{}
  \event{wx42}{\DW{x}{42}}{below=of ry42}
  \event{rx42}{\DR{x}{42}}{right=2.5em of ry42}
  \event{wy42}{\DW{y}{42}}{below=of rx42}
  \po{ry42}{wx42}
  \po{rx42}{wy42}
  \rf{wx42}{rx42}
  \rf{wy42}{ry42}
\end{tikzpicture}
&&
\begin{tikzpicture}[node distance=1em]
  \event{ry1}{\DR{y}{1}}{}
  \event{wx1}{\DW{x}{1}}{below=of ry1}
  \event{rx1}{\DR{x}{1}}{right=2.5em of ry1}
  \event{wy1}{\DW{y}{1}}{below=of rx1}
  \po{ry1}{wx1}
  \rf{wx1}{rx1}
  \rf{wy1}{ry1}
\end{tikzpicture}
\end{align*}
This cycle can be broken by removing a dependency, for example0
\((
  x\GETS y \PAR r\GETS x\SEMI y \GETS r+1-r
)\)
has the execution on the right above.
% \[\begin{tikzpicture}[node distance=1em]
%   \event{ry1}{\DR{y}{1}}{}
%   \event{wx1}{\DW{x}{1}}{below=of ry1}
%   \event{rx1}{\DR{x}{1}}{right=2.5em of ry1}
%   \event{wy1}{\DW{y}{1}}{below=of rx1}
%   \po{ry1}{wx1}
%   \rf{wx1}{rx1}
%   \rf{wy1}{ry1}
% \end{tikzpicture}\]
Note that $(\DR x1) \not\le (\DW y1)$, so this does not introduce a cycle.

Although it is not the primary focus of this paper, our model may be an
attractive model of relaxed memory.  Many prior models either permit
thin-air executions that our model forbids or forbid desirable executions
that our model permits.

In \S\ref{sec:logic}, we develop a logic which allows us to prove that our
semantics forbids thin air examples that are permitted by prior speculative
models
\cite{Manson:2005:JMM:1047659.1040336,DBLP:conf/esop/JagadeesanPR10,DBLP:conf/popl/KangHLVD17}.

Our model passes all of the expressible causality test cases
\cite{PughWebsite} (two test cases require loops).  Significantly, this
includes test case 9, which is forbidden by \cite{DBLP:conf/lics/JeffreyR16},
one of the few models that disallows the thin air example from
\S\ref{sec:logic}.  We present this test case in the appendix, where we also
discuss the thread inlining examples from
\cite{Manson:2005:JMM:1047659.1040336}.

In \cite{DBLP:conf/esop/BattyMNPS15},
\citeauthor{DBLP:conf/esop/BattyMNPS15} showed that the thin-air problem has
no per-candidate-execution solution for C++.  This result does not apply to
our model, as the semantics of a conditional can depend on the semantics
of both branches.

\subsection{Information flow attacks on relaxed memory}
\label{sec:info-flow-attack}

Consider an attacker program, again using security checks to
try to learn a \verb|SECRET|. Whereas \verb|SPECTRE| uses
hardware capabilities, which have to be modeled by adding
extra capabilities to the language, this new attacker works
by exploiting relaxed memory which can result in
unexpected information flows. The attacker program is:
\begin{verbatim}
  (
    x := 0; y := x;
  ) || (
    if (y == 0) { x := 1; }
    else if (canRead(SECRET)) { x := SECRET; }
    else { x := 1; z := 1; }
  )
\end{verbatim}
In the case where \verb|SECRET| is $2$, this has many executions,
one of which is:
\[\begin{tikzpicture}[node distance=1em]
  \event{wx0}{\DW{x}{0}}{}
  \event{wy0}{\DW{y}{0}}{right=of wx0}
  \event{ry0}{\DR{y}{0}}{right=2.5 em of wy0}
  \event{wx1}{\DW{x}{1}}{right=of ry0}
  \nonevent{wx2}{\DW{x}{2}}{right=of wx1}
  \nonevent{wz1}{\DW{z}{1}}{right=of wx2}
  \po{ry0}{wx1}
  \po[out=30,in=150]{ry0}{wz1}
  \po[out=25,in=150]{ry0}{wx2}
  \rf{wy0}{ry0}
\end{tikzpicture}\]
but there are no executions which exhibit
$(\DW{z}{1})$, since any attempt to do so
produces a cycle:
\[\begin{tikzpicture}[node distance=1em]
  \event{wx0}{\DW{x}{0}}{}
  \event{rx1}{\DR{x}{1}}{right=of wx0}
  \event{wy1}{\DW{y}{1}}{right=of rx1}
  \event{ry1}{\DR{y}{1}}{right=2.5 em of wy1}
  \event{wx1}{\DW{x}{1}}{right=of ry1}
  \nonevent{wx2}{\DW{x}{2}}{right=of wx1}
  \event{wz1}{\DW{z}{1}}{right=of wx2}
  \po{rx1}{wy1}
  \po{ry1}{wx1}
  \po[out=30,in=150]{ry1}{wz1}
  \po[out=25,in=150]{ry1}{wx2}
  \rf{wy1}{ry1}
  \rf[out=210,in=330]{wx1}{rx1}
\end{tikzpicture}\]
In the case where \verb|SECRET| is $1$, there is an execution:
\[\begin{tikzpicture}[node distance=1em]
  \event{wx0}{\DW{x}{0}}{}
  \event{rx1}{\DR{x}{1}}{right=of wx0}
  \event{wy1}{\DW{y}{1}}{right=of rx1}
  \event{ry1}{\DR{y}{1}}{right=2.5 em of wy1}
  \event{wx1}{\DW{x}{1}}{right=of ry1}
  \event{wz1}{\DW{z}{1}}{right=of wx1}
  \po{rx1}{wy1}
  \po[out=30,in=150]{ry1}{wz1}
  \rf{wy1}{ry1}
  \rf[out=210,in=330]{wx1}{rx1}
\end{tikzpicture}\]
Note that in this case, there is no dependency from
$(\DR{y}{1})$ to $(\DW{x}{1})$, which is what makes this
execution possible. Thus, if the attacker sees
an execution with $(\DW{z}{1})$, they can conclude
that \verb|SECRET| is $1$, which is an information flow
attack.

This attack is not just an artifact of the model,
since the same behavior can be exhibited by
compiler optimizations. Consider the program fragment:
\begin{verbatim}
  if (y == 0) { x := 1; }
  else if (canRead(SECRET)) { x := SECRET; }
  else { x := 1; z := 1; }
\end{verbatim}
Now, in the case where \verb|SECRET| is a constant \verb|1|,
the compiler can inline it:
\begin{verbatim}
  if (y == 0) { x := 1; }
  else if (canRead(SECRET)) { x := 1; }
  else { x := 1; z := 1; }
\end{verbatim}
and lift the assignment to \verb|x| out of the \verb|if| statement:
\begin{verbatim}
  x := 1;
  if (y == 0) { }
  else if (canRead(SECRET)) { }
  else { z := 1; }
\end{verbatim}
After these optimizations, a sequentially consistent execution
exhibits $(\DW{z}{1})$. We discuss the practicality of this attack
further in \S\ref{sec:experiments}.

This approach can be generalized to detect information flows in arbitrary
code.  If we replace the code fragment above with:
\begin{verbatim}
  if (y == 0) { x:=P(0); } else { x:=P(1); z:=1; }
\end{verbatim}
then $(\DW{z}{1})$ is possible only if \texttt{P} is independent of its
input.  The conditional is able to capture multiple executions, as in
\cite{Barthe:2004:SIF:1009380.1009669}.

\subsection{Dead store elimination}
\label{sec:dse}

A common compiler optimization is \emph{dead store elimination},
in which writes are omitted if they will be overwritten by a subsequent
write later in the same thread. We can model eliminated writes
by ones with an unsatisfiable precondition. For example,
one execution of \verb`(x := 1; x := 2 || r := x)` is:
\[\begin{tikzpicture}[node distance=1em]
  \nonevent{wx1}{\DW{x}{1}}{}
  \event{wx2}{\DW{x}{2}}{right=of wx1}
  \event{rx2}{\DR{x}{2}}{right=2.5em of wx2}
  \wk{wx1}{wx2}
  \rf{wx2}{rx2}
\end{tikzpicture}\]
Recall that for any satisfiable $\aEv$, if $\aEv$ reads $\aLoc$ from $\bLoc$
then $\bEv$ is a tautology. This means that, although we can eliminate
$(\DW{x}{1})$ we cannot eliminate $(\DW{x}{2})$.

One heuristic that a compiler might adopt is to only eliminate
writes that are guaranteed to be followed by another write
to the same variable. This can be formalized by saying that
$\bEv$ is eliminatable if there is a $\aEv \ltN \bEv$ such
that $\aEv$ is a tautology and $\bEv$ writes to every location
$\aEv$ writes to. A model of dead store elimination is one where,
in every pomset, every eliminatable event is unsatisfiable.
This simple model includes the examples above.

Note that if dead store
elimination is \emph{always} performed, then there is an information
flow attack similar to the one in \S\ref{sec:info-flow-attack}. Consider
the program:
\begin{verbatim}
  (
    r := x;
  ) || (
    x := 1;
    if (canRead(SECRET)) { if (SECRET) { x := 2; } }
    else { x := 2; }
  )
\end{verbatim}
In the case that \verb|SECRET| is $0$, there is an execution:
\[\begin{tikzpicture}[node distance=1em]
  \event{rx1}{\DR{x}{1}}{}
  \event{wx1}{\DW{x}{1}}{right=2.5em of rx1}
  \event{wx2}{\aForm \mid \DW{x}{2}}{right=of wx1}
  \rf{wx1}{rx1}
  \wk{wx1}{wx2}
\end{tikzpicture}\]
where $\aForm$ is ($\lnot$\verb|canRead(SECRET)|),
which is not a tautology, and so the $(\DW{x}{1})$ event is not eliminated.
In the case that \verb|SECRET| is not $0$, the matching execution
is:
\[\begin{tikzpicture}[node distance=1em]
  \event{rx2}{\DR{x}{2}}{}
  \nonevent{wx1}{\DW{x}{1}}{right=2.5em of rx2}
  \event{wx2}{\DW{x}{2}}{right=of wx1}
  \rf[out=160,in=20]{wx2}{rx2}
  \wk{wx1}{wx2}
\end{tikzpicture}\]
Now the $(\DW{x}{2})$ event is a guaranteed write, so the $(\DW{x}{1})$
is eliminated, and so cannot be read.
In the case that the attacker can rely on dead store
elimination taking place, this is an information flow: if the attacker observes
$x$ to be $1$, then they know \verb|SECRET| is $0$. We return to this attack
in \S\ref{sec:experiments}.


% Local Variables:
% TeX-master: "paper"
% End:

\subsection{Release/acquire synchronization}
\label{sec:ra}

In relaxed memory models, synchronization actions act as memory fences: that
is, they are a barrier to reordering memory accesses.  In this section, we
present a simple model of release/acquire fencing. In
\S\ref{sec:transactions}, we show that this can be scaled up to a model of
transactional memory.

We assume there are sets $\Rel$ and $\Acq \subseteq\Act$.  We say that
$\aAct$ is a \emph{release action} if $\aAct\in\Rel$ and $\aAct$ is an
\emph{acquire action} if $\aAct\in\Acq$.
In a pomset, a release event is one labelled with a release action,
and an acquire event is one labelled by an acquire action.
To give the semantics of fences, we add extra constraints
to Definition~\ref{def:prefix} of prefixing %$\aAct\prefix\aPSS$
(recalling that $\cEv$ is the %$\aAct$-labelled
event being introduced):
\begin{itemize}
\item $\cEv \le \aEv$ whenever $\cEv$ is an acquire event or $\aEv$ is a release event, and
\item if $\cEv$ is an acquire event then $\aEv$ is independent of $\aLoc$,
  for every $\aLoc$.
\end{itemize}
The first constraint ensures that events are ordered before a release and
after an acquire.  The second constraint ensures that thread-local reads do
not cross acquire fences.

In examples, we will use
releasing writes and acquiring reads:
\begin{itemize}
\item $(\DWRel{\aLoc}{\aVal})$, a release action that writes $\aVal$ to $\aLoc$, and
\item $(\DRAcq{\aLoc}{\aVal})$, an acquire action that reads $\aVal$ from $\aLoc$.
\end{itemize}
The semantics of programs with releasing write and acquiring read are similar
to regular write and read, with $\DWRel\aLoc\aVal$ replacing
$\DW\aLoc\aVal$ and $\DRAcq\aLoc\aVal$ replacing $\DR\aLoc\aVal$:
\begin{eqnarray*}
  \sem{\REL\aLoc\GETS\aExp\SEMI \aCmd} & = & \textstyle\bigcup_\aVal\; \bigl((\aExp=\aVal) \guard (\DWRel\aLoc\aVal) \prefix \sem{\aCmd}[\aExp/\aLoc]\bigr) \\
  \sem{\ACQ\aReg\GETS\aLoc\SEMI \aCmd} & = & \textstyle\bigcup_\aVal\; (\DRAcq\aLoc\aVal) \prefix \sem{\aCmd}[\aLoc/\aReg]
\end{eqnarray*}

To see the need for the first constraint on prefixing, consider the program:
\[
  \VAR x\GETS0\SEMI \VAR f\GETS0\SEMI
  (x\GETS 1\SEMI \REL f\GETS1 \PAR \ACQ r\GETS f; s\GETS x)
\]
This has an execution:
\[\begin{tikzpicture}[node distance=1em]
  \event{wx0}{\DW{x}{0}}{}
  \event{wf0}{\DW{f}{0}}{right=of wx0}
  \event{wx1}{\DW{x}{1}}{right=2.5em of wf0}
  \event{wf1}{\DWRel{f}{1}}{right=of wx1}
  \event{rf1}{\DRAcq{f}{1}}{right=2.5em of wf1}
  \event{rx1}{\DR{x}{1}}{right=of rf1}
  \po[out=20,in=160]{wx0}{wf1}
  \po[out=20,in=160]{wf0}{wf1}
  \po{wx1}{wf1}
  \po{rf1}{rx1}
  \rf{wf1}{rf1}
  \rf[out=20,in=160]{wx1}{rx1}
  \wk[out=20,in=160]{wx0}{wx1}
\end{tikzpicture}\]
but \emph{not}:
\[\begin{tikzpicture}[node distance=1em]
  \event{wx0}{\DW{x}{0}}{}
  \event{wf0}{\DW{f}{0}}{right=of wx0}
  \event{wx1}{\DW{x}{1}}{right=2.5em of wf0}
  \event{wf1}{\DWRel{f}{1}}{right=of wx1}
  \event{rf1}{\DRAcq{f}{1}}{right=2.5em of wf1}
  \event{rx0}{\DR{x}{0}}{right=of rf1}
  \po[out=20,in=160]{wx0}{wf1}
  \po[out=20,in=160]{wf0}{wf1}
  \po{wx1}{wf1}
  \po{rf1}{rx1}
  \rf{wf1}{rf1}
  \rf[out=20,in=160]{wx0}{rx0}
  \wk[out=20,in=160]{wx0}{wx1}
\end{tikzpicture}\]
since $(\DW x0) \gtN (\DW x1) < (\DR x0)$, so this pomset does not satisfy the
requirements to be $x$-closed.
If we replace the release
with a plain write, then the outcome $(\DRAcq f1)$ and $(\DR x0)$ is possible:
\[\begin{tikzpicture}[node distance=1em]
  \event{wx0}{\DW{x}{0}}{}
  \event{wf0}{\DW{f}{0}}{right=of wx0}
  \event{wx1}{\DW{x}{1}}{right=2.5em of wf0}
  \event{wf1}{\DW{f}{1}}{right=of wx1}
  \event{rf1}{\DRAcq{f}{1}}{right=2.5em of wf1}
  \event{rx0}{\DR{x}{0}}{right=of rf1}
  \wk[out=20,in=160]{wf0}{wf1}
  \po{rf1}{rx0}
  \rf{wf1}{rf1}
  \rf[out=20,in=160]{wx0}{rx0}
  \wk[out=20,in=160]{wx0}{wx1}
\end{tikzpicture}\]
since no order is required between $(\DW x1)$ and $(\DW f1)$.  
Symmetrically, if we replace the acquire of the original program
with a plain read, then the outcome $(\DR f1)$ and $(\DR x0)$ is possible.
% \begin{verbatim}
%   x := 0; rel f := 0; ||
%   acq r := f; if (r == 0) { x := x+1; rel f := 1; } ||
%   acq s := f; if (r == 1) { x := x+1; rel f := 2; }
% \end{verbatim}
% This has an execution:
% \[\begin{tikzpicture}[node distance=1em]
%   \event{wx0}{\DW{x}{0}}{}
%   \event{wf0}{\DWRel{f}{0}}{below=of wx0}
%   \event{rf0}{\DRAcq{f}{0}}{right=2.5 em of wx0}
%   \event{rx0}{\DR{x}{0}}{below=of rf0}
%   \event{wx1}{\DW{x}{1}}{below=of rx0}
%   \event{wf1}{\DWRel{f}{1}}{below=of wx1}
%   \event{rf1}{\DRAcq{f}{1}}{right=2.5 em of rf0}
%   \event{rx1}{\DR{x}{1}}{below=of rf1}
%   \event{wx2}{\DW{x}{2}}{below=of rx1}
%   \event{wf2}{\DWRel{f}{2}}{below=of wx2}
%   \po{wx0}{wf0}
%   \po{rf0}{rx0}
%   \po{rx0}{wx1}
%   \po{wx1}{wf1}
%   \po{rf1}{rx1}
%   \po{rx1}{wx2}
%   \po{wx2}{wf2}
%   \rf{wf0}{rf0}
%   \rf{wx0}{rx0}
%   \rf{wf1}{rf1}
%   \rf{wx1}{rx1}
% \end{tikzpicture}\]
% but \emph{not}:
% \[\begin{tikzpicture}[node distance=1em]
%   \event{wx0}{\DW{x}{0}}{}
%   \event{wf0}{\DWRel{f}{0}}{below=of wx0}
%   \event{rf0}{\DRAcq{f}{0}}{right=2.5 em of wx0}
%   \event{rx0}{\DR{x}{0}}{below=of rf0}
%   \event{wx1}{\DW{x}{1}}{below=of rx0}
%   \event{wf1}{\DWRel{f}{1}}{below=of wx1}
%   \event{rf1}{\DRAcq{f}{1}}{right=2.5 em of rf0}
%   \event{rx0b}{\DR{x}{0}}{below=of rf1}
%   \event{wx1b}{\DW{x}{1}}{below=of rx0b}
%   \event{wf2}{\DWRel{f}{2}}{below=of wx1b}
%   \po{wx0}{wf0}
%   \po{rf0}{rx0}
%   \po{rx0}{wx1}
%   \po{wx1}{wf1}
%   \po{rf1}{rx0b}
%   \po{rx0b}{wx1b}
%   \po{wx1b}{wf2}
%   \rf{wf0}{rf0}
%   \rf{wx0}{rx0}
%   \rf{wf1}{rf1}
%   \rf{wx0}{rx0b}
% \end{tikzpicture}\]
% since $(\DW x0) < (\DW x1) < (\DR x0)$, so this pomset does not satisfy the
% requirements to be an rf-pomset.

% The notion rf-pomset is sufficient to capture hardware models and
% release/acquire access in C++, where reads-from implies happens-before
% \cite{alglave}.  To model C++ relaxed access, it
% would be necessary to use a more general notion of rf-pomset, where
% $(\bEv,\aLoc,\aEv) \in \RF$ does not necessarily imply $\bEv < \aEv$, instead
% requiring that $(\mathord< \cup \mathord\RF)$ be acyclic.

To see the need for the second constraint on prefixing, consider the program:
\begin{displaymath}
  (
  x\GETS1\SEMI
  \REL f\GETS 1\SEMI
  \ACQ r\GETS f\SEMI
  y\GETS x
  )
  \PAR
  (
  \ACQ s\GETS f\SEMI
  x\GETS2\SEMI
  \REL f\GETS 2\SEMI
  )
\end{displaymath}
whose semantics includes execution:
\begin{displaymath}
\begin{tikzpicture}[node distance=1em]
  \event{wx1}{\DW{x}{1}}{}
  \event{wf1}{\DWRel{f}{1}}{right=of wx1}
  \event{rf1}{\DRAcq{f}{2}}{below=of wf1}
  \event{wx2}{\DW{x}{2}}{right=of rf1}
  \event{wf2}{\DWRel{f}{1}}{right=of wx2}
  \event{rf2}{\DRAcq{f}{2}}{above=of wf2}
  \event{wy1}{\DW{y}{1}}{right=of rf2}
  \po{wx1}{wf1}
  \rf{wf1}{rf1}
  \po{rf1}{wx2}
  \po{wx2}{wf2}
  \rf{wf2}{rf2}
  \po{rf2}{wy1}
\end{tikzpicture}
\end{displaymath}
This execution exists because
\begin{math}
  \sem{y\GETS x}
\end{math}
includes
\begin{math}
  (x=1\mid \DW{y}{1})
\end{math}
and the precondition $x=1$ is fulfilled by the preceding write $x\GETS1$.  In
implementation term, this execution is reading $1$ from $x$ in a ``stale
cache.''  The alternative execution that attempts to read $1$ from the $x$ in
``main memory,'' has an explicit $(\DR{x}{1})$ between $(\DRAcq{f}{2})$ and
$(\DW{y}{1})$, and thus will fail to be $x$-closed.

To prevent thread-local writes from crossing release/acquire pairs, we
require that pomsets in the semantics of acquire have no free locations.
This corresponds to the idea that acquires flush the read cache, and
therefore reads must reload values from main memory after an acquire.

% In addition, we must change the semantics of write from
% \S\ref{sec:sets-of-pomsets} to ensure that an action is generated for every
% write that might be published by a subsequent release action.
% Formally, $\sem{\aLoc\GETS\aExp\SEMI \aCmd}$ only includes pomsets
% from $\sem{\aCmd}[\aExp/\aLoc]$ that contain a write to
% $\aLoc$ that is not preceded by a release.


\subsection{Transactions}
\label{sec:transactions}

We present a model of transactional memory~\cite{Larus:2007:TM:1207012} that is sufficient to capture
\textsc{Prime+Abort} attacks~\cite{DBLP:conf/uss/DisselkoenKPT17}.  We make
several simplifying assumptions: transactions are serializable, strongly
isolated, and only abort due to cache conflicts.
To model the latter, we assume that the set of locations $\Loc$ is
partitioned into \emph{cache sets}.

The action $(\DB{\aVal})\in\Acq$ represents the begin of a transaction with
id $\aVal$ and $(\DC{\aVal})\in \Rel$ represents the corresponding commit.
We model a language in which transactions have explicit identifiers (which we
elide in examples) and abort handlers (which we elide when they are empty):
\begin{eqnarray*}
  \sem{\BEGINVAL\SEMI \aCmd\SEMI \RECOVERYVAL \bCmd \ENDREC}
  & = &
  (\DB{\aVal}) \prefix \bigl(\sem{\aCmd} \cup \bigl((\FALSE \guard \sem{\aCmd}) \parallel \sem{\bCmd}\bigr)\bigr)
  \\
  \sem{\COMMITVAL\SEMI \bCmd}
  & = &
  (\DC{\aVal}) \prefix \sem{\bCmd}
\end{eqnarray*}
The semantics of a transaction has two cases: a committed case
(executing only the transaction body) and an aborted case (executing both the body and the
recovery code, where the body is marked unsatisfiable). For example, two executions of
\begin{math}
  (\BEGIN\SEMI \aLoc\GETS1\SEMI \aLoc\GETS2\SEMI \COMMIT\SEMI \RECOVERY \bLoc\GETS1\ENDREC)
\end{math}
are:
\[\begin{tikzpicture}[node distance=1em]
  \event{b0}{\DB{}}{}
  \event{wx0}{\DW{x}{0}}{right=of b0}
  \event{wx1}{\DW{x}{1}}{right=of wx0}
  \event{c0}{\DC{}}{right=of wx1}
  \po{b0}{wx0}
  \po[out=30,in=150]{b0}{wx1}
  \po[out=30,in=150]{wx0}{c0}
  \po{wx1}{c0}
  \wk{wx0}{wx1}
\end{tikzpicture}
\qquad\qquad
\begin{tikzpicture}[node distance=1em]
  \event{b0}{\DB{}}{}
  \nonevent{wx0}{\DW{x}{0}}{right=of b0}
  \nonevent{wx1}{\DW{x}{1}}{right=of wx0}
  \nonevent{c0}{\DC{}}{right=of wx1}
  \event{wy1}{\DW{y}{1}}{right=of c0}
  \po{b0}{wx0}
  \po[out=30,in=150]{b0}{wx1}
  \po[out=30,in=160]{b0}{wy1}
  \po[out=30,in=150]{wx0}{c0}
  \po{wx1}{c0}
  \wk{wx0}{wx1}
\end{tikzpicture}\]
At top level, we require that pomsets be \emph{serializable}, as defined below.
\begin{definition}
  We say that event $\comEv$ \emph{matches} $\begEv$ if
  $\labelling(\comEv)=(\DC{\aVal})$ and
  $\labelling(\begEv)=(\DB{\aVal})$. %, for some $\aVal$.
  % We say that a begin event \emph{aborts} if every matching commit is
  % unsatisfiable.
  We say that begin event $\begEv$ \emph{begins} $\aEv$ if
  $\begEv\le\aEv$ and there is no intervening matching commit; in this case
  $\aEv$ \emph{belongs to} $\begEv$.
  % event $\comEv$ such that $\begEv\le\comEv\le\aEv$
  We say that commit event $\comEv$ \emph{commits} $\aEv$ if $\aEv\le\comEv$
  and there is no intervening matching begin.
  % event $\begEv$ such that $\aEv\le\begEv\le\comEv$.
\end{definition}
\begin{definition}
  A pomset is \emph{serializable} if:
  \begin{enumerate}
  \item\label{tx:1} no two begins have the same id,
  \item\label{tx:2} every commit follows the matching begin,
  \item\label{tx:3} $\le$ totally orders tautological begins and commits,
  \item\label{tx:4} if $\begEv$ begins $\aEv$, but not $\bEv$, and $\bEv\le\aEv$ then $\bEv\le\begEv$,
  \item\label{tx:5} if $\comEv$ ends $\aEv$, but not $\bEv$, and $\aEv\le\bEv$ then $\comEv\le\bEv$,
  % \item\label{tx:4} if $\begEv$ begins $\aEv$, but not $\bEv$, then
  %   $\bEv\le\aEv$ implies $\bEv\le\begEv$ and $\aEv\ltN\bEv$ implies $\begEv\ltN\bEv$
  % \item\label{tx:5} if $\comEv$ ends $\aEv$, but not $\bEv$, then
  %   $\aEv\le\bEv$ implies $\comEv\le\bEv$ and $\bEv\ltN\aEv$ implies $\bEv\ltN\comEv$,
  \item\label{tx:6} if $\aEv$ and $\bEv$ belong to $\begEv$ and read the same
    location, then both read the same value, and
    % note that read events are optional, so we can assume they come from
    % outside the transaction.
  % 
  % \item\label{tx:6} if $\begEv$ begins $\aEv$ then some matching $\comEv$ both implies and ends $\aEv$,    
  % \item\label{tx:6} if $\begEv$ begins $\aEv$ then some matching $\comEv$
  %   ends $\aEv$ such that both $\aEv$ implies $\comEv$ and $\comEv$ implies $\aEv$,    
  \item\label{tx:7} if $\aEv$ belongs to $\begEv$, then $\aEv$ implies some
    matching $\comEv$ that ends $\aEv$.
  \end{enumerate}
\end{definition}
%In discussion, we identify transactions by their unique begin event.
%A transaction that does not abort is \emph{successful}.
%
Conditions \ref{tx:1}-\ref{tx:5} ensure serializability of committed
transactions.  Conditions \ref{tx:4}-\ref{tx:6} also ensure strong isolation
for non-transactional events
\cite{DBLP:journals/pacmpl/DongolJR18}. Condition \ref{tx:7} ensures that all
events in aborted transactions are unsatisfiable.
%
For example Conditions \ref{tx:5} and \ref{tx:7} rule out
executions (which violate strong isolation and atomicity):
\[\begin{tikzpicture}[node distance=1em]
  \event{b0}{\DB{}}{}
  \event{wx0}{\DW{x}{0}}{right=of b0}
  \event{wx1}{\DW{x}{1}}{right=of wx0}
  \event{c0}{\DC{}}{right=of wx1}
  \event{rx0}{\DR{x}{0}}{right=of c0}
  \po{b0}{wx0}
  \po[out=30,in=150]{b0}{wx1}
  \po[out=30,in=150]{wx0}{c0}
  \po{wx1}{c0}
  \wk{wx0}{wx1}
  \rf[out=30,in=150]{wx0}{rx0}
\end{tikzpicture}
\qquad\qquad
\begin{tikzpicture}[node distance=1em]
  \event{b0}{\DB{}}{}
  \event{wx0}{\DW{x}{0}}{right=of b0}
  \nonevent{wx1}{\DW{x}{1}}{right=of wx0}
  \nonevent{c0}{\DC{}}{right=of wx1}
  \event{wy1}{\DW{y}{1}}{right=of c0}
  \po{b0}{wx0}
  \po[out=30,in=150]{b0}{wx1}
  \po[out=30,in=160]{b0}{wy1}
  \po[out=30,in=150]{wx0}{c0}
  \po{wx1}{c0}
  \wk{wx0}{wx1}
\end{tikzpicture}\]

In order to model \textsc{Prime+Abort}, we need a mechanism for modeling
\emph{why} a transaction aborts, as this can be used as a back channel.
We model a simple form of concurrent transaction, which aborts when it
encounters a memory conflict---this is similar to
the treatment of $\TOUCHED$ in \S\ref{sec:spectre}.

\begin{definition}
  A commit event $\comEv$ matching $\begEv$ \emph{aborts due to memory conflict}
  if there is some $\aEv$ ended by $\comEv$, and some tautologous $\begEv\gtN\bEv\gtN\comEv$ that does not
  belong to $\begEv$ such that $\aEv$ and $\bEv$ touch locations in the
  same cache set.
\end{definition}

\textsc{Prime+Abort} requires an honest agent whose cache-set
access depends upon a secret.  If $a[0]$ and $a[1]$ belong to separate
cache sets, then such an honest agent is:
\[
  a[\SEC]\,\GETS\,1
\]
% \ignore{
% \begin{verbatim}
%   a[SECRET] := 1
% \end{verbatim}
% }
The attack relies on discovery of some $y$ which belongs to the cache-set of $a[1]$.
Then the program
\[
\BEGIN\SEMI y\GETS0\SEMI r\GETS\COMMIT\SEMI \RECOVERY x\GETS1\ENDREC
\]
% \ignore{
% \begin{verbatim}
%   begin; y:=0; commit; onabort; x:=1;
% \end{verbatim}
% }
can write $1$ to $x$ if the \texttt{SECRET} is $1$, in which case the
following execution is possible.
\[\begin{tikzpicture}[node distance=1em,baselinecenter]
  \event{wa1}{\DW{a[1]}{1}}{}
  \event{b}{\DB{}}{right=2.5em of wa1}
  \nonevent{e}{\DW{y}{0}}{right=of b}
  \nonevent{c}{\DC{}}{right=of e}
  \event{wx1}{\DW{x}{1}}{right=of c}
  \po{b}{e}
  \po{e}{c}
  \po[out=30,in=155]{b}{wx1}
  \wk{b}{wa1}
  \wk[out=25,in=155]{wa1}{c}
\end{tikzpicture}\]
If the attacker knows that commits only abort due to memory conflicts,
then this attack is an information flow, since the memory conflict only happens
when the \texttt{SECRET} is $1$.

% The definition handles simple examples:
% \begin{itemize}
% \item Single threaded example: $\DB_1 \DC_1 \DB_2 \DC_2$.  Because
%   $\DC_2$ is a release, we know that $\DC_1<\DC_2$.  Because
%   $\DB_1$ is an acquire, we know that $\DB_1<\DB_2$ By lifting, either of
%   these is sufficient to require that $\DC_1<\DB_2$.
% \[\begin{tikzpicture}[node distance=1em,baselinecenter]
%   \event{b1}{\DB_1}{}
%   \event{c1}{\DC_1}{right=of b1}
%   \event{b2}{\DB_2}{below right=of b1}
%   \event{c2}{\DC_2}{right=of b2}
%   \po{b1}{c1}
%   \po{b1}{b2}
%   \po{b2}{c2}
%   \po{c1}{c2}
% \end{tikzpicture}
% \;\text{implies}\;
% \begin{tikzpicture}[node distance=1em,baselinecenter]
%   \event{b1}{\DB_1}{}
%   \event{c1}{\DC_1}{right=of b1}
%   \event{b2}{\DB_2}{below right=of b1}
%   \event{c2}{\DC_2}{right=of b2}
%   \po{b1}{c1}
%   \po{c1}{b2}
%   \po{b2}{c2}
% \end{tikzpicture}\]
% \item Abort example:
% \[\begin{tikzpicture}[node distance=1em,baselinecenter]
%   \event{b1}{\DB}{}
%   \event{wx1}{\DW{x}{1}}{right=of b1}
%   \event{c1}{\DC_1}{above right=of wx1}
%   \nonevent{c2}{\DC_2}{below right=of wx1}
%   \event{rx1}{\DR{x}{1}}{right=2.5 em of wx1}
%   \po{b1}{wx1}
%   \po{wx1}{c1}
%   \po{wx1}{c2}
%   \rf{wx1}{rx1}
% \end{tikzpicture}
% \;\text{implies}\;
% \begin{tikzpicture}[node distance=1em,baselinecenter]
%   \event{b1}{\DB}{}
%   \event{wx1}{\DW{x}{1}}{right=of b1}
%   \event{c1}{\DC_1}{above right=of wx1}
%   \nonevent{c2}{\DC_2}{below right=of wx1}
%   \event{rx1}{\DR{x}{1}}{right=2.5 em of wx1}
%   \po{b1}{wx1}
%   \po{wx1}{c1}
%   \po{wx1}{c2}
%   \rf{wx1}{rx1}
%   \po{c1}{rx1}
% \end{tikzpicture}\]
  
% % \item Clause \eqref{xrf} stops transaction from reading two different values
% %   for the same variable from transactions (it is possible with no
% %   transactional writes).  
% % \item Clause \eqref{xrf} also stops transactional IRIW.
% \end{itemize}
% % \begin{definition}
% %   An rf-pomset is transaction-closed if the $\DB$ and $\DC$ actions with
% %   satisfiable preconditions are totally ordered by $<$.
% % \end{definition}

% Let ``$\END\SEMI \bCmd$'' be syntax sugar for
% ``$\IF\COMMIT\vec\aLoc\THEN\bCmd \ELSE \bCmd$'', where $\vec\aLoc$ are the
% free variables of $\bCmd$.

% The semantics of
% \begin{alltt}
%   x:=1; begin; x:=2; end; y:=x;
% \end{alltt}
% includes
% \[\begin{tikzpicture}[node distance=1em]
%   \event{wx1}{\DW{x}{1}}{}
%   \event{b}{\DB}{right=of wx1}
%   \event{wx2}{\DW{x}{2}}{right=of b}
%   \event{c}{\DC}{right=of wx2}
%   \event{wy2}{\DW{y}{2}}{right=of c}
%   \nonevent{wy1}{\DW{y}{1}}{below=of wy2}
%   \po{b}{wx2}
%   \po[bend right]{b}{wy1}
%   \po[bend left]{b}{wy2}
%   \po{wx2}{c}
%   \po[bend left]{wx1}{c}
%   %\po{rz0}{wy2}
% \end{tikzpicture}\]
% and
% \[\begin{tikzpicture}[node distance=1em]
%   \event{wx1}{\DW{x}{1}}{}
%   \event{b}{\DB}{right=of wx1}
%   \nonevent{wx2}{\DW{x}{2}}{right=of b}
%   \nonevent{c}{\DC}{right=of wx2}
%   \nonevent{wy2}{\DW{y}{2}}{right=of c}
%   \event{wy1}{\DW{y}{1}}{below=of wy2}
%   \po{b}{wx2}
%   \po[bend right]{b}{wy1}
%   \po[bend left]{b}{wy2}
%   \po{wx2}{c}
%   \po[bend left]{wx1}{c}
%   %\po{rz0}{wy2}
% \end{tikzpicture}\]

% Publication example:
% \begin{alltt}
%   var x; var f; x:=0; f:=0; 
%      x:=1; (begin; f:=1; end;) || (begin; r:=f; end; s:=x;)
% \end{alltt}

% Note: we could also include a transaction factory, and close the factory.
% \begin{alltt}
%   TransactionFactory T; var x; var f; x:=0; f:=0; fence; 
%      x:=1; (begin T; f:=1; f:=2; end T;) || (begin T; r:=f; end T; s:=x;)
% \end{alltt}

% Before defining atomicity, we provide some auxiliary notation.
%
% We say that $\aEv$ is a \emph{begin event} if
% $\labelling(\aEv)=(\aForm\mid\DB)$ and a \emph{commit event} if
% $\labelling(\aEv)=(\aForm\mid\DC)$.
%
% We write $\aForm_\aEv$ for the formula and $\aAct_\aEv$ for the
% action of $\aEv$; that is, when $\labelling(\aEv)=(\aForm_\aEv\mid\aAct_\aEv)$.
%
% We say that $\aEv$ is \emph{compatible with} $\bEv$ when
% $\aForm_\aEv\land\aForm_\bEv$ is satisfiable.
%
% \begin{definition}
%   A pomset is \emph{atomic} when for any $\aEv$ that belongs to $(\begEv,\vec\comEv)$:
%   \begin{enumerate}
%   \item\label{xcommitform} $\aForm_{\aEv}$ implies $\textstyle\bigvee_i\aForm_{\comEv_i}$,
%   \item\label{xliftb} if $\bEv<\aEv$ then $\bEv<\begEv$, 
%   \item\label{xliftc} if $\aEv<\bEv$ and $\bEv$ is compatible with
%     $\comEv_i$ then $\comEv_i<\bEv$, 
%   \item\label{xtotal} if $\aEv'\neq\aEv$ belongs to $(\begEv',\vec\comEv')$ but not
%     $(\begEv,\dontcare)$ and
%     \begin{itemize}
%     \item $\comEv'_j$ is compatible with $\aEv$, and 
%     \item $\comEv_i$ is compatible with $\aEv'$ 
%     \end{itemize}
%     then either
%     $\comEv'_j<\begEv$ or
%     $\comEv_i<\begEv'$,
%   % \item\label{xrf} if $\aEv'\neq\aEv$ belongs to $(\begEv,\dontcare)$ but not
%   %   $(\begEv,\dontcare)$ and
%   %   \begin{itemize}
%   %   \item $\aEv$ reads from $\bEv'$ that belongs to $(\begEv',\vec\comEv')$,
%   %   \item $\aEv'$ reads from $\bEv''\neq\beV'$ that belongs to $(\begEv'',\vec\comEv'')$,
%   %   \item $\comEv''_j$ is compatible with $\bEv'$, and 
%   %   \item $\comEv'_i$ is compatible with $\bEv''$ 
%   %   \end{itemize}
%   %   then either
%   %   $\comEv''_j<\begEv'$ or
%   %   $\comEv'_i<\begEv''$ .
%   \item\label{xcommitvars} if $\aEv$ writes $\aLoc$ and $\comEv_i$ writes
%     $\vec\aLoc$ then $\aLoc=\aLoc_i$, for some $i$, and
%   \item\label{xreadunique} if $\aEv$ reads $\aLoc$ and $\aEv'\neq\aEv$ reads
%     $\aLoc$, belongs to $(\begEv,\dontcare)$ and is compatible with $\aEv$
%     then $\aAct_{\aEv}=\aAct_{\aEv'}$.
%   \end{enumerate}
% \end{definition}
% Clause \eqref{xcommitform} requires that the precondition on $\aEv$ is false on an
% aborted transaction.
% The \emph{lifting clauses}, \eqref{xliftb} and \eqref{xliftc}, require order
% come in or out of $\aEv$ is lifted to the corresponding begin or commit event.
% % Clause \eqref{xrf} requires that whenever a transaction reads from two other
% % transactions, the other transactions must be ordered.
% Clause \eqref{xtotal} requires that transactions be totally ordered.
% Clause \eqref{xcommitvars} requires that all writes be committed.
% Clause \eqref{xreadunique} requires that multiple reads of a location in a
% single transaction must see the same value.
%
% The definition of atomicity guarantees strong isolation.  For weak isolation,
% clauses \eqref{xcommitvars} and \eqref{xreadunique} are unnecessary,
% \eqref{xliftb} only applies when $\bEv$ is a commit, and \eqref{xliftc} only
% applies when $\bEv$ is a begin.

% Local Variables:
% TeX-master: "x"
% End:

\section{Experiments}
\label{sec:experiments}

One theme of this paper is that optimizations not typically part of formal
abstractions can result in information-flow leaks.
This is typified by the Spectre attack, which leverages speculative execution,
a hardware optimization.
\S\ref{sec:info-flow-attack} and~\S\ref{sec:dse} presented other attacks
along the same line, which leverage relaxed memory models and dead store
elimination respectively.
These attacks may result,
not from hardware optimizations, but from common \emph{compiler} optimizations.
These attacks also, unlike Spectre, do not rely on timing side channels, or
indeed timers of any kind, bypassing many common Spectre mitigations~\cite{???}.
%%%%% FuzzyFox, Chrome's and Firefox's restrictions on precise timers, etc.

In this section we present concrete implementations of the attacks outlined
in~\S\ref{sec:info-flow-attack} and~\S\ref{sec:dse}, in both cases
leveraging compiler optimizations to construct an information flow attack.
The attacker model for these attacks (detailed in~\S\ref{subsec:attacker-model})
is currently unrealistic in a
real-world sense, rendering these attacks proof-of-concepts rather than
immediately exploitable vulnerabilities.
However, we believe the novelty of their general mechanisms may lead to
interesting discussion; and with much more development, these attacks may
evolve into genuine threats against real-world targets such as JIT compilers.
We demonstrate the efficacy of both of our concrete proof-of-concept
attacks against
the \verb|clang| and \verb|gcc| C compilers.

All of our experiments are performed on a \todo{describe machine} with
\verb|clang| version \todo{clang version} and \verb|gcc| version \todo{gcc
version}.

\subsection{Attacker model}
\label{subsec:attacker-model}

In our attacker model, we assume that there is a \verb|SECRET| which an
attacker wishes to learn; for instance, \verb|SECRET| may be a cryptographic
key hardcoded into the application.
This \verb|SECRET| is known at compile time, but may not be
accessed except behind a security check.
Since the attacker is running with low security privileges,
the security check always fails,
so the attacker can only access \verb|SECRET| in dead code.
The attacker has no capabilities other than writing and executing code --- in
particular the attacker may not disassemble the compiler or libraries to learn
the \verb|SECRET| directly; may not examine the internal state of the compiler;
may not access timers of any kind; and may not leverage hardware side channels.
The attacker's goal is to learn the value of the \verb|SECRET|.

As a hypothetical concrete example, suppose there is a library which contains
a hardcoded \verb|SECRET| such as an API or signing key, which cannot be accessed
directly, only through a function guarded by a security check:
\begin{verbatim}
  private static uint SECRET = 0x1234;
  public uint get_secret() {
    if (canRead(SECRET)) { return SECRET; }
    else { return 0; }
  }
\end{verbatim}
As noted above, this is not necessarily a realistic attacker model,
since in most cases secrets are only known at run time rather than compile time,
which means that the attacks presented in this section
are more of theoretical interest than practical concern.
However, the mechanism of the attack is novel and could potentially be applied
in other contexts.
For instance, many real-world contexts allow attackers (untrusted or
third-party entities) to write code in a scripting language which is then
compiled alongside and integrated into a larger application, often
using a just-in-time (JIT) compiler.
JavaScript code from third-party websites running in a browser is a common
example of this.
Our attack gives an attacker similar capabilities against a
compiler, except it considers the simpler setting of using C code against a C
compiler.
One could imagine a similar attack using JavaScript against browser JIT
compilers, where the compiler may have access to interesting secrets in the
browser itself, and may be able to optimize based on those secrets.
We plan to explore JavaScript attacks of this type as future work.

\subsection{Load-store reordering attack}
\label{subsec:exp-rel-mem}

We begin by examining the attack in Section~\ref{sec:info-flow-attack} in
more detail, subject to the attacker model given above.
In particular, we show that by exploiting compiler optimizations which perform
load-store reordering, an attacker can learn the value of a compile-time
\verb|SECRET| despite only being allowed to use it inside dead code, that is,
code that can never be executed at runtime.
This attack was tested and works against \verb|gcc| version \todo{gcc version}.

The form of the attack presented in Section~\ref{sec:info-flow-attack} works in
theory, but in practice, just because a compiler is \emph{allowed} to perform a
load-store reordering doesn't mean that it \emph{will}.
We found that \verb|gcc| and \verb|clang| chose to read \verb|y| into a
register first (before writing to \verb|x|), regardless of the value of
\verb|SECRET|.
However, we did find a related pattern in which \verb|gcc| will emit a
different ordering of the read of \verb|y| and the write of \verb|x| depending
on the value of a \verb|SECRET|:
\begin{verbatim}
    y := 0;
    (
      y := x;
    ) || (
      x := 1;
      if (canRead(SECRET)) { x := SECRET; }
      if (y) { return 0; }
      else { return 1; }
    )
\end{verbatim}
Figure~\ref{fig:lsr-asm} shows the assembly output of \verb|gcc| in the cases
where \verb|SECRET| is 0 and 1 respectively.
In the case that \verb|SECRET| is \verb|1|, \verb|gcc| removes the \verb|if|
statement entirely, and moves the read of \verb|y| above the write of \verb|x|.
However, when \verb|SECRET| is \verb|0|, the \verb|if| statement must remain
intact, and \verb|gcc| does not move the read of \verb|y|.
This means that if \verb|SECRET| is \verb|1|, the second thread will always
read \verb|y == 0| and always return \verb|1|.
However, if \verb|SECRET| is \verb|0|, it is possible that the first thread
may observe \verb|x == 1| and write \verb|y := 1| in time for the second thread
to observe \verb|y == 1| and thus return \verb|0|.
In this way, we leverage compiler load-store reordering to learn the value of
a compile-time \verb|SECRET|.

\begin{figure}
  \begin{tabular}[fragile]{p{3cm} | p{3cm}}
    \texttt{SECRET == 0} & \texttt{SECRET == 1} \\
\begin{verbatim}
  mov f(%rip), %eax
  mov $1, x(%rip)
  test %eax, %eax
  je label1
  mov %0, x(%rip)
label1:
  mov y(%rip), %eax
  test %eax, %eax
  sete %eax
  ret
\end{verbatim}
  &
\begin{verbatim}
  mov f(%rip), %eax
  mov y(%rip), %eax
  mov $1, x(%rip)
  test %eax, %eax
  sete %eax
  ret
\end{verbatim}
  \\
  \end{tabular}
  \caption{
    (Simplified) x86 assembly output from \texttt{gcc} for the main thread of
    the load-store reordering attack.
    In particular, note that the order between \texttt{mov \$1, x(\%rip)}
    and \texttt{mov y(\%rip), \%eax} is different in the two cases.
    The call to \texttt{canRead(SECRET)} has been inlined; we implemented
    \texttt{canRead(x)} as \texttt{return f;} where
    \texttt{volatile bool f = false;}.
    Thus, \texttt{gcc} preserves the read of \texttt{f} even when its value is
    unused, as in the case on the right.
  }
  \label{fig:lsr-asm}
\end{figure}

We extend this attack to leak a secret consisting of an arbitrary number
\verb|N| of bits.
To do this, we simply compile \verb|N| copies of the test function, each
performing a boolean test on a single bit of the secret.
The function used for reading the \verb|k|th bit is as follows (for
\verb|N <= 64|):
\begin{verbatim}
    (
      y := x;
    ) || (
      x := 1;
      if (canRead(SECRET)) { x := (SECRET & (1 << k)) ? 1 : 0; }
      if (y) { return 0; }
      else { return 1; }
    )
\end{verbatim}
Following the same analysis as above, this function will always return \verb|1|
if the appropriate bit of \verb|SECRET| is \verb|1|, but may return \verb|0| if
the appropriate bit of \verb|SECRET| is \verb|0|.
The extension of the attack to the general case with truly arbitrary \verb|N|
is straightforward; \verb|SECRET| becomes an array of 64-bit values, and we use
\verb|k / 64| and \verb|1 << (k & 63)| as the array index and bitmask
respectively.

We make three additional tweaks to improve the reliability so that the attacker
can confidently infer the value of \verb|SECRET| based on the observed return
values of the function.
First, rather than performing \verb|y := x| only once in the first thread, we
perform \verb|y := x| continuously in a loop.
This maximizes the probability that, once \verb|x := 1| occurs in the second
thread, \verb|y| will be immediately assigned \verb|1| by the first thread
and the second thread will be able to read \verb|y == 1|.

Second, we wish to lengthen the timing window between \verb|x := 1| and the
read of \verb|y| in the second thread (in the case where the appropriate bit of
\verb|SECRET| is \verb|0| and the read of \verb|y| remains below
\verb|x := 1|).
However, we wish to do this in a way that does not block the reordering of the
read of \verb|y| upwards in the case where the appropriate bit of \verb|SECRET|
is \verb|1|.
We do this by inserting many copies of the line
\begin{verbatim}
    if (canRead(SECRET)) { x := (SECRET & (1 << k)) ? 1 : 0; }
\end{verbatim}
instead of just one.
In the case where the appropriate bit of \verb|SECRET| is \verb|0|, this
results in many calls to \verb|canRead(SECRET)| and many conditional jumps,
which in practice creates a timing window for the first thread to perform
\verb|y := x|.
However, in the case where the appropriate bit of \verb|SECRET| is \verb|1|,
all of these inserted lines can be removed just as a single copy could be.
In practice, we found that inserting too many copies of the line prevents
\verb|gcc| from reordering the read of \verb|y| above the write to \verb|x| as
desired; inserting \verb|30| copies was sufficient to create a timing window
while still allowing the desired reordering.

Finally, we redundantly execute the entire attack several times, noting the
return value of the function in each case.
We note that if \emph{any} of the redundant runs produces a return value of
\verb|0| for a particular bit position, we can be certain that the
corresponding bit of \verb|SECRET| \emph{must} be \verb|0|, as it implies the
read of \verb|y| was not reordered upwards in that particular function.
On the other hand, the more runs that produce a return value of \verb|1| for a
particular bit position, the more certain we can be that the read of \verb|y|
was reordered above the \verb|x := 1| assignment, and the appropriate bit of
\verb|SECRET| is \verb|1|.

Figure~\ref{fig:load-store-perf} gives the performance results for this attack
against \verb|gcc| version \todo{ver}.
The attack can sustain hundreds of thousands of bits per second leaked with
near-perfect accuracy, or millions of bits per second with error rates of a
few percent.
Note that this bandwidth assumes that all copies of the attack function are
already compiled; the cost of compilation is not included here.

\begin{figure}
  \begin{tabular}{ r | l | l | l }
    Redundancy & Bandwidth (bits/s) & Bitwise Acc & Per-run Acc \\ \hline
    1          & 3.17 million       & 90.13\%     & 0.0\%       \\
    2          & 1.62 million       & 96.77\%     & 0.7\%       \\
    3          & 1.07 million       & 98.84\%     & 3.9\%       \\
    4          & 812 thousand       & 99.55\%     & 13.5\%      \\
    5          & 652 thousand       & 99.83\%     & 34.0\%      \\
    7          & 466 thousand       & 99.97\%     & 71.8\%      \\
    10         & 322 thousand       & 99.998\%    & 96.6\%      \\
    15         & 216 thousand       & 100.00\%    & 100.0\%     \\
  \end{tabular}
  \caption{
    Performance results for the load-store reordering attack when leaking a
    2048-bit secret.
    `Redundancy' is the number of redundant runs performed for error
    correction; more redundant runs improves accuracy but reduces bandwidth.
    `Bandwidth' is the number of bits leaked per second after accounting for
    any error correction.
    `Bitwise Accuracy' is the percentage of bits that were correct, while
    `Per-run Accuracy' is the percentage of full 2048-bit secrets that were
    correct in all bit positions.
    \todo{Note: numbers are not final (collected on Craig's machine inside a
    VM), but give an idea of where we stand.}
  }
  \label{fig:load-store-perf}
\end{figure}

\subsection{Dead store elimination attack}
\label{subsec:exp-dse}

In this section we return to the attack in Section~\ref{sec:dse} based on
dead store elimination.
We show that in our attacker model (given in
Section~\ref{subsec:attacker-model}), the attacker is able to exploit dead
store elimination to again learn the value of a compile-time \verb|SECRET|
despite only being allowed to use it inside dead code, that is, code that can
never be executed at runtime.
This attack is even more efficient than the attack on load-store reordering,
and further, we were able to demonstrate its effectiveness against both
\verb|gcc| and \verb|clang|.

We start from the simple form of the attack presented in
Section~\ref{sec:dse}, and extend it to leak a secret consisting of an
arbitrary number \verb|N| of bits.
As we did in the load-store reordering attack, we again compile \verb|N| copies
of the test function, each performing a boolean test on a single bit of the
secret.
The function used for reading the \verb|k|th bit is as follows (for
\verb|N <= 64|):
\begin{verbatim}
    (
      r := x;
    ) || (
      x := 1;
      if (canRead(SECRET)) {
        if (SECRET & (1 << k)) { x := 2; }
      } else {
        x := 2;
      }
    )
\end{verbatim}
Then, we test each function in turn, each time noting the value of \verb|r|
observed by the `listening' thread.
If the appropriate bit of \verb|SECRET| is 1, the \verb|x := 2| assignment is
guaranteed to happen, so the compiler can eliminate the \verb|x := 1|
assignment as a dead store and we will observe \verb|r == 2|; however, if the
appropriate bit of \verb|SECRET| is 0, the \verb|x := 1| assignment cannot be
eliminated, and we will observe \verb|r == 1| with some probability.
The extension of the attack to the general case with truly arbitrary \verb|N|
is straightforward and proceeds exactly as it did for the attack on
load-store reordering.

We make three additional tweaks to improve the reliability so that the attacker
can confidently infer the value of \verb|SECRET| based on the observed values
of \verb|r|.
These three tweaks strongly resemble the reliability tweaks we made to the
load-store reordering attack and differ only in a few details.

First, rather than simply observing \verb|x| with \verb|r := x| in the
`listening' thread, we continuously load \verb|x| in a loop until a
nonzero value is observed --- i.e., we perform
\begin{verbatim}
    do {
      r := x;
    } while(r == 0);
\end{verbatim}
This remedies the case where \verb|r := x| could observe a value of \verb|x|
from `before' either of the two possible writes performed by the other thread.
\todo{do we need to explicitly say that we ensure x is initialized to 0 --- and
coherently seen as such by both threads --- before starting the attack?}

Second, we insert additional time-consuming computation immediately following
the \verb|x := 1| operation in the `main' thread.
This lengthens the timing window in which \verb|x| has the value \verb|1|,
increasing the likelihood that the `listening' thread will be able to observe
\verb|x == 1| (unless the \verb|x := 1| write was eliminated, of course).
Inserting this computation can be done without interfering with the dead store
elimination process itself, so that the compiler will continue to eliminate
the \verb|x := 1| write if and only if the appropriate bit of \verb|SECRET|
was 1.
For \verb|gcc|, we have a fair amount of freedom with the time-consuming
computation --- for instance, we can use an arbitrarily long loop.
In fact, we can perform a further optimization by monitoring the value of the
variable \verb|r| (written to by the listening thread) and breaking out of the
loop early if we see that the listening thread has already observed
\verb|x == 1|.
However, with \verb|clang|, we cannot use a loop at all --- the time-consuming
computation must be branch-free, and furthermore must not consist of too many
instructions.
This is because \verb|clang|'s dead store elimination pass operates only
within basic blocks, and uses a heuristic to stop scanning the basic block
early if it is too large.
Nonetheless, we find that even with these restrictions, we are able to
construct a reliable and fast attack against both \verb|clang| and \verb|gcc|.

Finally, we redundantly execute the entire attack several times, noting the
final value of \verb|r| (the first observed nonzero value of \verb|x|) in each
case.
We note that if \emph{any} of the redundant runs produces \verb|r == 1| for a
particular bit position, we can be certain that the corresponding bit of
\verb|SECRET| \emph{must} be $0$, as it implies that the \verb|x := 1| write
was not eliminated in that particular function.
On the other hand, the more runs that observe \verb|r == 2| in a particular bit
position despite our other reliability-increasing measures taken above, the
more certain we can be that the \verb|x := 1| write was eliminated in that
function, and the appropriate bit of \verb|SECRET| is $1$.

Unlike the load-store reordering attack, our implementation of the dead store
elimination attack has two important ``knobs'' which trade off reliability
vs.\@ performance, rather than only one.
First, we have the length of time which the writing thread attempts to
``stall'' immediately after the \verb|x := 1| write.
Second, we have the number of entire redundant runs of the attack that are
performed before the attacker reaches her conclusion.
Increased reliability can be achieved by adjusting either of these knobs,
and they each have (different) effects on the overall performance of the
attack.
After exploring the parameter space, we found that $3$ redundant runs is
sufficient to provide near-100\% accuracy while allowing us to maximize the
speed of the attack.
Specifically, on our machine, our attack on \verb|gcc| reaches speeds of
\todo{exact gcc leak speed} bits leaked per second
(\todo{exact gcc raw leak speed} `raw' bits leaked per second, that is, before
error correction) with \todo{exact gcc accuracy}, while our attack on
\verb|clang| reaches speeds of \todo{exact clang leak speed} bits leaked per
second (\todo{exact clang raw leak speed} `raw' bits leaked per second) with
\todo{exact clang accuracy}.
In particular, this means our attack can leak a 2048-bit cryptographic key in
under \todo{exact speed} ms, on either \verb|gcc| or \verb|clang|, with
probability \todo{exact probability} that there are exactly zero bit errors in
the leaked key, or probability \todo{exact probability} that there is at most
one bit error in the leaked key.

\section{Logic}
\label{sec:logic}

Syntax:
\begin{verbatim}
  phi ::= Wxv | Rxv | F(phi) | false | phi1 \/ phi2 | ~ phi
\end{verbatim}
Given a labelled rf-poset (E,lbl,<,rf), define satisfaction:
\begin{verbatim}
  e |= Wxv if lbl(e) = (true,Wxv)
  e |= Rxv if lbl(e) = (true,Rxv)
  e |= F(phi) if exists d < e, d |= phi
\end{verbatim}
Denote rf-poset by its carrier set E.
Then a set of rf-poset is denoted as Es.
\begin{verbatim}
  Es |= phi if forall E in Es: forall e in E: e |= phi
\end{verbatim}
Coinduction for F [sound proof rule]:
\begin{verbatim}
  phi => F(phi)
  ------------
     ~ phi
\end{verbatim}
Superset closed formula:
\begin{verbatim}
  Define: phi is superset closed if (Es |= phi) and (Es' supseteq Es) imply (Es' |= phi)
\end{verbatim}
Parallel composition [sound proof rule]:
\begin{verbatim}
  phi is superset closed
  Es1 |= phi  
  Es2 |= phi
  ----------------
  Es1 || Es2  |= phi
\end{verbatim}
Closed at x:
\begin{verbatim}
  Define closed(x) = (Rxv => F(Wxv))
\end{verbatim}
Local declaration [sound proof rule]:
\begin{verbatim}
  x notin phi
  Es |= closed(x) => phi
  ----------------------
  var x; Es |= phi
\end{verbatim}
Conditional TAR example:
\begin{verbatim}
  var x,y,z;
  y:=0; y:=x  ||  x:=0; if(~z){x:=1}else{x:=y;a:=y}  ||  z:=0; z:=1
\end{verbatim}
\begin{verbatim}
Goal: Es |= ~ F(Wa1)   [impossible to write a=1]
\end{verbatim}
Invariant:
\begin{verbatim}
     F(Wy1) => F(Rx1)
  /\ F(Wa1) => F(Ry1) /\ G(Wx1 => F(Ry1))
\end{verbatim}
Closing y:
\begin{verbatim}
  F(Wa1) => F(Rx1) /\ G(Wx1 => F(Rx1))
\end{verbatim}
Closing x:
\begin{verbatim}
  F(Wa1) => F(Wx1) /\ G(Wx1 => F(Wx1))
\end{verbatim}
Using coinduction for F:  
\begin{verbatim}
  F(Wa1) => F(Wx1) /\ G(~ Wx1)
\end{verbatim}
Simplifying:  
\begin{verbatim}
  F(Wa1) => false
\end{verbatim}
\section{Conclusions and future work}

In this paper, we have presented a model of speculative evaluation and
shown that it captures non-trivial properties of speculations produced
by hardware, compiler optimizations, and transactions. These properties
include information flow attacks: in the case of hardware and transactions
this is modeling known attacks~\cite{DBLP:journals/corr/abs-1801-01203,DBLP:conf/uss/DisselkoenKPT17},
but in the case of compiler optimizations the attacks are new, and were
discovered as a direct result of developing the model. We have experimentally
validated that the attacks can be carried out against \verb|gcc| and \verb|clang|,
though only against secrets known at compile time.

The model of relaxed memory used in this paper is deliberately
simplified, compared for example to
C11~\cite{Boehm:2008:FCC:1375581.1375591,Batty:2011:MCC:1926385.1926394}.
In particular our model of reads-from is strong, and could be weakened
by replacing the requirement $\bEv<\aEv$ in Definition~\ref{def:rf}
by $\aEv\ltN\bEv$. It remains to be seen how this impacts the model,
in particular the logical formulation of $\aLoc$-closure in
\S\ref{sec:logic} as
$((\DR{\aLoc}{\aVal}) \Rightarrow \once(\DW{\aLoc}{\aVal}))$
would no longer be sound.
% The model is also not considering coherence, though we speculate it
% can be added by requiring that for each $\aLoc$, $\ltN$ form a total
% order when restricted to events that write to $\aLoc$.

The design space for transactions is very rich~\cite{DBLP:journals/pacmpl/DongolJR18}.
We have only presented one design choice, and it remains to be seen how other
design choices could be adopted. For example we have chosen not to distinguish
commits that are aborted due to transaction failure from commits which are aborted
for other reasons, such as failed speculation.

One interesting feature of this model is that (in the language
of~\cite{Pichon-Pharabod:2016:CSR:2837614.2837616}) it is a
\emph{per-candidate execution model}, in that the correctness of an
execution only requires looking at that one execution, not at
others. This is explicit in memory models such
as~\cite{Jagadeesan:2010:GOS:2175486.2175503,Kang:2017:PSR:3009837.3009850} in which
``alternative futures'' are explored, in a style reminiscent of
Abramsky's bisimulation as a testing equivalence~\cite{ABRAMSKY1987225}. Models of
information flow are similar, in that they require comparing different
runs to test for the presence of dependencies~\cite{Clarkson:2010:HYP:1891823.1891830}. In contrast, the model
presented here explicitly captures dependency in the pomset order, and
models multiple runs by giving the semantics of $\IF$ in terms of a
concurrent semantics of both branches.
In the parlance of information flow~\cite{Barthe:2004:SIF:1009380.1009669},
the humble conditional suffices to construct a composition operator to detect information flow  in the presence of speculation.

In future work, it would be interesting to see if full-abstraction
results for pomsets \cite{Plotkin:1997:TSP:266557.266600} can be extended to
3-valued pomsets.


\bibliography{bib}
\appendix
\section{Memory model examples}
\label{sec:appendix}

\citet{PughWebsite} developed a set of twenty {causality test cases} in the
process of revising the Java Memory Model (JMM)
\cite{Manson:2005:JMM:1047659.1040336}.  Using hand calculation, we have
confirmed that our model gives the desired result for all twenty cases,
unrolling loops as necessary.  Our model also gives the desired results for
all of the examples in \citet[\textsection 4]{DBLP:conf/esop/BattyMNPS15} and
all but one in \citet[\textsection 5.3]{SevcikThesis}:
redundant-write-after-read-elimination---this counterexample applies to any
sensible non-coherent semantics.  Our model agrees with the JMM on the
``surprising and controversial behaviors'' of \citet[\textsection
8]{Manson:2005:JMM:1047659.1040336}, and thus fails to validate thread
inlining.

In this section, we discuss three of the causality test cases and the thread
inlining from \cite{Manson:2005:JMM:1047659.1040336}.  In presenting the
examples, we unroll loops, correct typos and simplify the code.  

\subsection{Causality test case 8}

Test case 8 asks whether:
\begin{displaymath}
  \VAR x\GETS 0\SEMI
  \VAR y\GETS 0\SEMI
  (\IF(x<2)\THEN y\GETS 1\FI 
  \PAR
  x\GETS y)
\end{displaymath}
may read $1$ for both $x$ and $y$.  This behavior is allowed, since
``interthread analysis could determine that $x$ and $y$ are always either $0$
or $1$.''  This breaks the dependency between the read of $x$ and the write
to $y$ in the first thread, allowing the write to be moved earlier.

The semantics of TC8 includes
\[\begin{tikzpicture}[node distance=1em]
  \event{ix}{\DW{x}{0}}{}
  \event{iy}{\DW{y}{0}}{right=of ix}
  \event{rx1}{\DR{x}{1}}{right=2.5 em of iy}
  \event{wy1}{\DW{y}{1}}{right=of rx1}
  \event{ry1}{\DR{y}{1}}{right=2.5em of wy0}
  \event{wx1}{\DW{x}{1}}{right=of ry0}
  \po{ry1}{wx1}
  \po[out=30,in=150]{ix}{rx1}
  \rf[in=-25,out=-160]{wx1}{rx1}
  \rf[out=20,in=160]{wy1}{ry1}
  \wk[out=-25,in=-150]{ix}{wx1}
  \wk[out=25,in=155]{iy}{wy1}
\end{tikzpicture}\]
Where we require $(\DW{x}{0})<(\DR{x}{1})$ but not $(\DR{x}{1})<(\DW{y}{1})$.
To see why this execution exists, consider the left thread with syntax sugar
removed:
\begin{displaymath}
  r\GETS x\SEMI \IF(r<2)\THEN y\GETS 1\FI
\end{displaymath}
\begin{math}
  \sem{\IF(r<2)\THEN y\GETS 1\FI}
\end{math}
includes
\begin{math}
  (r<2\mid\DW{y}{1}).
\end{math}
% \[\begin{tikzpicture}[node distance=1em]
%   \event{wy1}{r<2\mid\DW{y}{1}}{}
% \end{tikzpicture}\]
Thus, by definition~\ref{def:programs}, 
\begin{math}
  \sem{r\GETS x\SEMI \IF(r<2)\THEN y\GETS 1\FI}
\end{math}
includes
\begin{math}
  (\DR{x}{1}) \prefix (r<2\mid\DW{y}{1})[x/r]
\end{math}
which simplifies to
\begin{math}
  (\DR{x}{1}) \prefix (x<2\mid\DW{y}{1}),
\end{math}
which, by definition~\ref{def:prefix}, includes:
\[\begin{tikzpicture}[node distance=1em,baselinecenter]
    \event{rx1}{\DR{x}{1}}{}
    \event{wy1}{x<2\mid\DW{y}{1}}{right=of rx1}
  \end{tikzpicture}\]
Here we have used the \emph{non-ordering read} clause of definition~\ref{def:prefix}:
``$\bForm'$ implies $\bForm[\aVal/\aLoc] \land \bForm$, if $\aAct$ reads $\aVal$ from $\aLoc$,''
where $a=(\DR{x}{1})$,  $\bForm=\bForm'=(x<2)$.  We can use this case since
$x<2$ implies $1<2\land x<2$.

Prefixing with $(\DW{x}{0})$ allows us to discharge the assumption $x<2$,
arriving at:
\[\begin{tikzpicture}[node distance=1em,baselinecenter]
    \event{ix}{\DW{x}{0}}{}
    \event{rx1}{\DR{x}{1}}{right=2.5 em of ix}
    \event{wy1}{\DW{y}{1}}{right=of rx1}
    \po{ix}{rx1}
  \end{tikzpicture}\]
Here we have used the \emph{ordering read}
clause of \ref{def:prefix}:
``$\bForm'$ implies $\bForm[\aVal/\aLoc]$, if $\aAct$ reads $\aVal$ from $\aLoc$ and $\cEv<'\aEv$,''
where $a=(\DW{x}{0})$,  $\bForm=(x<2)$ and $\bForm'=\TRUE$.  As long as
require
\begin{math}
  (\DW{x}{0})<
  (\DR{x}{1}),
\end{math}
we can use this case since $\TRUE$ implies $0<2$.

\subsection{Causality test case 9}

Test case 9 asks whether:
\begin{displaymath}
  \VAR x\GETS 0\SEMI
  \VAR y\GETS 0\SEMI
  (\IF(x<2)\THEN y\GETS 1\FI 
  \PAR
  x\GETS y
  \PAR
  y\GETS 2\SEMI)
\end{displaymath}
may read $1$ for both $x$ and $y$.  This behavior is also allowed.  This is
``similar to test case $8$, except that $x$ is not always $0$ or
$1$. However, a compiler might determine that the read of $x$ by thread $1$
will never see the write by thread $3$ (perhaps because thread $3$ will be
scheduled after thread $1$)''

Reasoning as for test case 8, the semantics of test case 9 includes:
\[\begin{tikzpicture}[node distance=1em]
  \event{ix}{\DW{x}{0}}{}
  \event{iy}{\DW{y}{0}}{right=of ix}
  \event{rx1}{\DR{x}{1}}{right=2.5 em of iy}
  \event{wy1}{\DW{y}{1}}{right=of rx1}
  \event{ry1}{\DR{y}{1}}{right=2.5em of wy0}
  \event{wx1}{\DW{x}{1}}{right=of ry0}
  \event{wx2}{\DW{x}{2}}{right=2.5em of wx1}
  \po{ry1}{wx1}
  \po[out=30,in=150]{ix}{rx1}
  \rf[in=-25,out=-160]{wx1}{rx1}
  \rf[out=20,in=160]{wy1}{ry1}
  \wk[out=-25,in=-150]{ix}{wx1}
  \wk[out=25,in=155]{iy}{wy1}
  \wk[out=-25,in=-150]{ix}{wx2}
\end{tikzpicture}\]

Thus, with respect to the introduction of new threads, our model appears to
be more robust than the event structures semantics of
\cite{DBLP:conf/lics/JeffreyR16}, which fails on this test case.

\subsection{Causality test case 14}

Test case 14 asks whether:
\begin{displaymath}
  \VAR a\GETS 0\SEMI
  \VAR b\GETS 0\SEMI
  \VAR y\GETS 0\SEMI
  (\IF(a)\THEN b\GETS 1\ELSE y\GETS 1\FI 
  \PAR
  \WHILE(y+b==0) \THEN\SKIP\FI\; a\GETS1)
\end{displaymath}
may read $1$ for $a$ and $b$, yet $0$ for $y$.  Here $a$ and $b$ are regular
variables and $y$ is volatile, which is equivalent to release/acquire in this
example.  This behavior is also disallowed, since ``in all sequentially
consistent executions, [the read of $a$ gets $0$] and the program is
correctly synchronized. Since the program is correctly synchronized in all SC
executions, no non-SC behaviors are allowed.''

Unrolling the loop once, we have:
\begin{displaymath}
  \VAR a\GETS 0\SEMI
  \VAR b\GETS 0\SEMI
  \VAR y\GETS 0\SEMI
  (\IF(a)\THEN b\GETS 1\ELSE y\GETS 1\FI 
  \PAR
  \IF(y\lor b)\THEN a\GETS 1\FI)
\end{displaymath}
We argue that any execution with $(\DR{a}{1})$, $(\DR{b}{1})$, and
$(\DR{y}{0})$ must be cyclic.  The closure requirements require that
\begin{math}
  (\DW{a}{1})<(\DR{a}{1})
  \;\text{and}\;
  (\DR{b}{1})<(\DR{b}{1}).
\end{math}
Ignoring initialization, least ordered execution that includes all of these
actions is:
\[\begin{tikzpicture}[node distance=1em]
  \event{ra1}{\DR{a}{1}}{}
  \event{wb1}{\DW{b}{1}}{below=of ra1}
  \nonevent{wy1}{\DW{y}{1}}{left=of wb1}
  \event{rb1}{\DR{b}{1}}{right=4.5em of ra1}
  \event{ry0}{\DR{y}{0}}{right=of rb1}
  \event{wa1}{\DW{a}{1}}{below=of rb1}
  \po{ra1}{wb1}
  \po{rb1}{wa1}
  \rf{wa1}{ra1}
  \rf{wb1}{rb1}
\end{tikzpicture}\]
where the read of $a$ is ordering for $(\DW{b}{1})$ but
not $(\DW{y}{1})$, and the read of $b$ is ordering for $(\DW{a}{1})$ but the
read of $y$ is not.  $(\DW{y}{1})$ is crossed out, since its
precondition must imply $(\lnot a)[1/a]$, which is equivalent to $\FALSE$.
To avoid order from $(\DR{y}{0})$ to $(\DW{a}{1})$, we
have strengthened the predicate on $(\DW{a}{1})$ from $(y\lor b)$ to
$(y=0\land b=1)$.  Note that we cannot use this trick symmetrically to remove
the order from $(\DR{b}{1})$ to $(\DW{a}{1})$, since $b=1$ does not follow
from the initialization of $b$.


\subsection{Thread inlining}

One property one could ask of a model of shared memory is thread
inlining: any execution of $\sem{P\SEMI Q}$ is an execution of $\sem{P
  \PAR Q}$. This is \emph{not} a goal of our model, and indeed is not
satisfied, due to the different semantics of concurrent and sequential
memory accesses. We demonstrate this by considering an example from
the Java Memory Model~\cite{Manson:2005:JMM:1047659.1040336}, which shows that Java does not
satisfy thread inlining either.

The lack of thread inlining is related to the different dependency
relations introduced by sequential and concurrent access.
Recall from \S\ref{sec:sequential-memory} that the program
\verb`(x := 0; y := x+1;)` has execution:
\[\begin{tikzpicture}[node distance=1em]
  \event{wx0}{\DW{x}{0}}{}
  \event{wy1}{\DW{y}{1}}{right=of wx0}
\end{tikzpicture}\]
but that \verb`(x := 1; || y := x+1;)` has:
\[\begin{tikzpicture}[node distance=1em]
  \event{wx1}{\DW{x}{1}}{}
  \event{rx1}{\DR{x}{1}}{right=2.5em of wx1}
  \event{wy2}{\DW{y}{2}}{right=of rx1}
  \rf{wx1}{rx1}
  \po{rx1}{wy2}
\end{tikzpicture}\]
That is, in the sequential case there is no dependency from the
write of $x$ to the write of $y$, but in the concurrent case there
is such a dependency.

This can be used to construct a counter-example to thread inlining, based on~\cite[Ex~11]{Manson:2005:JMM:1047659.1040336}:
\begin{verbatim}
  x := 0; if (x == 1) { z := 1; } else { x := 1; } || y := x; || x := y;
\end{verbatim}
This has no execution containing $(\DW z1)$. Any attempt to build such an execution
results in a cycle:
\[\begin{tikzpicture}[node distance=1em]
  \event{rx1a}{\DR{x}{1}}{}
  \event{wz1}{\DW{z}{1}}{right=of rx1a}
  \nonevent{wx1a}{\DW{x}{1}}{right=of wz1}
  \event{rx1b}{\DR{x}{1}}{right=2.5em of wx1a}
  \event{wy1}{\DW{y}{1}}{right=of rx1b}
  \event{ry1}{\DR{y}{1}}{right=2.5em of wy1}
  \event{wx1b}{\DW{x}{1}}{right=of ry1}
  \po{rx1a}{wz1}
  \po[out=25, in=150]{rx1a}{wx1a}
  \po{rx1b}{wy1}
  \po{ry1}{wx1b}
  \rf{wy1}{ry1}
  \rf[out=160, in=30]{wx1b}{rx1a}
  \rf[out=160, in=30]{wx1b}{rx1b}
\end{tikzpicture}\]
Inlining the thread \verb|(y := x)| gives~\cite[Ex~12]{Manson:2005:JMM:1047659.1040336}:
\begin{verbatim}
  x := 0; if (x == 1) { z := 1; } else { x := 1; } y := x; || x := y;
\end{verbatim}
with execution:
\[\begin{tikzpicture}[node distance=1em]
  \event{rx1a}{\DR{x}{1}}{}
  \event{wz1}{\DW{z}{1}}{right=of rx1a}
  \nonevent{wx1a}{\DW{x}{1}}{right=of wz1}
  \event{wy1}{\DW{y}{1}}{right=of wx1a}
  \event{ry1}{\DR{y}{1}}{right=2.5em of wy1}
  \event{wx1b}{\DW{x}{1}}{right=of ry1}
  \po{rx1a}{wz1}
  \po[out=25, in=150]{rx1a}{wx1a}
  \po{ry1}{wx1b}
  \rf{wy1}{ry1}
  \rf[out=160, in=30]{wx1b}{rx1a}
\end{tikzpicture}\]
To see why this execution exists, consider the program fragment:
\begin{verbatim}
  if (x == 1) { z := 1; } else { x := 1; } y := x;
\end{verbatim}
Removing the syntax sugar, this is:
\begin{verbatim}
  r1 := x; if (r1 == 1) {
    z := 1; r2 := x; y := r2; skip
  } else {
    x := 1; r3 := x; y := r3; skip
  }
\end{verbatim}
Now, $\sem{z := 1\SEMI r_2 := x\SEMI y := r_2\SEMI \SKIP}$
includes pomset:
\[\begin{tikzpicture}[node distance=1em]
  \event{wz1}{r_1=1 \mid \DW{z}{1}}{}
  \event{wy1}{r_1=x=1 \mid \DW{y}{1}}{right=of wz1}
\end{tikzpicture}\]
and $\sem{x := 1\SEMI r_3 := x\SEMI y := r_3\SEMI \SKIP}$
includes pomset:
\[\begin{tikzpicture}[node distance=1em]
  \event{wx1a}{r_1\neq 1 \mid \DW{x}{1}}{}
  \event{wy1}{r_1\neq 1 \mid \DW{y}{1}}{right=of wx1a}
\end{tikzpicture}\]
so  $\sem{\IF (r_1 = 1) \THEN z := 1\SEMI r_2 := x\SEMI y := r_2\SEMI \SKIP \ELSE x := 1\SEMI r_3 := x\SEMI y := r_3\SEMI \SKIP \FI}$ includes:
\[\begin{tikzpicture}[node distance=1em]
  \event{wz1}{r_1=1 \mid \DW{z}{1}}{}
  \event{wx1a}{r_1\neq1 \mid \DW{x}{1}}{right=of wz1}
  \event{wy1}{(r_1=x=1) \lor (r_1\neq1) \mid \DW{y}{1}}{right=of wx1a}
\end{tikzpicture}\]
which means $\sem{\IF (r_1 = 1) \THEN z := 1\SEMI r_2 := x\SEMI y := r_2\SEMI \SKIP \ELSE x := 1\SEMI r_3 := x\SEMI y := r_3\SEMI \SKIP \FI}[x/r_1]$ includes:
\[\begin{tikzpicture}[node distance=1em]
  \event{wz1}{x=1 \mid \DW{z}{1}}{}
  \event{wx1a}{x\neq1 \mid \DW{x}{1}}{right=of wz1}
  \event{wy1}{(x=x=1) \lor (x\neq1)) \mid \DW{y}{1}}{right=of wx1a}
\end{tikzpicture}\]
Now $(x=x=1) \lor (x\neq1)$ is a tautology, so this is just:
\[\begin{tikzpicture}[node distance=1em]
  \event{wz1}{x=1 \mid \DW{z}{1}}{}
  \event{wx1a}{x\neq1 \mid \DW{x}{1}}{right=of wz1}
  \event{wy1}{\DW{y}{1}}{right=of wx1a}
\end{tikzpicture}\]
and so $\sem{r_1 \GETS x\SEMI \IF (r_1 = 1) \THEN z := 1\SEMI r_2 := x\SEMI y := r_2\SEMI \SKIP \ELSE x := 1\SEMI r_3 := x\SEMI y := r_3\SEMI \SKIP \FI}$ includes:
\[\begin{tikzpicture}[node distance=1em]
  \event{rx1a}{\DR{x}{1}}{}
  \event{wz1}{1=1 \mid \DW{z}{1}}{right=of rx1a}
  \event{wx1a}{1\neq1 \mid \DW{x}{1}}{right=of wz1}
  \event{wy1}{\DW{y}{1}}{right=of wx1a}
  \po{rx1a}{wz1}
  \po[out=25, in=150]{rx1a}{wx1a}
\end{tikzpicture}\]
which simplifies to:
\[\begin{tikzpicture}[node distance=1em]
  \event{rx1a}{\DR{x}{1}}{}
  \event{wz1}{\DW{z}{1}}{right=of rx1a}
  \nonevent{wx1a}{\DW{x}{1}}{right=of wz1}
  \event{wy1}{\DW{y}{1}}{right=of wx1a}
  \po{rx1a}{wz1}
  \po[out=25, in=150]{rx1a}{wx1a}
\end{tikzpicture}\]
as required. The rest of the example is straightforward, and shows that our semantics
agrees with the JMM in not supporting thread inlining.



% \subsection{Word tearing}

% \todo{Remove this section, since it's not needed for transactions?}

% In \S\ref{sec:transactions}, we shall be considering transactional memory,
% and in \S\ref{sec:transactions} show that we can model a simplified version
% of an information flow attack on transactions. In order to model transactions,
% we need to consider actions that can write many memory locations at once,
% since this is part of the semantics of commitment. To lead up to this, we first
% consider a simpler scenario of many-location writes and reads, which is word
% tearing.

% In word tearing, a program contains a write instruction with data larger
% than the hardware word size, for example copying a byte array, or assigning
% a 64-bit float on a 32-bit architecture. For example, consider the program:
% \begin{verbatim}
%   (x := [0, 0];) || (x := [1, 1];) || (r := x;)
% \end{verbatim}
% This has executions in which the read of $x$ only reads from one of the writes,
% for example:
% \[\begin{tikzpicture}[node distance=1em]
%   \event{wx00}{\DW{x}{[0,0]}}{}
%   \event{wx11}{\DW{x}{[1,1]}}{right=2.5em of wx00}
%   \event{rx00}{\DR{x}{[0,0]}}{right=2.5em of wx11}
%   \rf[out=20, in=160]{wx00}{rx00}
% \end{tikzpicture}\]
% but also has executions in which the read of $x$ reads from both writes,
% for example:
% \[\begin{tikzpicture}[node distance=1em]
%   \event{wx00}{\DW{x}{[0,0]}}{}
%   \event{wx11}{\DW{x}{[1,1]}}{right=2.5em of wx00}
%   \event{rx01}{\DR{x}{[0,1]}}{right=2.5em of wx11}
%   \rfx[out=20, in=160]{wx00}{x[0]}{rx01}
%   \rfx[out=-20, in=-160]{wx11}{x[1]}{rx01}
% \end{tikzpicture}\]
% Word tearing can occur, for example, in Java extended floating point~\cite{jmm},
% LLVM 64-bit instructions on 32-bit hardware~\cite{llvm}, or in
% JavaScript SharedArrayBuffers~\cite{js-sab}.

% \newcommand{\rfControl}[4][]{\draw[rf,#1](#2) .. controls (#3) .. (#4);}
% \[\begin{tikzpicture}[node distance=1em]
%   \event{wx0}{\DW{x}{0}}{}
%   \event{wx1}{\DW{x}{1}}{right=of wx0}
%   \event{wy0}{\DW{y}{0}}{right=2.5em of wx1}
%   \event{wy1}{\DW{y}{1}}{right=of wy0}
%   \event{rx1}{\DR{x}{1}}{right=2.5 em of wy1}
%   \event{ry0}{\DR{y}{0}}{right=of rx1}
%   \event{ry1}{\DR{y}{1}}{right=2.5 em of ry0}
%   \event{rx0}{\DR{x}{0}}{right=of ry1}
%   \rf[out=20,in=160]{wx1}{rx1}
%   \rf[out=20,in=160]{wy0}{ry0}
%   \rf[out=340,in=200]{wy1}{ry1}
%   \coordinate (a) [below=of wy1];
%   \rfControl[out=340,in=200]{wx0}{a}{rx0}
%   \wk{wx0}{wx1}
%   \wk{wy0}{wy1}
%   \po{rx1}{ry0}
%   \po{ry1}{rx0}
% \end{tikzpicture}\]


% Batty section 4:
% \cite[\S4]{DBLP:conf/esop/BattyMNPS15},
% Example LB+ctrldata+ctrl-double (language must allow)
% r1=loadrlx(x) //reads 42
% if (r1 == 42)
%   storerlx(y,r1)

% r2=loadrlx(y) //reads 42
% if (r2 == 42)
%   storerlx (x,42)
% else
% storerlx (x,42)

% a:RRLX x=42 sb,dd,cd
% c:RRLX y=42 sb,cd
%   This is forbidden on hardware if compiled naively, as the architectures respect read-to-write control dependencies, but in practice compilers will collapse con- ditionals like that of the second thread, removing the control dependencies from the read of y to the writes of x and making the code identical to the previous example. As that example is allowed and observable on hardware (and we pre- sume that it would be impractical to outlaw such optimisation for C or C++), the language must also allow this execution. But this execution has a cycle in the union of reads-from and dependency, so we cannot simply exclude all those.
% Then one might hope for some other adaptation of the C/C++11 model, but the following example shows at least that there is no per-candidate-execution solution.
% Example LB+ctrldata+ctrl-single (language can and should forbid)
% r1=loadrlx(x) //reads 42 if (r1 == 42)
% storerlx (y,r1) r2=loadrlx (y) //reads 42 if (r2 == 42)
% a:RRLX x=42 sb,dd,cd
% rf
% b:WRLX y=42
% c:RRLX y=42 sb,cd
% rf
% d:WRLX x=42
% rf rf
% b:WRLX y=42 d:WRLX x=42
%   storerlx (x,42)
\end{document}
