\documentclass[conference]{IEEEtran}
\IEEEoverridecommandlockouts

\usepackage[numbers]{natbib}
\usepackage{amsmath}
\usepackage{amssymb}
\usepackage{stmaryrd}
\usepackage{hyperref}
\newcommand{\email}[1]{\href{mailto:#1}{{\UrlFont #1}}}
\usepackage{amsthm}
\theoremstyle{plain}
\newtheorem{theorem}{Theorem}[section]
\newtheorem{proposition}[theorem]{Proposition}
\theoremstyle{definition}
\newtheorem{definition}[theorem]{Definition}
% \usepackage{verbatim}
% \makeatletter
% \def\verbatim@font{\sffamily\upshape}
% \makeatother

\bibliographystyle{plainnat}
\usepackage{macros}
\newcommand{\ignore}[1]{}
\newcommand{\todo}[1]{{\color{red}\textbf{\{#1\}}}}

\begin{document}

\title{The Code That Never Ran\\Modeling Attacks on Speculative Evaluation
  \thanks{Identify applicable funding agency here. If none, delete this.}
}

\author{
\IEEEauthorblockN{Craig Disselkoen}
\IEEEauthorblockA{\textit{University of California San Diego}\\
  \textit{Mozilla Research Internship}\\
  \email{cdisselk@cs.ucsd.edu}}
\and
\IEEEauthorblockN{Radha Jagadeesan}
\IEEEauthorblockA{\textit{DePaul University}\\
  \email{rjagadeesan@cs.depaul.edu}}
\and
\IEEEauthorblockN{Alan Jeffrey}
\IEEEauthorblockA{\textit{Mozilla Research}\\
  \email{ajeffrey@mozilla.com}}
\and
\IEEEauthorblockN{James Riely}
\IEEEauthorblockA{\textit{DePaul University}\\
  \email{jriely@cs.depaul.edu}}
}


\maketitle

\begin{abstract}
  This paper studies information flow caused by speculation mechanisms
  in hardware and software.  The Spectre attack shows that there are
  practical information flow attacks which use an interaction of
  dynamic security checks, speculative evaluation and cache timing.
  Previous formal models of program execution have not been designed
  to model speculative evaluation, and so do not capture attacks such
  as Spectre. In this paper, we propose a model based on pomsets which
  is designed to model speculative evaluation. The model provides a
  compositional semantics for a simple shared-memory concurrent
  language, which captures features such as data and control
  dependencies, relaxed memory and transactions. We provide models for
  existing information flow attacks based on speculative evaluation
  and transactions.  We also model new information flow attacks based on compiler
  optimizations. The new attacks are experimentally validated against
  gcc and clang.  We develop a simple temporal logic that supports 
  invariant reasoning.
\end{abstract}

\section{Introduction}

This paper studies information flow caused by speculation mechanisms
in hardware and software.

Information flow provides a formal
foundation for end-to-end security.  Informally, a program is secure
if there is no observable dependency of low-security observables on high-security inputs.
The precise formalization of this intuitive idea has been the topic of
extensive research \cite{Sabelfeld:2006:LIS:2312191.2314769}, encompassing a variety of language
features such as non-determinism~\cite{Wittbold1990InformationFI},
concurrency~\cite{Smith:1998:SIF:268946.268975}, reactivity~\cite{O'Neill:2006:ISI:1155442.1155677}, and
probability~\cite{Gray:1992:TMF:2699806.2699811}. The static and dynamic enforcement
of these definitions in general purpose languages~\cite{myers-popl99} has % also
% been studied extensively and has
influenced language design and implementation.

A key parameter in defining information flow is the \emph{observational power} of the attacker model. Whereas the classical
input-output behavior is often an adequate foundation,
it has long been known~\cite{Lampson:1973:NCP:362375.362389,Biswas:2017:STC:3058791.3023872} that side-channels that leak
information arise from other observables such as execution time and
power consumption.
Recently, the Spectre family of attacks~\cite{DBLP:journals/corr/abs-1801-01203} has
shown that side-channels arise from speculative evaluation.
%In this paper, we develop a formal model of such attacks.

There are several sources of speculative evaluation.  Each of these is
designed so that failed speculation does not affect the input-output behavior
of the program, but may affect the behavior, opening an opportunity
for a side-channel attack:
\begin{itemize}
\item To facilitate pipelining, microprocessors use heuristics to predict the
  outcome of instructions. Execution proceeds until these predictions can be
  validated, at which point the relevant instructions are either committed or
  aborted.  The Spectre family of
  attacks~\cite{DBLP:journals/corr/abs-1801-01203} exploit the timing
  differences between successful and unsuccessful predictions.  This means,
  for example, that a single execution of
  $(\IF(\aExp)\THEN \aCmd \ELSE \bCmd \FI)$ may depend on both $\aCmd$ and
  $\bCmd$.  This differs from the standard semantics of the conditional, in
  which executions of $\aCmd$ and $\bCmd$ are disjoint.
\item Some modern microprocessors also support transactional
  memory~\cite{ChongSW18}, where aborted transactions are meant to be
  unobservable.  Transactions may abort due to cache conflicts, however, and
  this mechanism can be exploited to improve the efficacy of Spectre-like
  attacks~\cite{DBLP:conf/uss/DisselkoenKPT17}.
\item Relaxed memory models
  \cite{Manson:2005:JMM:1047659.1040336,Boehm:2008:FCC:1375581.1375591,DBLP:conf/popl/ZhaoNMZ12}
  allow for the observation of control and data dependencies. This creates an
  opportunity for information flows caused by optimizing compilers, whose
  behavior is driven by dependency analysis.  For example,
  $(\IF(\aReg)\THEN \aLoc\GETS1 \ELSE \aLoc\GETS1 \FI)$ can be optimized to
  $(\aLoc\GETS1)$, whereas
  $(\IF(\aReg)\THEN \aLoc\GETS1 \ELSE \aLoc\GETS2 \FI)$ cannot be so
  optimized.  
\end{itemize}

We develop a model to capture such attacks.
%% This line of research was initiated by~\citet{Zhang:2012:LCM:2345156.2254078}.  
%% Whereas they explore static annotations to address side channels in the context of hardware description languages, we explore a model of programs
%% that captures enough detail to reveal and analyze the presence of side
%% channels revealed by speculative execution.  
%
Our model is based on \emph{partially ordered multisets}~\cite{GISCHER1988199,Plotkin:1997:TSP:266557.266600}
(``pomsets''), whose labels are given by read and write actions. These can be
visualized as a graph where the edges indicate dependencies, for example
$(\aReg\GETS\aLoc\SEMI \bLoc\GETS1\SEMI \cLoc\GETS\aReg+1)$
has an execution modeled by the pomset:
\[\begin{tikzpicture}[node distance=1em]
  \event{rx1}{\DR{\aLoc}{1}}{}
  \event{wy1}{\DW{\bLoc}{1}}{right=of rx1}
  \event{wz2}{\DW{\cLoc}{2}}{right=of wy1}
  \po[out=25,in=155]{rx1}{wz2}
\end{tikzpicture}\]
The edge from $(\DR{\aLoc}{1})$ to $(\DW{\cLoc}{2})$ indicates a
data dependency. The novel aspect of the model is that events have
\emph{preconditions} which may be false. These are used in giving the
semantics of conditionals, for example
$(\IF(\aLoc)\THEN \bLoc\GETS1\SEMI\cLoc\GETS1 \ELSE \bLoc\GETS2\SEMI\cLoc\GETS1\FI)$
has an execution:
\[\begin{tikzpicture}[node distance=1em]
  \event{rx1}{\DR{\aLoc}{1}}{}
  \event{wy1}{\DW{\bLoc}{1}}{right=of rx1}
  \nonevent{wy2}{\DW{\bLoc}{2}}{below=of wy1}
  \event{wz1}{\DW{\cLoc}{1}}{right=of wy1}
  \po{rx1}{wy1}
  \po{rx1}{wy2}
\end{tikzpicture}\]
The edges from $(\DR{\aLoc}{1})$ to $(\DW{\bLoc}{1})$ and
$(\DW{\bLoc}{2})$ indicate control dependencies. The presence of
a crossed out $(\DW{\bLoc}{2})$ indicates an event with an unsatisfiable precondition,

The novel contributions of this paper are:
\begin{itemize}

\item a model of program execution that includes speculation (\S\ref{sec:model}),

\item examples showing how the model can be applied,
  including information flow attacks on
  hardware, optimizing compilers, and transactional memory (\S\ref{sec:examples}),

\item a new class of attacks targeting optimizing compilers and relaxed memory
  (\S\ref{sec:info-flow-attack} and \S\ref{sec:dse}),

\item experimental evidence about how practical it is to mount
  the new class of attacks (\S\ref{sec:experiments}), and

\item a temporal logic which supports compositional proof (\S\ref{sec:logic}).

\end{itemize}
Readers who wish to focus on the impact of the model can skip past \S\ref{sec:model}
on first reading, and refer back to it when needed.

\paragraph*{Speculation in Information flow}

The information flow literature has largely, with the exceptions noted below, not addressed matters of speculation and the side channels that arise from the observations  of efficiency of executions such as execution time and  power consumption.  For example, the well-known Smith-Volpano type system (which guarantees noninterference) will allow the Prime-Abort attack as formalized
in the paper (and Spectre attack as well, if we were to add cache sets to the semantics of a ``touch'' primitive that we study). 

~\citet{Zhang:2012:LCM:2345156.2254078}  models hardware (and thus the
microarchitecture) with timing attacks, and explore explicit and static annotations to address timing side channels for hardware description languages.  However, this paper does not address speculative execution. 

This paper's primary focus is not weak memory.  However, hardware relaxed memory (such as Total and Partial Store ordering~\cite{SparcV9}) supports differing views of the memory between threads, that can be viewed as a form of thread-specific speculation.  So, the research into the impact of hardware relaxed memory on information flow in programs~\cite{6957104,Vaughan:2012:SIF} is relevant.  These papers show that the information flow exhibited by a program  depends crucially on the particular model of relaxed memory; for example, they demonstrate progams that have no information flow when executed in the in usual {\em sequentially consistent} memory, and yet exhibit information flow when executed in the TSO model. 

In contrast to the hardware relaxed memory models addressed in these papers, software relaxed memory models (such as the JMM~\cite{Manson:2005:JMM:1047659.1040336} and C11~\cite{Boehm:2008:FCC:1375581.1375591} ) also incorporate speculation on conditionals.   

Our model captures enough detail  enough to reveal and analyze the presence of side
channels revealed by speculative executions that are implicit in the literature on hardware and software relaxed memory models. Thus, we are able to demonstrate example attacks that show that ordinary compiler optimizations can
violate some intuitively expected informal information flow guarantees.  Furthermore,  as far as we know, the observation that compiler optimizations can result in information flows that can be observed without timers, is new to this paper.  

Self-composition~\cite{Barthe:2004:SIF:1009380.1009669} reduces the problem of secure information flow of a program to a {\em safety} property of a program derived by composing the program with a renaming of itself.  Our work shows that the traditional conditional serves as a self-composition operator in the presence of speculation.  


\section{Model}
\label{sec:model}

The model used in this paper is one of sets of pomsets with event labels of the form
$(\aForm \mid \aAct)$, where $\aForm$ is the event's precondition
(such as $\aExp=\aVal$) and $\aAct$ is the event's action (such as $\DW\aLoc\aVal$).
For example the semantics of the program $(\aLoc\GETS\aExp)$ includes the case
where $\aExp$ is $\aVal$, which is written to $\aLoc$, and is captured
by the one-event pomset:
\[\begin{tikzpicture}[node distance=1em]
  \event{wxv}{\aExp=\aVal \mid \DW{\aLoc}{\aVal}}{}
\end{tikzpicture}\]
We make few requirements of the logic of preconditions, save that it
has includes equalities between expressions, is closed under substitution,
and supports a notion of implication.

% For example, the set of pomsets $\sem{\aReg\GETS\bLoc\SEMI \aLoc\GETS\aReg+1}$ contains:
% \[\begin{tikzpicture}[node distance=1em]
%   \event{ry1}{\DR{\bLoc}{1}}{}
%   \event{wx2}{\DW{\aLoc}{2}}{right=of ry1}
%   \po{ry1}{wx2}
% \end{tikzpicture}\]
The semantics is defined compositionally. As an example, we show how to 
construct one of the pomsets in
$\sem{\aReg\GETS\bLoc\SEMI \aLoc\GETS\aReg+1}$.
First, $\sem{\aLoc\GETS\aReg+1}$
contains the pomset:
\[\begin{tikzpicture}[node distance=1em]
  \event{wx2}{\aReg=1 \mid \DW{\aLoc}{2}}{}
\end{tikzpicture}\]
Next, we perform the substitution of $\aReg$ with $1$ in every precondition,
to get that $\sem{\aLoc\GETS\aReg+1}[1/\aReg]$
contains the pomset:
\[\begin{tikzpicture}[node distance=1em]
  \event{wx2}{1=1 \mid \DW{\aLoc}{2}}{}
\end{tikzpicture}\]
and since $(1=1)$ is a tautology, we elide it:
\[\begin{tikzpicture}[node distance=1em]
  \event{wx2}{\DW{\aLoc}{2}}{}
\end{tikzpicture}\]
This substitution is performed in defining
$\sem{\aReg\GETS\bLoc\SEMI \aLoc\GETS\aReg+1}$, which contains
the pomset:
\[\begin{tikzpicture}[node distance=1em]
  \event{ry1}{\DR{\bLoc}{1}}{}
  \event{wx2}{\DW{\aLoc}{2}}{right=of ry1}
  \po{ry1}{wx2}
\end{tikzpicture}\]
There is an ordering $(\DR{\bLoc}{1}) < (\DW{\aLoc}{2})$ (represented pictorially as an arrow)
because the precondition $(\aReg=1)$ depends on $\aReg$. If the precondition
was independent of $\aReg$ then there would be no ordering, for example
$\sem{\aReg\GETS\bLoc\SEMI \aLoc\GETS\aReg+1-\aReg}$ contains
the pomset:
\[\begin{tikzpicture}[node distance=1em]
  \event{ry1}{\DR{\bLoc}{1}}{}
  \event{wx1}{\DW{\aLoc}{1}}{right=of ry1}
\end{tikzpicture}\]
since the precondition $(\aReg+1-\aReg=1)$ is independent of $\aReg$.

The main novelty of our semantics is the use of preconditions, which allow us
to provide an unusual model of conditionals. In most
models, an execution of
$\sem{\IF(\aExp)\THEN \aCmd \ELSE \bCmd \FI}$ would either be
given by an execution from $\sem{\aCmd}$ or from $\sem{\bCmd}$, but not both.
In our semantics, a pomset
in $\sem{\IF(\aExp)\THEN \aCmd \ELSE \bCmd \FI}$ may include
both a pomset from $\sem{\aCmd}$ \emph{and} a pomset from $\sem{\bCmd}$.
For example, $\sem{\IF(\aExp)\THEN \aLoc\GETS1 \ELSE \aLoc\GETS2 \FI}$
contains:
\[\begin{tikzpicture}[node distance=1em]
  \event{wx1}{\aExp\neq0 \mid \DW{\aLoc}{1}}{}
  \event{wx2}{\aExp=0    \mid \DW{\aLoc}{2}}{right=of wx1}
\end{tikzpicture}\]
that is we have behavior from both branches of execution.

Moreover, two events representing the same action on both sides of a
conditional can be merged, producing a single event.
The precondition of the merged event is the disjunction of the preconditions
of the original events.
For example
$\sem{\IF(\aExp)\THEN \aLoc\GETS1\SEMI \bLoc\GETS3 \ELSE \aLoc\GETS2\SEMI \bLoc\GETS3 \FI}$
contains:
\[\begin{tikzpicture}[node distance=1em]
  \event{wx1}{\aExp\neq0 \mid \DW{\aLoc}{1}}{}
  \event{wx2}{\aExp=0    \mid \DW{\aLoc}{2}}{right=of wx1}
  \event{wy3}{(\aExp\neq0) \lor (\aExp=0) \mid \DW{\bLoc}{3}}{below=3ex of $(wx1)!0.5!(wx2)$}
\end{tikzpicture}\]
and since $(\aExp\neq0) \lor (\aExp=0)$ is a tautology, this is:
\[\begin{tikzpicture}[node distance=1em]1
  \event{wx1}{\aExp\neq0 \mid \DW{\aLoc}{1}}{}
  \event{wx2}{\aExp=0    \mid \DW{\aLoc}{2}}{right=of wx1}
  \event{wy3}{\DW{\bLoc}{3}}{right=of wx2}
\end{tikzpicture}\]

Combining this model of conditionals with the previously discussed model of memory using substitutions
gives that
$\sem{\IF(\cLoc)\THEN \aLoc\GETS1\SEMI \bLoc\GETS3 \ELSE \aLoc\GETS2\SEMI \bLoc\GETS3 \FI}$
contains:
\[\begin{tikzpicture}[node distance=1em]
  \event{rz1}{\DR{\cLoc}{1}}{}
  \event{wx1}{1\neq0 \mid \DW{\aLoc}{1}}{right=of rz1}
  \event{wx2}{1=0    \mid \DW{\aLoc}{2}}{right=of wx1}
  \event{wy3}{\DW{\bLoc}{3}}{right=of wx2}
  \po{rz1}{wx1}
  \po[out=25,in=155]{rz1}{wx2}
\end{tikzpicture}\]
and we visualize unsatisfiable preconditions as crossed out:
\[\begin{tikzpicture}[node distance=1em]
  \event{rz1}{\DR{\cLoc}{1}}{}
  \event{wx1}{\DW{\aLoc}{1}}{right=of rz1}
  \nonevent{wx2}{\DW{\aLoc}{2}}{right=of wx1}
  \event{wy3}{\DW{\bLoc}{3}}{right=of wx2}
  \po{rz1}{wx1}
  \po[out=25,in=155]{rz1}{wx2}
\end{tikzpicture}\]
Note that this semantics captures control dependencies
such as $(\DR\cLoc1)<(\DW\aLoc1)$, independencies
such as $(\DR\cLoc1)\not<(\DW\bLoc3)$, and failed
speculations such as the crossed out $(\DW\aLoc2)$.

In summary, the features we need of the underlying data model are:
\begin{itemize}
\item \emph{actions}, which may read or write memory locations, and
\item \emph{preconditions}, which form a logic closed under substitution.
\end{itemize}
We make data models precise in~\S\ref{sec:preliminaries},
and define pomsets in \S\ref{sec:pomsets}.
In \refapp{sets-of-pomsets}, we define
operations on sets of pomsets, which are used in~\S\ref{sec:semantics}
to give a compositional semantics for
a simple imperative language.

\subsection{Data models}
\label{sec:preliminaries}

A \emph{data model} consists of:
\begin{itemize}
\item a set of \emph{memory locations} $\Loc$, ranged over by
  $\aLoc$ and $\bLoc$,
\item a set of \emph{registers} $\Reg$, ranged over by
  $\aReg$ and $\bReg$,
\item a set of \emph{values} $\Val$, ranged over by
  $\aVal$ and $\bVal$,
\item a set of \emph{expressions} $\Exp$, ranged over by
  $\aExp$ and $\bExp$,
\item a set of \emph{logical formulae} $\Formulae$, ranged over by
  $\aForm$ and $\bForm$, and
\item a set of \emph{actions} $\Act$, ranged over by $\aAct$ and $\bAct$,
\end{itemize}
such that:
\begin{itemize}
\item values include at least the constants $0$ and $1$,
\item expressions include at least registers and values,
\item expressions are closed under substitutions of the form $\aExp[\bExp/\aReg]$,
\item formulae include at least $\TRUE$, $\FALSE$, and equalities of the form $(\aExp=\bExp)$ and $(\aLoc=\bExp)$,
\item formulae are closed under negation, conjunction, disjunction,
\item formulae are closed under substitutions of the form $\aForm[\aLoc/\aReg]$ or $\aForm[\bExp/\aLoc]$,
\item there is a relation $\vDash$ between formulae, and
\item there are partial functions $\rreads$ and $\rwrites: \Act \fun (\Loc \times \Val)$.
\end{itemize}
We shall say $\aAct$ \emph{reads} $\aVal$ \emph{from} $\aLoc$ whenever
$\rreads(\aAct) = (\aLoc,\aVal)$, and
$\aAct$ \emph{writes} $\aVal$ \emph{to} $\aLoc$ whenever
$\rwrites(\aAct) = (\aLoc,\aVal)$.
We shall say $\aForm$ \emph{implies} $\bForm$ whenever $\aForm\vDash\bForm$,
$\aForm$ is a \emph{tautology} whenever $\TRUE\vDash\aForm$,
$\aForm$ is \emph{unsatisfiable} whenever $\aForm\vDash\FALSE$, and
$\aForm$ is \emph{independent of $\aLoc$} whenever $\aForm \vDash \aForm[\aVal/\aLoc] \vDash \aForm$ for every $\aVal$.
In examples, the actions are of the form $(\DR{\aLoc}{\aVal})$, which reads $\aVal$ from $\aLoc$,
and $(\DW{\aLoc}{\aVal})$, which writes $\aVal$ to $\aLoc$.

\subsection{3-valued pomsets}
\label{sec:pomsets}

Recall the definition of a pomset from~\cite{GISCHER1988199}:
\begin{definition}
  A \emph{pomset} $(\Event, {\le}, \labelling)$ with alphabet $\Alphabet$
  is a partial order $(\Event, {\le})$ together with
  $\labelling: \Event \fun \Alphabet$.
\end{definition}
Going forward, we fix the alphabet $\Alphabet=(\Formulae\times\Act)$.
We will write $(\aForm \mid \aAct)$ for the pair $(\aForm,\aAct)$,
elide $\aForm$ when $\aForm$ is a tautology, and write $\aAct$ crossed-out ($\NEVER\aAct$)
when $\aForm$ is unsatisfiable.
We lift terminology from logical formulae and actions to events,
for example if $\labelling(\aEv)=(\aForm\mid\aAct)$
then we say
$\aEv$ is unsatisfiable whenever $\aForm$ is unsatisfiable,
$\aEv$ writes $\aVal$ to $\aLoc$ whenever $\aAct$ writes $\aVal$ to $\aLoc$, and
so forth.
We visualize a pomset as a graph where the nodes are drawn from
$\Event$, each node $\aEv$ is labelled with $\labelling(\aEv)$,
and an edge $\bEv \rightarrow \aEv$ corresponds to an ordering
$\bEv\le\aEv$. For example:
\[\begin{tikzpicture}[node distance=1em]
  \event{rx1}{\DR{\aLoc}{1}}{}
  \nonevent{wy0}{\DW{\bLoc}{0}}{right=of rx1}
  \event{wy1}{\DW{\bLoc}{1}}{right=of wy0}
  \po{rx1}{wy0}
  \po[out=30,in=150]{rx1}{wy1}
\end{tikzpicture}\]
is a visualization of the pomset where:
\[\begin{array}{c}
  E = \{ 0,1,2 \} \quad
  0 \le 1 \quad
  0 \le 2 \quad
  \labelling(0) = (\TRUE, \DR{\aLoc}{1}) \\
  \labelling(1) = (\FALSE, \DW{\bLoc}{0}) \quad
  \labelling(2) = (\TRUE, \DW{\bLoc}{1}) \quad
\end{array}\]

We are building a compositional semantics of shared memory
concurrency, which means we require a notion of when
a read has a matching write. This is a property we require
of closed programs, but \emph{not} of open programs.
For example a program whose semantics includes:
\[\begin{tikzpicture}[node distance=1em]
  \event{wx1}{\DW{\aLoc}{1}}{}
  \event{rx0}{\DR{\aLoc}{0}}{right=2.5em of wx1}
  \event{wy0}{\DW{\bLoc}{0}}{right=of rx0}
  \nonevent{wy1}{\DW{\bLoc}{1}}{right=of wy0}
  \po{rx0}{wy0}
  \po[out=30,in=150]{rx0}{wy1}
\end{tikzpicture}\]
may be put in parallel
with another program which writes $0$ to $\aLoc$.
If the program is closed with respect to $\aLoc$ though, such an execution cannot exist,
so we need each read of $\aLoc$ to have a matching write.
This is captured by defining when $\aEv$ \emph{reads $\aLoc$ from} $\bEv$~\cite{alglave}.
A preliminary definition (which, as we shall see, needs to be strengthened) is:
\begin{itemize}
\item $\bEv < \aEv$,
\item $\aEv$ implies $\bEv$,
\item $\bEv$ writes $\aVal$ to $\aLoc$,
  and $\aEv$ reads $\aVal$ from $\aLoc$, and
\item there is no $\bEv < \cEv < \aEv$ such that
  $\cEv$ writes to $\aLoc$.
\end{itemize}
% In diagrams, for readability we often highlight the reads-from edges,
% for example:
%% We visualize rf-pomsets by drawing a dashed edge between nodes in $\RF$,
%% labelled with the memory location,
%% for example:
%% \[\begin{tikzpicture}[node distance=1em]
%%   \event{wx1}{\DW{\aLoc}{1}}{}
%%   \event{x1}{\DR{\aLoc}{1}}{right=5em of wx1}
%%   \nonevent{y0}{\DW{\bLoc}{0}}{below left=of x1}
%%   \event{y1}{\DW{\bLoc}{1}}{below right=of x1}
%%   \rfx{wx1}{x}{x1}
%%   \po{x1}{y0}
%%   \po{x1}{y1}
%% \end{tikzpicture}\]
%% In most cases, the memory location is obvious from context,
%% so we elide it:
% \[\begin{tikzpicture}[node distance=1em]
%   \event{wx1}{\DW{\aLoc}{1}}{}
%   \event{x1}{\DR{\aLoc}{1}}{right=2.5em of wx1}
%   \nonevent{y0}{\DW{\bLoc}{0}}{right=of x1}
%   \event{y1}{\DW{\bLoc}{1}}{right=of y0}
%   \rf{wx1}{x1}
%   \po{x1}{y0}
%   \po[out=30,in=150]{x1}{y1}
% \end{tikzpicture}\]
Unfortunately by itself, this is not enough. The problem is the final
clause saying that there does not exist an $\aLoc$-\emph{blocking}
event $\cEv$ between $\bEv$ and $\aEv$. Unfortunately, concurrency can
turn events that were not $\aLoc$-blockers into an $\aLoc$-blocker,
\emph{even if the new thread does not mention $\aLoc$.}
We give an example to show this in \refapp{blockers}.
This is a problem in that it means the preliminary model violates
\emph{scope extrusion}~\cite{Milner:1999:CMS:329902},
in that we can find programs $\aCmd$ and $\bCmd$ such that
$\sem{\VAR\aLoc\SEMI(\aCmd\PAR\bCmd)}$ is not the same as
$\sem{(\VAR\aLoc\SEMI\aCmd)\PAR\bCmd}$, even if $\bCmd$ does not mention~$\aLoc$.

There are a number of ways this can be addressed; for example,
in models such as~\cite{Batty:2011:MCC:1926385.1926394} the reads-from relation is taken
as a primitive. In this paper, we propose \emph{3-valued pomsets}
as a solution. These are pomsets in which, in addition to positive statements
$(\bEv < \aEv)$ (interpreted as $\aEv$ depends on $\bEv$),
we also have negative statements $(\bEv \ltN \aEv)$
(interpreted as $\aEv$ cannot depend on $\bEv$).

\begin{definition}
  A \emph{3-valued pomset} $(\Event, {\le}, {\ltN}, \labelling)$ 
  is a \emph{pomset} $(\Event, {\le}, \labelling)$
  together with ${\ltN} \subseteq (\Event\times\Event)$ such that:
  \begin{itemize}
  \item if $\bEv \le \aEv$ then $\aEv \ltN \bEv$,
  \item if $\bEv \le \aEv$ and $\bEv \ltN \aEv$ then $\bEv = \aEv$,
  \item if $\cEv \ge \bEv \ltN \aEv$ or $\cEv \ltN \bEv \ge \aEv$ then $\cEv \ltN \aEv$.
  \end{itemize}
\end{definition}

% \begin{definition}
%   A \emph{3-valued poset} $(\Event,{\le},{\ltN})$ is a poset $(\Event,{\le})$
%   together with ${\ltN} \subseteq (\Event\times\Event)$ such that:
%   \begin{itemize}
%   \item if $\bEv \le \aEv$ then $\aEv \ltN \bEv$,
%   \item if $\bEv \le \aEv$ and $\bEv \ltN \aEv$ then $\bEv = \aEv$,
%   \item if $\cEv \ge \bEv \ltN \aEv$ or $\cEv \ltN \bEv \ge \aEv$ then $\cEv \ltN \aEv$.
%   \end{itemize}
% \end{definition}

% \begin{definition}
%   A \emph{3-valued pomset} $(\Event, {\le}, {\ltN}, \labelling)$
%   is a 3-valued poset $(\Event, {\le}, {\ltN})$ and
%   a pomset $(\Event, {\le}, \labelling)$.
% \end{definition}

Structures similar to 3-valued pomsets have come up in many guises, for example
rough sets~\cite{Pawlak1982} or ultrametrics over
$\{0,{}^1\!/_2,1\}$. They correspond to axioms A1--A3 of Lamport's
\emph{system executions}~\cite{DBLP:journals/dc/Lamport86}.
They are the notion of pomset given by interpreting
$\bEv\le\aEv$ in a 3-valued logic~\cite{Urquhart1986}. 

In diagrams, we visualize $(\aEv \ltN \bEv)$ as a dashed
arrow from $\bEv$ to $\aEv$ (note the change of direction).
We refer to edges introduced by $(\bEv < \aEv)$ as
\emph{strong edges} and by $(\aEv \ltN \bEv)$
as \emph{weak edges}.
For readability, we often highlight the reads-from edges as well.
% for example:
For example:
\[\begin{tikzpicture}[node distance=1em]
  \event{wx0}{\DW{\aLoc}{0}}{}
  \event{wx1}{\DW{\aLoc}{1}}{right=of wx0}
  \event{rx1}{\DR{\aLoc}{1}}{right=2.5 em of wx1}
  \event{wx2}{\DW{\aLoc}{2}}{right=of rx1}
  \rf{wx1}{rx1}
  \wk{wx0}{wx1}
  \wk{rx1}{wx2}
\end{tikzpicture}\]
We strengthen the definition of reads-from to require not just that
no blocker exists, but that any candidate blocker must either
have $\bEv \ltN \cEv$ or $\cEv \ltN \aEv$. This ensures that any
further concurrency cannot turn a non-blocker into a blocker.
\begin{definition}\label{def:rf}
  In a 3-valued pomset, $\aEv$ \emph{can read $\aLoc$ from} $\bEv$ whenever: 
  \begin{itemize}
  \item $\bEv < \aEv$,
  \item if $\aEv$ is satisfiable, then $\bEv$ is a tautology,
  \item $\bEv$ writes $\aVal$ to $\aLoc$,
    and $\aEv$ reads $\aVal$ from $\aLoc$, and
  \item if $\cEv$ writes to $\aLoc$
    then either $\bEv \ltN \cEv$ or $\cEv \ltN \aEv$.
  \end{itemize}
\end{definition}
One of the requirements of closed programs is that
every read event reads from a write event.


\section{Semantics of programs}
\label{sec:semantics}

\begin{figure*}
\begin{eqnarray*}
  \sem{\SKIP} & = & \{ \emptyset \} \\
  \sem{\aLoc\GETS\aExp\SEMI \aCmd} & = & \textstyle\bigcup_\aVal\; \bigl((\aExp=\aVal) \guard (\DW\aLoc\aVal) \prefix \sem{\aCmd}[\aExp/\aLoc]\bigr) \\
  \sem{\aReg\GETS\aLoc\SEMI \aCmd} & = & \sem{\aCmd}[\aLoc/\aReg] \cup \textstyle\bigcup_\aVal\; (\DR\aLoc\aVal) \prefix \sem{\aCmd}[\aLoc/\aReg] \\
  \sem{\IF (\aExp) \THEN \aCmd \ELSE \bCmd \FI} & = & \bigl((\aExp \neq 0) \guard \sem{\aCmd}\bigr) \parallel \bigl((\aExp=0) \guard \sem{\bCmd}\bigr) \\
  \sem{\aCmd \PAR \bCmd} & = & \sem{\aCmd} \parallel \sem{\bCmd} \\
  \sem{\VAR\aLoc\SEMI \aCmd} & = & \nu \aLoc \st \sem{\aCmd}
\end{eqnarray*}
\caption{Semantics of a concurrent shared-memory language}
\label{fig:programs}
\end{figure*}

We give the semantics of programs as sets of 3-valued pomsets.  A pomset in
the semantics of a program should be seen as one possible execution of that
program.  The semantics does not include prefixes, and thus a pomset models
a completed execution.

The definitions result in sets of pomsets that are closed with respect to
augmentation, which may create additional order and strengthening
preconditions:
\begin{definition}
  $\aPS'$ is an augmentation of $\aPS$ if $\Event'=\Event$, $\aEv\le\bEv$
  implies $\aEv\le'\bEv$, $\aEv\ltN\bEv$ implies $\aEv\ltN'\bEv$, and
  % $\labelling'(\aEv)=\labelling(\aEv)$
  if $\labelling(\aEv) = (\bForm \mid \bAct)$ then
  $\labelling'(\aEv) = (\bForm' \mid \bAct)$ where $\bForm'$ implies
  $\bForm$.
\end{definition}

In \refapp{sets-of-pomsets} we define the operations
needed to define the semantics, which are:
\begin{itemize}
\item \emph{prefixing} $\aAct\prefix\aPSS$, which adds an event
  with action $\aAct$ to pomsets in $\aPSS$,
\item \emph{guarding} $\aForm\guard\aPSS$, which filters $\aPSS$,
  keeping pomsets whose events have preconditions which imply $\aForm$,
\item \emph{substitution} $\aPSS[\aExp/\aLoc]$, which performs a substitution
  on every precondition in $\aPSS$,
\item \emph{concurrency} $\aPSS_1\parallel\aPSS_2$, which unions pomsets from
  $\aPSS_1$ and $\aPSS_2$, allowing events to be merged, and
\item \emph{restriction} $\nu\aLoc\st\aPSS$, which requires $\aLoc$ to
  satisfy the requirements of a memory location, in particular that
  any event which reads from $\aLoc$ must have a write event which
  it reads from.
\end{itemize}
These operations are similar to those from models of concurrency such
as~\cite{Brookes:1984:TCS:828.833}, but adapted here to the setting of
speculative evaluation.
We can use them to give the
semantics of a simple shared-memory concurrent language
in Figure~\ref{fig:programs}.

We use $\aPSS_1 \parallel \aPSS_2$ in defining the semantics of conditionals
and concurrency.
It contains the union of pomsets from $\aPSS_1$ and $\aPSS_2$,
allowing overlap as long as they agree on actions. For example, if
$\aPSS_1$ and $\aPSS_2$ contain:
\[\begin{tikzpicture}[node distance=1em]
  \event{a}{\aForm \mid \aAct}{}
  \event{b}{\bForm_1 \mid \bAct}{right=of a}
  \po{a}{b}
\end{tikzpicture}\qquad\qquad\begin{tikzpicture}[node distance=1em]
  \event{b}{\bForm_2 \mid \bAct}{}
  \event{c}{\cForm \mid \cAct}{right=of b}
  \wk{b}{c}
\end{tikzpicture}\]
then $\aPSS_1 \parallel \aPSS_2$ contains:
\[\begin{tikzpicture}[node distance=1em]
  \event{a}{\aForm \mid \aAct}{}
  \event{b}{\bForm_1 \lor \bForm_2 \mid \bAct}{right=of a}
  \event{c}{\cForm \mid \cAct}{right=of b}
  \po{a}{b}
  \wk{b}{c}
\end{tikzpicture}\]

Prefixing is used to define the semantics of reads and writes, and
adds a new event $\cEv$ with action $\aAct$.  As in the definition
of parallel composition, the definition allows the new event to overlap with
events in $\aPSS$ as long as they agree on the action.

The tricky parts of the
definition are the requirements on read
dependencies.  If $\aAct$ reads $\aVal$ from $\aLoc$, we have to
decide whether $\aEv$ depends on $\cEv$ for some $\aEv$ with old
precondition $\bForm$ and new precondition $\bForm'$. The first case
\textsc{[dependent read]} is that the dependency exists, in which case
$\bForm'$ just has to imply $\bForm[\aVal/\aLoc]$. The more interesting 
case is \textsc{[independent read]}, in which case $\bForm'$ has to imply
$\bForm[\aVal/\aLoc]$ and $\bForm$. This corresponds to a case where
$\aEv$ can be performed with or without $\cEv$.
In particular, if $\bForm$ is independent of $\aLoc$ then we can pick
$\bForm'$ to be $\bForm$, and the independent read case will apply.
For example,
if $\aAct$ and $\bAct$ write to the same location, $\aAct$ reads $\aVal$ from $\aLoc$, $\bForm$ is independent of $\aLoc$,
and
$\aPSS$ contains:
\[\begin{tikzpicture}[node distance=1em]
  \event{b}{\bForm \mid \bAct}{}
  \event{c}{\cForm \mid \cAct}{right=of b}
  \po{b}{c}
\end{tikzpicture}\]
then $\aAct\prefix\aPSS$ contains:
\[\begin{tikzpicture}[node distance=1em]
  \event{a}{\aForm \mid \aAct}{}
  \event{b}{\bForm \mid \bAct}{right=of a}
  \event{c}{\cForm[\vec\aVal/\vec\aLoc] \mid \cAct}{right=of b}
  \po[out=25,in=155]{a}{c}
  \wk{a}{b}
  \po{b}{c}
\end{tikzpicture}\]

%% A write generates a write event that may be visible
%% to other threads.  A read may see a
%% thread-local value, or it may generate a read event that must be justified by
%% another thread.  In the latter case, occurrences of $\aReg$ are replaced with
%% $\aLoc$ (rather than $\aVal$) to ensure that dependencies are tracked
%% properly.  The subsequent substitution of $\aVal$ for $\aLoc$ occurs in
%% Definition~\ref{def:prefix} of prefixing.

% We have completed the formal definition of our model of speculative
% evaluation, and now turn to examples.

\subsection{Sequential memory accesses}
\label{sec:sequential-memory}

In the semantics of memory, there are two very different ways memory
can be accessed: sequentially or concurrently. These are modeled
differently, since hardware and compilers give very different
guarantees about their behavior.
In this section, we discuss the sequential semantics, and leave
the concurrent semantics to \S\ref{sec:concurrent-memory}.

Consider the program $(\aLoc\GETS0\SEMI \bLoc\GETS\aLoc+1)$.  One execution of
this program is where the write to $y$ uses the sequential value of
$x$, which is $0$:
\[\begin{tikzpicture}[node distance=1em]
  \event{wx0}{\DW{x}{0}}{}
  \event{wy1}{\DW{y}{1}}{right=of wx0}
\end{tikzpicture}\]
To see how this execution is modeled, we first
expand out the syntax sugar to get the program
$(\aLoc\GETS0\SEMI \aReg\GETS\aLoc\SEMI \bLoc\GETS\aReg+1\SEMI\SKIP)$
Now $\sem{\SKIP}$ is just $\{\emptyset\}$, and
$\sem{y \GETS r+1\SEMI \SKIP}$ includes:
\[
   (r+1=1) \guard (\DW y1) \prefix \sem{\SKIP}[1/y]
\]
which contains the pomset:
\[\begin{tikzpicture}[node distance=1em]
  \event{wy1}{r+1=1 \mid \DW{y}{1}}{}
\end{tikzpicture}\]
expressing that this program can write $1$ to $y$,
as long as the precondition $(r+1=1)$ is satisfied.
Now $\sem{r \GETS x\SEMI y \GETS r+1\SEMI \SKIP}$
has two cases, the sequential case
(which does not introduce a read action)
and the concurrent case (which does).
For the moment, we are interested in the sequential case:
\[
   \sem{y \GETS r+1\SEMI \SKIP}[x/r]
\]
which contains the pomset:
\[\begin{tikzpicture}[node distance=1em]
  \event{wy1}{x+1=1 \mid \DW{y}{1}}{}
\end{tikzpicture}\]
In this pomset, the precondition is $(x+1=1)$, which specifies a property
of the thread-local value of $x$.
Finally $\sem{x \GETS 0\SEMI r \GETS x\SEMI y \GETS r+1\SEMI \SKIP}$ includes:
\[
   (0=0) \guard (\DW x0) \prefix \sem{r \GETS x\SEMI y \GETS r+1\SEMI \SKIP}[0/x]
\]
which contains the pomset:
\[\begin{tikzpicture}[node distance=1em]
  \event{wx0}{0=0 \mid \DW{x}{0}}{}
  \event{wy1}{0=0\land0+1=1 \mid \DW{y}{1}}{right=of wx0}
\end{tikzpicture}\]
all of whose preconditions are tautologies, so this has the expected behavior:
\[\begin{tikzpicture}[node distance=1em]
  \event{wx0}{\DW{x}{0}}{}
  \event{wy1}{\DW{y}{1}}{right=of wx0}
\end{tikzpicture}\]
There is no dependency between $(\DW x0)$ and $(\DW y1)$,
since $(0=0\land0+1=1)$ is independent of $\aLoc$.

This example demonstrates how preconditions
capture the sequential semantics of memory.
In an execution containing an event with label
$(\aForm \mid \aAct)$, one way the precondition $\aForm$
can be discharged is by an assignment $\aLoc\GETS\aExp$,
which performs a substitution $[\aExp/\aLoc]$.
This is a variant of the Hoare semantics of
assignment \cite{Hoare:1969:ABC:363235.363259}, where if $\aCmd$ has precondition $\aForm$
then $\aLoc\GETS\aExp\SEMI\aCmd$ has precondition
$\aForm[\aExp/\aLoc]$.

\subsection{Concurrent memory accesses}
\label{sec:concurrent-memory}

We now turn to the case of concurrent accesses to memory.
Consider the program %a concurrent version of the program from \S\ref{sec:sequential-memory}:
$(\aLoc\GETS1 \PAR \bLoc\GETS\aLoc+1)$.
In executions of this program, it is possible for the second thread to 
perform a concurrent read of $x$:
\[\begin{tikzpicture}[node distance=1em]
  \event{wx1}{\DW{x}{1}}{}
  \event{rx1}{\DR{x}{1}}{right=2.5em of wx1}
  \event{wy2}{\DW{y}{2}}{right=of rx1}
  \rf{wx1}{rx1}
  \po{rx1}{wy2}
\end{tikzpicture}\]
To see how this execution is modeled, we first
expand out the syntax sugar to get the program
$(\aLoc\GETS1\SEMI\SKIP \PAR \aReg\GETS\aLoc\SEMI \bLoc\GETS\aReg+1\SEMI\SKIP)$.
As before, $\sem{y \GETS r+1\SEMI \SKIP}$ includes:
\[
   (r+1=2) \guard (\DW y2) \prefix \sem{\SKIP}[2/y]
\]
which contains the pomset:
\[\begin{tikzpicture}[node distance=1em]
  \event{wy2}{r+1=2 \mid \DW{y}{2}}{}
\end{tikzpicture}\]
As before, $\sem{r \GETS x\SEMI y \GETS r+1\SEMI \SKIP}$ has two cases.
We are now interested in the concurrent case, which includes:
\[
   (\DR x1) \prefix \sem{y \GETS r+1\SEMI \SKIP}[x/r]
\]
which contains the pomset:
\[\begin{tikzpicture}[node distance=1em]
  \event{rx1}{\DR{x}{1}}{}
  \event{wy2}{\DW{y}{2}}{right=of rx1}
  \po{rx1}{wy2}
\end{tikzpicture}\]
Note that $(\DR x1)$ reads $1$ from $x$, and while
$(x+1=2)[1/x]$ is a tautology,
$(x+1=2)$ is not,
and so there is a dependency
$(\DR x1) < (\DW y2)$.

Now, $\sem{x \GETS 1\SEMI \SKIP}$ includes the pomset:
\[\begin{tikzpicture}[node distance=1em]
  \event{wx1}{\DW{x}{1}}{}
\end{tikzpicture}\]
and so $\sem{x \GETS 1\SEMI \SKIP \PAR r \GETS x\SEMI y \GETS r+1\SEMI \SKIP}$ includes:
\[\begin{tikzpicture}[node distance=1em]
  \event{wx1}{\DW{x}{1}}{}
  \event{rx1}{\DR{x}{1}}{right=2.5em of wx1}
  \event{wy2}{\DW{y}{2}}{right=of rx1}
  \rf{wx1}{rx1}
  \po{rx1}{wy2}
\end{tikzpicture}\]
as expected, including a reads-from dependency
$(\DW x1) < (\DR x1)$.

This example demonstrates how read and write events
capture the concurrent semantics of memory.
In an execution containing an event with label
$(\DR \aLoc\aVal)$, if the execution is
$\aLoc$-closed, then there must be an event
it reads from, for example one labelled
$(\DW \aLoc\aVal)$.

\subsection{Control dependencies}
\label{sec:control-dep}

Conditionals introduce control dependencies, for example consider the program:
\[
  \aReg\GETS\cLoc\SEMI
  \IF(\aReg)\THEN \aLoc\GETS1 \ELSE \bLoc\GETS2 \FI
\]
This includes executions in which the false branch is taken:
\[\begin{tikzpicture}[node distance=1em]
  \event{rz0}{\DR{z}{0}}{}
  \nonevent{wx1}{\DW{x}{1}}{right=of rz0}
  \event{wy2}{\DW{y}{2}}{right=of wx1}
  \po{rz0}{wx1}
  \po[out=30,in=150]{rz0}{wy2}
\end{tikzpicture}\]
and ones where the true branch is taken:
\[\begin{tikzpicture}[node distance=1em]
  \event{rz1}{\DR{z}{1}}{}
  \event{wx1}{\DW{x}{1}}{right=of rz1}
  \nonevent{wy2}{\DW{y}{2}}{right=of wx1}
  \po{rz1}{wx1}
  \po[out=30,in=150]{rz1}{wy2}
\end{tikzpicture}\]
In both cases, we record the actions in the branch that was
not taken. This is a novel feature of this model, and is
intended to capture speculative evaluation. In \S\ref{sec:spectre}
we will show how this model captures Spectre-like information
flow attacks, once the attacker is provided with the ability to
observe such speculations.

To see how these executions are modeled, consider the semantics of
$\sem{x\GETS 1\SEMI\SKIP}$, which contains any pomset of the form:
\[\begin{tikzpicture}[node distance=1em]
  \event{wx1}{\aForm \mid \DW{x}{1}}{}
\end{tikzpicture}\]
in particular it contains:
\[\begin{tikzpicture}[node distance=1em]
  \event{wx1}{r\neq0 \mid \DW{x}{1}}{}
\end{tikzpicture}\]
Similarly $\sem{y\GETS 2\SEMI\SKIP}$ contains:
\[\begin{tikzpicture}[node distance=1em]
  \event{wy2}{r=0 \mid \DW{y}{2}}{}
\end{tikzpicture}\]
and so $\sem{\IF(r)\THEN x\GETS 1\SEMI\SKIP \ELSE y\GETS 2\SEMI\SKIP \FI}$
contains:
\[\begin{tikzpicture}[node distance=1em]
  \event{wx1}{r\neq0 \mid \DW{x}{1}}{}
  \event{wy2}{r=0 \mid \DW{y}{2}}{right=of wx1}
\end{tikzpicture}\]
Now, the semantics of concurrent read performs substitutions, for example:
\[\begin{tikzpicture}[node distance=1em]
  \event{rz0}{\DR{z}{0}}{}
  \event{wx1}{0\neq0 \mid \DW{x}{1}}{right=of rz0}
  \event{wy2}{0=0 \mid \DW{y}{2}}{right=of wx1}
  \po{rz0}{wx1}
  \po[out=25,in=155]{rz0}{wy2}
\end{tikzpicture}\]
which gives the required pomset:
\[\begin{tikzpicture}[node distance=1em]
  \event{rz0}{\DR{z}{0}}{}
  \nonevent{wx1}{\DW{x}{1}}{right=of rz0}
  \event{wy2}{\DW{y}{2}}{right=of wx1}
  \po{rz0}{wx1}
  \po[out=30,in=150]{rz0}{wy2}
\end{tikzpicture}\]
Note that the precondition $r=0$ is dependent on $r$,
and so there is a dependency $(\DR z0) < (\DW y2)$,
modeling the control dependency introduced by the conditional.

\subsection{Control independencies}

In most models of control dependencies, the dependency relation
is syntactic, based on whether the action occurs inside syntactically
inside a conditional. In contrast, the notion in this model is
semantic: if an action can occur on both sides of a conditional,
there is no control dependency. Consider a variant of the example
from \S\ref{sec:control-dep}:
\[
  \aReg\GETS\cLoc\SEMI
  \IF(\aReg)\THEN \aLoc\GETS1 \ELSE \aLoc\GETS1 \FI
\]
This has the expected execution in which the control
dependencies exist:
\[\begin{tikzpicture}[node distance=1em]
  \event{rz0}{\DR{z}{0}}{}
  \nonevent{nwx1}{\DW{x}{1}}{right=of rz0}
  \event{wx1}{\DW{x}{1}}{right=of nwx1}
  \po{rz0}{nwx1}
  \po[out=30,in=150]{rz0}{wx1}
\end{tikzpicture}\]
but it also has an execution in which the two writes
of $1$ to $x$ are merged, resulting in no dependency:
\[\begin{tikzpicture}[node distance=1em]
  \event{rz0}{\DR{z}{0}}{}
  \event{wx1}{\DW{x}{1}}{right=of rz0}
\end{tikzpicture}\]
To see how this arises,
consider the definition of $\sem{\IF(r)\THEN x\GETS1\SEMI\SKIP \ELSE x\GETS1\SEMI\SKIP \FI}$:
\[\begin{array}{rl}
   \aPSS_1 \parallel \aPSS_2 \quad\mbox{where}\quad&
   \aPSS_1 = (r\neq 0) \guard \sem{x\GETS1\SEMI\SKIP} \\&
   \aPSS_2 = (r=0) \guard \sem{x\GETS1\SEMI\SKIP}
\end{array}\]
Now, one pomset in $\aPSS_1$ is:
\[\begin{tikzpicture}[node distance=1em]
  \event{wx1}{r\neq0 \mid \DW{x}{1}}{}
\end{tikzpicture}\]
that is $\aPS_1$ where:
\[
  \Event_1 = \{\aEv\} \quad
  \labelling_1(\aEv) = (r\neq 0, \DW x1)
\]
and similarly, one pomset in $\aPSS_2$ is:
\[\begin{tikzpicture}[node distance=1em]
  \event{wx1}{r=0 \mid \DW{x}{1}}{}
\end{tikzpicture}\]
that is $\aPS_2$ where:
\[
  \Event_2 = \{\aEv\} \quad
  \labelling_2(\aEv) = (r= 0, \DW x1)
\]
Crucially, in the definition of $\aPSS_1 \parallel \aPSS_2$
there is \emph{no} requirement that $\Event_1$ and $\Event_2$ are disjoint,
and in this case they overlap at $\aEv$. As a result, one pomset in
$\aPSS_1\parallel\aPSS_2$ is $\aPS_0$ where:
\[
  \Event_0 = \{\aEv\} \quad
  \labelling_0(\aEv) = (r\neq0 \lor r=0, \DW x1)
\]
that is:
\[\begin{tikzpicture}[node distance=1em]
  \event{wx1}{\DW{x}{1}}{}
\end{tikzpicture}\]
Note that this pomset has no precondition dependent on $r$,
since $(r\neq0 \lor r=0)$ does not depend on $r$, which is why
we end up with an execution without a control dependency:
\[\begin{tikzpicture}[node distance=1em]
  \event{rz0}{\DR{z}{0}}{}
  \event{wx1}{\DW{x}{1}}{right=of rz0}
\end{tikzpicture}\]
This semantics captures compiler optimizations which may, for example,
merge code executed on both branches of a conditional, or hoist
constant assignments out of loops.

We can now see the counterintuitive behavior of conditionals
in the presence of control dependencies.
There are programs such as
\(
  (\IF(\cLoc)\THEN \aLoc\GETS1 \ELSE \aLoc\GETS1 \FI)
\)
with executions in which  $(\DW x1)$ is independent of $(\DR z1)$:
\[\begin{tikzpicture}[node distance=1em]
  \event{rz1}{\DR{z}{1}}{}
  \event{wx1}{\DW{x}{1}}{right=of rz1}
\end{tikzpicture}\]
while programs such as
\(
  (\IF(\cLoc)\THEN \aLoc\GETS1 \ELSE \bLoc\GETS2 \FI)
\)
only have executions in which $(\DW x1)$ is dependent on $(\DR z1)$:
\[\begin{tikzpicture}[node distance=1em]
  \event{rz1}{\DR{z}{1}}{}
  \event{wx1}{\DW{x}{1}}{right=of rz1}
  \nonevent{wy2}{\DW{y}{2}}{right=of wx1}
  \po{rz1}{wx1}
  \po[out=30,in=150]{rz1}{wy2}
\end{tikzpicture}\]
These programs have executions with different dependency relations, depending only
on conditional branches that were \emph{not} taken. In \S\ref{sec:info-flow-attack}
we shall see that this has security implications, since relaxed
memory can observe dependency. The attack is similar to Spectre, so
we shall take a detour to see how Spectre can be modeled in this
setting.

\subsection{Relaxed memory}
\label{sec:relaxed-memory}

In \S\ref{sec:info-flow-attack} we present an information flow attack
on relaxed memory, similar to Spectre in that it relies on speculative
evaluation. Unlike Spectre it does not depend on timing attacks,
but instead is based on the sensitivity of relaxed memory to data
dependencies. % For this reason, we present a simple model of relaxed
% memory, which is strong enough to capture this attack.

Our model includes concurrent memory accesses, which can introduce concurrent
reads-from. 
Since we are allowing events to be partially ordered, this gives a simple
model of relaxed memory.  For example an independent read independent write
(IRIW) example is:
\[\begin{array}{l}
  x\GETS0\SEMI x\GETS x+1
  \PAR
  y\GETS0\SEMI y\GETS y+1
\\{}
  \PAR
  r_1\GETS x\SEMI r_2\GETS y
  \PAR
  s_1\GETS y\SEMI s_2\GETS x
\end{array}\]
which includes the execution:
\[\begin{tikzpicture}[node distance=1em]
  \event{wx0}{\DW{x}{0}}{}
  \event{wx1}{\DW{x}{1}}{right=of wx0}
  \event{wy0}{\DW{y}{0}}{right=2.5em of wx1}
  \event{wy1}{\DW{y}{1}}{right=of wy0}
  \event{ry1}{\DR{y}{1}}{below=4ex of wx0}
  \event{rx0}{\DR{x}{0}}{right=of ry1}
  \event{rx1}{\DR{x}{1}}{right=2.5 em of rx0}
  \event{ry0}{\DR{y}{0}}{right=of rx1}
  \rf{wx1}{rx1}
  \rf{wy0}{ry0}
  \rf[out=210,in=30]{wy1}{ry1}
  \rf{wx0}{rx0}
  \wk{rx0}{wx1}
  \wk{ry0}{wy1}
\end{tikzpicture}\]
This model does not introduce thin-air reads (TAR).
For example the TAR pit
\((
  x\GETS y \PAR y \GETS x
)\)
fails to produce a value for $x$ from thin air
since this produces a cycle in $\le$, as shown on the left below:
\begin{align*}
\begin{tikzpicture}[node distance=1em]
  \event{ry42}{\DR{y}{42}}{}
  \event{wx42}{\DW{x}{42}}{below=of ry42}
  \event{rx42}{\DR{x}{42}}{right=2.5em of ry42}
  \event{wy42}{\DW{y}{42}}{below=of rx42}
  \po{ry42}{wx42}
  \po{rx42}{wy42}
  \rf{wx42}{rx42}
  \rf{wy42}{ry42}
\end{tikzpicture}
&&
\begin{tikzpicture}[node distance=1em]
  \event{ry1}{\DR{y}{1}}{}
  \event{wx1}{\DW{x}{1}}{below=of ry1}
  \event{rx1}{\DR{x}{1}}{right=2.5em of ry1}
  \event{wy1}{\DW{y}{1}}{below=of rx1}
  \po{ry1}{wx1}
  \rf{wx1}{rx1}
  \rf{wy1}{ry1}
\end{tikzpicture}
\end{align*}
This cycle can be broken by removing a dependency. For example
\((
  x\GETS y \PAR r\GETS x\SEMI y \GETS r+1-r
)\)
has the execution on the right above.
% \[\begin{tikzpicture}[node distance=1em]
%   \event{ry1}{\DR{y}{1}}{}
%   \event{wx1}{\DW{x}{1}}{below=of ry1}
%   \event{rx1}{\DR{x}{1}}{right=2.5em of ry1}
%   \event{wy1}{\DW{y}{1}}{below=of rx1}
%   \po{ry1}{wx1}
%   \rf{wx1}{rx1}
%   \rf{wy1}{ry1}
% \end{tikzpicture}\]
Note that $(\DR x1) \not\le (\DW y1)$, so this does not introduce a cycle.

Although it is not the primary focus of this paper, our model may be an
attractive model of relaxed memory.  Many prior models either permit
thin-air executions that our model forbids or forbid desirable executions
that our model permits.
%% In \S\ref{sec:logic}, we develop a logic which allows us to prove that our
%% semantics forbids thin air examples that are permitted by prior speculative
%% models
%% \cite{Manson:2005:JMM:1047659.1040336,DBLP:conf/esop/JagadeesanPR10,DBLP:conf/popl/KangHLVD17}.
% Our model passes all of the causality test cases
% \cite{PughWebsite}.
%% Significantly, this
%% includes test case 9, which is forbidden by \cite{DBLP:conf/lics/JeffreyR16},
%% one of the few models that disallows the thin air example from
%% \S\ref{sec:logic}.  We present this test case in the appendix, where we also
%% discuss the thread inlining examples from
%% \cite{Manson:2005:JMM:1047659.1040336}.

In \refapp{logic}, we present a variant of the TAR-pit
example %from \S\ref{sec:relaxed-memory}
that is allowed under prior speculative semantics
\cite{Manson:2005:JMM:1047659.1040336,DBLP:conf/esop/JagadeesanPR10,DBLP:conf/popl/KangHLVD17}.
We develop a logic that allows us to prove that the problematic execution is
forbidden in our model.  \citet{DBLP:conf/esop/BattyMNPS15} showed that the
thin-air problem has no per-candidate-execution solution for C++.  This
result does not apply to our model, which has a different notion of
dependency.
% as the semantics of a conditional can depend on the semantics
% of both branches.

\citet{PughWebsite} developed a set of twenty {causality test cases} in the
process of revising the Java Memory Model (JMM)
\cite{Manson:2005:JMM:1047659.1040336}.  Using hand calculation, we have
confirmed that our model gives the desired result for all twenty cases,
unrolling loops as necessary.  Our model also gives the desired results for
all of the examples in \citet[\textsection 4]{DBLP:conf/esop/BattyMNPS15} and
all but one in \citet[\textsection 5.3]{SevcikThesis}:
redundant-write-after-read-elimination fails for any
sensible non-coherent semantics.  Our model agrees with the JMM on the
``surprising and controversial behaviors'' of \citet[\textsection
8]{Manson:2005:JMM:1047659.1040336}, and thus fails to validate thread
inlining.

In \refapp{tc}, we discuss three of the causality test cases and the thread
inlining from \cite{Manson:2005:JMM:1047659.1040336}.  In presenting the
examples, we unroll loops, correct typos and simplify the code.  


\section{Examples}
\label{sec:examples}

In this section, we shall start off by giving some basic examples,
and then show how three different information flow attacks can be
modeled. We cover Spectre in \S\ref{sec:spectre}, new attacks
on compiler optimizations in \S\ref{sec:info-flow-attack}--\ref{sec:dse},
and attacks on transactions in \S\ref{sec:transactions}.

\subsection{Sequential memory accesses}
\label{sec:sequential-memory}

In the semantics of memory, there are two very different ways memory
can be accessed: sequentially or concurrently. These are modeled
differently, since hardware and compilers give very different
guarantees about their behavior.
In this section, we discuss the sequential semantics, and leave
the concurrent semantics to \S\ref{sec:concurrent-memory}.

Consider the program $(\aLoc\GETS0\SEMI \bLoc\GETS\aLoc+1)$.  One execution of
this program is where the write to $y$ uses the sequential value of
$x$, which is $0$:
\[\begin{tikzpicture}[node distance=1em]
  \event{wx0}{\DW{x}{0}}{}
  \event{wy1}{\DW{y}{1}}{right=of wx0}
\end{tikzpicture}\]
To see how this execution is modeled, we first
expand out the syntax sugar to get the program
$(\aLoc\GETS0\SEMI \aReg\GETS\aLoc\SEMI \bLoc\GETS\aReg+1\SEMI\SKIP)$
Now $\sem{\SKIP}$ is just $\{\emptyset\}$, and
$\sem{y \GETS r+1\SEMI \SKIP}$ includes:
\[
   (r+1=1) \guard (\DW y1) \prefix \sem{\SKIP}[1/y]
\]
which contains the pomset:
\[\begin{tikzpicture}[node distance=1em]
  \event{wy1}{r+1=1 \mid \DW{y}{1}}{}
\end{tikzpicture}\]
expressing that this program can write $1$ to $y$,
as long as the precondition $(r+1=1)$ is satisfied.
Now $\sem{r \GETS x\SEMI y \GETS r+1\SEMI \SKIP}$
has two cases, the sequential case
(which does not introduce a read action)
and the concurrent case (which does).
For the moment, we are interested in the sequential case:
\[
   \sem{y \GETS r+1\SEMI \SKIP}[x/r]
\]
which contains the pomset:
\[\begin{tikzpicture}[node distance=1em]
  \event{wy1}{x+1=1 \mid \DW{y}{1}}{}
\end{tikzpicture}\]
In this pomset, the precondition is $(x+1=1)$, which specifies a property
of the thread-local value of $x$.
Finally $\sem{x \GETS 0\SEMI r \GETS x\SEMI y \GETS r+1\SEMI \SKIP}$ includes:
\[
   (0=0) \guard (\DW x0) \prefix \sem{r \GETS x\SEMI y \GETS r+1\SEMI \SKIP}[0/x]
\]
which contains the pomset:
\[\begin{tikzpicture}[node distance=1em]
  \event{wx0}{0=0 \mid \DW{x}{0}}{}
  \event{wy1}{0=0\land0+1=1 \mid \DW{y}{1}}{right=of wx0}
\end{tikzpicture}\]
all of whose preconditions are tautologies, so this has the expected behavior:
\[\begin{tikzpicture}[node distance=1em]
  \event{wx0}{\DW{x}{0}}{}
  \event{wy1}{\DW{y}{1}}{right=of wx0}
\end{tikzpicture}\]
There is no dependency between $(\DW x0)$ and $(\DW y1)$,
since $(0=0\land0+1=1)$ is independent of $\aLoc$.

This example demonstrates how preconditions
capture the sequential semantics of memory.
In an execution containing an event with label
$(\aForm \mid \aAct)$, one way the precondition $\aForm$
can be discharged is by an assignment $\aLoc\GETS\aExp$,
which performs a substitution $[\aExp/\aLoc]$.
This is a variant of the Hoare semantics of
assignment \cite{Hoare:1969:ABC:363235.363259}, where if $\aCmd$ has precondition $\aForm$
then $\aLoc\GETS\aExp\SEMI\aCmd$ has precondition
$\aForm[\aExp/\aLoc]$.

\subsection{Concurrent memory accesses}
\label{sec:concurrent-memory}

We now turn to the case of concurrent accesses to memory.
Consider the program %a concurrent version of the program from \S\ref{sec:sequential-memory}:
$(\aLoc\GETS1 \PAR \bLoc\GETS\aLoc+1)$.
In executions of this program, it is possible for the second thread to 
perform a concurrent read of $x$:
\[\begin{tikzpicture}[node distance=1em]
  \event{wx1}{\DW{x}{1}}{}
  \event{rx1}{\DR{x}{1}}{right=2.5em of wx1}
  \event{wy2}{\DW{y}{2}}{right=of rx1}
  \rf{wx1}{rx1}
  \po{rx1}{wy2}
\end{tikzpicture}\]
To see how this execution is modeled, we first
expand out the syntax sugar to get the program
$(\aLoc\GETS1\SEMI\SKIP \PAR \aReg\GETS\aLoc\SEMI \bLoc\GETS\aReg+1\SEMI\SKIP)$.
As before, $\sem{y \GETS r+1\SEMI \SKIP}$ includes:
\[
   (r+1=2) \guard (\DW y2) \prefix \sem{\SKIP}[2/y]
\]
which contains the pomset:
\[\begin{tikzpicture}[node distance=1em]
  \event{wy2}{r+1=2 \mid \DW{y}{2}}{}
\end{tikzpicture}\]
As before, $\sem{r \GETS x\SEMI y \GETS r+1\SEMI \SKIP}$ has two cases.
We are now interested in the concurrent case, which includes:
\[
   (\DR x1) \prefix \sem{y \GETS r+1\SEMI \SKIP}[x/r]
\]
which contains the pomset:
\[\begin{tikzpicture}[node distance=1em]
  \event{rx1}{\DR{x}{1}}{}
  \event{wy2}{\DW{y}{2}}{right=of rx1}
  \po{rx1}{wy2}
\end{tikzpicture}\]
Note that $(\DR x1)$ reads $1$ from $x$, and while
$(x+1=2)[1/x]$ is a tautology,
$(x+1=2)$ is not,
and so there is a dependency
$(\DR x1) < (\DW y2)$.

Now, $\sem{x \GETS 1\SEMI \SKIP}$ includes the pomset:
\[\begin{tikzpicture}[node distance=1em]
  \event{wx1}{\DW{x}{1}}{}
\end{tikzpicture}\]
and so $\sem{x \GETS 1\SEMI \SKIP \PAR r \GETS x\SEMI y \GETS r+1\SEMI \SKIP}$ includes:
\[\begin{tikzpicture}[node distance=1em]
  \event{wx1}{\DW{x}{1}}{}
  \event{rx1}{\DR{x}{1}}{right=2.5em of wx1}
  \event{wy2}{\DW{y}{2}}{right=of rx1}
  \rf{wx1}{rx1}
  \po{rx1}{wy2}
\end{tikzpicture}\]
as expected, including a reads-from dependency
$(\DW x1) < (\DR x1)$.

This example demonstrates how read and write events
capture the concurrent semantics of memory.
In an execution containing an event with label
$(\DR \aLoc\aVal)$, if the execution is
$\aLoc$-closed, then there must be an event
it reads from, for example one labelled
$(\DW \aLoc\aVal)$.

\subsection{Control dependencies}
\label{sec:control-dep}

Conditionals introduce control dependencies, for example consider the program:
\[
  \aReg\GETS\cLoc\SEMI
  \IF(\aReg)\THEN \aLoc\GETS1 \ELSE \bLoc\GETS2 \FI
\]
This includes executions in which the false branch is taken:
\[\begin{tikzpicture}[node distance=1em]
  \event{rz0}{\DR{z}{0}}{}
  \nonevent{wx1}{\DW{x}{1}}{right=of rz0}
  \event{wy2}{\DW{y}{2}}{right=of wx1}
  \po{rz0}{wx1}
  \po[out=30,in=150]{rz0}{wy2}
\end{tikzpicture}\]
and ones where the true branch is taken:
\[\begin{tikzpicture}[node distance=1em]
  \event{rz1}{\DR{z}{1}}{}
  \event{wx1}{\DW{x}{1}}{right=of rz1}
  \nonevent{wy2}{\DW{y}{2}}{right=of wx1}
  \po{rz1}{wx1}
  \po[out=30,in=150]{rz1}{wy2}
\end{tikzpicture}\]
In both cases, we record the actions in the branch that was
not taken. This is a novel feature of this model, and is
intended to capture speculative evaluation. In \S\ref{sec:spectre}
we will show how this model captures Spectre-like information
flow attacks, once the attacker is provided with the ability to
observe such speculations.

To see how these executions are modeled, consider the semantics of
$\sem{x\GETS 1\SEMI\SKIP}$, which contains any pomset of the form:
\[\begin{tikzpicture}[node distance=1em]
  \event{wx1}{\aForm \mid \DW{x}{1}}{}
\end{tikzpicture}\]
in particular it contains:
\[\begin{tikzpicture}[node distance=1em]
  \event{wx1}{r\neq0 \mid \DW{x}{1}}{}
\end{tikzpicture}\]
Similarly $\sem{y\GETS 2\SEMI\SKIP}$ contains:
\[\begin{tikzpicture}[node distance=1em]
  \event{wy2}{r=0 \mid \DW{y}{2}}{}
\end{tikzpicture}\]
and so $\sem{\IF(r)\THEN x\GETS 1\SEMI\SKIP \ELSE y\GETS 2\SEMI\SKIP \FI}$
contains:
\[\begin{tikzpicture}[node distance=1em]
  \event{wx1}{r\neq0 \mid \DW{x}{1}}{}
  \event{wy2}{r=0 \mid \DW{y}{2}}{right=of wx1}
\end{tikzpicture}\]
Now, the semantics of concurrent read performs substitutions, for example:
\[\begin{tikzpicture}[node distance=1em]
  \event{rz0}{\DR{z}{0}}{}
  \event{wx1}{0\neq0 \mid \DW{x}{1}}{right=of rz0}
  \event{wy2}{0=0 \mid \DW{y}{2}}{right=of wx1}
  \po{rz0}{wx1}
  \po[out=25,in=155]{rz0}{wy2}
\end{tikzpicture}\]
which gives the required pomset:
\[\begin{tikzpicture}[node distance=1em]
  \event{rz0}{\DR{z}{0}}{}
  \nonevent{wx1}{\DW{x}{1}}{right=of rz0}
  \event{wy2}{\DW{y}{2}}{right=of wx1}
  \po{rz0}{wx1}
  \po[out=30,in=150]{rz0}{wy2}
\end{tikzpicture}\]
Note that the precondition $r=0$ is dependent on $r$,
and so there is a dependency $(\DR z0) < (\DW y2)$,
modeling the control dependency introduced by the conditional.

\subsection{Control independencies}

In most models of control dependencies, the dependency relation
is syntactic, based on whether the action occurs inside syntactically
inside a conditional. In contrast, the notion in this model is
semantic: if an action can occur on both sides of a conditional,
there is no control dependency. Consider a variant of the example
from \S\ref{sec:control-dep}:
\[
  \aReg\GETS\cLoc\SEMI
  \IF(\aReg)\THEN \aLoc\GETS1 \ELSE \aLoc\GETS1 \FI
\]
This has the expected execution in which the control
dependencies exist:
\[\begin{tikzpicture}[node distance=1em]
  \event{rz0}{\DR{z}{0}}{}
  \nonevent{nwx1}{\DW{x}{1}}{right=of rz0}
  \event{wx1}{\DW{x}{1}}{right=of nwx1}
  \po{rz0}{nwx1}
  \po[out=30,in=150]{rz0}{wx1}
\end{tikzpicture}\]
but it also has an execution in which the two writes
of $1$ to $x$ are merged, resulting in no dependency:
\[\begin{tikzpicture}[node distance=1em]
  \event{rz0}{\DR{z}{0}}{}
  \event{wx1}{\DW{x}{1}}{right=of rz0}
\end{tikzpicture}\]
To see how this arises,
consider the definition of $\sem{\IF(r)\THEN x\GETS1\SEMI\SKIP \ELSE x\GETS1\SEMI\SKIP \FI}$:
\[\begin{array}{rl}
   \aPSS_1 \parallel \aPSS_2 \quad\mbox{where}\quad&
   \aPSS_1 = (r\neq 0) \guard \sem{x\GETS1\SEMI\SKIP} \\&
   \aPSS_2 = (r=0) \guard \sem{x\GETS1\SEMI\SKIP}
\end{array}\]
Now, one pomset in $\aPSS_1$ is:
\[\begin{tikzpicture}[node distance=1em]
  \event{wx1}{r\neq0 \mid \DW{x}{1}}{}
\end{tikzpicture}\]
that is $\aPS_1$ where:
\[
  \Event_1 = \{\aEv\} \quad
  \labelling_1(\aEv) = (r\neq 0, \DW x1)
\]
and similarly, one pomset in $\aPSS_2$ is:
\[\begin{tikzpicture}[node distance=1em]
  \event{wx1}{r=0 \mid \DW{x}{1}}{}
\end{tikzpicture}\]
that is $\aPS_2$ where:
\[
  \Event_2 = \{\aEv\} \quad
  \labelling_2(\aEv) = (r= 0, \DW x1)
\]
Crucially, in the definition of $\aPSS_1 \parallel \aPSS_2$
there is \emph{no} requirement that $\Event_1$ and $\Event_2$ are disjoint,
and in this case they overlap at $\aEv$. As a result, one pomset in
$\aPSS_1\parallel\aPSS_2$ is $\aPS_0$ where:
\[
  \Event_0 = \{\aEv\} \quad
  \labelling_0(\aEv) = (r\neq0 \lor r=0, \DW x1)
\]
that is:
\[\begin{tikzpicture}[node distance=1em]
  \event{wx1}{\DW{x}{1}}{}
\end{tikzpicture}\]
Note that this pomset has no precondition dependent on $r$,
since $(r\neq0 \lor r=0)$ does not depend on $r$, which is why
we end up with an execution without a control dependency:
\[\begin{tikzpicture}[node distance=1em]
  \event{rz0}{\DR{z}{0}}{}
  \event{wx1}{\DW{x}{1}}{right=of rz0}
\end{tikzpicture}\]
This semantics captures compiler optimizations which may, for example,
merge code executed on both branches of a conditional, or hoist
constant assignments out of loops.

We can now see the counterintuitive behavior of conditionals
in the presence of control dependencies.
There are programs such as
\(
  (\IF(\cLoc)\THEN \aLoc\GETS1 \ELSE \aLoc\GETS1 \FI)
\)
with executions in which  $(\DW x1)$ is independent of $(\DR z1)$:
\[\begin{tikzpicture}[node distance=1em]
  \event{rz1}{\DR{z}{1}}{}
  \event{wx1}{\DW{x}{1}}{right=of rz1}
\end{tikzpicture}\]
while programs such as
\(
  (\IF(\cLoc)\THEN \aLoc\GETS1 \ELSE \bLoc\GETS2 \FI)
\)
only have executions in which $(\DW x1)$ is dependent on $(\DR z1)$:
\[\begin{tikzpicture}[node distance=1em]
  \event{rz1}{\DR{z}{1}}{}
  \event{wx1}{\DW{x}{1}}{right=of rz1}
  \nonevent{wy2}{\DW{y}{2}}{right=of wx1}
  \po{rz1}{wx1}
  \po[out=30,in=150]{rz1}{wy2}
\end{tikzpicture}\]
These programs have executions with different dependency relations, depending only
on conditional branches that were \emph{not} taken. In \S\ref{sec:info-flow-attack}
we shall see that this has security implications, since relaxed
memory can observe dependency. The attack is similar to Spectre, so
we shall take a detour to see how Spectre can be modeled in this
setting.

\subsection{Spectre}
\label{sec:spectre}

We give a simplified model of Spectre attacks, ignoring the details of
cache timing.  In this model, we extend programs with the ability to tell
whether a memory location has been touched (in practice this is
implemented using timing attacks on the cache). For example,
we can model Spectre by:
\[\begin{array}{l}
  \VAR a\SEMI \IF(\CANREAD(\SEC))\THEN a[\SEC]\GETS1
  \brELIF(\TOUCHED a[0])\THEN x\GETS0
  \brELIF(\TOUCHED a[1])\THEN x\GETS1 \FI
\end{array}\]
This is a low-security program, which is attempting to discover the
value of a high-security variable $\SEC$. The low-security program
is allowed to attempt to escalate its privileges by checking that it is
allowed to read a high-security variable:
\[\begin{array}{l}
  \IF(\CANREAD(\SEC))\THEN \mbox{code allowed to read $\SEC$}
  \brELSE \mbox{code not allowed to read $\SEC$} \FI
\end{array}\]
In this case, $\CANREAD(\SEC)$ is false, so the fallback code
is executed. Unfortunately, the escalated code is speculatively
evaluated, which allows information to leak by testing for which
memory locations have been touched.

We model the $\TOUCHED$ test by introducing a new action
$(\DT{\aLoc})$, and defining:
\[\begin{array}{l}
  \sem{\IF (\TOUCHED\aLoc) \THEN \aCmd \ELSE \bCmd \FI} \\[\jot]\quad =  ((\DT\aLoc) \prefix \sem{\aCmd}) \cup \sem{\bCmd}
\end{array}\]
Implementations of $\TOUCHED$ use cache timing, but their success can be modeled
without needing to be precise about such microarchitectural details:
\begin{itemize}
\item if $\labelling(\aEv)=(\aForm \mid \DT{\aLoc})$
  then there is $\bEv\gtN\aEv$
  where $\bEv$ reads or writes $\aLoc$.
\end{itemize}
Note that there is no requirement that $\bEv$ be satisfiable,
and indeed Spectre has the execution:
\[\begin{tikzpicture}[node distance=1em]
  \nonevent{rs}{\DR{\SEC}{1}}{}
  \nonevent{wa}{\DW{a[1]}{1}}{right=of rs}
  \event{ta}{\DT{a[1]}}{right=of wa}
  \event{wx}{\DW{x}{1}}{right=of ta}
  \po{rs}{wa}
  \wk{wa}{ta}
  \po{ta}{wx}
\end{tikzpicture}\]
but (assuming a successful implementation of $\TOUCHED$) \emph{not}:
\[\begin{tikzpicture}[node distance=1em]
  \nonevent{rs}{\DR{\SEC}{0}}{}
  \nonevent{wa}{\DW{a[0]}{1}}{right=of rs}
  \event{ta}{\DT{a[1]}}{right=of wa}
  \event{wx}{\DW{x}{1}}{right=of ta}
  \po{rs}{wa}
  \wk{wa}{ta}
  \po{ta}{wx}
\end{tikzpicture}\]
Thus, the attacker has managed to leak the value of a high-security
location to a low-security one: if $(\DW x1)$ is observed, the \verb|SECRET|
must have been 1.

This shows how our model of speculation can express
the way in which Spectre-like attacks bypass dynamic security checks,
without giving a treatment of microarchitecture.

\subsection{Relaxed memory}
\label{sec:relaxed-memory}

In \S\ref{sec:info-flow-attack} we present an information flow attack
on relaxed memory, similar to Spectre in that it relies on speculative
evaluation. Unlike Spectre it does not depend on timing attacks,
but instead is based on the sensitivity of relaxed memory to data
dependencies. % For this reason, we present a simple model of relaxed
% memory, which is strong enough to capture this attack.

Our model includes concurrent memory accesses, which can introduce concurrent
reads-from. 
Since we are allowing events to be partially ordered, this gives a simple
model of relaxed memory.  For example an independent read independent write
(IRIW) example is:
\[\begin{array}{l}
  x\GETS0\SEMI x\GETS x+1
  \PAR
  y\GETS0\SEMI y\GETS y+1
\\{}
  \PAR
  r_1\GETS x\SEMI r_2\GETS y
  \PAR
  s_1\GETS y\SEMI s_2\GETS x
\end{array}\]
which includes the execution:
\[\begin{tikzpicture}[node distance=1em]
  \event{wx0}{\DW{x}{0}}{}
  \event{wx1}{\DW{x}{1}}{right=of wx0}
  \event{wy0}{\DW{y}{0}}{right=2.5em of wx1}
  \event{wy1}{\DW{y}{1}}{right=of wy0}
  \event{ry1}{\DR{y}{1}}{below=4ex of wx0}
  \event{rx0}{\DR{x}{0}}{right=of ry1}
  \event{rx1}{\DR{x}{1}}{right=2.5 em of rx0}
  \event{ry0}{\DR{y}{0}}{right=of rx1}
  \rf{wx1}{rx1}
  \rf{wy0}{ry0}
  \rf[out=210,in=30]{wy1}{ry1}
  \rf{wx0}{rx0}
  \wk{rx0}{wx1}
  \wk{ry0}{wy1}
\end{tikzpicture}\]
This model does not introduce thin-air reads (TAR).
For example the TAR pit
\((
  x\GETS y \PAR y \GETS x
)\)
fails to produce a value for $x$ from thin air
since this produces a cycle in $\le$, as shown on the left below:
\begin{align*}
\begin{tikzpicture}[node distance=1em]
  \event{ry42}{\DR{y}{42}}{}
  \event{wx42}{\DW{x}{42}}{below=of ry42}
  \event{rx42}{\DR{x}{42}}{right=2.5em of ry42}
  \event{wy42}{\DW{y}{42}}{below=of rx42}
  \po{ry42}{wx42}
  \po{rx42}{wy42}
  \rf{wx42}{rx42}
  \rf{wy42}{ry42}
\end{tikzpicture}
&&
\begin{tikzpicture}[node distance=1em]
  \event{ry1}{\DR{y}{1}}{}
  \event{wx1}{\DW{x}{1}}{below=of ry1}
  \event{rx1}{\DR{x}{1}}{right=2.5em of ry1}
  \event{wy1}{\DW{y}{1}}{below=of rx1}
  \po{ry1}{wx1}
  \rf{wx1}{rx1}
  \rf{wy1}{ry1}
\end{tikzpicture}
\end{align*}
This cycle can be broken by removing a dependency. For example
\((
  x\GETS y \PAR r\GETS x\SEMI y \GETS r+1-r
)\)
has the execution on the right above.
% \[\begin{tikzpicture}[node distance=1em]
%   \event{ry1}{\DR{y}{1}}{}
%   \event{wx1}{\DW{x}{1}}{below=of ry1}
%   \event{rx1}{\DR{x}{1}}{right=2.5em of ry1}
%   \event{wy1}{\DW{y}{1}}{below=of rx1}
%   \po{ry1}{wx1}
%   \rf{wx1}{rx1}
%   \rf{wy1}{ry1}
% \end{tikzpicture}\]
Note that $(\DR x1) \not\le (\DW y1)$, so this does not introduce a cycle.

Although it is not the primary focus of this paper, our model may be an
attractive model of relaxed memory.  Many prior models either permit
thin-air executions that our model forbids or forbid desirable executions
that our model permits.
%% In \S\ref{sec:logic}, we develop a logic which allows us to prove that our
%% semantics forbids thin air examples that are permitted by prior speculative
%% models
%% \cite{Manson:2005:JMM:1047659.1040336,DBLP:conf/esop/JagadeesanPR10,DBLP:conf/popl/KangHLVD17}.
Our model passes all of the causality test cases
\cite{PughWebsite}.
%% Significantly, this
%% includes test case 9, which is forbidden by \cite{DBLP:conf/lics/JeffreyR16},
%% one of the few models that disallows the thin air example from
%% \S\ref{sec:logic}.  We present this test case in the appendix, where we also
%% discuss the thread inlining examples from
%% \cite{Manson:2005:JMM:1047659.1040336}.

\citet{DBLP:conf/esop/BattyMNPS15} showed that the thin-air problem has
no per-candidate-execution solution for C++.  This result does not apply to
our model, which has a different notion of dependency.
% as the semantics of a conditional can depend on the semantics
% of both branches.

\subsection{Information flow attacks on relaxed memory}
\label{sec:info-flow-attack}

Consider an attacker program, again using dynamic security checks to
try to learn a \verb|SECRET|. Whereas \verb|SPECTRE| uses
hardware capabilities, which have to be modeled by adding
extra capabilities to the language, this new attacker works
by exploiting relaxed memory which can result in
unexpected information flows. The attacker program is:
\[\begin{array}[t]{@{}l}
  \VAR x\GETS0\SEMI \VAR y\GETS0\SEMI\\\quad
    y\GETS x
  \PAR\begin{array}[t]{@{}l}
    \IF(y\EQ0)\THEN x\GETS1
    \brELIF(\CANREAD(\SEC))\THEN x\GETS\SEC
    \brELSE x\GETS1\SEMI z\GETS1 \FI
\end{array}\end{array}\]
In the case where $\SEC$ is $2$, this has many executions,
one of which is:
\[\begin{tikzpicture}[node distance=1em]
  \event{ix}{\DW{x}{0}}{}
  \event{iy}{\DW{y}{0}}{right=of ix}
  \event{rx0}{\DR{x}{0}}{below=of wx0}
  \event{wy0}{\DW{y}{0}}{right=of rx0}
  \event{ry0}{\DR{y}{0}}{below=of wy0}
  \event{wx1}{\DW{x}{1}}{right=of ry0}
  \nonevent{wx2}{\DW{x}{2}}{right=of wx1}
  \nonevent{wz1}{\DW{z}{1}}{right=of wx2}
  \po{rx0}{wy0}
  \po{ry0}{wx1}
  \po[out=30,in=150]{ry0}{wz1}
  \po[out=25,in=155]{ry0}{wx2}
  \rf{ix}{rx0}
  \rf{wy0}{ry0}
  \wk{iy}{wy0}
\end{tikzpicture}\]
but there are no executions which exhibit
$(\DW{z}{1})$, since any attempt to do so
produces a cycle:
\[\begin{tikzpicture}[node distance=1em]
  \event{ix}{\DW{x}{0}}{}
  \event{iy}{\DW{y}{0}}{right=of ix}
  \event{rx1}{\DR{x}{1}}{below=of ix}
  \event{wy1}{\DW{y}{1}}{right=of rx1}
  \event{ry1}{\DR{y}{1}}{below=of wy1}
  \event{wx1}{\DW{x}{1}}{right=of ry0}
  \nonevent{wx2}{\DW{x}{2}}{right=of wx1}
  \event{wz1}{\DW{z}{1}}{right=of wx2}
  \po{rx1}{wy1}
  \po{ry1}{wx1}
  \po[out=30,in=150]{ry1}{wz1}
  \po[out=25,in=155]{ry1}{wx2}
  \rf[in=-90,out=-150]{wx1}{rx1}
  \rf{wy1}{ry1}
  \wk[out=-20,in=90]{ix}{wx1}
  \wk[out=-20,in=120]{ix}{wx2}
  \wk{iy}{wy1}
\end{tikzpicture}\]\vskip-\bigskipamount\noindent
In the case where \verb|SECRET| is $1$, there is an execution:
\[\begin{tikzpicture}[node distance=1em]
  \event{ix}{\DW{x}{0}}{}
  \event{iy}{\DW{y}{0}}{right=of ix}
  \event{rx1}{\DR{x}{1}}{below=of ix}
  \event{wy1}{\DW{y}{1}}{right=of rx1}
  \event{ry1}{\DR{y}{1}}{below=of wy1}
  \event{wx1}{\DW{x}{1}}{right=of ry0}
  \event{wz1}{\DW{z}{1}}{right=of wx1}
  \po{rx1}{wy1}
  \po[out=30,in=150]{ry1}{wz1}
  \rf[in=-90,out=-150]{wx1}{rx1}
  \rf{wy1}{ry1}
  \wk[out=-20,in=90]{ix}{wx1}
  \wk[out=-20,in=120]{ix}{wx2}
  \wk{iy}{wy1}
\end{tikzpicture}\]\vskip-\bigskipamount\noindent
Note that in this case, there is no dependency from
$(\DR{y}{1})$ to $(\DW{x}{1})$.  This lack of dependency makes the
execution possible. Thus, if the attacker sees
an execution with $(\DW{z}{1})$, they can conclude
that \verb|SECRET| is $1$, which is an information flow
attack.

This attack is not just an artifact of the model,
since the same behavior can be exhibited by
compiler optimizations. Consider the program fragment:
\[\begin{array}{l}
    \IF(y = 0)\THEN x\GETS1
    \brELIF(\CANREAD(\SEC))\THEN x\GETS\SEC
    \brELSE x\GETS1\SEMI z\GETS1 \FI
\end{array}\]
In the case where \verb|SECRET| is a constant \verb|1|,
the compiler can inline it
and lift the assignment to $x$ out of the $\IF$ statement:
\[\begin{array}{l}
    x\GETS1\SEMI
    \IF(y = 0)\THEN
    \brELIF(\CANREAD(\SEC))\THEN
    \brELSE z\GETS1 \FI
\end{array}\]
After these optimizations, a sequentially consistent execution
exhibits $(\DW{z}{1})$. We discuss the practicality of this attack
further in \S\ref{sec:experiments}.

\subsection{Dead store elimination}
\label{sec:dse}

A common compiler optimization is \emph{dead store elimination},
in which writes are omitted if they will be overwritten by a subsequent
write later in the same thread. We can model eliminated writes
by ones with an unsatisfiable precondition. For example,
one execution of $(x \GETS 1\SEMI x \GETS 2) \PAR (r \GETS x)$ is:
\[\begin{tikzpicture}[node distance=1em]
  \nonevent{wx1}{\DW{x}{1}}{}
  \event{wx2}{\DW{x}{2}}{right=of wx1}
  \event{rx2}{\DR{x}{2}}{right=2.5em of wx2}
  \wk{wx1}{wx2}
  \rf{wx2}{rx2}
\end{tikzpicture}\]
Recall that for any satisfiable $\aEv$, if $\aEv$ reads $\aLoc$ from $\bLoc$
then $\bEv$ is satisfiable. This means that, although we can eliminate
$(\DW{x}{1})$ we cannot eliminate $(\DW{x}{2})$.

One heuristic that a compiler might adopt is to only eliminate
writes that are guaranteed to be followed by another write
to the same variable. This can be formalized by saying that
a write event $\bEv$ is eliminable if
there is a tautology $\aEv \ltN \bEv$
which writes to the same location.
A model of dead store elimination is one where,
in every pomset, every eliminable event is unsatisfiable.
This model includes the example above.

Note that if dead store
elimination is \emph{always} performed, then there is an information
flow attack similar to the one in \S\ref{sec:info-flow-attack}. Consider
the program:
\[\begin{array}[t]{@{}l}
    y\GETS x
  \PAR\begin{array}[t]{@{}l}
    x\GETS 1\SEMI\\
    \IF(\CANREAD(\SEC))\THEN \IF(\SEC)\THEN x\GETS 2\FI
    \brELSE x\GETS 2\FI
\end{array}\end{array}\]
In the case that \verb|SECRET| is $0$, there is an execution:
\[\begin{tikzpicture}[node distance=1em]
  \event{rx1}{\DR{x}{1}}{}
  \event{wy1}{\DW{y}{1}}{right=of rx1}
  \event{wx1}{\DW{x}{1}}{right=2.5em of wy1}
  \event{wx2}{\aForm \mid \DW{x}{2}}{right=of wx1}
  \rf[out=160,in=20]{wx1}{rx1}
  \po{rx1}{wy1}
  \wk{wx1}{wx2}
\end{tikzpicture}\]
where $\aForm$ is ($\lnot$\verb|canRead(SECRET)|),
which is not a tautology, and so the $(\DW{x}{1})$ event is not eliminated.
In the case that \verb|SECRET| is not $0$, the matching execution
is:
\[\begin{tikzpicture}[node distance=1em]
  \event{rx2}{\DR{x}{2}}{}
  \event{wy2}{\DW{y}{2}}{right=of rx2}
  \nonevent{wx1}{\DW{x}{1}}{right=2.5em of wy2}
  \event{wx2}{\DW{x}{2}}{right=of wx1}
  \rf[out=160,in=20]{wx2}{rx2}
  \po{rx2}{wy2}
  \wk{wx1}{wx2}
\end{tikzpicture}\]
Now the $(\DW{x}{2})$ event is a guaranteed write, so the $(\DW{x}{1})$
is eliminated, and so cannot be read.
In the case that the attacker can rely on dead store
elimination taking place, this is an information flow: if the attacker observes
$x$ to be $1$, then they know \verb|SECRET| is $0$. We return to this attack
in \S\ref{sec:experiments}.


% Local Variables:
% TeX-master: "paper"
% End:

\subsection{Fences and release/acquire synchronization}

We assume there are sets $\Rel$ and $\Acq \subseteq\Act$.  We say that
$\aAct$ is a \emph{release action} if $\aAct\in\Rel$ and $\aAct$ is an
\emph{acquire action} if $\aAct\in\Rel$.  

We model release/acquire synchronization by introducing the actions:
\begin{itemize}
%\item $(\DF)$ is both a release and an acquire action
\item $(\DWRel{\aLoc}{\aVal})$, a release action that writes $\aVal$ to $\aLoc$, and
\item $(\DRAcq{\aLoc}{\aVal})$, an acquire action that reads $\aVal$ from $\aLoc$,
\end{itemize}
with semantics:
\begin{eqnarray*}
  %\sem{\FENCE\SEMI \bCmd} & = & (\TRUE \mid \DF) \prefix \sem{\bCmd} \\
  \sem{\REL\aLoc\GETS\aExp\SEMI \bCmd}
  & = & \textstyle\bigcup_\aVal\; (\aExp=\aVal \mid \DWRel\aLoc\aVal) \prefix \sem{\bCmd}[\aExp/\aLoc]
  \\
  \sem{\aReg\GETS\ACQ\aLoc\SEMI \bCmd}
  & = & \textstyle\bigcup_\aVal\; (\TRUE \mid \DRAcq\aLoc\aVal) \prefix \sem{\bCmd}[\aLoc/\aReg] 
\end{eqnarray*}
%There are no additional requirements for $\aLoc$-closure is.

Publication example:
\begin{alltt}
    var x; var f; x:=0; f:=0; (x:=1; \REL{f}:=1;  ||  r:=\ACQ{f}; s:=x;)
\end{alltt}
We disallow the execution where \texttt{r==1} and \texttt{s!=1}
Note that $\RF$ is only augmented in the parallel rule, so neither thread can have an $\RF$ from an init action.

\subsection{Transactions}

We model transactions by introducing the actions:
\begin{itemize}
\item $(\DB)$, an acquire action, and
\item $(\DC{\vec\aLoc}{\vec\aVal})$, a release action that writes $\vec\aVal$ to $\vec\aLoc$,
\end{itemize}
with semantics:
\begin{eqnarray*}
  \sem{\BEGIN\SEMI \bCmd}
  & = & %\mathit{atomic}(
  (\TRUE \mid \DB) \prefix \sem{\bCmd}
  \\
  \sem{\IF\COMMIT\vec\aLoc\THEN \bCmd_1 \ELSE \bCmd_2}
  & = & \textstyle\bigcup_{\vec\aVal,\,\aForm\,\text{implies}\,\vec\aLoc=\vec\aVal}\;
        ((\aForm \mid \DC{\vec\aLoc}{\vec\aVal}) \prefix (\aForm \mid \sem{\bCmd_1}))
        \sqcup  (\lnot\aForm \mid \sem{\bCmd_2})
\end{eqnarray*}
where we require that all pomsets in the semantics of programs be
\emph{atomic}.

Before defining atomicity, we provide some auxiliary notation.

We say that $\aEv$ is a \emph{begin event} if
$\labelling(\aEv)=(\aForm\mid\DB)$ and a \emph{commit event} if
$\labelling(\aEv)=(\aForm\mid\DC{\vec\aLoc}{\vec\aVal})$.

We write $\aForm_\aEv$ for the formula of $\aEv$; that is, when
$\labelling(\aEv)=(\aForm_\aEv\mid\aAct)$.

We say that $\aForm$ is \emph{compatible with} $\bForm$ when
$\aForm\land\bForm$ is satisfiable.

We say that  event $\aEv$ \emph{happens between} $\begEv$ and $\vec\comEv$ when
\begin{itemize}
\item $\begEv$ is a begin event, $\begEv<\aEv$, and
  there is no commit event $\bEv$ such that $\begEv<\bEv<\aEv$,
\item $\vec\comEv$ are the commit events $\comEv_i$ such that $\aEv<\comEv_i$ and
  there is no begin event $\bEv$ such that $\aEv<\bEv<\comEv_i$.
\end{itemize}

\begin{definition}
  A pomset is \emph{atomic} when for any $\aEv$ that happens between $\begEv$ and $\vec\comEv$:
  \begin{enumerate}
  \item\label{xcommit} $\aForm_{\aEv}$ implies $\textstyle\bigcup_i\aForm_{\comEv_i}$,
  \item\label{xliftb} if $\bEv<\aEv$ then $\bEv<\begEv$, 
  \item\label{xliftc} if $\aEv<\bEv$ and $\aForm_\bEv$ is compatible with
    $\aForm_{\comEv_i}$ then $\comEv_i<\bEv$, 
  \item\label{xrf} if $\aEv'\neq\aEv$ also happens between $\begEv$ and $\vec\comEv$ and
    \begin{itemize}
    \item $\aEv$ reads from $\bEv'$ that happens between $\begEv'$ and $\vec\comEv'$,
    \item $\aEv'$ reads from $\bEv''$ that happens between $\begEv''$ and $\vec\comEv''$,
    \item $\comEv''_j$ is compatible with $\bEv'$, and 
    \item $\comEv'_i$ is compatible with $\bEv''$ 
    \end{itemize}
    then either
    $\comEv''_j<\begEv'$ or
    $\comEv'_i<\begEv''$ .
  \end{enumerate}
\end{definition}
Clause \eqref{xcommit} requires that the precondition on $\aEv$ is false on an
aborted transaction.
The \emph{lifting clauses}, \eqref{xliftb} and \eqref{xliftc}, require order
come in or out of $\aEv$ is lifted to the corresponding begin or commit event.
Clause \eqref{xrf} requires that whenever a transaction reads from two other
transactions, the other transactions must be ordered.

The definition of atomicity guarantees strong isolation.  For weak isolation,
the \eqref{xliftb} only applies when $\bEv$ is a commit and \eqref{xliftc} only
applies when $\bEv$ is a begin.

The definition handles simple examples:
\begin{itemize}
\item Single threaded example: $\DB_1 \DC{}{}_1 \DB_2 \DC{}{}_2$.  By release semantics of
  $\DC{}{}_2$, we know that $\DC{}{}_1<\DC{}{}_2$. By lifting, we now that $\DC{}{}_1<\DB_2$.
\item Clause \eqref{xrf} stops transaction from reading two different values
  for the same variable from transactions (it is possible with no
  transactional writes).  
\item Clause \eqref{xrf} also stops transactional IRIW.
\end{itemize}
% \begin{definition}
%   An rf-pomset is transaction-closed if the $\DB$ and $\DC{}{}$ actions with
%   satisfiable preconditions are totally ordered by $<$.
% \end{definition}

Let ``$\END\SEMI \bCmd$'' be syntax sugar for
``$\IF\COMMIT\vec\aLoc\THEN\bCmd \ELSE \bCmd$'', where $\vec\aLoc$ are the
free variables of $\bCmd$.

The semantics of
\begin{alltt}
  x:=1; begin; x:=2; end; y:=x;
\end{alltt}
includes
\[\begin{tikzpicture}[node distance=1em]
  \event{wx1}{\DW{x}{1}}{}
  \event{b}{\DB}{right=of wx1}
  \event{wx2}{\DW{x}{2}}{right=of b}
  \event{c}{\DC{x}{2}}{right=of wx2}
  \event{wy2}{\DW{y}{2}}{right=of c}
  \nonevent{wy1}{\DW{y}{1}}{below=of wy2}
  \po{b}{wx2}
  \po[bend right]{b}{wy1}
  \po[bend left]{b}{wy2}
  \po{wx2}{c}
  \po[bend left]{wx1}{c}
  %\po{rz0}{wy2}
\end{tikzpicture}\]
and
\[\begin{tikzpicture}[node distance=1em]
  \event{wx1}{\DW{x}{1}}{}
  \event{b}{\DB}{right=of wx1}
  \nonevent{wx2}{\DW{x}{2}}{right=of b}
  \nonevent{c}{\DC{x}{2}}{right=of wx2}
  \nonevent{wy2}{\DW{y}{2}}{right=of c}
  \event{wy1}{\DW{y}{1}}{below=of wy2}
  \po{b}{wx2}
  \po[bend right]{b}{wy1}
  \po[bend left]{b}{wy2}
  \po{wx2}{c}
  \po[bend left]{wx1}{c}
  %\po{rz0}{wy2}
\end{tikzpicture}\]

Publication example:
\begin{alltt}
  var x; var f; x:=0; f:=0; 
     x:=1; (begin; f:=1; end;) || (begin; r:=f; end; s:=x;)
\end{alltt}

\subsection{Picky details}

The relations $\rreads$ and $\rwrites$ and $\RF$ should be restricted
to elements of $(\Act \times \Loc \times \Val)$ that are in
$(\Act \times (\Loc \rightarrow \Val))$.

$(\DT{\aLoc})$ neither reads nor writes $\aLoc$.

% Note: we could also include a transaction factory, and close the factory.
% \begin{alltt}
%   TransactionFactory T; var x; var f; x:=0; f:=0; fence; 
%      x:=1; (begin T; f:=1; f:=2; end T;) || (begin T; r:=f; end T; s:=x;)
% \end{alltt}

% Local Variables:
% TeX-master: "x"
% End:

\section{Experiments}
\label{sec:experiments}

One theme of this paper is that optimizations not typically part of formal
abstractions can result in information flow leaks.
This is typified by the Spectre attack, which leverages speculative execution,
a hardware optimization.
\S\ref{sec:info-flow-attack} and~\S\ref{sec:dse} presented other attacks
along the same line, which leverage compiler optimizations.
These attacks also, unlike Spectre, do not rely on timing side channels, or
indeed timers of any kind, bypassing many common Spectre mitigations~\cite{???}.
%%%%% FuzzyFox, Chrome's and Firefox's restrictions on precise timers, etc.

In this section we present implementations of the attacks described
in~\S\ref{sec:info-flow-attack} and~\S\ref{sec:dse}, in both cases
exploiting compiler optimizations to construct an information flow attack.
\ignore{
The attacker model (detailed in~\S\ref{subsec:attacker-model})
is currently unrealistic, as we attack C compilers rather than scripting
languages, and we require the secret to be a compile-time constant which
the compiler can optimize on.
This renders these attacks proof-of-concepts rather than
immediately exploitable vulnerabilities.
However, we believe their novelty may lead to
interesting discussion, and with much more development, these attacks may
evolve into genuine threats against targets such as JIT compilers.
}
We demonstrate the efficacy of our proof-of-concept attacks against
the {\CLANG} and {\GCC} C compilers.
All of our experiments are performed on a \todo{describe machine} with
{\CLANG} version \todo{clang version} and {\GCC} version \todo{gcc
version}.

\subsection{Attacker model}
\label{subsec:attacker-model}

In our attacker model, we assume that there is a {\SEC} hardcoded into an
application; for instance, {\SEC} may be an API key.
This {\SEC} is known at compile time, but may not be
accessed except behind a security check.
Since the attacker is running with low security privileges,
the security check always fails,
so the attacker can only access {\SEC} in dead code.
\ignore{
The attacker has no capabilities other than writing and executing code --- in
particular the attacker may not disassemble the compiler or libraries to learn
the {\SEC} directly; may not examine the internal state of the compiler;
may not access timers of any kind; and may not leverage hardware side channels.
}
The attacker's goal is to learn the value of the {\SEC}.

As a hypothetical example, suppose there is a library which contains
a hardcoded {\SEC} such as an API or signing key, which cannot be accessed
directly, only through a function guarded by a security check:
\todo{change real code to match this}
\begin{verbatim}
  private static const uint SECRET = 0x1234;
  private static volatile uint securityLevel = 0;
  public uint get_secret() {
    if (securityLevel > 0) { return SECRET; }
    else { return 0; }
  }
\end{verbatim}
This is not necessarily a realistic attacker model,
since in most cases secrets are only known at run time rather than compile time,
which means that the attacks presented in this section
are more proof-of-concepts rather than immediately exploitable vulnerabilities.
However, the mechanisms we use are novel and could potentially be applied
in other contexts.
For instance, many real-world contexts allow untrusted or
third-party entities to write code in a scripting language which is then
compiled alongside and integrated into a larger application, often
using a just-in-time (JIT) compiler.
JavaScript code from third-party websites running in a browser is a common
example of this.
We give an attacker similar capabilities against a
compiler, except that we consider the simpler setting of using C code against a C
compiler.
One could imagine a similar attack using JavaScript against browser JIT
compilers, where the compiler may have access to interesting secrets such as the
browser's cookie store, and may be able to optimize based on those secrets.
We plan to explore JavaScript attacks of this type as future work.

\subsection{Load-store reordering attack}
\label{subsec:exp-rel-mem}

We begin by examining the attack in~\S\ref{sec:info-flow-attack} in
more detail, subject to the attacker model given above.
In particular, we show that by exploiting compiler optimizations which perform
load-store reordering, an attacker can learn the value of a compile-time
{\SEC} despite only being allowed to use it inside dead code, that is,
code that can never be executed at runtime.
This attack was tested and works against {\GCC} version \todo{gcc version}.

The form of the attack presented in~\S\ref{sec:info-flow-attack} works in
theory, but in practice, just because a compiler is \emph{allowed} to perform a
load-store reordering doesn't mean that it \emph{will}.
We found that {\GCC} and {\CLANG} chose to read $y$ into a
register first (before writing to $x$), regardless of the value of
{\SEC}.
However, we did find a related pattern in which {\GCC} will emit a
different ordering of the read of $y$ and the write of $x$ depending
on the value of a {\SEC}:
\[\begin{array}[t]{@{}l}
  \VAR x\GETS0\SEMI \VAR y\GETS0\SEMI\\\quad
    y\GETS x
  \PAR\begin{array}[t]{@{}l}
    x\GETS 1\SEMI\\
    \IF(\CANREAD(\SEC))\THEN x\GETS\SEC\SEMI\FI\\
    \IF(y > 0)\THEN \RETURN 0
    \brELSE \RETURN 1\FI
\end{array}\end{array}\]
\ignore{
\begin{verbatim}
    x := 0; y := 0;
    (
      y := x;
    ) || (
      x := 1;
      if (canRead(SECRET)) { x := SECRET; }
      if (y) { return 0; }
      else { return 1; }
    )
\end{verbatim}
}

Figure~\ref{fig:lsr-asm} shows the assembly output of {\GCC} in the cases
where {\SEC} is 0 and 1 respectively.
In the case that {\SEC} is $1$, {\GCC} removes the \IF
statement entirely, and moves the read of $y$ above the write of $x$.
However, when {\SEC} is $0$, the \IF statement must remain
intact, and {\GCC} does not move the read of $y$.
This means that if {\SEC} is $1$, the second thread will always
read $y\EQ0$ and always return $1$.
However, if {\SEC} is $0$, it is possible that the first thread
may observe $x\EQ1$ and write $y\GETS1$ in time for the second thread
to observe $y\EQ1$ and thus return $0$.
In this way, we leverage compiler load-store reordering to learn the value of
a compile-time {\SEC}.

\begin{figure}
  \begin{tabular}[fragile]{p{3cm} | p{3cm}}
    \texttt{SECRET == 0} & \texttt{SECRET == 1} \\
\begin{verbatim}
  mov f(%rip), %eax
  mov $1, x(%rip)
  test %eax, %eax
  je label1
  mov %0, x(%rip)
label1:
  mov y(%rip), %eax
  test %eax, %eax
  sete %eax
  ret
\end{verbatim}
  &
\begin{verbatim}
  mov f(%rip), %eax
  mov y(%rip), %eax
  mov $1, x(%rip)
  test %eax, %eax
  sete %eax
  ret
\end{verbatim}
  \\
  \end{tabular}
  \caption{
    (Simplified) x86 assembly output from \texttt{gcc} for the main thread of
    the load-store reordering attack.
    In particular, note that the order between \texttt{mov \$1, x(\%rip)}
    and \texttt{mov y(\%rip), \%eax} is different in the two cases.
    The call to \texttt{canRead(SECRET)} has been inlined; we implemented
    \texttt{canRead(x)} as \texttt{return f;} where
    \texttt{volatile bool f = false;}.
    Thus, \texttt{gcc} preserves the read of \texttt{f} even when its value is
    unused, as in the case on the right.
  }
  \label{fig:lsr-asm}
\end{figure}

We extend this attack to leak a secret consisting of an arbitrary number
\verb|N| of bits.
To do this, we simply compile \verb|N| copies of the test function, each
performing a boolean test on a single bit of the secret.
The function used for reading the \verb|k|th bit is as follows (for
\verb|N <= 64|):
\[\begin{array}[t]{@{}l}
  \VAR x\GETS0\SEMI \VAR y\GETS0\SEMI\\\quad
    y\GETS x
  \PAR\begin{array}[t]{@{}l}
    x\GETS 1\SEMI\\
    \IF(\CANREAD(\SEC))\THEN x\GETS\texttt{(\SEC\, \& (1 << k)) ? 1 : 0}\SEMI\FI\\
    \IF(y > 0)\THEN \RETURN 0
    \brELSE \RETURN 1\FI
\end{array}\end{array}\]
\ignore{
\begin{verbatim}
    x := 0; y := 0;
    (
      y := x;
    ) || (
      x := 1;
      if (canRead(SECRET)) { x := (SECRET & (1 << k)) ? 1 : 0; }
      if (y) { return 0; }
      else { return 1; }
    )
\end{verbatim}
}
Following the same analysis as above, this function will always return $1$
if the appropriate bit of {\SEC} is $1$, but may return $0$ if
the appropriate bit of {\SEC} is $0$.
The extension of the attack to the general case with truly arbitrary \verb|N|
is straightforward; {\SEC} becomes an array of 64-bit values, and we use
\verb|k / 64| and \verb|1 << (k & 63)| as the array index and bitmask
respectively.

We make three additional tweaks to improve the reliability so that the attacker
can confidently infer the value of {\SEC} based on the observed return
values of the function.
First, rather than performing $y\GETS x$ only once in the first thread, we
perform $y\GETS x$ continuously in a loop.
This maximizes the probability that, once $x\GETS 1$ occurs in the second
thread, $y$ will be immediately assigned $1$ by the first thread
and the second thread will be able to read $y\EQ 1$.

Second, we wish to lengthen the timing window between $x\GETS 1$ and the
read of $y$ in the second thread (in the case where the appropriate bit of
{\SEC} is $0$ and the read of $y$ remains below $x\GETS 1$).
However, we wish to do this in a way that does not block the reordering of the
read of $y$ upwards in the case where the appropriate bit of {\SEC}
is $1$.
We do this by inserting many copies of the line
\begin{verbatim}
    if (canRead(SECRET)) { x := (SECRET & (1 << k)) ? 1 : 0; }
\end{verbatim}
instead of just one.
In the case where the appropriate bit of {\SEC} is $0$, this
results in many calls to $\CANREAD(\SEC)$ and many conditional jumps,
which in practice creates a timing window for the first thread to perform
$y\GETS x$.
However, in the case where the appropriate bit of {\SEC} is $1$,
all of these inserted lines can be removed just as a single copy could be.
In practice, we found that inserting too many copies of the line prevents
{\GCC} from reordering the read of $y$ above the write to $x$ as
desired; inserting $30$ copies was sufficient to create a timing window
while still allowing the desired reordering.

Finally, we redundantly execute the entire attack several times, noting the
return value of the function in each case.
We note that if \emph{any} of the redundant runs produces a return value of
$0$ for a particular bit position, we can be certain that the
corresponding bit of {\SEC} \emph{must} be $0$, as it implies the
read of $y$ was not reordered upwards in that particular function.
On the other hand, the more runs that produce a return value of $1$ for a
particular bit position, the more certain we can be that the read of $y$
was reordered above the $x\GETS 1$ assignment, and the appropriate bit of
{\SEC} is $1$.

\begin{figure}
  \small
  \begin{tabular}{ r | l | l | l }
    Redundancy & Bandwidth (bits/s) & Bitwise Acc & Per-run Acc \\ \hline
    1          & 3.17 million       & 90.13\%     & 0.0\%       \\
    2          & 1.62 million       & 96.77\%     & 0.7\%       \\
    3          & 1.07 million       & 98.84\%     & 3.9\%       \\
    4          & 812 thousand       & 99.55\%     & 13.5\%      \\
    5          & 652 thousand       & 99.83\%     & 34.0\%      \\
    7          & 466 thousand       & 99.97\%     & 71.8\%      \\
    10         & 322 thousand       & 99.998\%    & 96.6\%      \\
    15         & 216 thousand       & 100.00\%    & 100.0\%     \\
  \end{tabular}
  \caption{
    Performance results for the load-store reordering attack when leaking a
    2048-bit secret.
    `Redundancy' is the number of redundant runs performed for error
    correction; more redundant runs improves accuracy but reduces bandwidth.
    `Bandwidth' is the number of bits leaked per second after accounting for
    any error correction.
    `Bitwise Accuracy' is the percentage of bits that were correct, while
    `Per-run Accuracy' is the percentage of full 2048-bit secrets that were
    correct in all bit positions.
    \todo{Note: numbers are not final (collected on Craig's machine inside a
    VM), but give an idea of where we stand.}
  }
  \label{fig:load-store-perf}
\end{figure}

Figure~\ref{fig:load-store-perf} gives the performance results for this attack
against {\GCC} version \todo{ver}.
The attack can sustain hundreds of thousands of bits per second leaked with
near-perfect accuracy, or millions of bits per second with error rates of a
few percent.
This means that an attacker can leak a 2048-bit secret with near-perfect
accuracy in under $10$ ms.
Note that this bandwidth assumes that all copies of the attack function are
already compiled; the cost of compilation is not included here.

\subsection{Dead store elimination attack}
\label{subsec:exp-dse}

In this section we return to the attack in~\S\ref{sec:dse} based on
dead store elimination.
We show that in our attacker model (given in~\S\ref{subsec:attacker-model}),
the attacker is able to exploit dead
store elimination to again learn the value of a compile-time {\SEC}
despite only being allowed to use it inside dead code, that is, code that can
never be executed at runtime.
This attack is even more efficient than the attack on load-store reordering,
and further, we were able to demonstrate its effectiveness against both
{\GCC} and {\CLANG}.

We start from the simple form of the attack presented in~\S\ref{sec:dse},
and extend it to leak a secret consisting of an
arbitrary number \verb|N| of bits.
As we did in the load-store reordering attack, we again compile \verb|N| copies
of the test function, each performing a boolean test on a single bit of the
secret.
The function used for reading the \verb|k|th bit is as follows (for
\verb|N <= 64|):
\[\begin{array}[t]{@{}l}
  \VAR x\GETS0\SEMI\\\quad
    r\GETS x
  \PAR\begin{array}[t]{@{}l}
    x\GETS 1\SEMI\\
    \IF(\CANREAD(\SEC))\THEN \IF(\SEC\,\texttt{ \& (1 << k)}\NOTEQ0)\THEN x\GETS 2\FI
    \brELSE x\GETS 2\FI
\end{array}\end{array}\]
\ignore{
\begin{verbatim}
    (
      r := x;
    ) || (
      x := 1;
      if (canRead(SECRET)) {
        if (SECRET & (1 << k)) { x := 2; }
      } else {
        x := 2;
      }
    )
\end{verbatim}
}
Then, we test each function in turn, each time noting the value of $r$
observed by the `listening' thread.
If the appropriate bit of {\SEC} is 1, the $x\GETS 2$ assignment is
guaranteed to happen, so the compiler can eliminate the $x \GETS 1$
assignment as a dead store and we will observe $r\EQ 2$; however, if the
appropriate bit of {\SEC} is 0, the $x\GETS 1$ assignment cannot be
eliminated, and we will observe $r\EQ 1$ with some probability.
The extension of the attack to the general case with truly arbitrary \verb|N|
is straightforward and proceeds exactly as it did for the attack on
load-store reordering.

We make three additional tweaks to improve the reliability so that the attacker
can confidently infer the value of {\SEC} based on the observed values
of $r$.
These three tweaks strongly resemble the reliability tweaks we made to the
load-store reordering attack and differ only in a few details.

First, rather than simply observing $x$ with $r\GETS x$ in the
`listening' thread, we continuously load $x$ in a loop until a
nonzero value is observed --- i.e., we perform
$\DO{r\GETS x} \WHILE(r\EQ0)$.
\ignore{
\begin{verbatim}
    do {
      r := x;
    } while(r == 0);
\end{verbatim}
}
This remedies the case where $r\GETS x$ could observe a value of $x$
from `before' either of the two possible writes performed by the other thread.

Second, we insert additional time-consuming computation immediately following
the $x\GETS 1$ operation in the `main' thread.
This lengthens the timing window in which $x$ has the value $1$,
increasing the likelihood that the `listening' thread will be able to observe
$x\EQ 1$ (unless the $x\GETS 1$ write was eliminated, of course).
Inserting this computation can be done without interfering with the dead store
elimination process itself, so that the compiler will continue to eliminate
the $x\GETS 1$ write if and only if the appropriate bit of {\SEC}
was 1.
For {\GCC}, we have a fair amount of freedom with the time-consuming
computation --- for instance, we can use an arbitrarily long loop.
In fact, we can perform a further optimization by monitoring the value of the
variable $r$ (written to by the listening thread) and breaking out of the
loop early if we see that the listening thread has already observed $x\EQ 1$.
However, with {\CLANG}, we cannot use a loop at all --- the time-consuming
computation must be branch-free, and furthermore must not consist of too many
instructions.
This is because {\CLANG}'s dead store elimination pass operates only
within basic blocks, and uses a heuristic to stop scanning the basic block
early if it is too large.
Nonetheless, we find that even with these restrictions, we are able to
construct a reliable and fast attack against both {\CLANG} and {\GCC}.

Finally, we redundantly execute the entire attack several times, noting the
final value of $r$ (the first observed nonzero value of $x$) in each
case.
We note that if \emph{any} of the redundant runs produces $r\EQ 1$ for a
particular bit position, we can be certain that the corresponding bit of
{\SEC} \emph{must} be $0$, as it implies that the $x\GETS 1$ write
was not eliminated in that particular function.
On the other hand, the more runs that observe $r\EQ 2$ in a particular bit
position despite our other reliability-increasing measures taken above, the
more certain we can be that the $x\GETS 1$ write was eliminated in that
function, and the appropriate bit of {\SEC} is $1$.

\begin{figure}
  \small
  \begin{tabular}{ r | l | l | l }
    Redundancy & Bandwidth (bits/s) & Bitwise Acc & Per-run Acc \\ \hline
    1          & 1.40 million       & 99.92\%     & 66.8\%      \\
    2          & 702 thousand       & 99.999\%    & 97.5\%      \\
    3          & 470 thousand       & 99.99999\%  & 99.9\%       \\
    4          & 352 thousand       & 100.0\%     & 100.0\%      \\
  \end{tabular}
  \caption{
    Performance results for the dead store elimination attack on {\CLANG} when
    leaking a 2048-bit secret.
    Terms are the same as defined in the caption for Figure~\ref{fig:load-store-perf}.
    \todo{Note: numbers are not final (collected on Craig's Mac with Apple Clang),
    but give an idea of where we stand.}
  }
  \label{fig:clang-dse-perf}
\end{figure}

Performance results for the dead store elimination attack against {\CLANG}
are given in Figure~\ref{fig:clang-dse-perf}.
This attack is faster than the load-store reordering attack against {\GCC} for
any given desired accuracy.
\todo{however, compared to the load-store reordering attack we don't have an
even-faster-but-even-less-accurate setting.
This could be accomplished by reducing the amount of `time-consuming computation'
which for clang we have been holding constant near the max value clang will
tolerate.
Really, we could make this a 2D table like Figure~\ref{fig:gcc-dse-perf}, we'll
just have a strict maximum on how far to the right you can go.}

\begin{figure}
  \small
  \begin{tabular}{ r | c | c | c | c | c | c }
    Stall amount & 10 & 20 & 50 & 100 & 200 & 500 \\ \hline
    Redundancy 1 & \makecell{2.75 million\\97.13\%} &
                   \makecell{2.55 million\\99.72\%} &
                   \makecell{2.27 million\\99.98\%} &
                   \makecell{1.83 million\\99.99\%} &
                   \makecell{1.34 million\\99.99\%} &
                   \makecell{728 thousand\\99.99\%} \\ \hline
    Redundancy 2 & \makecell{1.40 million\\99.01\%} &
                   \makecell{1.30 million\\99.99\%} &
                   \makecell{1.14 million\\100.0\%} &
                   \makecell{902 thousand\\100.0\%} &
                   \makecell{663 thousand\\100.0\%} &
                   \makecell{382 thousand\\100.0\%} \\ \hline
    Redundancy 3 & \makecell{926 thousand\\99.44\%} &
                   \makecell{870 thousand\\100.0\%} &
                   \makecell{763 thousand\\100.0\%} &
                   \makecell{610 thousand\\100.0\%} &
                   \makecell{443 thousand\\100.0\%} &
                   \makecell{246 thousand\\100.0\%} \\ \hline
    Redundancy 4 & \makecell{694 thousand\\99.71\%} &
                   \makecell{645 thousand\\100.0\%} &
                   \makecell{571 thousand\\100.0\%} &
                   \makecell{451 thousand\\100.0\%} &
                   \makecell{330 thousand\\100.0\%} &
                   \makecell{186 thousand\\100.0\%} \\
  \end{tabular}
  \caption{
    Performance results for the dead store elimination attack on {\GCC} when
    leaking a 2048-bit secret.
    Rows give different values of `redundancy' (as defined in previous figures),
    while columns give amounts of stall time immediately following the
    $x\GETS 1$ write (as measured in loop iterations).
    Each table cell gives the leak bandwidth in bits/sec, followed by the
    bitwise accuracy.
    \todo{Note: numbers are not final (collected on Craig's machine inside a
    VM), but give an idea of where we stand.}
  }
  \label{fig:gcc-dse-perf}
\end{figure}

Figure~\ref{fig:gcc-dse-perf} shows the performance results for the dead
store elimination attack against {\GCC}.
Unlike the other attacks we've discussed, our implementation of this attack
has two important ``knobs'' which trade off reliability vs.\@ performance,
rather than only one.
First, we have the length of time which the writing thread attempts to
``stall'' immediately after the $x\GETS 1$ write (which is configurable in this
case because {\GCC} will perform the dead store elimination even with a loop
here).
Second, as in the other attacks, we have the number of entire redundant runs of
the attack that are performed before the attacker reaches her conclusion.
Increased reliability can be achieved by adjusting either of these knobs,
and they each have (different) effects on the overall performance of the
attack.
Although Redundancy 1 never reaches perfect accuracy (probably because of rare
catastrophic events like OS interrupts), once we move to Redundancy 2,
increasing the ``stall'' time is much more advantageous than further
redundancy.
Perfect accuracy (in our tests) can be achieved with a bandwidth of over 1.1
million bits per second at Redundancy 2, making this our most efficient and
accurate attack by a considerable margin.
In particular, this means that this attack can leak a 2048-bit cryptographic
key with perfect accuracy (in our tests) in under $2$ ms.

\section{Logic}
\label{sec:logic}

\newcommand{\pLTL}{\textsf{PLTL}}
\newcommand{\once}{\Diamond^{-1}}
\newcommand{\always}{\Box^{-1}}
\newcommand{\afo}{\phi}
\newcommand{\bfo}{\psi}
\newcommand{\mods}{\texsf{Models}}

We adapt past linear temporal logic (\pLTL)
\cite{Lichtenstein:1985:GP:648065.747612} to pomsets by dropping the previous
instant operator and adopting strict versions of the temporal operators.
The atoms of our logic are write and read events.
% \begin{displaymath}
%   \afo \QUAD::=\QUAD
%   \DR{\aLoc}{\aVal}
%   \mid
%   \DW\aLoc\aVal
%   \afo \wedge\bfo
%   \mid \neg \afo
%   \once\afo
%   \mid \always\afo
% \end{displaymath}
\begin{definition} %[Satisfaction]
  Given an pomset $\aPS$ and event $\aEv$, define:
  \begin{displaymath}
    \begin{array}{lrl}
      \aPS,\aEv &\models& \DW{\aLoc}{\aVal}, \text{ if } \labelling(\aEv) =  (\TRUE, \DW{\aLoc}{\aVal}) \\
      \aPS,\aEv &\models& \DR{\aLoc}{\aVal}, \text{ if } \labelling(\aEv) =  (\TRUE, \DR{\aLoc}{\aVal}) \\
      \aPS,\aEv &\models& \afo\land\bfo, \text{ if } \aPS,\aEv \models  \afo \text{ and } \aPS,\aEv \models  \bfo \\
      \aPS,\aEv &\models& \neg\afo, \text{ if } \aPS,\aEv \not\models \afo \\
      %\aPS,\aEv &\models& \once\afo, \text{if } (\exists \bEv \le \aEv, \bEv\not=\aEv)  \aPS,\bEv \models \afo \\
      \aPS,\aEv &\models& \always\afo, \text{ if } (\forall \bEv \le \aEv  \bEv\not=\aEv)\; \aPS,\bEv \models \afo
    \end{array} 
  \end{displaymath}
  Define $\aPS \models \afo$ if
  $(\forall \aEv \in \Event) \;\aPS,\aEv \models\afo$ and $\aPSS\models \afo$
  if $(\forall \aPS \in \aPSS)\; \aPS \models\afo$.
\end{definition}
As usual, we write
$\once\afo$ for $\neg(\always\neg\afo)$,
$\afo\lor\bfo$ for $\neg(\neg \afo \land \neg \bfo)$,
and $\afo \Rightarrow \bfo$ for $\neg \afo \lor \bfo$.

The past operators do not include the current instant, and thus 
they do \emph{not} satisfy the rule
\begin{math}
  \always\afo\Rightarrow\once\afo.
\end{math}
However, they do satisfy:
% \begin{align*}  
%   \frac{\aPS \models \afo \Rightarrow\once{\afo}}{\aPS \models \neg \afo}\text{(Coinduction)}
%   &&
%   \frac{\aPS \models \always\afo \Rightarrow\afo}{\aPS \models \afo}\text{(Induction)}
% \end{align*}
\begin{gather*}
  \tag{Coinduction}
  (\afo \Rightarrow\once{\afo}) \Rightarrow\neg \afo
  \\
  \tag{Induction}
  (\always\afo \Rightarrow\afo) \Rightarrow\afo
\end{gather*}
% \begin{description}
% \item[Coinduction.]
%   \begin{math}
%     (\afo \Rightarrow\once{\afo}) \Rightarrow\neg \afo
%   \end{math}
% \item[Induction.] 
%   \begin{math}
%     (\always\afo \Rightarrow\afo) \Rightarrow\afo
%   \end{math}
% \end{description}
Note that $\aPS \models \afo \land \always\afo$ whenever $\aPS \models \afo$.

We state a composition result in the style of Abadi and
Lamport~\cite{Abadi:1993:CS:151646.151649}.  To simplify the presentation, we
present the special case with a single invariant.
% We view the
% composition result as capturing key aspects of no-ThinAirRead, as will become
% clearer in the examples below.
In order to state the theorem, we generalize the satisfaction relation to
include environment assumptions.  Let
\begin{math}
  \mods{(\afo)} = \{ \aPSS \mid \aPSS \models \afo \}
\end{math}
be the set of pomsets that
satisfy $\afo$.
We say that $\afo$ is prefix closed if $\mods{(\afo)}$
is prefix-closed\footnote{$\aPS_1$ is a prefix of $\aPS$ if the carrier set
  of $\aPS_1$ is a downwards closed subset of $\aPS$; i.e if
  $\aEv \in \aPS_1$ and $ \bEv \le_{\aPS} \aEv$, then $\bEv$ also in
  $\aPS_1$.}.
\begin{definition}
  \begin{math}
    \afo, \aPSS \models \bfo  \text{ if } \mods{(\afo)} \parallel \aPSS \models \bfo
  \end{math}
\end{definition}
\begin{proposition}%[Composition]
  Let $\afo$ be prefix-closed.  Let $\aPSS_1, \aPSS_2$ be
  augmentation-closed\footnote{$\aPS_1$ is a augmentation of $\aPS$ if their
    carrier sets are the same and if $ \bEv \le_{\aPS} \aEv$, then
    $\bEv \le \aEv$ also in $\aPS_1$.}.  Then:
  \begin{displaymath}
    \frac{
      \afo, \aPSS_1 \models\afo
      \quad
      \afo, \aPSS_2 \models\afo
    }{\aPSS_1 \parallel \aPSS_2 \models \afo}
  \end{displaymath}
\end{proposition}
\begin{proof}[Sketch]
  We will show that all prefixes in the prefix closures of
  $\aPSS_1 \parallel \aPSS_2$ satisfy the required property.  Proof proceeds
  by induction on prefixes of $\aPS \in \aPSS_1 \parallel \aPSS_2$.

  The case for empty prefix  follows from assumption that  $\afo$ is prefix closed.  

  For the inductive case, consider $\aPS$ in the prefix closure of
  $\aPSS_1 \parallel \aPSS_2$, i.e. $\aPS = \aPS_1 \parallel \aPS_2$ where
  $\aPS_i \in \aPSS_i$.  Since $\aPSS_1$ and $\aPSS_2$ are augmentation
  closed, we can assume that the restriction of $\aPS$ to the events of
  $\aPS_i$ coincides with $\aPS_i$, for $i=1,2$.

  Consider a prefix (say $\aPS'$) got by deleting a maximal element, say
  $\aEv$, of $\aPS$.  There are two cases depending on whether $\aEv$ comes
  from $\aPS_1$ or $\aPS_2$.  In the case when $\aEv$ comes from $\aPS_1$,
  since $\aPS_2$ is a prefix of $\aPS'$ and $\aPS' \models \afo$ by induction
  hypothesis, we deduce that $\aPS_2 \models \afo$.  Thus,
  $\aPS_2 \in \mods{(\afo)}$.  Since $\aPS_1 \in \aPSS_1$, assumption
  $\afo, \aPSS_1 \models\afo$, we deduce that
  $\aPS_1 \parallel \aPS_2 \models \afo$.
\end{proof}


Closed at x:
\begin{verbatim}
  Define closed(x) = (Rxv => F(Wxv))
\end{verbatim}
Local declaration [sound proof rule]:
\begin{verbatim}
  x notin phi
  Es |= closed(x) => phi
  ----------------------
  var x; Es |= phi
\end{verbatim}
Conditional TAR example:
\begin{verbatim}
  var x,y,z;
  y:=0; y:=x  ||  x:=0; if(~z){x:=1}else{x:=y;a:=y}  ||  z:=0; z:=1
\end{verbatim}
\begin{verbatim}
Goal: Es |= ~ F(Wa1)   [impossible to write a=1]
\end{verbatim}
Invariant:
\begin{verbatim}
     F(Wy1) => F(Rx1)
  /\ F(Wa1) => F(Ry1) /\ G(Wx1 => F(Ry1))
\end{verbatim}
Closing y:
\begin{verbatim}
  F(Wa1) => F(Rx1) /\ G(Wx1 => F(Rx1))
\end{verbatim}
Closing x:
\begin{verbatim}
  F(Wa1) => F(Wx1) /\ G(Wx1 => F(Wx1))
\end{verbatim}
Using coinduction for F:  
\begin{verbatim}
  F(Wa1) => F(Wx1) /\ G(~ Wx1)
\end{verbatim}
Simplifying:  
\begin{verbatim}
  F(Wa1) => false
\end{verbatim}
\section{Conclusions and future work}

One oddity of the model is that
$\sem{r \GETS x\SEMI y \GETS r\SEMI\SKIP}$ includes:
\[\begin{tikzpicture}[node distance=1em]
  \event{rx0}{\DR{x}{0}}{}
  \nonevent{wy1}{\DW{y}{1}}{right=of rx0}
  \po{rx0}{wy1}
\end{tikzpicture}\]
where the write action guessed its value incorrectly, and therefore has
precondition $0=1$.   This form of speculative
execution does not appear to be used in practice. In order to disallow it,
one could change the semantics of $\SKIP$ to introduce a tick 
action denoting successful completion of the thread and only consider
executions in which the precondition of every tick action is satisfiable.  We
leave the elaboration of this idea as future work.

\todo{Comment on the following:}
\begin{itemize}
\item coherence = per location total order on $\not<$

\item Validation of write removal requires some tricks to ensure that thread
  does not rf its own write

\item Definition of rf can use $\not<$ in first clause, rather than $\leq$.
  We chose the stronger definition because it makes some of the examples
  simpler: in particular, the example motivating rf needs a volatile in the
  other thread if you don't have rf-implies-hb.

  Old text: The notion rf-pomset is sufficient to capture hardware models and
  release/acquire access in C++, where reads-from implies happens-before
  \cite{alglave}.  To model C++ relaxed access, it would be necessary to use
  a more general notion of rf-pomset, where $(\bEv,\aLoc,\aEv) \in \RF$ does
  not necessarily imply $\bEv < \aEv$, instead requiring that
  $(\mathord< \cup \mathord\RF)$ be acyclic.

\item The design space for transactions is very rich
  \cite{DBLP:journals/pacmpl/DongolJR18}.  We have only presented one option.
  we pun between abort and false commit
  
\end{itemize}



% \todo{Integrate this example on information flow?}

% Let {\tt P(hi)} be a function with domain $\{0, . . . ,n \}$  and codomain $\{1, . . . ,n \}$. Our aim is to write a
% program that tests whether there is information flow from input {\tt  hi} to the return value {\tt lo}. Consider:
% \begin{alltt}
%  s:=y; if (s!=0) { x=P(s); z:=1; } else { x=P(0); }
% \end{alltt}
% When there is a dependency of the output on the input, there are at least two different possible
% assignments to {\tt z}. In this case, the only possible execution, as validated by our model, of the above
% program is the SC execution that reads {\tt m} as $0$ and leaves {\tt noflow} untouched.
% When there is no dependency from input to output the value of {\tt z} is the same, say {\tt k} in all cases,
% non-zero by assumption. In this case, the compiler can hoist the assignment of  to  {\tt z} outside the
% conditional and swap with the independent asignment to {\tt hi}, 
% thus rewriting T2 to:
% \begin{alltt}
% T2: z=k; s=m; if (s!=0) { noflow=true; }
% \end{alltt}
% In this case, the program has an execution, validated by our model, that sets {\tt noflow} to {\tt true}.

% In the parlance
% of information flow, the humble conditional suffices to construct a composition operator to detect information flow  in the presence of speculation.

\bibliography{bib}

\clearpage
\appendix
\subsection{Blockers}
\label{sec:blockers}

Recall the preliminary definition of reads-from in \S\ref{sec:pomsets}, which
defined an $\aLoc$-blocker to be and event $\cEv$ that writes to $\aLoc$ such that
$\bEv < \cEv < \aEv$.  Were we to adopt this definition, then concurrent
threads could turn events that were not $\aLoc$-blockers into an
$\aLoc$-blocker, even if the new thread does not mention $\aLoc$.

To see this, consider the program
\begin{math}
  (
  \aLoc\GETS1\SEMI
  \bLoc\GETS\aLoc
  \PAR
  \aLoc\GETS\cLoc+1\SEMI
  \bLoc\GETS\aLoc
  \PAR
  \IF(z=2)\THEN\aReg\GETS\aLoc\FI
  )
\end{math}
with execution:
\[\begin{tikzpicture}[node distance=1em]
  \event{wx1}{\DW{\aLoc}{1}}{}
  \event{rz1}{\DR{\cLoc}{1}}{right=of wx1}
  \event{wx2}{\DW{\aLoc}{2}}{right=of rz1}
  \event{rz2}{\DR{\cLoc}{2}}{right=of wx2}
  \event{rx1}{\DR{\aLoc}{1}}{right=of rz2}
  \event{rx1a}{\DR{\aLoc}{1}}{below=of wx1}
  \event{wy1}{\DW{\bLoc}{1}}{below=of rx1a}
  \event{rx2a}{\DR{\aLoc}{2}}{below=of wx2}
  \event{wy2}{\DW{\bLoc}{2}}{below=of rx2a}
  \rf{wx1}{rx1a}
  \po{rx1a}{wy1}
  \rf{wx2}{rx2a}
  \po{rx2a}{wy2}
  \po{rz1}{wx2}
  \po{rz2}{rx1}
  \rf[out=20,in=160]{wx1}{rx1}
\end{tikzpicture}\]
and the program
\begin{math}
  (
  \cLoc\GETS\bLoc\SEMI
  \cLoc\GETS\bLoc
  )
\end{math}
with execution:
\[\begin{tikzpicture}[node distance=1em]
  \event{ry1}{\DR{\bLoc}{1}}{}
  \event{wz1}{\DW{\cLoc}{1}}{right=of ry1}
  \event{ry2}{\DR{\bLoc}{2}}{right=of wz1}
  \event{wz2}{\DW{\cLoc}{2}}{right=of ry2}
  \po{ry1}{wz1}
  \po{ry2}{wz2}
\end{tikzpicture}\]
If these are placed in parallel, then a possible execution is:
\[\begin{tikzpicture}[node distance=1em]
  \event{wx1}{\DW{\aLoc}{1}}{}
  \event{rz1}{\DR{\cLoc}{1}}{right=of wx1}
  \event{wx2}{\DW{\aLoc}{2}}{right=of rz1}
  \event{rz2}{\DR{\cLoc}{2}}{right=of wx2}
  \event{rx1}{\DR{\aLoc}{1}}{right=of rz2}
  \event{rx1a}{\DR{\aLoc}{1}}{below=of wx1}
  \event{wy1}{\DW{\bLoc}{1}}{below=of rx1a}
  \event{rx2a}{\DR{\aLoc}{2}}{below=of wx2}
  \event{wy2}{\DW{\bLoc}{2}}{below=of rx2a}
  \rf{wx1}{rx1a}
  \po{rx1a}{wy1}
  \rf{wx2}{rx2a}
  \po{rx2a}{wy2}
  \po{rz1}{wx2}
  \po{rz2}{rx1}
  \event{ry1}{\DR{\bLoc}{1}}{below=of wy1}
  \event{wz1}{\DW{\cLoc}{1}}{right=of ry1}
  \event{ry2}{\DR{\bLoc}{2}}{below=of wy2}
  \event{wz2}{\DW{\cLoc}{2}}{right=of ry2}
  \po{ry1}{wz1}
  \po{ry2}{wz2}
  \rf{wy1}{ry1}
  \rf{wz1}{rz1}
  \rf{wy2}{ry2}
  \rf{wz2}{rz2}
\end{tikzpicture}\]
and now the $(\DW{\aLoc}{2})$ event is an $\aLoc$-blocker,
so $(\DR{\aLoc}{1})$ cannot
read from $(\DW{\aLoc}{1})$.

In the final definition of reads-from in \S\ref{sec:pomsets} we
ruled out $\aLoc$-blockers by requring that any
event $\cEv$ that writes to $\aLoc$ has
either $\bEv \ltN \cEv$ or $\cEv \ltN \aEv$.
With this definition, in order for $(\DR{\aLoc}{1})$ to read from
$(\DW{\aLoc}{1})$, we either need $(\DW{\aLoc}{1}) \ltN (\DW{\aLoc}{2})$
or $(\DW{\aLoc}{2}) \ltN (\DR{\aLoc}{1})$, for example:
\[\begin{tikzpicture}[node distance=1em]
  \event{wx1}{\DW{\aLoc}{1}}{}
  \event{rz1}{\DR{\cLoc}{1}}{right=of wx1}
  \event{wx2}{\DW{\aLoc}{2}}{right=of rz1}
  \event{rz2}{\DR{\cLoc}{2}}{right=of wx2}
  \event{rx1}{\DR{\aLoc}{1}}{right=of rz2}
  \event{rx1a}{\DR{\aLoc}{1}}{below=of wx1}
  \event{wy1}{\DW{\bLoc}{1}}{below=of rx1a}
  \event{rx2a}{\DR{\aLoc}{2}}{below=of wx2}
  \event{wy2}{\DW{\bLoc}{2}}{below=of rx2a}
  \rf{wx1}{rx1a}
  \po{rx1a}{wy1}
  \rf{wx2}{rx2a}
  \po{rx2a}{wy2}
  \po{rz1}{wx2}
  \po{rz2}{rx1}
  \rf[out=20,in=160]{wx1}{rx1}
  \wk[out=-150,in=-30]{rx1}{wx2}
  \wk{wy1}{wy2}
\end{tikzpicture}\]
then putting this in parallel as before results in:
\[\begin{tikzpicture}[node distance=1em]
  \event{wx1}{\DW{\aLoc}{1}}{}
  \event{rz1}{\DR{\cLoc}{1}}{right=of wx1}
  \event{wx2}{\DW{\aLoc}{2}}{right=of rz1}
  \event{rz2}{\DR{\cLoc}{2}}{right=of wx2}
  \event{rx1}{\DR{\aLoc}{1}}{right=of rz2}
  \event{rx1a}{\DR{\aLoc}{1}}{below=of wx1}
  \event{wy1}{\DW{\bLoc}{1}}{below=of rx1a}
  \event{rx2a}{\DR{\aLoc}{2}}{below=of wx2}
  \event{wy2}{\DW{\bLoc}{2}}{below=of rx2a}
  \rf{wx1}{rx1a}
  \po{rx1a}{wy1}
  \rf{wx2}{rx2a}
  \po{rx2a}{wy2}
  \po{rz1}{wx2}
  \po{rz2}{rx1}
  \rf[out=20,in=160]{wx1}{rx1}
  \wk[out=-150,in=-30]{rx1}{wx2}
  \wk{wy1}{wy2}
  \event{ry1}{\DR{\bLoc}{1}}{below=of wy1}
  \event{wz1}{\DW{\cLoc}{1}}{right=of ry1}
  \event{ry2}{\DR{\bLoc}{2}}{below=of wy2}
  \event{wz2}{\DW{\cLoc}{2}}{right=of ry2}
  \po{ry1}{wz1}
  \po{ry2}{wz2}
  \rf{wy1}{ry1}
  \rf{wz1}{rz1}
  \rf{wy2}{ry2}
  \rf{wz2}{rz2}
  \wk[out=30,in=150]{wz1}{wz2}
\end{tikzpicture}\]
but this is \emph{not} a valid 3-valued pomset,
since $(\DW{\aLoc}{2}) < (\DR{\aLoc}{1})$ but also $(\DW{\aLoc}{2}) \ltN (\DR{\aLoc}{1})$,
which is a contradiction.


\section{Causality test cases}
\label{sec:tc}

\citet{PughWebsite} developed a set of twenty {causality test cases} in the
process of revising the Java Memory Model (JMM)
\cite{Manson:2005:JMM:1047659.1040336}.  Using hand calculation, we have
confirmed that our model gives the desired result for all twenty cases,
unrolling loops as necessary.  Our model also gives the desired results for
all of the examples in \citet[\textsection 4]{DBLP:conf/esop/BattyMNPS15} and
all but one in \citet[\textsection 5.3]{SevcikThesis}:
redundant-write-after-read-elimination fails for any
sensible non-coherent semantics.  Our model agrees with the JMM on the
``surprising and controversial behaviors'' of \citet[\textsection
8]{Manson:2005:JMM:1047659.1040336}, and thus fails to validate thread
inlining.

In this section, we discuss three of the causality test cases and the thread
inlining from \cite{Manson:2005:JMM:1047659.1040336}.  In presenting the
examples, we unroll loops, correct typos and simplify the code.  

\subsection{Causality test case 8}

Test case 8 asks whether:
\begin{displaymath}
  \VAR x\GETS 0\SEMI
  \VAR y\GETS 0\SEMI
  (\IF(x<2)\THEN y\GETS 1\FI 
  \PAR
  x\GETS y)
\end{displaymath}
may read $1$ for both $x$ and $y$.  This behavior is allowed, since
``interthread analysis could determine that $x$ and $y$ are always either $0$
or $1$.''  This breaks the dependency between the read of $x$ and the write
to $y$ in the first thread, allowing the write to be moved earlier.

The semantics of TC8 includes
\[\begin{tikzpicture}[node distance=1em]
  \event{ix}{\DW{x}{0}}{}
  \event{iy}{\DW{y}{0}}{right=of ix}
  \event{rx1}{\DR{x}{1}}{right=2.5 em of iy}
  \event{wy1}{\DW{y}{1}}{right=of rx1}
  \event{ry1}{\DR{y}{1}}{right=2.5em of wy0}
  \event{wx1}{\DW{x}{1}}{right=of ry0}
  \po{ry1}{wx1}
  \po[out=30,in=150]{ix}{rx1}
  \rf[in=-25,out=-160]{wx1}{rx1}
  \rf[out=20,in=160]{wy1}{ry1}
  \wk[out=-25,in=-150]{ix}{wx1}
  \wk[out=25,in=155]{iy}{wy1}
\end{tikzpicture}\]
Where we require $(\DW{x}{0})<(\DR{x}{1})$ but not $(\DR{x}{1})<(\DW{y}{1})$.
To see why this execution exists, consider the left thread with syntax sugar
removed:
\begin{displaymath}
  r\GETS x\SEMI \IF(r<2)\THEN y\GETS 1\FI
\end{displaymath}
\begin{math}
  \sem{\IF(r<2)\THEN y\GETS 1\FI}
\end{math}
includes
\begin{math}
  (r<2\mid\DW{y}{1}).
\end{math}
% \[\begin{tikzpicture}[node distance=1em]
%   \event{wy1}{r<2\mid\DW{y}{1}}{}
% \end{tikzpicture}\]
Thus, by Definition~\ref{def:programs}, 
\begin{math}
  \sem{r\GETS x\SEMI \IF(r<2)\THEN y\GETS 1\FI}
\end{math}
includes
\begin{math}
  (\DR{x}{1}) \prefix (r<2\mid\DW{y}{1})[x/r]
\end{math}
which simplifies to
\begin{math}
  (\DR{x}{1}) \prefix (x<2\mid\DW{y}{1}),
\end{math}
which, by Definition~\ref{def:prefix}, includes:
\[\begin{tikzpicture}[node distance=1em,baselinecenter]
    \event{rx1}{\DR{x}{1}}{}
    \event{wy1}{x<2\mid\DW{y}{1}}{right=of rx1}
  \end{tikzpicture}\]
Here we have used the \textsc{[non-ordering read]} clause of Definition~\ref{def:prefix}:
``$\bForm'$ implies $\bForm[\aVal/\aLoc] \land \bForm$, if $\aAct$ reads $\aVal$ from $\aLoc$,''
where $a=(\DR{x}{1})$,  $\bForm=\bForm'=(x<2)$.  We can use this case since
$x<2$ implies $1<2\land x<2$.

Prefixing with $(\DW{x}{0})$ allows us to discharge the assumption $x<2$,
arriving at:
\[\begin{tikzpicture}[node distance=1em,baselinecenter]
    \event{ix}{\DW{x}{0}}{}
    \event{rx1}{\DR{x}{1}}{right=2.5 em of ix}
    \event{wy1}{\DW{y}{1}}{right=of rx1}
    \po{ix}{rx1}
  \end{tikzpicture}\]
Here we have used the \textsc{[ordering read]}
clause of \ref{def:prefix}:
``$\bForm'$ implies $\bForm[\aVal/\aLoc]$, if $\aAct$ reads $\aVal$ from $\aLoc$ and $\cEv<'\aEv$,''
where $a=(\DW{x}{0})$,  $\bForm=(x<2)$ and $\bForm'=\TRUE$.  As long as
require
\begin{math}
  (\DW{x}{0})<
  (\DR{x}{1}),
\end{math}
we can use this case since $\TRUE$ implies $0<2$.

\subsection{Causality test case 9}

Test case 9 asks whether:
\begin{displaymath}
  \VAR x\GETS 0\SEMI
  \VAR y\GETS 0\SEMI
  (\IF(x<2)\THEN y\GETS 1\FI 
  \PAR
  x\GETS y
  \PAR
  y\GETS 2\SEMI)
\end{displaymath}
may read $1$ for both $x$ and $y$.  This behavior is also allowed.  This is
``similar to test case $8$, except that $x$ is not always $0$ or
$1$. However, a compiler might determine that the read of $x$ by thread $1$
will never see the write by thread $3$ (perhaps because thread $3$ will be
scheduled after thread $1$)''

Reasoning as for test case 8, the semantics of test case 9 includes:
\[\begin{tikzpicture}[node distance=1em]
  \event{ix}{\DW{x}{0}}{}
  \event{iy}{\DW{y}{0}}{right=of ix}
  \event{rx1}{\DR{x}{1}}{right=2.5 em of iy}
  \event{wy1}{\DW{y}{1}}{right=of rx1}
  \event{ry1}{\DR{y}{1}}{right=2.5em of wy0}
  \event{wx1}{\DW{x}{1}}{right=of ry0}
  \event{wx2}{\DW{x}{2}}{right=2.5em of wx1}
  \po{ry1}{wx1}
  \po[out=30,in=150]{ix}{rx1}
  \rf[in=-25,out=-160]{wx1}{rx1}
  \rf[out=20,in=160]{wy1}{ry1}
  \wk[out=-25,in=-150]{ix}{wx1}
  \wk[out=25,in=155]{iy}{wy1}
  \wk[out=-25,in=-150]{ix}{wx2}
\end{tikzpicture}\]

Thus, with respect to the introduction of new threads, our model appears to
be more robust than the event structures semantics of
\cite{DBLP:conf/lics/JeffreyR16}, which fails on this test case.

\subsection{Causality test case 14}

Test case 14 asks whether:
\begin{displaymath}
  \VAR a\GETS 0\SEMI
  \VAR b\GETS 0\SEMI
  \VAR y\GETS 0\SEMI
  (\IF(a)\THEN b\GETS 1\ELSE y\GETS 1\FI 
  \PAR
  \WHILE(y+b==0) \THEN\SKIP\FI\; a\GETS1)
\end{displaymath}
may read $1$ for $a$ and $b$, yet $0$ for $y$.  Here $a$ and $b$ are regular
variables and $y$ is volatile, which is equivalent to release/acquire in this
example.  This behavior is also disallowed, since ``in all sequentially
consistent executions, [the read of $a$ gets $0$] and the program is
correctly synchronized. Since the program is correctly synchronized in all SC
executions, no non-SC behaviors are allowed.''

Unrolling the loop once, we have:
\begin{displaymath}
  \VAR a\GETS 0\SEMI
  \VAR b\GETS 0\SEMI
  \VAR y\GETS 0\SEMI
  (\IF(a)\THEN b\GETS 1\ELSE y\GETS 1\FI 
  \PAR
  \IF(y\lor b)\THEN a\GETS 1\FI)
\end{displaymath}
We argue that any execution with $(\DR{a}{1})$, $(\DR{b}{1})$, and
$(\DR{y}{0})$ must be cyclic.  The closure requirements require that
\begin{math}
  (\DW{a}{1})<(\DR{a}{1})
  \;\text{and}\;
  (\DR{b}{1})<(\DR{b}{1}).
\end{math}
Ignoring initialization, least ordered execution that includes all of these
actions is:
\[\begin{tikzpicture}[node distance=1em]
  \event{ra1}{\DR{a}{1}}{}
  \event{wb1}{\DW{b}{1}}{below=of ra1}
  \nonevent{wy1}{\DW{y}{1}}{left=of wb1}
  \event{rb1}{\DR{b}{1}}{right=4.5em of ra1}
  \event{ry0}{\DR{y}{0}}{right=of rb1}
  \event{wa1}{\DW{a}{1}}{below=of rb1}
  \po{ra1}{wb1}
  \po{rb1}{wa1}
  \rf{wa1}{ra1}
  \rf{wb1}{rb1}
\end{tikzpicture}\]
where the read of $a$ is ordering for $(\DW{b}{1})$ but
not $(\DW{y}{1})$, and the read of $b$ is ordering for $(\DW{a}{1})$ but the
read of $y$ is not.  $(\DW{y}{1})$ is crossed out, since its
precondition must imply $(\lnot a)[1/a]$, which is equivalent to $\FALSE$.
To avoid order from $(\DR{y}{0})$ to $(\DW{a}{1})$, we
have strengthened the predicate on $(\DW{a}{1})$ from $(y\lor b)$ to
$(y=0\land b=1)$.  Note that we cannot use this trick symmetrically to remove
the order from $(\DR{b}{1})$ to $(\DW{a}{1})$, since $b=1$ does not follow
from the initialization of $b$.


\subsection{Thread inlining}

One property one could ask of a model of shared memory is thread
inlining: any execution of $\sem{P\SEMI Q}$ is an execution of $\sem{P
  \PAR Q}$. This is \emph{not} a goal of our model, and indeed is not
satisfied, due to the different semantics of concurrent and sequential
memory accesses. We demonstrate this by considering an example from
the Java Memory Model~\cite{Manson:2005:JMM:1047659.1040336}, which shows that Java does not
satisfy thread inlining either.

The lack of thread inlining is related to the different dependency
relations introduced by sequential and concurrent access.
Recall from \S\ref{sec:sequential-memory} that the program
\verb`(x := 0; y := x+1;)` has execution:
\[\begin{tikzpicture}[node distance=1em]
  \event{wx0}{\DW{x}{0}}{}
  \event{wy1}{\DW{y}{1}}{right=of wx0}
\end{tikzpicture}\]
but that \verb`(x := 1; || y := x+1;)` has:
\[\begin{tikzpicture}[node distance=1em]
  \event{wx1}{\DW{x}{1}}{}
  \event{rx1}{\DR{x}{1}}{right=2.5em of wx1}
  \event{wy2}{\DW{y}{2}}{right=of rx1}
  \rf{wx1}{rx1}
  \po{rx1}{wy2}
\end{tikzpicture}\]
That is, in the sequential case there is no dependency from the
write of $x$ to the write of $y$, but in the concurrent case there
is such a dependency.

This can be used to construct a counter-example to thread inlining, based on~\cite[Ex~11]{Manson:2005:JMM:1047659.1040336}:
\begin{verbatim}
  x := 0; if (x == 1) { z := 1; } else { x := 1; } || y := x; || x := y;
\end{verbatim}
This has no execution containing $(\DW z1)$. Any attempt to build such an execution
results in a cycle:
\[\begin{tikzpicture}[node distance=1em]
  \event{rx1a}{\DR{x}{1}}{}
  \event{wz1}{\DW{z}{1}}{right=of rx1a}
  \nonevent{wx1a}{\DW{x}{1}}{right=of wz1}
  \event{rx1b}{\DR{x}{1}}{right=2.5em of wx1a}
  \event{wy1}{\DW{y}{1}}{right=of rx1b}
  \event{ry1}{\DR{y}{1}}{right=2.5em of wy1}
  \event{wx1b}{\DW{x}{1}}{right=of ry1}
  \po{rx1a}{wz1}
  \po[out=25, in=150]{rx1a}{wx1a}
  \po{rx1b}{wy1}
  \po{ry1}{wx1b}
  \rf{wy1}{ry1}
  \rf[out=160, in=30]{wx1b}{rx1a}
  \rf[out=160, in=30]{wx1b}{rx1b}
\end{tikzpicture}\]
Inlining the thread \verb|(y := x)| gives~\cite[Ex~12]{Manson:2005:JMM:1047659.1040336}:
\begin{verbatim}
  x := 0; if (x == 1) { z := 1; } else { x := 1; } y := x; || x := y;
\end{verbatim}
with execution:
\[\begin{tikzpicture}[node distance=1em]
  \event{rx1a}{\DR{x}{1}}{}
  \event{wz1}{\DW{z}{1}}{right=of rx1a}
  \nonevent{wx1a}{\DW{x}{1}}{right=of wz1}
  \event{wy1}{\DW{y}{1}}{right=of wx1a}
  \event{ry1}{\DR{y}{1}}{right=2.5em of wy1}
  \event{wx1b}{\DW{x}{1}}{right=of ry1}
  \po{rx1a}{wz1}
  \po[out=25, in=150]{rx1a}{wx1a}
  \po{ry1}{wx1b}
  \rf{wy1}{ry1}
  \rf[out=160, in=30]{wx1b}{rx1a}
\end{tikzpicture}\]
To see why this execution exists, consider the program fragment:
\begin{verbatim}
  if (x == 1) { z := 1; } else { x := 1; } y := x;
\end{verbatim}
Removing the syntax sugar, this is:
\begin{verbatim}
  r1 := x; if (r1 == 1) {
    z := 1; r2 := x; y := r2; skip
  } else {
    x := 1; r3 := x; y := r3; skip
  }
\end{verbatim}
Now, $\sem{z := 1\SEMI r_2 := x\SEMI y := r_2\SEMI \SKIP}$
includes pomset:
\[\begin{tikzpicture}[node distance=1em]
  \event{wz1}{r_1=1 \mid \DW{z}{1}}{}
  \event{wy1}{r_1=x=1 \mid \DW{y}{1}}{right=of wz1}
\end{tikzpicture}\]
and $\sem{x := 1\SEMI r_3 := x\SEMI y := r_3\SEMI \SKIP}$
includes pomset:
\[\begin{tikzpicture}[node distance=1em]
  \event{wx1a}{r_1\neq 1 \mid \DW{x}{1}}{}
  \event{wy1}{r_1\neq 1 \mid \DW{y}{1}}{right=of wx1a}
\end{tikzpicture}\]
so  $\sem{\IF (r_1 = 1) \THEN z := 1\SEMI r_2 := x\SEMI y := r_2\SEMI \SKIP \ELSE x := 1\SEMI r_3 := x\SEMI y := r_3\SEMI \SKIP \FI}$ includes:
\[\begin{tikzpicture}[node distance=1em]
  \event{wz1}{r_1=1 \mid \DW{z}{1}}{}
  \event{wx1a}{r_1\neq1 \mid \DW{x}{1}}{right=of wz1}
  \event{wy1}{(r_1=x=1) \lor (r_1\neq1) \mid \DW{y}{1}}{right=of wx1a}
\end{tikzpicture}\]
which means $\sem{\IF (r_1 = 1) \THEN z := 1\SEMI r_2 := x\SEMI y := r_2\SEMI \SKIP \ELSE x := 1\SEMI r_3 := x\SEMI y := r_3\SEMI \SKIP \FI}[x/r_1]$ includes:
\[\begin{tikzpicture}[node distance=1em]
  \event{wz1}{x=1 \mid \DW{z}{1}}{}
  \event{wx1a}{x\neq1 \mid \DW{x}{1}}{right=of wz1}
  \event{wy1}{(x=x=1) \lor (x\neq1)) \mid \DW{y}{1}}{right=of wx1a}
\end{tikzpicture}\]
Now $(x=x=1) \lor (x\neq1)$ is a tautology, so this is just:
\[\begin{tikzpicture}[node distance=1em]
  \event{wz1}{x=1 \mid \DW{z}{1}}{}
  \event{wx1a}{x\neq1 \mid \DW{x}{1}}{right=of wz1}
  \event{wy1}{\DW{y}{1}}{right=of wx1a}
\end{tikzpicture}\]
and so $\sem{r_1 \GETS x\SEMI \IF (r_1 = 1) \THEN z := 1\SEMI r_2 := x\SEMI y := r_2\SEMI \SKIP \ELSE x := 1\SEMI r_3 := x\SEMI y := r_3\SEMI \SKIP \FI}$ includes:
\[\begin{tikzpicture}[node distance=1em]
  \event{rx1a}{\DR{x}{1}}{}
  \event{wz1}{1=1 \mid \DW{z}{1}}{right=of rx1a}
  \event{wx1a}{1\neq1 \mid \DW{x}{1}}{right=of wz1}
  \event{wy1}{\DW{y}{1}}{right=of wx1a}
  \po{rx1a}{wz1}
  \po[out=25, in=150]{rx1a}{wx1a}
\end{tikzpicture}\]
which simplifies to:
\[\begin{tikzpicture}[node distance=1em]
  \event{rx1a}{\DR{x}{1}}{}
  \event{wz1}{\DW{z}{1}}{right=of rx1a}
  \nonevent{wx1a}{\DW{x}{1}}{right=of wz1}
  \event{wy1}{\DW{y}{1}}{right=of wx1a}
  \po{rx1a}{wz1}
  \po[out=25, in=150]{rx1a}{wx1a}
\end{tikzpicture}\]
as required. The rest of the example is straightforward, and shows that our semantics
agrees with the JMM in not supporting thread inlining.



% \subsection{Word tearing}

% \todo{Remove this section, since it's not needed for transactions?}

% In \S\ref{sec:transactions}, we shall be considering transactional memory,
% and in \S\ref{sec:transactions} show that we can model a simplified version
% of an information flow attack on transactions. In order to model transactions,
% we need to consider actions that can write many memory locations at once,
% since this is part of the semantics of commitment. To lead up to this, we first
% consider a simpler scenario of many-location writes and reads, which is word
% tearing.

% In word tearing, a program contains a write instruction with data larger
% than the hardware word size, for example copying a byte array, or assigning
% a 64-bit float on a 32-bit architecture. For example, consider the program:
% \begin{verbatim}
%   (x := [0, 0];) || (x := [1, 1];) || (r := x;)
% \end{verbatim}
% This has executions in which the read of $x$ only reads from one of the writes,
% for example:
% \[\begin{tikzpicture}[node distance=1em]
%   \event{wx00}{\DW{x}{[0,0]}}{}
%   \event{wx11}{\DW{x}{[1,1]}}{right=2.5em of wx00}
%   \event{rx00}{\DR{x}{[0,0]}}{right=2.5em of wx11}
%   \rf[out=20, in=160]{wx00}{rx00}
% \end{tikzpicture}\]
% but also has executions in which the read of $x$ reads from both writes,
% for example:
g% \[\begin{tikzpicture}[node distance=1em]
%   \event{wx00}{\DW{x}{[0,0]}}{}
%   \event{wx11}{\DW{x}{[1,1]}}{right=2.5em of wx00}
%   \event{rx01}{\DR{x}{[0,1]}}{right=2.5em of wx11}
%   \rfx[out=20, in=160]{wx00}{x[0]}{rx01}
%   \rfx[out=-20, in=-160]{wx11}{x[1]}{rx01}
% \end{tikzpicture}\]
% Word tearing can occur, for example, in Java extended floating point~\cite{jmm},
% LLVM 64-bit instructions on 32-bit hardware~\cite{llvm}, or in
% JavaScript SharedArrayBuffers~\cite{js-sab}.

% \newcommand{\rfControl}[4][]{\draw[rf,#1](#2) .. controls (#3) .. (#4);}
% \[\begin{tikzpicture}[node distance=1em]
%   \event{wx0}{\DW{x}{0}}{}
%   \event{wx1}{\DW{x}{1}}{right=of wx0}
%   \event{wy0}{\DW{y}{0}}{right=2.5em of wx1}
%   \event{wy1}{\DW{y}{1}}{right=of wy0}
%   \event{rx1}{\DR{x}{1}}{right=2.5 em of wy1}
%   \event{ry0}{\DR{y}{0}}{right=of rx1}
%   \event{ry1}{\DR{y}{1}}{right=2.5 em of ry0}
%   \event{rx0}{\DR{x}{0}}{right=of ry1}
%   \rf[out=20,in=160]{wx1}{rx1}
%   \rf[out=20,in=160]{wy0}{ry0}
%   \rf[out=340,in=200]{wy1}{ry1}
%   \coordinate (a) [below=of wy1];
%   \rfControl[out=340,in=200]{wx0}{a}{rx0}
%   \wk{wx0}{wx1}
%   \wk{wy0}{wy1}
%   \po{rx1}{ry0}
%   \po{ry1}{rx0}
% \end{tikzpicture}\]


% Batty section 4:
% \cite[\S4]{DBLP:conf/esop/BattyMNPS15},
% Example LB+ctrldata+ctrl-double (language must allow)
% r1=loadrlx(x) //reads 42
% if (r1 == 42)
%   storerlx(y,r1)

% r2=loadrlx(y) //reads 42
% if (r2 == 42)
%   storerlx (x,42)
% else
% storerlx (x,42)

% a:RRLX x=42 sb,dd,cd
% c:RRLX y=42 sb,cd
%   This is forbidden on hardware if compiled naively, as the architectures respect read-to-write control dependencies, but in practice compilers will collapse con- ditionals like that of the second thread, removing the control dependencies from the read of y to the writes of x and making the code identical to the previous example. As that example is allowed and observable on hardware (and we pre- sume that it would be impractical to outlaw such optimisation for C or C++), the language must also allow this execution. But this execution has a cycle in the union of reads-from and dependency, so we cannot simply exclude all those.
% Then one might hope for some other adaptation of the C/C++11 model, but the following example shows at least that there is no per-candidate-execution solution.
% Example LB+ctrldata+ctrl-single (language can and should forbid)
% r1=loadrlx(x) //reads 42 if (r1 == 42)
% storerlx (y,r1) r2=loadrlx (y) //reads 42 if (r2 == 42)
% a:RRLX x=42 sb,dd,cd
% rf
% b:WRLX y=42
% c:RRLX y=42 sb,cd
% rf
% d:WRLX x=42
% rf rf
% b:WRLX y=42 d:WRLX x=42
%   storerlx (x,42)


\end{document}
