\section{Examples}

\subsection{Spectre}

We give a simplified model of Spectre attacks, ignoring the details of
timing.  In this model, we extend programs with the ability to tell
whether a memory location has been touched (in practice this is
implemented using timing attacks on the cache). For example,
we can write a \verb|SPECTRE| program as:
\begin{verbatim}
    var a;
    if (isCapability(0)) { a[SECRET] := 1; }
    else if (touched a[0]) { x := 0; }
    else if (touched a[1]) { x := 1; }
\end{verbatim}
This is a low-security program, which is attempting to discover the
value of a high-security variable \verb|SECRET|. The low-security program
is allowed to attempt to escalate its privileges by providing a capability
which demonstrates that they are entitled to run high-security code:
\begin{verbatim}
    if (isCapability(c)) { ... escalated code ... }
    else { ... fallback code ... }
\end{verbatim}
In this case, the \verb|isCapability(0)| is false, so the fallback code
is executed. Unfortunately, the escalated code is speculatively
evaluated, which allows information to leak by testing for which
memory locations have been touched.

We model the \verb|touched| test by introducing a new read action
$(\DT{\aLoc})$ and defining:
\begin{eqnarray*}
  \sem{\IF \TOUCHED\aLoc\THEN C \ELSE D} & = & ((\DT\aLoc) \prefix \sem{C}) \sqcup \sem{D}
\end{eqnarray*}
The additional requirement we need to add for $\aLoc$-closure is:
\begin{itemize}
\item if $\labelling(\aEv)=(\aForm \mid \DT{\aLoc})$
  then there is $\bEv\not>\aEv$ with $\labelling(\bEv)=(\bForm \mid \DR{\aLoc}{\aVal})$
  or $\labelling(\bEv)=(\bForm \mid \DW{\aLoc}{\aVal})$.
\end{itemize}
For example, one execution of \verb|SPECTRE| is:
\[\begin{tikzpicture}[node distance=1em]
  \nonevent{rs}{\DR{\SEC}{1}}{}
  \nonevent{wa}{\DW{a[1]}{1}}{right=of rs}
  \event{ta}{\DT{a[1]}}{right=of wa}
  \event{wx}{\DW{x}{1}}{right=of ta}
  \po{rs}{wa}
  \po{ta}{wx}
\end{tikzpicture}\]
Putting this in parallel with a high-security write to \verb|SECRET| gives:
\[\begin{tikzpicture}[node distance=1em]
  \event{ws}{\DW{\SEC}{1}}{}
  \nonevent{rs}{\DR{\SEC}{1}}{right=2.5em of ws}
  \nonevent{wa}{\DW{a[1]}{1}}{right=of rs}
  \event{ta}{\DT{a[1]}}{right=of wa}
  \event{wx}{\DW{x}{1}}{right=of ta}
  \rf{ws}{rs}
  \po{rs}{wa}
  \po{ta}{wx}
\end{tikzpicture}\]
but due the requirement of \verb|a|-closure we do \emph{not} have:
\[\begin{tikzpicture}[node distance=1em]
  \event{ws}{\DW{\SEC}{0}}{}
  \nonevent{rs}{\DR{\SEC}{0}}{right=2.5em of ws}
  \nonevent{wa}{\DW{a[1]}{0}}{right=of rs}
  \event{ta}{\DT{a[1]}}{right=of wa}
  \event{wx}{\DW{x}{1}}{right=of ta}
  \rf{ws}{rs}
  \po{rs}{wa}
  \po{ta}{wx}
\end{tikzpicture}\]
Thus, the attacker has managed to leak the value of a high-security
location to a low-security one.

This shows how a (very abstract, untimed) model of Spectre attacks
using speculative evaluation can be modelled.
