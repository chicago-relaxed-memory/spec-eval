\section{Examples}
\label{sec:examples}

In this section, we show how four different information flow attacks can be
modeled. We cover Spectre in \S\ref{sec:spectre}, new attacks
on compiler optimizations in \S\ref{sec:info-flow-attack}--\ref{sec:dse},
and attacks on transactions in \S\ref{sec:transactions}.

\subsection{Spectre}
\label{sec:spectre}

We give a simplified model of Spectre attacks, ignoring the details of
cache timing.  In this model, we extend programs with the ability to tell
whether a memory location has been touched (in practice this is
implemented using timing attacks on the cache). For example,
we can model Spectre by:
\[\begin{array}{l}
  \VAR a\SEMI \IF(\CANREAD(\SEC))\THEN a[\SEC]\GETS1
  \brELIF(\TOUCHED a[0])\THEN x\GETS0
  \brELIF(\TOUCHED a[1])\THEN x\GETS1 \FI
\end{array}\]
This is a low-security program, which is attempting to discover the
value of a high-security variable $\SEC$. The low-security program
is allowed to attempt to escalate its privileges by checking that it is
allowed to read a high-security variable:
\[\begin{array}{l}
  \IF(\CANREAD(\SEC))\THEN \mbox{code allowed to read $\SEC$}
  \brELSE \mbox{code not allowed to read $\SEC$} \FI
\end{array}\]
In this case, $\CANREAD(\SEC)$ is false, so the fallback code
is executed. Unfortunately, the escalated code is speculatively
evaluated, which allows information to leak by testing for which
memory locations have been touched.

Attacks may realize the abstract notions in various ways.  For example, in
variant 1 of Spectre, the dynamic security check is implemented as an array
bounds check.

We model the $\TOUCHED$ test by introducing a new action
$(\DT{\aLoc})$, and defining:
\[\begin{array}{l}
  \sem{\IF (\TOUCHED\aLoc) \THEN \aCmd \ELSE \bCmd \FI} \\[\jot]\quad =  ((\DT\aLoc) \prefix \sem{\aCmd}) \cup \sem{\bCmd}
\end{array}\]
Implementations of $\TOUCHED$ use cache timing, but their success can be modeled
without needing to be precise about such microarchitectural details:
\begin{itemize}
\item if $\labelling(\aEv)=(\aForm \mid \DT{\aLoc})$
  then there is $\bEv\gtN\aEv$
  where $\bEv$ reads or writes $\aLoc$.
\end{itemize}
Note that there is no requirement that $\bEv$ be satisfiable,
and indeed Spectre has the execution:
\[\begin{tikzpicture}[node distance=1em]
  \nonevent{rs}{\DR{\SEC}{1}}{}
  \nonevent{wa}{\DW{a[1]}{1}}{right=of rs}
  \event{ta}{\DT{a[1]}}{right=of wa}
  \event{wx}{\DW{x}{1}}{right=of ta}
  \po{rs}{wa}
  \wk{wa}{ta}
  \po{ta}{wx}
\end{tikzpicture}\]
but (assuming a successful implementation of $\TOUCHED$) \emph{not}:
\[\begin{tikzpicture}[node distance=1em]
  \nonevent{rs}{\DR{\SEC}{0}}{}
  \nonevent{wa}{\DW{a[0]}{1}}{right=of rs}
  \event{ta}{\DT{a[1]}}{right=of wa}
  \event{wx}{\DW{x}{1}}{right=of ta}
  \po{rs}{wa}
  \wk{wa}{ta}
  \po{ta}{wx}
\end{tikzpicture}\]
Thus, the attacker has managed to leak the value of a high-security
location to a low-security one: if $(\DW x1)$ is observed, the \verb|SECRET|
must have been 1.

This shows how our model of speculation can express
the way in which Spectre-like attacks bypass dynamic security checks,
without giving a treatment of microarchitecture.

\subsection{Information flow attacks on relaxed memory}
\label{sec:info-flow-attack}

Consider an attacker program, again using dynamic security checks to
try to learn a \verb|SECRET|. Whereas \verb|SPECTRE| uses
hardware capabilities, which have to be modeled by adding
extra capabilities to the language, this new attacker works
by exploiting relaxed memory which can result in
unexpected information flows. The attacker program is:
\[\begin{array}[t]{@{}l}
  \VAR x\GETS0\SEMI \VAR y\GETS0\SEMI\\\quad
    y\GETS x
  \PAR\begin{array}[t]{@{}l}
    \IF(y\EQ0)\THEN x\GETS1
    \brELIF(\CANREAD(\SEC))\THEN x\GETS\SEC
    \brELSE x\GETS1\SEMI z\GETS1 \FI
\end{array}\end{array}\]
In the case where $\SEC$ is $2$, this has many executions,
one of which is:
\[\begin{tikzpicture}[node distance=1em]
  \event{ix}{\DW{x}{0}}{}
  \event{iy}{\DW{y}{0}}{right=of ix}
  \event{rx0}{\DR{x}{0}}{below=of wx0}
  \event{wy0}{\DW{y}{0}}{right=of rx0}
  \event{ry0}{\DR{y}{0}}{below=of wy0}
  \event{wx1}{\DW{x}{1}}{right=of ry0}
  \nonevent{wx2}{\DW{x}{2}}{right=of wx1}
  \nonevent{wz1}{\DW{z}{1}}{right=of wx2}
  \po{rx0}{wy0}
  \po{ry0}{wx1}
  \po[out=30,in=150]{ry0}{wz1}
  \po[out=25,in=155]{ry0}{wx2}
  \rf{ix}{rx0}
  \rf{wy0}{ry0}
  \wk{iy}{wy0}
\end{tikzpicture}\]
but there are no executions which exhibit
$(\DW{z}{1})$, since any attempt to do so
produces a cycle:
\[\begin{tikzpicture}[node distance=1em]
  \event{ix}{\DW{x}{0}}{}
  \event{iy}{\DW{y}{0}}{right=of ix}
  \event{rx1}{\DR{x}{1}}{below=of ix}
  \event{wy1}{\DW{y}{1}}{right=of rx1}
  \event{ry1}{\DR{y}{1}}{below=of wy1}
  \event{wx1}{\DW{x}{1}}{right=of ry0}
  \nonevent{wx2}{\DW{x}{2}}{right=of wx1}
  \event{wz1}{\DW{z}{1}}{right=of wx2}
  \po{rx1}{wy1}
  \po{ry1}{wx1}
  \po[out=30,in=150]{ry1}{wz1}
  \po[out=25,in=155]{ry1}{wx2}
  \rf[in=-90,out=-150]{wx1}{rx1}
  \rf{wy1}{ry1}
  \wk[out=-20,in=90]{ix}{wx1}
  \wk[out=-20,in=120]{ix}{wx2}
  \wk{iy}{wy1}
\end{tikzpicture}\]\vskip-\bigskipamount\noindent
In the case where \verb|SECRET| is $1$, there is an execution:
\[\begin{tikzpicture}[node distance=1em]
  \event{ix}{\DW{x}{0}}{}
  \event{iy}{\DW{y}{0}}{right=of ix}
  \event{rx1}{\DR{x}{1}}{below=of ix}
  \event{wy1}{\DW{y}{1}}{right=of rx1}
  \event{ry1}{\DR{y}{1}}{below=of wy1}
  \event{wx1}{\DW{x}{1}}{right=of ry0}
  \event{wz1}{\DW{z}{1}}{right=of wx1}
  \po{rx1}{wy1}
  \po[out=30,in=150]{ry1}{wz1}
  \rf[in=-90,out=-150]{wx1}{rx1}
  \rf{wy1}{ry1}
  \wk[out=-20,in=90]{ix}{wx1}
  \wk[out=-20,in=120]{ix}{wx2}
  \wk{iy}{wy1}
\end{tikzpicture}\]\vskip-\bigskipamount\noindent
Note that in this case, there is no dependency from
$(\DR{y}{1})$ to $(\DW{x}{1})$.  This lack of dependency makes the
execution possible. Thus, if the attacker sees
an execution with $(\DW{z}{1})$, they can conclude
that \verb|SECRET| is $1$, which is an information flow
attack.

This attack is not just an artifact of the model,
since the same behavior can be exhibited by
compiler optimizations. Consider the program fragment:
\[\begin{array}{l}
    \IF(y = 0)\THEN x\GETS1
    \brELIF(\CANREAD(\SEC))\THEN x\GETS\SEC
    \brELSE x\GETS1\SEMI z\GETS1 \FI
\end{array}\]
In the case where \verb|SECRET| is a constant \verb|1|,
the compiler can inline it
and lift the assignment to $x$ out of the $\IF$ statement:
\[\begin{array}{l}
    x\GETS1\SEMI
    \IF(y = 0)\THEN
    \brELIF(\CANREAD(\SEC))\THEN
    \brELSE z\GETS1 \FI
\end{array}\]
After these optimizations, a sequentially consistent execution
exhibits $(\DW{z}{1})$. We discuss the practicality of this attack
further in \S\ref{sec:experiments}.

\subsection{Dead store elimination}
\label{sec:dse}

A common compiler optimization is \emph{dead store elimination},
in which writes are omitted if they will be overwritten by a subsequent
write later in the same thread. We can model eliminated writes
by ones with an unsatisfiable precondition. For example,
one execution of $(x \GETS 1\SEMI x \GETS 2) \PAR (r \GETS x)$ is:
\[\begin{tikzpicture}[node distance=1em]
  \nonevent{wx1}{\DW{x}{1}}{}
  \event{wx2}{\DW{x}{2}}{right=of wx1}
  \event{rx2}{\DR{x}{2}}{right=2.5em of wx2}
  \wk{wx1}{wx2}
  \rf{wx2}{rx2}
\end{tikzpicture}\]
Recall that for any satisfiable $\aEv$, if $\aEv$ reads $\aLoc$ from $\bLoc$
then $\bEv$ is satisfiable. This means that, although we can eliminate
$(\DW{x}{1})$ we cannot eliminate $(\DW{x}{2})$.

One heuristic that a compiler might adopt is to only eliminate
writes that are guaranteed to be followed by another write
to the same variable. This can be formalized by saying that
a write event $\bEv$ is eliminable if
there is a tautology $\aEv \ltN \bEv$
which writes to the same location.
A model of dead store elimination is one where,
in every pomset, every eliminable event is unsatisfiable.
This model includes the example above.

Note that if dead store
elimination is \emph{always} performed, then there is an information
flow attack similar to the one in \S\ref{sec:info-flow-attack}. Consider
the program:
\[\begin{array}[t]{@{}l}
    y\GETS x
  \PAR\begin{array}[t]{@{}l}
    x\GETS 1\SEMI\\
    \IF(\CANREAD(\SEC))\THEN \IF(\SEC)\THEN x\GETS 2\FI
    \brELSE x\GETS 2\FI
\end{array}\end{array}\]
In the case that \verb|SECRET| is $0$, there is an execution:
\[\begin{tikzpicture}[node distance=1em]
  \event{rx1}{\DR{x}{1}}{}
  \event{wy1}{\DW{y}{1}}{right=of rx1}
  \event{wx1}{\DW{x}{1}}{right=2.5em of wy1}
  \event{wx2}{\aForm \mid \DW{x}{2}}{right=of wx1}
  \rf[out=160,in=20]{wx1}{rx1}
  \po{rx1}{wy1}
  \wk{wx1}{wx2}
\end{tikzpicture}\]
where $\aForm$ is ($\lnot$\verb|canRead(SECRET)|),
which is not a tautology, and so the $(\DW{x}{1})$ event is not eliminated.
In the case that \verb|SECRET| is not $0$, the matching execution
is:
\[\begin{tikzpicture}[node distance=1em]
  \event{rx2}{\DR{x}{2}}{}
  \event{wy2}{\DW{y}{2}}{right=of rx2}
  \nonevent{wx1}{\DW{x}{1}}{right=2.5em of wy2}
  \event{wx2}{\DW{x}{2}}{right=of wx1}
  \rf[out=160,in=20]{wx2}{rx2}
  \po{rx2}{wy2}
  \wk{wx1}{wx2}
\end{tikzpicture}\]
Now the $(\DW{x}{2})$ event is a guaranteed write, so the $(\DW{x}{1})$
is eliminated, and so cannot be read.
In the case that the attacker can rely on dead store
elimination taking place, this is an information flow: if the attacker observes
$x$ to be $1$, then they know \verb|SECRET| is $0$. We return to this attack
in \S\ref{sec:experiments}.


% Local Variables:
% TeX-master: "paper"
% End:
