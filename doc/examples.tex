\section{Examples}

\subsection{Sequential memory accesses}

In the semantics of memory, there are two very different ways memory
can be accessed: sequentially or concurrently. These are modelled
differently, since hardware and compilers give very different
guarantees about their behaviour.
In this section, we discuss the sequential semantics, and leave
the concurrent semantics to \S\ref{sec:concurrent-memory}.

Consider the program \verb`(x := 0; y := x+1;)`.  One execution of
this program is where the write to $y$ uses the sequential value of
$x$, which is $0$:
\[\begin{tikzpicture}[node distance=1em]
  \event{wx0}{\DW{x}{0}}{}
  \event{wy1}{\DW{y}{1}}{right=of wx0}
\end{tikzpicture}\]
To see how this execution is modelled, we first
expand out the syntax sugar to get the program \verb`(x := 0; r := x; y := r+1; skip)`.
Now $\sem{\SKIP}$ is just $\{\emptyset\}$, and
$\sem{y \GETS r+1\SEMI \SKIP}$ is:
\[
   \bigcup_v\; (r+1=v \mid \DW yv) \prefix \sem{\SKIP}[v/y]
\]
which includes the case where $v$ is $1$:
\[
   (r+1=1 \mid \DW y1) \prefix \sem{\SKIP}[1/y]
\]
which contains the pomset:
\[\begin{tikzpicture}[node distance=1em]
  \event{wy1}{r+1=1 \mid \DW{y}{1}}{}
\end{tikzpicture}\]
Now $\sem{r \GETS x\SEMI y \GETS r+1\SEMI \SKIP}$ is:
\[
   \sem{y \GETS r+1\SEMI \SKIP}[x/r] \cup
   \bigcup_v\; (\DR xv) \prefix \sem{y \GETS r+1\SEMI \SKIP}[x/r]
\]
This has two cases, the sequential case
(which does not introduce a read action)
and the concurrent case (which does).
For the moment, we are interested in the sequential case, which is:
\[
   \sem{y \GETS r+1\SEMI \SKIP}[x/r]
\]
which contains the pomset:
\[\begin{tikzpicture}[node distance=1em]
  \event{wy1}{x+1=1 \mid \DW{y}{1}}{}
\end{tikzpicture}\]
Finally $\sem{x \GETS 0\SEMI r \GETS x\SEMI y \GETS r+1\SEMI \SKIP}$ is:
\[
   \bigcup_v\; (0=v \mid \DW xv) \prefix \sem{r \GETS x\SEMI y \GETS r+1\SEMI \SKIP}[v/x]
\]
which includes the case where $v$ is $0$:
\[
   (0=0 \mid \DW x0) \prefix \sem{r \GETS x\SEMI y \GETS r+1\SEMI \SKIP}[0/x]
\]
which contains the pomset:
\[\begin{tikzpicture}[node distance=1em]
  \event{wx0}{0=0 \mid \DW{x}{0}}{}
  \event{wy1}{0+1=1 \mid \DW{y}{1}}{right=of wx0}
\end{tikzpicture}\]
all of whose preconditions are tautologies, so this has the expected behaviour:
\[\begin{tikzpicture}[node distance=1em]
  \event{wx0}{\DW{x}{0}}{}
  \event{wy1}{\DW{y}{1}}{right=of wx0}
\end{tikzpicture}\]
Note that $(\DW x0)$ does not read anything, and so there is no
requirement of order between $(\DW x0)$ and $(\DW y1)$.

This example demonstrates how preconditions
capture the sequential semantics of memory.
In an execution containing an event with label
$(\aForm \mid \aAct)$, one way the precondition $\aForm$
can be discharged is by a write $\aLoc\GETS\aExp$,
which performs a substitution $[\aExp/\aLoc]$.
This is a variant of the usual Hoare semantics for
assignment, where if $\aCmd$ has preconditon $\aForm$
then $\aLoc\GETS\aExp\SEMI\aCmd$ has precondition
$\aForm[\aExp/\aLoc]$.

\subsection{Concurrent memory accesses}
\label{sec:concurrent-memory}

We now turn to the case of concurrent accesses to memory.
Consider the program \verb`x := 1; || y := x+1;`.
One execution of this program is where the write to $y$
performs a concurrent read of $x$:
\[\begin{tikzpicture}[node distance=1em]
  \event{wx1}{\DW{x}{1}}{}
  \event{rx1}{\DR{x}{1}}{right=2.5em of wx1}
  \event{wy2}{\DW{y}{2}}{right=of rx1}
  \rf{wx1}{rx1}
  \po{rx1}{wy2}
\end{tikzpicture}\]
To see how this execution is modelled, we first
expand out the syntax sugar to get the program \verb`(x := 0; skip || r := x; y := r+1; skip)`.
As before, $\sem{y \GETS r+1\SEMI \SKIP}$ is:
\[
   \bigcup_v\; (r+1=v \mid \DW yv) \prefix \sem{\SKIP}[v/y]
\]
which includes the case where $v$ is $2$:
\[
   (r+1=2 \mid \DW y2) \prefix \sem{\SKIP}[2/y]
\]
which contains the pomset:
\[\begin{tikzpicture}[node distance=1em]
  \event{wy2}{r+1=2 \mid \DW{y}{2}}{}
\end{tikzpicture}\]
Now $\sem{r \GETS x\SEMI y \GETS r+1\SEMI \SKIP}$ is:
\[
   \sem{y \GETS r+1\SEMI \SKIP}[x/r] \cup
   \bigcup_v\; (\DR xv) \prefix \sem{y \GETS r+1\SEMI \SKIP}[x/r]
\]
This has two cases, the sequential case
(which does not introduce a read action)
and the concurrent case (which does).
We are now interested in the concurrent case, which is:
\[
   \bigcup_v\; (\DR xv) \prefix \sem{y \GETS r+1\SEMI \SKIP}[x/r]
\]
which includes the case where $v$ is $1$:
\[
   (\DR x1) \prefix \sem{y \GETS r+1\SEMI \SKIP}[x/r]
\]
which contains the pomset:
\[\begin{tikzpicture}[node distance=1em]
  \event{rx1}{\DR{x}{1}}{}
  \event{wy2}{\DW{y}{2}}{right=of rx1}
  \po{rx1}{wy2}
\end{tikzpicture}\]
Note that $(\DR x1)$ reads $1$ from $x$, and while
$\TRUE$ implies $(x+1=2)[1/2]$ it does \emph{not}
imply $(x+1=2)$, and so there is an ordering
$(\DR x1) < (\DW y2)$.

Now, $\sem{x \GETS 1\SEMI \SKIP}$ includes the pomset:
\[\begin{tikzpicture}[node distance=1em]
  \event{wx1}{\DW{x}{1}}{}
\end{tikzpicture}\]
and so $\sem{x \GETS 1\SEMI \SKIP \PAR r \GETS x\SEMI y \GETS r+1\SEMI \SKIP}$ includes:
\[\begin{tikzpicture}[node distance=1em]
  \event{wx1}{\DW{x}{1}}{}
  \event{rx1}{\DR{x}{1}}{right=2.5em of wx1}
  \event{wy2}{\DW{y}{2}}{right=of rx1}
  \rf{wx1}{rx1}
  \po{rx1}{wy2}
\end{tikzpicture}\]
as expected.

This example demonstrates how read and write events
capture the concurrent semantics of memory.
In an execution containing an event with label
$(\DR \aLoc\aVal)$, if the execution is
$\aLoc$-closed, then there must be an event
it reads from, for example one labelled
$(\DW \aLoc\aVal)$.

\subsection{Independent writes}

Consider an example with two independent writes
\verb|(x := 1; y := 2;)|.
This has the semantics:
\[
  \bigcup_\aVal\; (1=\aVal \mid \DW x\aVal) \prefix \left(
    \bigcup_\bVal\; (2=\bVal \mid \DW y\aVal) \prefix \left(
      \{\emptyset\}
    \right)[2/y]
  \right)[1/x]
\]
which is the same as:
\[
  \bigcup_\aVal\; (1=\aVal \mid \DW x\aVal) \prefix
    \bigcup_\bVal\; (2=\bVal \mid \DW y\aVal) \prefix
      \{\emptyset\}
\]
which includes the case where $\aVal=1$ and $\bVal=2$:
\[
  (1=1 \mid \DW x1) \prefix
    (2=2 \mid \DW y2) \prefix
      \{\emptyset\}
\]
One of the executions this contains is:
\[\begin{tikzpicture}[node distance=1em]
  \event{wx1}{\DW{x}{1}}{}
  \event{wy2}{\DW{y}{2}}{right=of wx1}
  \po{wx1}{wy2}
\end{tikzpicture}\]
but it also contains:
\[\begin{tikzpicture}[node distance=1em]
  \event{wx1}{\DW{x}{1}}{}
  \event{wy2}{\DW{y}{2}}{right=of wx1}
\end{tikzpicture}\]
and:
\[\begin{tikzpicture}[node distance=1em]
  \event{wx1}{\DW{x}{1}}{}
  \event{wy2}{\DW{y}{2}}{right=of wx1}
  \po{wy2}{wx1}
\end{tikzpicture}\]
since there is no requirement that
$(\DW{x}{1}) \le (\DW{y}{2})$.

Thus, the semantics of \verb|(x := 1; y := 2;)|
is the same as the semantics of \verb|(y := 2; x := 1;)|.

\subsection{Independent reads and writes}

Whereas write prefixing introduces no new dependencies,
read prefixing can. For example
in \S\ref{sec:concurrent-memory} we saw that the program
\verb|(r := x; y := r+1;)| includes the pomset:
\[\begin{tikzpicture}[node distance=1em]
  \event{rx1}{\DR{x}{1}}{}
  \event{wy2}{\DW{y}{2}}{right=of rx1}
  \po{rx1}{wy2}
\end{tikzpicture}\]
but since $\TRUE$ does not imply $(x+1=2)$,
we have the requirement that $(\DR{x}{1}) \le (\DW{y}{2})$.

This is in contrast to the program 
\verb|(r := x; y := r+2-r;)|.
Since $\TRUE$ implies $(x+2-x=2)[1/x]$, this contains:
\[\begin{tikzpicture}[node distance=1em]
  \event{rx1}{\DR{x}{1}}{}
  \event{wy2}{\DW{y}{2}}{right=of rx1}
  \po{rx1}{wy2}
\end{tikzpicture}\]
but also $\TRUE$ implies $(x+2-x=2)$
(at least for integer arithmetic)
and so this also contains:
\[\begin{tikzpicture}[node distance=1em]
  \event{rx1}{\DR{x}{1}}{}
  \event{wy2}{\DW{y}{2}}{right=of rx1}
\end{tikzpicture}\]
Thus, the semantics of \verb|(r := x; y := r+2-r;)|
is the same as the semantics of \verb|(y := 2; r := x;)|.

Note this this example shows that we are not just dealing
with a syntactic notion of dependency, which is common
in hardware models of memory. In syntactic dependency,
since $r$ occurs free in $(y\GETS r+2-r)$, there would be
a dependency between $(r\GETS x)$ and $(y\GETS r+2-r)$.
In contrast, this model is based on logical implication,
which can be interpreted semantically.

\subsection{Control dependencies}
\label{sec:control-dep}

Conditionals introduce control dependencies, for example consider the program:
\begin{verbatim}
  r := z; if (r) { x := 1; } else { y := 2; }
\end{verbatim}
This includes executions in which the false branch is taken:
\[\begin{tikzpicture}[node distance=1em]
  \event{rz0}{\DR{z}{0}}{}
  \nonevent{wx1}{\DW{x}{1}}{right=of rz0}
  \event{wy2}{\DW{y}{2}}{right=of wx1}
  \po{rz0}{wx1}
  \po[out=30,in=150]{rz0}{wy2}
\end{tikzpicture}\]
and ones where the true branch is taken:
\[\begin{tikzpicture}[node distance=1em]
  \event{rz1}{\DR{z}{1}}{}
  \event{wx1}{\DW{x}{1}}{right=of rz1}
  \nonevent{wy2}{\DW{y}{2}}{right=of wx1}
  \po{rz1}{wx1}
  \po[out=30,in=150]{rz1}{wy2}
\end{tikzpicture}\]
In both cases, we record the actions in the branch that was
not taken. This is a novel feature of this model, and is
intended to capture speculative evaluation. In \S\ref{sec:spectre}
we will show how this model captures Spectre-like information
flow attacks, once the attacker is provided with the ability to
observe such speculations.

To see how these executions is modelled, consider the semantics of
$\sem{x\GETS 1\SEMI\SKIP}$, which contains any pomset of the form:
\[\begin{tikzpicture}[node distance=1em]
  \event{wx1}{\aForm \mid \DW{x}{1}}{}
\end{tikzpicture}\]
in particular it contains:
\[\begin{tikzpicture}[node distance=1em]
  \event{wx1}{r\neq0 \mid \DW{x}{1}}{}
\end{tikzpicture}\]
Similarly $\sem{y\GETS 2\SEMI\SKIP}$ contains:
\[\begin{tikzpicture}[node distance=1em]
  \event{wy2}{r=0 \mid \DW{y}{2}}{}
\end{tikzpicture}\]
and so $\sem{\IF(r)\{ x\GETS 1\SEMI\SKIP \} \ELSE \{ y\GETS 2\SEMI\SKIP \}}$
contains:
\[\begin{tikzpicture}[node distance=1em]
  \event{wx1}{r\neq0 \mid \DW{x}{1}}{}
  \event{wy2}{r=0 \mid \DW{y}{2}}{right=of wx1}
\end{tikzpicture}\]
Now, the semantics of concurrent read performs substitutions, for example
a read of $0$ from $x$ results in:
\[\begin{tikzpicture}[node distance=1em]
  \event{rz0}{\DR{z}{0}}{}
  \event{wx1}{0\neq0 \mid \DW{x}{1}}{right=of rz0}
  \event{wy2}{0=0 \mid \DW{y}{2}}{right=of wx1}
  \po{rz0}{wx1}
  \po[out=30,in=150]{rz0}{wy2}
\end{tikzpicture}\]
which gives the required pomset:
\[\begin{tikzpicture}[node distance=1em]
  \event{rz0}{\DR{z}{0}}{}
  \nonevent{wx1}{\DW{x}{1}}{right=of rz0}
  \event{wy2}{\DW{y}{2}}{right=of wx1}
  \po{rz0}{wx1}
  \po[out=30,in=150]{rz0}{wy2}
\end{tikzpicture}\]
Note that the precondition $r=0$ is dependent on $r$,
and so there is a dependency $(\DR z0) < (\DW y2)$,
modelling the control dependency introduced by the conditional.

\subsection{Control independencies}

In most models of control dependencies, the dependency relation
is syntactic, based on whether the action occurs inside syntactically
inside a conditional. In contrast, the notion in this model is
semantic: if an action can occur on both sides of a conditional,
there is no control dependency. Consider a variant of the example
from \S\ref{sec:control-dep}:
\begin{verbatim}
  r := z; if (r) { x := 1; } else { x := 1; }
\end{verbatim}
This has the expected execution in which the control
dependencies exist:
\[\begin{tikzpicture}[node distance=1em]
  \event{rz0}{\DR{z}{0}}{}
  \nonevent{nwx1}{\DW{x}{1}}{right=of rz0}
  \event{wx1}{\DW{x}{1}}{right=of nwx1}
  \po{rz0}{nwx1}
  \po[out=30,in=150]{rz0}{wx1}
\end{tikzpicture}\]
but it also has an execution in which the two writes
of $1$ to $x$ are merged, resulting in no dependency:
\[\begin{tikzpicture}[node distance=1em]
  \event{rz0}{\DR{z}{0}}{}
  \event{wx1}{\DW{x}{1}}{right=of rz0}
\end{tikzpicture}\]
To see how this arises in the model,
consider the definition of $\sem{\IF(r)\{ x\GETS1\SEMI\SKIP \} \ELSE \{ x\GETS1\SEMI\SKIP \}}$:
\[
   \aPSS_1 \sqcup \aPSS_2 \quad\mbox{where}\quad
   \aPSS_1 = (r\neq 0 \mid \sem{x\GETS1\SEMI\SKIP})  \quad\mbox{and}\quad
   \aPSS_2 = (r=0 \mid \sem{x\GETS1\SEMI\SKIP})
\]
Now, one pomset in $\aPSS_1$ is:
\[\begin{tikzpicture}[node distance=1em]
  \event{wx1}{r\neq0 \mid \DW{x}{1}}{}
\end{tikzpicture}\]
that is $\aPS_1$ where:
\[
  \Event_1 = \{\aEv\} \quad
  \labelling_1(\aEv) = (r\neq 0, \DW x1)
\]
and similarly, one pomset in $\aPSS_2$ is:
\[\begin{tikzpicture}[node distance=1em]
  \event{wx1}{r=0 \mid \DW{x}{1}}{}
\end{tikzpicture}\]
that is $\aPS_2$ where:
\[
  \Event_2 = \{\aEv\} \quad
  \labelling_2(\aEv) = (r= 0, \DW x1)
\]
Crucuially, in the definition of $\aPSS_1 \sqcup \aPSS_2$
there is \emph{no} requirement that $\Event_1$ and $\Event_2$ are disjoint,
and in this case they overlap at $\aEv$. As a result, one pomset in
$\aPSS_1\sqcup\aPSS_2$ is $\aPS_0$ where:
\[
  \Event_0 = \{\aEv\} \quad
  \labelling_0(\aEv) = (r\neq0 \lor r=0, \DW x1)
\]
that is:
\[\begin{tikzpicture}[node distance=1em]
  \event{wx1}{\DW{x}{1}}{}
\end{tikzpicture}\]
Note that this pomset has no precondition dependent on $r$,
since $(r\neq0 \lor r=0)$ does not depend on $r$, which is why
we end up with an execution without a control dependency:
\[\begin{tikzpicture}[node distance=1em]
  \event{rz0}{\DR{z}{0}}{}
  \event{wx1}{\DW{x}{1}}{right=of rz0}
\end{tikzpicture}\]
This semantics captures compiler optimizations which may, for example
merge code executed on both branches of a conditional, or hoist
constant assignments out of loops.

We can now see the counterintuitive behavior of conditionals
in the presence of control dependencies.
There are programs such as
\verb|r := z; if (r) { x := 1; } else { x := 1; }| with
executions in which  $(\DW x1)$ is independent of $(\DR z1)$:
\[\begin{tikzpicture}[node distance=1em]
  \event{rz1}{\DR{z}{1}}{}
  \event{wx1}{\DW{x}{1}}{right=of rz1}
\end{tikzpicture}\]
while programs such as
\verb|r := z; if (r) { x := 1; } else { y := 2; }|
only have executions in which $(\DW x1)$ is dependent on $(\DR z1)$:
\[\begin{tikzpicture}[node distance=1em]
  \event{rz1}{\DR{z}{1}}{}
  \event{wx1}{\DW{x}{1}}{right=of rz1}
  \nonevent{wy2}{\DW{y}{2}}{right=of wx1}
  \po{rz1}{wx1}
  \po[out=30,in=150]{rz1}{wy2}
\end{tikzpicture}\]
so these programs have different dependency relations, depending
on conditional branches that were not taken. In \S\ref{sec:info-flow-attack}
we shall see that this has security implications, since relaxed
memory can observe dependency. The attack is similar to Spectre, so
we shall take a detour to see how Spectre can be modeled in this
setting.

\subsection{Spectre}
\label{sec:spectre}

We give a simplified model of Spectre attacks, ignoring the details of
timing.  In this model, we extend programs with the ability to tell
whether a memory location has been touched (in practice this is
implemented using timing attacks on the cache). For example,
we can write a \verb|SPECTRE| program as:
\begin{verbatim}
    var a;
    if (isCapability(0)) { a[SECRET] := 1; }
    else if (touched a[0]) { x := 0; }
    else if (touched a[1]) { x := 1; }
\end{verbatim}
This is a low-security program, which is attempting to discover the
value of a high-security variable \verb|SECRET|. The low-security program
is allowed to attempt to escalate its privileges by providing a capability
which demonstrates that they are entitled to run high-security code:
\begin{verbatim}
    if (isCapability(c)) { ... escalated code ... }
    else { ... fallback code ... }
\end{verbatim}
In this case, the \verb|isCapability(0)| is false, so the fallback code
is executed. Unfortunately, the escalated code is speculatively
evaluated, which allows information to leak by testing for which
memory locations have been touched.

We model the \verb|touched| test by introducing a new read action
\textcolor{red}{(Why a read?)}
$(\DT{\aLoc})$ and defining:
\begin{eqnarray*}
  \sem{\IF \TOUCHED\aLoc\THEN \aCmd \ELSE \bCmd} & = & ((\DT\aLoc) \prefix \sem{\aCmd}) \sqcup \sem{\bCmd}
\end{eqnarray*}
The additional requirement we need to add for $\aLoc$-closure is:
\begin{itemize}
\item if $\labelling(\aEv)=(\aForm \mid \DT{\aLoc})$
  then there is $\bEv\not>\aEv$ with $\labelling(\bEv)=(\bForm \mid \DR{\aLoc}{\aVal})$
  or $\labelling(\bEv)=(\bForm \mid \DW{\aLoc}{\aVal})$.
\end{itemize}
For example, one execution of \verb|SPECTRE| is:
\[\begin{tikzpicture}[node distance=1em]
  \nonevent{rs}{\DR{\SEC}{1}}{}
  \nonevent{wa}{\DW{a[1]}{1}}{right=of rs}
  \event{ta}{\DT{a[1]}}{right=of wa}
  \event{wx}{\DW{x}{1}}{right=of ta}
  \po{rs}{wa}
  \po{ta}{wx}
\end{tikzpicture}\]
Putting this in parallel with a high-security write to \verb|SECRET| gives:
\[\begin{tikzpicture}[node distance=1em]
  \event{ws}{\DW{\SEC}{1}}{}
  \nonevent{rs}{\DR{\SEC}{1}}{right=2.5em of ws}
  \nonevent{wa}{\DW{a[1]}{1}}{right=of rs}
  \event{ta}{\DT{a[1]}}{right=of wa}
  \event{wx}{\DW{x}{1}}{right=of ta}
  \rf{ws}{rs}
  \po{rs}{wa}
  \po{ta}{wx}
\end{tikzpicture}\]
but due the requirement of \verb|a|-closure we do \emph{not} have:
\[\begin{tikzpicture}[node distance=1em]
  \event{ws}{\DW{\SEC}{0}}{}
  \nonevent{rs}{\DR{\SEC}{0}}{right=2.5em of ws}
  \nonevent{wa}{\DW{a[1]}{0}}{right=of rs}
  \event{ta}{\DT{a[1]}}{right=of wa}
  \event{wx}{\DW{x}{1}}{right=of ta}
  \rf{ws}{rs}
  \po{rs}{wa}
  \po{ta}{wx}
\end{tikzpicture}\]
Thus, the attacker has managed to leak the value of a high-security
location to a low-security one.

This shows how a (very abstract, untimed) model of Spectre attacks
using speculative evaluation can be modeled.

\subsection{Relaxed memory}

The model includes concurrent memory accesses, which can introduce concurrent
reads-from. For example, the program:
\begin{verbatim}
  x := 1; || r := x;
\end{verbatim}
has an execution in which the write to $x$ does not justify the read from $x$:
\[\begin{tikzpicture}[node distance=1em]
  \event{wx1}{\DW{x}{1}}{}
  \event{rxv}{\DR{x}{v}}{right=2.5 em of wx1}
\end{tikzpicture}\]
but also has an execution in which the write to $x$ does justify the read from $x$:
\[\begin{tikzpicture}[node distance=1em]
  \event{wx1}{\DW{x}{1}}{}
  \event{rx1}{\DR{x}{1}}{right=2.5 em of wx1}
  \rf{wx1}{rx1}
\end{tikzpicture}\]
Since we are allowing events to be partially ordered, this gives a simple
model of relaxed memory, for example an independent read independent write
(IRIW) example is:
\begin{verbatim}
  x := 0; x := x+1; || y := 0; y := y+1; || if (x) { r := y; } || if (y) { s := x; }
\end{verbatim}
which includes the execution:
\[\begin{tikzpicture}[node distance=1em]
  \event{wx0}{\DW{x}{0}}{}
  \event{wx1}{\DW{x}{1}}{right=of wx0}
  \event{wy0}{\DW{y}{0}}{right=2.5em of wx1}
  \event{wy1}{\DW{y}{1}}{right=of wy0}
  \event{rx1}{\DR{x}{1}}{right=2.5 em of wy1}
  \event{ry0}{\DR{y}{0}}{right=of rx1}
  \event{ry1}{\DR{y}{1}}{right=2.5 em of ry0}
  \event{rx0}{\DR{x}{0}}{right=of ry1}
  \rf[out=30,in=150]{wx1}{rx1}
  \rf[out=30,in=150]{wy0}{ry0}
  \rf[out=340,in=200]{wy1}{ry1}
  \rf[out=340,in=200]{wx0}{rx0}
  \po{wx0}{wx1}
  \po{wy0}{wy1}
  \po{rx1}{ry0}
  \po{ry1}{rx0}
\end{tikzpicture}\]
This model does not introduce thin-air reads (TAR),
for example the TAR pit is:
\begin{verbatim}
  x := y; || y := x;
\end{verbatim}
but an attempt to produce a value from thin air fails,
for the usual reason of producing a cycle in $\le$:
\[\begin{tikzpicture}[node distance=1em]
  \event{ry42}{\DR{y}{42}}{}
  \event{wx42}{\DW{x}{42}}{below=of ry42}
  \event{rx42}{\DR{x}{42}}{right=2.5em of ry42}
  \event{wy42}{\DW{y}{42}}{below=of rx42}
  \po{ry42}{wx42}
  \po{rx42}{wy42}
  \rf{wx42}{rx42}
  \rf{wy42}{ry42}
\end{tikzpicture}\]
This cycle can be broken if one of the writes does not depend on the read, for example:
\begin{verbatim}
  x := y; || r := x; y := r+1-r;
\end{verbatim}
has execution:
\[\begin{tikzpicture}[node distance=1em]
  \event{ry1}{\DR{y}{1}}{}
  \event{wx1}{\DW{x}{1}}{below=of ry1}
  \event{rx1}{\DR{x}{1}}{right=2.5em of ry1}
  \event{wy1}{\DW{y}{1}}{below=of rx1}
  \po{ry1}{wx1}
  \rf{wx1}{rx1}
  \rf{wy1}{ry1}
\end{tikzpicture}\]
Note that $(\DR x1) \not\le (\DW y1)$, so this does not introduce a cycle.

This model of relaxed memory is simple, and does not model many features
such as coherence, non-release-acquire access, or memory fences, but it is
good enough for the examples in this paper. In particular, we use the fact that
relaxed memory is sensitive to data dependency to give a Spectre attack which
does not depend on timing in \S\ref{sec:info-flow-attack}.

\subsection{Information flow attacks on relaxed memory}
\label{sec:info-flow-attack}

Consider an attacker program, again using security checks to
try to learn a \verb|SECRET|. Whereas \verb|SPECTRE| uses
hardware capabilities, which have to be modeled by adding
extra capabilities to the language, this new attacker works
by exploiting relaxed memory which can result in
unexpected information flows. The attacker program is:
\begin{verbatim}
  var x := 0; var y := 0;
  (
    y := x;
  ) || (
    r := y;
    if (isCapability(r)) { x := SECRET; }
    else { x := 1; z := r; }
  )
\end{verbatim}
In the case where \verb|SECRET| is $2$, this has many executions,
one of which is:
\[\begin{tikzpicture}[node distance=1em]
  \event{wx0}{\DW{x}{0}}{}
  \event{wy0}{\DW{y}{0}}{right=of wx0}
  \event{rx1}{\DR{x}{1}}{right=2.5 em of wy0}
  \event{wy1}{\DW{y}{1}}{right=of rx1}
  \event{ry0}{\DR{y}{0}}{right=2.5 em of wy1}
  \event{wx1}{\DW{x}{1}}{right=of ry0}
  \event{wz0}{\DW{z}{0}}{right=of wx1}
  \nonevent{wx2}{\DW{x}{2}}{right=of wz0}
  \po{rx1}{wy1}
  \po{ry0}{wx1}
  \po[out=25,in=150]{ry0}{wz0}
  \po[out=30,in=150]{ry0}{wx2}
  \rf[out=30,in=150]{wy0}{ry0}
  \rf[out=210,in=330]{wx1}{rx1}
\end{tikzpicture}\]
but there are no executions which exhibit
$(\DW{z}{1})$, since any attempt to do so
produces a cycle:
\[\begin{tikzpicture}[node distance=1em]
  \event{wx0}{\DW{x}{0}}{}
  \event{wy0}{\DW{y}{0}}{right=of wx0}
  \event{rx1}{\DR{x}{1}}{right=2.5 em of wy0}
  \event{wy1}{\DW{y}{1}}{right=of rx1}
  \event{ry1}{\DR{y}{1}}{right=2.5 em of wy1}
  \event{wx1}{\DW{x}{1}}{right=of ry1}
  \event{wz1}{\DW{z}{1}}{right=of wx1}
  \nonevent{wx2}{\DW{x}{2}}{right=of wz1}
  \po{rx1}{wy1}
  \po{ry1}{wx1}
  \po[out=25,in=150]{ry1}{wz1}
  \po[out=30,in=150]{ry1}{wx2}
  \rf{wy1}{ry1}
  \rf[out=210,in=330]{wx1}{rx1}
\end{tikzpicture}\]
In the case where \verb|SECRET| is $1$, there is an execution:
\[\begin{tikzpicture}[node distance=1em]
  \event{wx0}{\DW{x}{0}}{}
  \event{wy0}{\DW{y}{0}}{right=of wx0}
  \event{rx1}{\DR{x}{1}}{right=2.5 em of wy0}
  \event{wy1}{\DW{y}{1}}{right=of rx1}
  \event{ry1}{\DR{y}{1}}{right=2.5 em of wy1}
  \event{wx1}{\DW{x}{1}}{right=of ry1}
  \event{wz1}{\DW{z}{1}}{right=of wx1}
  \po{rx1}{wy1}
  \po[out=30,in=150]{ry1}{wz1}
  \rf{wy1}{ry1}
  \rf[out=210,in=330]{wx1}{rx1}
\end{tikzpicture}\]
Note that in this case, there is no dependency from
$(\DR{y}{1})$ to $(\DW{x}{1})$, which is what makes this
execution possible. Thus, if the attacker sees
an execution with $(\DW{z}{1})$, they can conclude
that \verb|SECRET| is $1$, which is an information flow
attack.

This attack is not just an artifact of the model,
since the same behavior can be exhibited by
compiler optimizations. Consider the program fragment:
\begin{verbatim}
  r := y;
  if (isCapability(r)) { x := SECRET; }
  else { x := 1; z := r; }
\end{verbatim}
Now, in the case where \verb|SECRET| is a constant \verb|1|,
the compiler can inline it:
\begin{verbatim}
  r := y;
  if (isCapability(r)) { x := 1; }
  else { x := 1; z := r; }
\end{verbatim}
lift the assignment to \verb|x| out of the \verb|if| statement:
\begin{verbatim}
  r := y; x := 1;
  if (isCapability(r)) { }
  else { z := r; }
\end{verbatim}
and then perform independent read/write reordering:
\begin{verbatim}
  x := 1; r := y;
  if (isCapability(r)) { }
  else { z := r; }
\end{verbatim}
After these optimizations, a sequentially consistent execution
exhibits $(\DW{z}{1})$.

% Local Variables:
% TeX-master: "paper"
% End:
