\section{Attacks on speculative execution}
\label{sec:examples}

In this section, we show how known attacks on speculative execution can be
modeled.  In \S\ref{sec:spectre}, we discuss Spectre. In
\S\ref{sec:spec-barriers}, we describe \emph{speculation barriers} for
defense against Spectre.  In \S\ref{sec:transactions}, we discuss attacks on
transactions.

In each attack, there is a high-security variable $\SEC$,
and the goal of the attacker is to learn one bit of information
from $\SEC$. The Spectre and \textsc{Prime+Abort}
attacks exploit optimizations in hardware, and so can be mounted
against a dynamic $\SEC$.

\subsection{Spectre}
\label{sec:spectre}

We give a simplified model of Spectre attacks, ignoring the details of
cache timing.  In this model, we extend programs with the ability to tell
whether a memory location has been touched (in practice this is
implemented using timing attacks on the cache). For example,
we can model Spectre by:
\[\begin{array}{l}
  \VAR a\SEMI \IF(\CANREAD(\SEC))\THEN a[\SEC]\GETS1
  \brELIF(\TOUCHED a[0])\THEN x\GETS0
  \brELIF(\TOUCHED a[1])\THEN x\GETS1 \FI
\end{array}\]
This is a low-security program, which is attempting to discover the
value of a high-security variable $\SEC$. The low-security program
is allowed to attempt to escalate its privileges by checking that it is
allowed to read a high-security variable:
\[\begin{array}{l}
  \IF(\CANREAD(\SEC))\THEN \mbox{code allowed to read $\SEC$}
  \brELSE \mbox{code not allowed to read $\SEC$} \FI
\end{array}\]
In this case, $\CANREAD(\SEC)$ is false, so the fallback code
is executed. Unfortunately, the escalated code is speculatively
evaluated, which allows information to leak by testing for which
memory locations have been touched.

Attacks may realize the abstract notions in various ways.  For example, in
variant 1 of Spectre, the dynamic security check is implemented as an array
bounds check.

We model the $\TOUCHED$ test by introducing a new action
$(\DT{\aLoc})$, and defining:
\[\begin{array}{l}
  \sem{\IF (\TOUCHED\aLoc) \THEN \aCmd \ELSE \bCmd \FI} \\[\jot]\quad =  ((\DT\aLoc) \prefix \sem{\aCmd}) \cup \sem{\bCmd}
\end{array}\]
Implementations of $\TOUCHED$ use cache timing, but their success can be modeled
without needing to be precise about such microarchitectural details:
\begin{itemize}
\item if $\labelling(\aEv)=(\aForm \mid \DT{\aLoc})$
  then there is $\bEv\gtN\aEv$
  where $\bEv$ reads or writes $\aLoc$.
\end{itemize}
Note that there is no requirement that $\bEv$ be satisfiable,
and indeed Spectre has the execution:
\[\begin{tikzpicture}[node distance=1em]
  \nonevent{rs}{\DR{\SEC}{1}}{}
  \nonevent{wa}{\DW{a[1]}{1}}{right=of rs}
  \event{ta}{\DT{a[1]}}{right=of wa}
  \event{wx}{\DW{x}{1}}{right=of ta}
  \po{rs}{wa}
  \wk{wa}{ta}
  \po{ta}{wx}
\end{tikzpicture}\]
but (assuming a successful implementation of $\TOUCHED$) \emph{not}:
\[\begin{tikzpicture}[node distance=1em]
  \nonevent{rs}{\DR{\SEC}{0}}{}
  \nonevent{wa}{\DW{a[0]}{1}}{right=of rs}
  \event{ta}{\DT{a[1]}}{right=of wa}
  \event{wx}{\DW{x}{1}}{right=of ta}
  \po{rs}{wa}
  \wk{wa}{ta}
  \po{ta}{wx}
\end{tikzpicture}\]
Thus, the attacker has managed to leak the value of a high-security
location to a low-security one: if $(\DW x1)$ is observed, the \verb|SECRET|
must have been 1.

This shows how our model of speculation can express
the way in which Spectre-like attacks bypass dynamic security checks,
without giving a treatment of microarchitecture.

\subsection{Speculation barriers}
\label{sec:spec-barriers}

The ability to model Spectre is useful, but really we would
like to model defences against such attacks, and provide some
confidence in the correctness of the defence. One such defence
which fits naturally in our model is \emph{speculation barriers},
which prevent code from being speculatively executed. For example,
we could introduce such a $\BARRIER$, and require that
a barrier is introduced on each security check:
\[\begin{array}{l}
\IF(\CANREAD(\SEC))\THEN \BARRIER\SEMI\cdots
  \ELSE \cdots \FI
\end{array}\]
To model barriers, we introduce a new action $\DSB$
and define:
\[\begin{array}{l}
  \sem{\BARRIER\SEMI\aCmd} =  \{\emptyset\} \cup ((\DSB) \prefix \sem{\aCmd})
\end{array}\]
Implementations of $\BARRIER$ make use of hardware barriers which
halt speculative execution until all instructions up to the barrier
have been retired. Such barriers are successful when:
\begin{itemize}
\item if $\labelling(\aEv)=(\aForm \mid \DSB)$
  then $\aForm$ is satisfiable.
\end{itemize}
For example, a successful implementation of barriers disallows
the execution of Spectre:
\[\begin{tikzpicture}[node distance=1em]
  \nonevent{sb}{\DSB}{}
  \nonevent{rs}{\DR{\SEC}{1}}{right=of sb}
  \nonevent{wa}{\DW{a[1]}{1}}{right=of rs}
  \event{ta}{\DT{a[1]}}{right=of wa}
  \event{wx}{\DW{x}{1}}{right=of ta}
  \po{rs}{wa}
  \wk{wa}{ta}
  \po{ta}{wx}
\end{tikzpicture}\]
One might expect that this is a successful (albeit expensive) defence
against Spectre, but it is not, unless the compiler is aware that
$\BARRIER$ cannot be lifted out of a conditional. An unaware compiler
might perform common subexpression elimination on barriers, allowing
the attacker to introduce a barrier to fool a compiler into optimizing
the safe:
\[\begin{array}{l}
  \IF(\CANREAD(\SEC))\THEN \BARRIER\SEMI \cdots
  \ELSE \BARRIER\SEMI \cdots \FI
\end{array}\]
into the unsafe:
\[\begin{array}{l}
  \BARRIER\SEMI \IF(\CANREAD(\SEC))\THEN \cdots
  \ELSE \cdots \FI
\end{array}\]
To model the requirement that barriers are not moved past
conditionals, we make them \emph{unmergeable}: $\DSB$ events
on different arms of a conditional cannot be merged.
With this requirement, we can show that barriers act as a
defence against Spectre by first showing that $\sem\bCmd$ has
the same successful executions as:
\[
  \sem{\IF(\CANREAD(\SEC)) \BARRIER\SEMI\aCmd \ELSE \bCmd \FI}
\]
and then showing that the semantics which only looks at successful
executions is \emph{compositional}: if $\sem\bCmd$ has the same
successful executions as $\sem{\bCmd'}$ then $\sem{\aCmd[\bCmd]}$ has
the same successful executions as $\sem{\aCmd[\bCmd']}$ for any
``program with a hole'' $\aCmd[{\bullet}]$. Compositional reasoning
is what fails when $\DSB$ is mergeable, as shown by the
attack against a compiler which blindly performs common subexpression
elimination.

To realize a speculation barrier in microarchitecture, it is likely
sufficient for the barrier to stop any further speculation until the barrier
is known to succeed.  There is experimental evidence that Intel’s
\texttt{mfence} instruction has the effect of a speculation barrier in some contexts
\cite[\S{VII-\textit{D}}]{DBLP:conf/micro/TrippelLM18}.
