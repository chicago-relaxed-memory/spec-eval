\section{Logic}
\label{sec:logic}

Syntax:
\begin{verbatim}
  phi ::= Wxv | Rxv | F(phi) | false | phi1 \/ phi2 | ~ phi
\end{verbatim}
Given a labelled rf-poset (E,lbl,<,rf), define satisfaction:
\begin{verbatim}
  e |= Wxv if lbl(e) = (true,Wxv)
  e |= Rxv if lbl(e) = (true,Rxv)
  e |= F(phi) if exists d < e, d |= phi
\end{verbatim}
Denote rf-poset by its carrier set E.
Then a set of rf-poset is denoted as Es.
\begin{verbatim}
  Es |= phi if forall E in Es: forall e in E: e |= phi
\end{verbatim}
Coinduction for F [sound proof rule]:
\begin{verbatim}
  phi => F(phi)
  ------------
     ~ phi
\end{verbatim}
Superset closed formula:
\begin{verbatim}
  Define: phi is superset closed if (Es |= phi) and (Es' supseteq Es) imply (Es' |= phi)
\end{verbatim}
Parallel composition [sound proof rule]:
\begin{verbatim}
  phi is superset closed
  Es1 |= phi  
  Es2 |= phi
  ----------------
  Es1 || Es2  |= phi
\end{verbatim}
Closed at x:
\begin{verbatim}
  Define closed(x) = (Rxv => F(Wxv))
\end{verbatim}
Local declaration [sound proof rule]:
\begin{verbatim}
  x notin phi
  Es |= closed(x) => phi
  ----------------------
  var x; Es |= phi
\end{verbatim}
Conditional TAR example:
\begin{verbatim}
  var x,y,z;
  y:=0; y:=x  ||  x:=0; if(~z){x:=1}else{x:=y;a:=y}  ||  z:=0; z:=1
\end{verbatim}
\begin{verbatim}
Goal: Es |= ~ F(Wa1)   [impossible to write a=1]
\end{verbatim}
Invariant:
\begin{verbatim}
     F(Wy1) => F(Rx1)
  /\ F(Wa1) => F(Ry1) /\ G(Wx1 => F(Ry1))
\end{verbatim}
Closing y:
\begin{verbatim}
  F(Wa1) => F(Rx1) /\ G(Wx1 => F(Rx1))
\end{verbatim}
Closing x:
\begin{verbatim}
  F(Wa1) => F(Wx1) /\ G(Wx1 => F(Wx1))
\end{verbatim}
Using coinduction for F:  
\begin{verbatim}
  F(Wa1) => F(Wx1) /\ G(~ Wx1)
\end{verbatim}
Simplifying:  
\begin{verbatim}
  F(Wa1) => false
\end{verbatim}