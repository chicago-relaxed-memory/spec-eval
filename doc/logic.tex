\section{Logic}
\label{sec:logic}

\newcommand{\since}[2]{{#1} \  {\tt  S } \  {#2}}
\newcommand{\pLTL}{{\tt PLTL}}
\newcommand{\once}{\Diamond^{-1}}
\newcommand{\pastalways}{\Box^{-1}}
\newcommand{\afo}{\phi}
\newcommand{\bfo}{\psi}
\newcommand{\mods}{\tt Models}



We work with a variant of \pLTL\ \footnote{ Our presentation differs from
  standard presentations of past
  LTL\citet{Lichtenstein:1985:GP:648065.747612} in two ways.  First, we
  eschew the previous instant operator, to account for the current setting of
  partial orders; second, we consider strictly past versions of the ``once''
  and ``always in the past'' operators, by not evaluating the formula at the
  current instant.}  interpreted over pomsets.  The atoms of our logic are
write and read events, and we use strictly in the past modal operators in
addition to the usual boolean connectives.
\begin{displaymath}
  \begin{array}{lrl}
    \afo &::=&\DR{\aLoc}{\aVal}  \mid \DW\aLoc\aVal \\
    &&  \afo \wedge\ \bfo \mid \neg \afo  \\
    && \once\afo \mid\ \pastalways\afo
  \end{array} 
\end{displaymath}
As usual, we write $\afo\vee\bfo$ for $\neg(\neg \afo \wedge\ \neg \bfo)$,
and $\afo \Rightarrow \bfo$ for $\neg \afo \vee \bfo$.

\begin{definition} [Satisfaction]
  Given an rf-pomset $\aPS= (\Event, {\le}, \labelling)$ with alphabet
  $\Alphabet$, and $\aEv \in \Event$, define:
  \begin{displaymath}
    \begin{array}{lrl}
      \aPS,\aEv & \models& \DR{\aLoc}{\aVal}, \ \mbox{if } \labelling(\aEv) =  (\TRUE, \DR{\aLoc}{\aVal}) \\
      \aPS,\aEv &\models& \DW{\aLoc}{\aVal}, \ \mbox{if } \labelling(\aEv) =  (\TRUE, \DW{\aLoc}{\aVal}) \\
      \aPS,\aEv &\models&  \afo\wedge\bfo, \ \mbox{if } \aPS,\aEv \models  \afo\ \mbox{ and } \aPS,\aEv \models  \bfo\  \\
      \aPS,\aEv &\models&  \neg\afo\ , \ \mbox{if } \aPS,\aEv \not\models \afo \\
      \aPS,\aEv &\models& \once\afo, \mbox{if } (\exists \bEv \le \aEv, \bEv\not=\aEv) \  \aPS,\bEv \models \afo\  \\
      \aPS,\aEv &\models& \pastalways\afo, \mbox{if } (\forall \bEv \le \aEv, \bEv\not=\aEv) \  \aPS,\bEv \models \afo\ 
    \end{array} 
  \end{displaymath}

  Define $\aPS \models \afo$ if $(\forall \aEv \in \Event) \ \aPS,\aEv  \models\ \afo$.

  For a set of rf-pomsets $\aPSS$, define $\aPSS\models \afo$ if $(\forall \aPS \in \aPSS) \ \aPS \models\ \afo$.
\end{definition}
The definition of satisfaction of formulas by pomsets validates the rule: if
$\aPS \models \afo$ then $\aPS \models \afo \wedge \pastalways\afo$.

While the boolean connectives are interpreted as expected, the strict past
operators explore the past of the current event as determined by the pomset
ordering.  Their interpretation does not include the current instant.  Thus,
while they are connected by a DeMorgan law
\begin{math}
  \pastalways\afo = \neg\once(\neg\afo)
\end{math}
they do {\emph not} satisfy the rule
\begin{math}
  \pastalways\afo\ \Rightarrow\ \once\afo.
\end{math}
However, they do satisfy:
\begin{description}
\item[Coinduction.]
  \begin{math}
    (\afo \Rightarrow\ \once{\afo}) \Rightarrow\ \neg \afo
  \end{math}
\item[Induction.] 
  \begin{math}
    (\pastalways\afo \Rightarrow\ \afo) \Rightarrow\ \afo
  \end{math}
\end{description}

We next state a composition result in the style of Abadi and
Lamport~\cite{Abadi:1993:CS:151646.151649}.  Our statement and proof is
intentionally less general to simplify the presentation.  We view the
composition result as capturing key aspects of no-ThinAirRead, as will become
clearer in the examples below.

The statement of the theorem requires us to incorporate environment
assumptions in the satisfaction relation.
\begin{definition}
  Let $\afo \in \pLTL$.  Then, define:
  \begin{displaymath} \mods{(\afo)} = \{ \aPSS \mid \ \aPSS \models \afo \} \end{displaymath}
\end{definition}
$\mods{(\afo)}$ are the models of the formula $\afo$, i.e. the pomsets that
satisfy the formula.  We say that $\afo$ is prefix closed if $\mods{(\afo)}$
is prefix-closed\footnote{$\aPS_1$ is a prefix of $\aPS$ if the carrier set
  of $\aPS_1$ is a downwards closed subset of $\aPS$; i.e if
  $\aEv \in \aPS_1$ and $ \bEv \le_{\aPS} \aEv$, then $\bEv$ also in
  $\aPS_1$.}.

%\begin{definition}
Let $\afo \in \pLTL$ be prefix-closed.  Let $ \bfo \in \pLTL$.  Then define
\begin{math}
  \afo, \aPSS \models \bfo  \mbox{ if } \mods{(\afo)} \parallel \aPSS \models \bfo.
\end{math}
% \end{definition}

We are now ready to state the composition theorem.  In the vocabulary of
~\citet{Abadi:1993:CS:151646.151649}, we are in the special case of
invariants without environment assumptions.
\begin{lemma}[Composition]
  Let $\afo \in \pLTL$ be prefix-closed.  Let $\aPSS_1, \aPSS_2$ be
  augmentation-closed\footnote{$\aPS_1$ is a augmentation of $\aPS$ if their
    carrier sets are the same and if $ \bEv \le_{\aPS} \aEv$, then
    $\bEv \le \aEv$ also in $\aPS_1$.}.  Then:
  \begin{displaymath}
    \frac{
      \afo, \aPSS_1 \models\ \afo
      \quad
      \afo, \aPSS_2 \models\ \afo
    }{\aPSS_1 \parallel \aPSS_2 \models \afo}
  \end{displaymath}
\end{lemma}
\begin{proof}[Sketch]
  We will show that all prefixes in the prefix closures of
  $\aPSS_1 \parallel \aPSS_2$ satisfy the required property.  Proof proceeds
  by induction on prefixes of $\aPS \in \aPSS_1 \parallel \aPSS_2$.

  The case for empty prefix  follows from assumption that  $\afo$ is prefix closed.  

  For the inductive case, consider $\aPS$ in the prefix closure of
  $\aPSS_1 \parallel \aPSS_2$, i.e. $\aPS = \aPS_1 \parallel \aPS_2$ where
  $\aPS_i \in \aPSS_i$.  Since $\aPSS_1$ and $\aPSS_2$ are augmentation
  closed, we can assume that the restriction of $\aPS$ to the events of
  $\aPS_i$ coincides with $\aPS_i$, for $i=1,2$.

  Consider a prefix (say $\aPS'$) got by deleting a maximal element, say
  $\aEv$, of $\aPS$.  There are two cases depending on whether $\aEv$ comes
  from $\aPS_1$ or $\aPS_2$.  In the case when $\aEv$ comes from $\aPS_1$,
  since $\aPS_2$ is a prefix of $\aPS'$ and $\aPS' \models \afo$ by induction
  hypothesis, we deduce that $\aPS_2 \models \afo$.  Thus,
  $\aPS_2 \in \mods{(\afo)}$.  Since $\aPS_1 \in \aPSS_1$, assumption
  $\afo, \aPSS_1 \models\ \afo$, we deduce that
  $\aPS_1 \parallel \aPS_2 \models \afo$.
\end{proof}

\endinput

Closed at x:
\begin{verbatim}
  Define closed(x) = (Rxv => F(Wxv))
\end{verbatim}
Local declaration [sound proof rule]:
\begin{verbatim}
  x notin phi
  Es |= closed(x) => phi
  ----------------------
  var x; Es |= phi
\end{verbatim}
Conditional TAR example:
\begin{verbatim}
  var x,y,z;
  y:=0; y:=x  ||  x:=0; if(~z){x:=1}else{x:=y;a:=y}  ||  z:=0; z:=1
\end{verbatim}
\begin{verbatim}
Goal: Es |= ~ F(Wa1)   [impossible to write a=1]
\end{verbatim}
Invariant:
\begin{verbatim}
     F(Wy1) => F(Rx1)
  /\ F(Wa1) => F(Ry1) /\ G(Wx1 => F(Ry1))
\end{verbatim}
Closing y:
\begin{verbatim}
  F(Wa1) => F(Rx1) /\ G(Wx1 => F(Rx1))
\end{verbatim}
Closing x:
\begin{verbatim}
  F(Wa1) => F(Wx1) /\ G(Wx1 => F(Wx1))
\end{verbatim}
Using coinduction for F:  
\begin{verbatim}
  F(Wa1) => F(Wx1) /\ G(~ Wx1)
\end{verbatim}
Simplifying:  
\begin{verbatim}
  F(Wa1) => false
\end{verbatim}