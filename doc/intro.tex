\section{Introduction}

This paper studies information flow caused by speculation mechanisms
in hardware and software.

Information flow from high (hi) to low (lo) provides a formal
foundation for end-to-end security.  Informally, a program is secure
if there is no observable dependency of lo-observables on hi-inputs.
The precise formalization of this intutive idea has been the topic of
extensive research (e.g., see~\cite{Sabelfeld:2006:LIS:2312191.2314769}  for a detailed survey till
2006), and can be generalized to account for a variety of language
features and observables such as non-determinisim~\cite{Wittbold1990InformationFI},
concurrency~\cite{Smith:1998:SIF:268946.268975}, reactivity~\cite{O'Neill:2006:ISI:1155442.1155677}, and
probability~\cite{Gray:1992:TMF:2699806.2699811}. The static and dynamic enforcement
of these definitions in general purpose languages~\cite{myers-popl99} has also
been studied extensively and has influenced language design and
implementation.

A key parameter in the definitions cited above is the notion of
the \emph{observational power} of the attacker model. Whereas the classical
input-output behavior is often an adequate foundation,
it has long been known~\cite{Lampson:1973:NCP:362375.362389,Biswas:2017:STC:3058791.3023872} that side-channels that leak
information arise from other observables such as execution time and
power consumption.

The focus of this paper is the development of a compositional semantic model of
executions of programs to explicate side channel attacks that arise
from speculation. In particular, we model conditionals such that it is
possible for the behaviour of a conditional
$(\IF(\aExp)\THEN \aCmd \ELSE \bCmd \FI)$ to depend on the behaviour
of $\bCmd$ even when the condition $\aExp$ is true.

Our study addresses several sources of speculation in the concurrent
execution of programs.
\begin{itemize}

\item Pipelined microprocessors use heuristics to predict the path of a
  program, for example by guessing memory dependencies, or the outcome of
  conditionals. Execution proceeds along the predicted path until it
  is possible to validate the prediction; the execution is committed
  if the prediction is correct; otherwise it is rolled
  back.  The speculation does not affect the observable input-output
  behavior of the program, thus ensuring correctness with respect to
  the usual intended semantics of the program. However, since there are
  clear timing differences between the cases where the prediction is
  and is not successful, it raises the concern that the predictive
  mechanisms themselves could cause information leaks via timing
  side-channels; a fear that is realized with devastating impact by
  the Spectre family of attacks~\cite{DBLP:journals/corr/abs-1801-01203}.

\item Several modern microprocessors~\cite{ChongSW18} support transactions
  to aid in the design of correct and efficient concurrent
  programs. Transactions are executed optimistically; they are
  committed if there are no conflicts, and aborted otherwise. All
  memory effects of an aborted transaction are rolled back; so there
  is no way for a concurrent observer to detect an aborted
  transaction.  However, the thread of the aborted transaction, via
  abort-handler code, gets notified of the cancellation of the
  transaction.

  Thus, when a transaction of a thread aborts, the thread can deduce
  plausible information about specific memory accesses of a
  concurrently executing transaction. In combination with the
  techniques used in Spectre, this potential has been exploited
  recently to accelerate and scale up the efficacy of the attacker in
  the Spectre family of attacks~\cite{DBLP:conf/uss/DisselkoenKPT17}.

\item Compiler optimizations may depend on both branches of
  a conditional, for example one that replaces
  $(\IF(\aExp)\THEN \aCmd \ELSE \aCmd \FI)$ by $\aCmd$.
  In particular, $(\IF(\aReg)\THEN \aLoc\GETS1 \ELSE \aLoc\GETS1 \FI)$
  can be optimized to $(\aLoc\GETS1)$, which does not have a control
  dependency from $\aReg$ to the assignment to $\aLoc$.
  In contrast, the program $(\IF(\aReg)\THEN \aLoc\GETS1 \ELSE \aLoc\GETS2 \FI)$
  cannot be optimized, so the control dependency cannot be removed.

  This would be fine if there were no program constructs
  which could observe control dependencies, but unfortunately
  the relaxed memory models for Java~\cite{Manson:2005:JMM:1047659.1040336}, C++~\cite{Boehm:2008:FCC:1375581.1375591} and LLVM~\cite{DBLP:conf/popl/ZhaoNMZ12}
  allow for such observations. This means there is the possibility
  for information flows caused by optimizing compilers, which
  we investigate in this paper.

  Whereas information flows caused by speculation in hardware,
  or by transactional memory are known~\cite{DBLP:journals/corr/abs-1801-01203,DBLP:conf/uss/DisselkoenKPT17},
  these attacks on compiler optimizations, and the relaxed memory
  models that justify them, are new. In this paper we provide
  both a formal model for such attacks, and experimental evidence
  about their practicality.

\end{itemize}
Information flow attacks motivate the main technical development of this
paper.
%% This line of research was initiated by~\citet{Zhang:2012:LCM:2345156.2254078}.  
%% Whereas they explore static annotations to address sidechannels in the context of hardware description languages, we explore a model of programs
%% that captures enough detail to reveal and analyze the presence of side
%% channels revealed by speculative execution.  
%
Our model is based on \emph{partially ordered multisets}~\cite{GISCHER1988199,Plotkin:1997:TSP:266557.266600}
(``pomsets''), whose labels are given by read and write actions. These can be
visualized as a graph where the edges indicate dependencies, for example
$(\aReg\GETS\aLoc\SEMI \bLoc\GETS1\SEMI \cLoc\GETS\aReg+1)$
has an execution modelled by the pomset:
\[\begin{tikzpicture}[node distance=1em]
  \event{rx1}{\DR{\aLoc}{1}}{}
  \event{wy1}{\DW{\bLoc}{1}}{right=of rx1}
  \event{wz2}{\DW{\cLoc}{2}}{right=of wy1}
  \po[out=25,in=155]{rx1}{wz2}
\end{tikzpicture}\]
The edge from $(\DR{\aLoc}{1})$ to $(\DW{\cLoc}{2})$ indicates a
data dependency. The novel aspect of the model is that events have
\emph{preconditions} which may be false. These are used in giving the
semantics of conditionals, for example
$(\IF(\aLoc)\THEN \bLoc\GETS1\SEMI\cLoc\GETS1 \ELSE \bLoc\GETS2\SEMI\cLoc\GETS1\FI)$
has an execution:
\[\begin{tikzpicture}[node distance=1em]
  \event{rx1}{\DR{\aLoc}{1}}{}
  \event{wy1}{\DW{\bLoc}{1}}{right=of rx1}
  \nonevent{wy2}{\DW{\bLoc}{2}}{below=of wy1}
  \event{wz1}{\DW{\cLoc}{1}}{right=of wy1}
  \po{rx1}{wy1}
  \po{rx1}{wy2}
\end{tikzpicture}\]
Th edges from $(\DR{\aLoc}{1})$ to $(\DW{\bLoc}{1})$ and
$(\DW{\bLoc}{2})$ indicate control dependencies. The presence of
a crossed out $(\DW{\bLoc}{2})$ indicates an event with an unsatisfiable precondition,

The novel contributions of this paper are:
\begin{itemize}

\item a model of program execution that includes speculation (\S\ref{sec:model}),

\item examples showing how the model can be applied,
  including information flow attacks on
  hardware, optimizing compilers, and transactional memory (\S\ref{sec:examples}), and

\item experimental evidence about how practical it is to mount
  the new class of attacks (\S\ref{sec:experiments}).

\end{itemize}
