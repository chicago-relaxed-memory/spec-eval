\section{Model}
\label{sec:model}

The model used in this paper is one of sets of pomsets with event labels of the form
$(\aForm \mid \aAct)$, where $\aForm$ is the event's precondition
(such as $\aExp=\aVal$) and $\aAct$ is the event's action (such as $\DW\aLoc\aVal$).
For example the semantics of the program $(\aLoc\GETS\aExp)$ includes the case
where $\aExp$ is $\aVal$, which is written to $\aLoc$, and is captured
by the one-event pomset:
\[\begin{tikzpicture}[node distance=1em]
  \event{wxv}{\aExp=\aVal \mid \DW{\aLoc}{\aVal}}{}
\end{tikzpicture}\]
We make few requirements of the logic of preconditions, save that it
has includes equalities between expressions, is closed under substitution,
and supports a notion of implication.

% For example, the set of pomsets $\sem{\aReg\GETS\bLoc\SEMI \aLoc\GETS\aReg+1}$ contains:
% \[\begin{tikzpicture}[node distance=1em]
%   \event{ry1}{\DR{\bLoc}{1}}{}
%   \event{wx2}{\DW{\aLoc}{2}}{right=of ry1}
%   \po{ry1}{wx2}
% \end{tikzpicture}\]
The semantics is defined compositionally. As an example, we show how to 
construct one of the pomsets in
$\sem{\aReg\GETS\bLoc\SEMI \aLoc\GETS\aReg+1}$.
First, $\sem{\aLoc\GETS\aReg+1}$
contains the pomset:
\[\begin{tikzpicture}[node distance=1em]
  \event{wx2}{\aReg=1 \mid \DW{\aLoc}{2}}{}
\end{tikzpicture}\]
Next, we perform the substitution of $\aReg$ with $1$ in every precondition,
to get that $\sem{\aLoc\GETS\aReg+1}[1/\aReg]$
contains the pomset:
\[\begin{tikzpicture}[node distance=1em]
  \event{wx2}{1=1 \mid \DW{\aLoc}{2}}{}
\end{tikzpicture}\]
and since $(1=1)$ is a tautology, we elide it:
\[\begin{tikzpicture}[node distance=1em]
  \event{wx2}{\DW{\aLoc}{2}}{}
\end{tikzpicture}\]
This substitution is performed in defining
$\sem{\aReg\GETS\bLoc\SEMI \aLoc\GETS\aReg+1}$, which contains
the pomset:
\[\begin{tikzpicture}[node distance=1em]
  \event{ry1}{\DR{\bLoc}{1}}{}
  \event{wx2}{\DW{\aLoc}{2}}{right=of ry1}
  \po{ry1}{wx2}
\end{tikzpicture}\]
There is an ordering $(\DR{\bLoc}{1}) < (\DW{\aLoc}{2})$ (represented pictorially as an arrow)
because the precondition $(\aReg=1)$ depends on $\aReg$. If the precondition
was independent of $\aReg$ then there would be no ordering, for example
$\sem{\aReg\GETS\bLoc\SEMI \aLoc\GETS\aReg+1-\aReg}$ contains
the pomset:
\[\begin{tikzpicture}[node distance=1em]
  \event{ry1}{\DR{\bLoc}{1}}{}
  \event{wx1}{\DW{\aLoc}{1}}{right=of ry1}
\end{tikzpicture}\]
since the precondition $(\aReg+1-\aReg=1)$ is independent of $\aReg$.

The main novelty of our semantics is the use of preconditions, which allow us
to provide an unusual model of conditionals. In most
models, an execution of
$\sem{\IF(\aExp)\THEN \aCmd \ELSE \bCmd \FI}$ would either be
given by an execution from $\sem{\aCmd}$ or from $\sem{\bCmd}$, but not both.
In our semantics, a pomset
in $\sem{\IF(\aExp)\THEN \aCmd \ELSE \bCmd \FI}$ may include
both a pomset from $\sem{\aCmd}$ \emph{and} a pomset from $\sem{\bCmd}$.
For example, $\sem{\IF(\aExp)\THEN \aLoc\GETS1 \ELSE \aLoc\GETS2 \FI}$
contains:
\[\begin{tikzpicture}[node distance=1em]
  \event{wx1}{\aExp\neq0 \mid \DW{\aLoc}{1}}{}
  \event{wx2}{\aExp=0    \mid \DW{\aLoc}{2}}{right=of wx1}
\end{tikzpicture}\]
that is we have behavior from both branches of execution.

Moreover, two events representing the same action on both sides of a
conditional can be merged, producing a single event.
The precondition of the merged event is the disjunction of the preconditions
of the original events.
For example
$\sem{\IF(\aExp)\THEN \aLoc\GETS1\SEMI \bLoc\GETS3 \ELSE \aLoc\GETS2\SEMI \bLoc\GETS3 \FI}$
contains:
\[\begin{tikzpicture}[node distance=1em]
  \event{wx1}{\aExp\neq0 \mid \DW{\aLoc}{1}}{}
  \event{wx2}{\aExp=0    \mid \DW{\aLoc}{2}}{right=of wx1}
  \event{wy3}{(\aExp\neq0) \lor (\aExp=0) \mid \DW{\bLoc}{3}}{below=3ex of $(wx1)!0.5!(wx2)$}
\end{tikzpicture}\]
and since $(\aExp\neq0) \lor (\aExp=0)$ is a tautology, this is:
\[\begin{tikzpicture}[node distance=1em]1
  \event{wx1}{\aExp\neq0 \mid \DW{\aLoc}{1}}{}
  \event{wx2}{\aExp=0    \mid \DW{\aLoc}{2}}{right=of wx1}
  \event{wy3}{\DW{\bLoc}{3}}{right=of wx2}
\end{tikzpicture}\]

Combining this model of conditionals with the previously discussed model of memory using substitutions
gives that
$\sem{\IF(\cLoc)\THEN \aLoc\GETS1\SEMI \bLoc\GETS3 \ELSE \aLoc\GETS2\SEMI \bLoc\GETS3 \FI}$
contains:
\[\begin{tikzpicture}[node distance=1em]
  \event{rz1}{\DR{\cLoc}{1}}{}
  \event{wx1}{1\neq0 \mid \DW{\aLoc}{1}}{right=of rz1}
  \event{wx2}{1=0    \mid \DW{\aLoc}{2}}{right=of wx1}
  \event{wy3}{\DW{\bLoc}{3}}{right=of wx2}
  \po{rz1}{wx1}
  \po[out=25,in=155]{rz1}{wx2}
\end{tikzpicture}\]
and we visualize unsatisfiable preconditions as crossed out:
\[\begin{tikzpicture}[node distance=1em]
  \event{rz1}{\DR{\cLoc}{1}}{}
  \event{wx1}{\DW{\aLoc}{1}}{right=of rz1}
  \nonevent{wx2}{\DW{\aLoc}{2}}{right=of wx1}
  \event{wy3}{\DW{\bLoc}{3}}{right=of wx2}
  \po{rz1}{wx1}
  \po[out=25,in=155]{rz1}{wx2}
\end{tikzpicture}\]
Note that this semantics captures control dependencies
such as $(\DR\cLoc1)<(\DW\aLoc1)$, independencies
such as $(\DR\cLoc1)\not<(\DW\bLoc3)$, and failed
speculations such as the crossed out $(\DW\aLoc2)$.

In summary, the features we need of the underlying data model are:
\begin{itemize}
\item \emph{actions}, which may read or write memory locations, and
\item \emph{preconditions}, which form a logic closed under substitution.
\end{itemize}
We make data models precise in~\S\ref{sec:preliminaries},
and define pomsets in \S\ref{sec:pomsets}.
In \refapp{sets-of-pomsets}, we define
operations on sets of pomsets, which are used in~\S\ref{sec:semantics}
to give a compositional semantics for
a simple imperative language.

\subsection{Data models}
\label{sec:preliminaries}

A \emph{data model} consists of:
\begin{itemize}
\item a set of \emph{memory locations} $\Loc$, ranged over by
  $\aLoc$ and $\bLoc$,
\item a set of \emph{registers} $\Reg$, ranged over by
  $\aReg$ and $\bReg$,
\item a set of \emph{values} $\Val$, ranged over by
  $\aVal$ and $\bVal$,
\item a set of \emph{expressions} $\Exp$, ranged over by
  $\aExp$ and $\bExp$,
\item a set of \emph{logical formulae} $\Formulae$, ranged over by
  $\aForm$ and $\bForm$, and
\item a set of \emph{actions} $\Act$, ranged over by $\aAct$ and $\bAct$,
\end{itemize}
such that:
\begin{itemize}
\item values include at least the constants $0$ and $1$,
\item expressions include at least registers and values,
\item expressions are closed under substitutions of the form $\aExp[\bExp/\aReg]$,
\item formulae include at least $\TRUE$, $\FALSE$, and equalities of the form $(\aExp=\bExp)$ and $(\aLoc=\bExp)$,
\item formulae are closed under negation, conjunction, disjunction,
\item formulae are closed under substitutions of the form $\aForm[\aLoc/\aReg]$ or $\aForm[\bExp/\aLoc]$,
\item there is a relation $\vDash$ between formulae, and
\item there are partial functions $\rreads$ and $\rwrites: \Act \fun (\Loc \times \Val)$.
\end{itemize}
We shall say $\aAct$ \emph{reads} $\aVal$ \emph{from} $\aLoc$ whenever
$\rreads(\aAct) = (\aLoc,\aVal)$, and
$\aAct$ \emph{writes} $\aVal$ \emph{to} $\aLoc$ whenever
$\rwrites(\aAct) = (\aLoc,\aVal)$.
We shall say $\aForm$ \emph{implies} $\bForm$ whenever $\aForm\vDash\bForm$,
$\aForm$ is a \emph{tautology} whenever $\TRUE\vDash\aForm$,
$\aForm$ is \emph{unsatisfiable} whenever $\aForm\vDash\FALSE$, and
$\aForm$ is \emph{independent of $\aLoc$} whenever $\aForm \vDash \aForm[\aVal/\aLoc] \vDash \aForm$ for every $\aVal$.
In examples, the actions are of the form $(\DR{\aLoc}{\aVal})$, which reads $\aVal$ from $\aLoc$,
and $(\DW{\aLoc}{\aVal})$, which writes $\aVal$ to $\aLoc$.

\subsection{3-valued pomsets}
\label{sec:pomsets}

Recall the definition of a pomset from~\cite{GISCHER1988199}:
\begin{definition}
  A \emph{pomset} $(\Event, {\le}, \labelling)$ with alphabet $\Alphabet$
  is a partial order $(\Event, {\le})$ together with
  $\labelling: \Event \fun \Alphabet$.
\end{definition}
Going forward, we fix the alphabet $\Alphabet=(\Formulae\times\Act)$.
We will write $(\aForm \mid \aAct)$ for the pair $(\aForm,\aAct)$,
elide $\aForm$ when $\aForm$ is a tautology, and write $\aAct$ crossed-out ($\NEVER\aAct$)
when $\aForm$ is unsatisfiable.
We lift terminology from logical formulae and actions to events,
for example if $\labelling(\aEv)=(\aForm\mid\aAct)$
then we say
$\aEv$ is unsatisfiable whenever $\aForm$ is unsatisfiable,
$\aEv$ writes $\aVal$ to $\aLoc$ whenever $\aAct$ writes $\aVal$ to $\aLoc$, and
so forth.
We visualize a pomset as a graph where the nodes are drawn from
$\Event$, each node $\aEv$ is labelled with $\labelling(\aEv)$,
and an edge $\bEv \rightarrow \aEv$ corresponds to an ordering
$\bEv\le\aEv$. For example:
\[\begin{tikzpicture}[node distance=1em]
  \event{rx1}{\DR{\aLoc}{1}}{}
  \nonevent{wy0}{\DW{\bLoc}{0}}{right=of rx1}
  \event{wy1}{\DW{\bLoc}{1}}{right=of wy0}
  \po{rx1}{wy0}
  \po[out=30,in=150]{rx1}{wy1}
\end{tikzpicture}\]
is a visualization of the pomset where:
\[\begin{array}{c}
  E = \{ 0,1,2 \} \quad
  0 \le 1 \quad
  0 \le 2 \quad
  \labelling(0) = (\TRUE, \DR{\aLoc}{1}) \\
  \labelling(1) = (\FALSE, \DW{\bLoc}{0}) \quad
  \labelling(2) = (\TRUE, \DW{\bLoc}{1}) \quad
\end{array}\]

We are building a compositional semantics of shared memory
concurrency, which means we require a notion of when
a read has a matching write. This is a property we require
of closed programs, but \emph{not} of open programs.
For example a program whose semantics includes:
\[\begin{tikzpicture}[node distance=1em]
  \event{wx1}{\DW{\aLoc}{1}}{}
  \event{rx0}{\DR{\aLoc}{0}}{right=2.5em of wx1}
  \event{wy0}{\DW{\bLoc}{0}}{right=of rx0}
  \nonevent{wy1}{\DW{\bLoc}{1}}{right=of wy0}
  \po{rx0}{wy0}
  \po[out=30,in=150]{rx0}{wy1}
\end{tikzpicture}\]
may be put in parallel
with another program which writes $0$ to $\aLoc$.
If the program is closed with respect to $\aLoc$ though, such an execution cannot exist,
so we need each read of $\aLoc$ to have a matching write.
This is captured by defining when $\aEv$ \emph{reads $\aLoc$ from} $\bEv$~\cite{alglave}.
A preliminary definition (which, as we shall see, needs to be strengthened) is:
\begin{itemize}
\item $\bEv < \aEv$,
\item $\aEv$ implies $\bEv$,
\item $\bEv$ writes $\aVal$ to $\aLoc$,
  and $\aEv$ reads $\aVal$ from $\aLoc$, and
\item there is no $\bEv < \cEv < \aEv$ such that
  $\cEv$ writes to $\aLoc$.
\end{itemize}
% In diagrams, for readability we often highlight the reads-from edges,
% for example:
%% We visualize rf-pomsets by drawing a dashed edge between nodes in $\RF$,
%% labelled with the memory location,
%% for example:
%% \[\begin{tikzpicture}[node distance=1em]
%%   \event{wx1}{\DW{\aLoc}{1}}{}
%%   \event{x1}{\DR{\aLoc}{1}}{right=5em of wx1}
%%   \nonevent{y0}{\DW{\bLoc}{0}}{below left=of x1}
%%   \event{y1}{\DW{\bLoc}{1}}{below right=of x1}
%%   \rfx{wx1}{x}{x1}
%%   \po{x1}{y0}
%%   \po{x1}{y1}
%% \end{tikzpicture}\]
%% In most cases, the memory location is obvious from context,
%% so we elide it:
% \[\begin{tikzpicture}[node distance=1em]
%   \event{wx1}{\DW{\aLoc}{1}}{}
%   \event{x1}{\DR{\aLoc}{1}}{right=2.5em of wx1}
%   \nonevent{y0}{\DW{\bLoc}{0}}{right=of x1}
%   \event{y1}{\DW{\bLoc}{1}}{right=of y0}
%   \rf{wx1}{x1}
%   \po{x1}{y0}
%   \po[out=30,in=150]{x1}{y1}
% \end{tikzpicture}\]
Unfortunately by itself, this is not enough. The problem is the final
clause saying that there does not exist an $\aLoc$-\emph{blocking}
event $\cEv$ between $\bEv$ and $\aEv$. Unfortunately, concurrency can
turn events that were not $\aLoc$-blockers into an $\aLoc$-blocker,
\emph{even if the new thread does not mention $\aLoc$.}
We give an example to show this in \refapp{blockers}.
This is a problem in that it means the preliminary model violates
\emph{scope extrusion}~\cite{Milner:1999:CMS:329902},
in that we can find programs $\aCmd$ and $\bCmd$ such that
$\sem{\VAR\aLoc\SEMI(\aCmd\PAR\bCmd)}$ is not the same as
$\sem{(\VAR\aLoc\SEMI\aCmd)\PAR\bCmd}$, even if $\bCmd$ does not mention~$\aLoc$.

There are a number of ways this can be addressed; for example,
in models such as~\cite{Batty:2011:MCC:1926385.1926394} the reads-from relation is taken
as a primitive. In this paper, we propose \emph{3-valued pomsets}
as a solution. These are pomsets in which, in addition to positive statements
$(\bEv < \aEv)$ (interpreted as $\aEv$ depends on $\bEv$),
we also have negative statements $(\bEv \ltN \aEv)$
(interpreted as $\aEv$ cannot depend on $\bEv$).

\begin{definition}
  A \emph{3-valued pomset} $(\Event, {\le}, {\ltN}, \labelling)$ 
  is a \emph{pomset} $(\Event, {\le}, \labelling)$
  together with ${\ltN} \subseteq (\Event\times\Event)$ such that:
  \begin{itemize}
  \item if $\bEv \le \aEv$ then $\aEv \ltN \bEv$,
  \item if $\bEv \le \aEv$ and $\bEv \ltN \aEv$ then $\bEv = \aEv$,
  \item if $\cEv \ge \bEv \ltN \aEv$ or $\cEv \ltN \bEv \ge \aEv$ then $\cEv \ltN \aEv$.
  \end{itemize}
\end{definition}

% \begin{definition}
%   A \emph{3-valued poset} $(\Event,{\le},{\ltN})$ is a poset $(\Event,{\le})$
%   together with ${\ltN} \subseteq (\Event\times\Event)$ such that:
%   \begin{itemize}
%   \item if $\bEv \le \aEv$ then $\aEv \ltN \bEv$,
%   \item if $\bEv \le \aEv$ and $\bEv \ltN \aEv$ then $\bEv = \aEv$,
%   \item if $\cEv \ge \bEv \ltN \aEv$ or $\cEv \ltN \bEv \ge \aEv$ then $\cEv \ltN \aEv$.
%   \end{itemize}
% \end{definition}

% \begin{definition}
%   A \emph{3-valued pomset} $(\Event, {\le}, {\ltN}, \labelling)$
%   is a 3-valued poset $(\Event, {\le}, {\ltN})$ and
%   a pomset $(\Event, {\le}, \labelling)$.
% \end{definition}

Structures similar to 3-valued pomsets have come up in many guises, for example
rough sets~\cite{Pawlak1982} or ultrametrics over
$\{0,{}^1\!/_2,1\}$. They correspond to axioms A1--A3 of Lamport's
\emph{system executions}~\cite{DBLP:journals/dc/Lamport86}.
They are the notion of pomset given by interpreting
$\bEv\le\aEv$ in a 3-valued logic~\cite{Urquhart1986}. 

In diagrams, we visualize $(\aEv \ltN \bEv)$ as a dashed
arrow from $\bEv$ to $\aEv$ (note the change of direction).
We refer to edges introduced by $(\bEv < \aEv)$ as
\emph{strong edges} and by $(\aEv \ltN \bEv)$
as \emph{weak edges}.
For readability, we often highlight the reads-from edges as well.
% for example:
For example:
\[\begin{tikzpicture}[node distance=1em]
  \event{wx0}{\DW{\aLoc}{0}}{}
  \event{wx1}{\DW{\aLoc}{1}}{right=of wx0}
  \event{rx1}{\DR{\aLoc}{1}}{right=2.5 em of wx1}
  \event{wx2}{\DW{\aLoc}{2}}{right=of rx1}
  \rf{wx1}{rx1}
  \wk{wx0}{wx1}
  \wk{rx1}{wx2}
\end{tikzpicture}\]
We strengthen the definition of reads-from to require not just that
no blocker exists, but that any candidate blocker must either
have $\bEv \ltN \cEv$ or $\cEv \ltN \aEv$. This ensures that any
further concurrency cannot turn a non-blocker into a blocker.
\begin{definition}\label{def:rf}
  In a 3-valued pomset, $\aEv$ \emph{can read $\aLoc$ from} $\bEv$ whenever: 
  \begin{itemize}
  \item $\bEv < \aEv$,
  \item if $\aEv$ is satisfiable, then $\bEv$ is a tautology,
  \item $\bEv$ writes $\aVal$ to $\aLoc$,
    and $\aEv$ reads $\aVal$ from $\aLoc$, and
  \item if $\cEv$ writes to $\aLoc$
    then either $\bEv \ltN \cEv$ or $\cEv \ltN \aEv$.
  \end{itemize}
\end{definition}
One of the requirements of closed programs is that
every read event reads from a write event.


\section{Semantics of programs}
\label{sec:semantics}

\begin{figure*}
\begin{eqnarray*}
  \sem{\SKIP} & = & \{ \emptyset \} \\
  \sem{\aLoc\GETS\aExp\SEMI \aCmd} & = & \textstyle\bigcup_\aVal\; \bigl((\aExp=\aVal) \guard (\DW\aLoc\aVal) \prefix \sem{\aCmd}[\aExp/\aLoc]\bigr) \\
  \sem{\aReg\GETS\aLoc\SEMI \aCmd} & = & \sem{\aCmd}[\aLoc/\aReg] \cup \textstyle\bigcup_\aVal\; (\DR\aLoc\aVal) \prefix \sem{\aCmd}[\aLoc/\aReg] \\
  \sem{\IF (\aExp) \THEN \aCmd \ELSE \bCmd \FI} & = & \bigl((\aExp \neq 0) \guard \sem{\aCmd}\bigr) \parallel \bigl((\aExp=0) \guard \sem{\bCmd}\bigr) \\
  \sem{\aCmd \PAR \bCmd} & = & \sem{\aCmd} \parallel \sem{\bCmd} \\
  \sem{\VAR\aLoc\SEMI \aCmd} & = & \nu \aLoc \st \sem{\aCmd}
\end{eqnarray*}
\caption{Semantics of a concurrent shared-memory language}
\label{fig:programs}
\end{figure*}

We give the semantics of programs as sets of 3-valued pomsets.  A pomset in
the semantics of a program should be seen as one possible execution of that
program.  The semantics does not include prefixes, and thus a pomset models
a completed execution.

The definitions result in sets of pomsets that are closed with respect to
augmentation, which may create additional order and strengthening
preconditions:
\begin{definition}
  $\aPS'$ is an augmentation of $\aPS$ if $\Event'=\Event$, $\aEv\le\bEv$
  implies $\aEv\le'\bEv$, $\aEv\ltN\bEv$ implies $\aEv\ltN'\bEv$, and
  % $\labelling'(\aEv)=\labelling(\aEv)$
  if $\labelling(\aEv) = (\bForm \mid \bAct)$ then
  $\labelling'(\aEv) = (\bForm' \mid \bAct)$ where $\bForm'$ implies
  $\bForm$.
\end{definition}

In \refapp{sets-of-pomsets} we define the operations
needed to define the semantics, which are:
\begin{itemize}
\item \emph{prefixing} $\aAct\prefix\aPSS$, which adds an event
  with action $\aAct$ to pomsets in $\aPSS$,
\item \emph{guarding} $\aForm\guard\aPSS$, which filters $\aPSS$,
  keeping pomsets whose events have preconditions which imply $\aForm$,
\item \emph{substitution} $\aPSS[\aExp/\aLoc]$, which performs a substitution
  on every precondition in $\aPSS$,
\item \emph{concurrency} $\aPSS_1\parallel\aPSS_2$, which unions pomsets from
  $\aPSS_1$ and $\aPSS_2$, allowing events to be merged, and
\item \emph{restriction} $\nu\aLoc\st\aPSS$, which requires $\aLoc$ to
  satisfy the requirements of a memory location, in particular that
  any event which reads from $\aLoc$ must have a write event which
  it reads from.
\end{itemize}
These operations are similar to those from models of concurrency such
as~\cite{Brookes:1984:TCS:828.833}, but adapted here to the setting of
speculative evaluation.
We can use them to give the
semantics of a simple shared-memory concurrent language
in Figure~\ref{fig:programs}.

We use $\aPSS_1 \parallel \aPSS_2$ in defining the semantics of conditionals
and concurrency.
It contains the union of pomsets from $\aPSS_1$ and $\aPSS_2$,
allowing overlap as long as they agree on actions. For example, if
$\aPSS_1$ and $\aPSS_2$ contain:
\[\begin{tikzpicture}[node distance=1em]
  \event{a}{\aForm \mid \aAct}{}
  \event{b}{\bForm_1 \mid \bAct}{right=of a}
  \po{a}{b}
\end{tikzpicture}\qquad\qquad\begin{tikzpicture}[node distance=1em]
  \event{b}{\bForm_2 \mid \bAct}{}
  \event{c}{\cForm \mid \cAct}{right=of b}
  \wk{b}{c}
\end{tikzpicture}\]
then $\aPSS_1 \parallel \aPSS_2$ contains:
\[\begin{tikzpicture}[node distance=1em]
  \event{a}{\aForm \mid \aAct}{}
  \event{b}{\bForm_1 \lor \bForm_2 \mid \bAct}{right=of a}
  \event{c}{\cForm \mid \cAct}{right=of b}
  \po{a}{b}
  \wk{b}{c}
\end{tikzpicture}\]

Prefixing is used to define the semantics of reads and writes, and
adds a new event $\cEv$ with action $\aAct$.  As in the definition
of parallel composition, the definition allows the new event to overlap with
events in $\aPSS$ as long as they agree on the action.

The tricky parts of the
definition are the requirements on read
dependencies.  If $\aAct$ reads $\aVal$ from $\aLoc$, we have to
decide whether $\aEv$ depends on $\cEv$ for some $\aEv$ with old
precondition $\bForm$ and new precondition $\bForm'$. The first case
\textsc{[dependent read]} is that the dependency exists, in which case
$\bForm'$ just has to imply $\bForm[\aVal/\aLoc]$. The more interesting 
case is \textsc{[independent read]}, in which case $\bForm'$ has to imply
$\bForm[\aVal/\aLoc]$ and $\bForm$. This corresponds to a case where
$\aEv$ can be performed with or without $\cEv$.
In particular, if $\bForm$ is independent of $\aLoc$ then we can pick
$\bForm'$ to be $\bForm$, and the independent read case will apply.
For example,
if $\aAct$ and $\bAct$ write to the same location, $\aAct$ reads $\aVal$ from $\aLoc$, $\bForm$ is independent of $\aLoc$,
and
$\aPSS$ contains:
\[\begin{tikzpicture}[node distance=1em]
  \event{b}{\bForm \mid \bAct}{}
  \event{c}{\cForm \mid \cAct}{right=of b}
  \po{b}{c}
\end{tikzpicture}\]
then $\aAct\prefix\aPSS$ contains:
\[\begin{tikzpicture}[node distance=1em]
  \event{a}{\aForm \mid \aAct}{}
  \event{b}{\bForm \mid \bAct}{right=of a}
  \event{c}{\cForm[\vec\aVal/\vec\aLoc] \mid \cAct}{right=of b}
  \po[out=25,in=155]{a}{c}
  \wk{a}{b}
  \po{b}{c}
\end{tikzpicture}\]

%% A write generates a write event that may be visible
%% to other threads.  A read may see a
%% thread-local value, or it may generate a read event that must be justified by
%% another thread.  In the latter case, occurrences of $\aReg$ are replaced with
%% $\aLoc$ (rather than $\aVal$) to ensure that dependencies are tracked
%% properly.  The subsequent substitution of $\aVal$ for $\aLoc$ occurs in
%% Definition~\ref{def:prefix} of prefixing.

% We have completed the formal definition of our model of speculative
% evaluation, and now turn to examples.

\subsection{Sequential memory accesses}
\label{sec:sequential-memory}

In the semantics of memory, there are two very different ways memory
can be accessed: sequentially or concurrently. These are modeled
differently, since hardware and compilers give very different
guarantees about their behavior.
In this section, we discuss the sequential semantics, and leave
the concurrent semantics to \S\ref{sec:concurrent-memory}.

Consider the program $(\aLoc\GETS0\SEMI \bLoc\GETS\aLoc+1)$.  One execution of
this program is where the write to $y$ uses the sequential value of
$x$, which is $0$:
\[\begin{tikzpicture}[node distance=1em]
  \event{wx0}{\DW{x}{0}}{}
  \event{wy1}{\DW{y}{1}}{right=of wx0}
\end{tikzpicture}\]
To see how this execution is modeled, we first
expand out the syntax sugar to get the program
$(\aLoc\GETS0\SEMI \aReg\GETS\aLoc\SEMI \bLoc\GETS\aReg+1\SEMI\SKIP)$
Now $\sem{\SKIP}$ is just $\{\emptyset\}$, and
$\sem{y \GETS r+1\SEMI \SKIP}$ includes:
\[
   (r+1=1) \guard (\DW y1) \prefix \sem{\SKIP}[1/y]
\]
which contains the pomset:
\[\begin{tikzpicture}[node distance=1em]
  \event{wy1}{r+1=1 \mid \DW{y}{1}}{}
\end{tikzpicture}\]
expressing that this program can write $1$ to $y$,
as long as the precondition $(r+1=1)$ is satisfied.
Now $\sem{r \GETS x\SEMI y \GETS r+1\SEMI \SKIP}$
has two cases, the sequential case
(which does not introduce a read action)
and the concurrent case (which does).
For the moment, we are interested in the sequential case:
\[
   \sem{y \GETS r+1\SEMI \SKIP}[x/r]
\]
which contains the pomset:
\[\begin{tikzpicture}[node distance=1em]
  \event{wy1}{x+1=1 \mid \DW{y}{1}}{}
\end{tikzpicture}\]
In this pomset, the precondition is $(x+1=1)$, which specifies a property
of the thread-local value of $x$.
Finally $\sem{x \GETS 0\SEMI r \GETS x\SEMI y \GETS r+1\SEMI \SKIP}$ includes:
\[
   (0=0) \guard (\DW x0) \prefix \sem{r \GETS x\SEMI y \GETS r+1\SEMI \SKIP}[0/x]
\]
which contains the pomset:
\[\begin{tikzpicture}[node distance=1em]
  \event{wx0}{0=0 \mid \DW{x}{0}}{}
  \event{wy1}{0=0\land0+1=1 \mid \DW{y}{1}}{right=of wx0}
\end{tikzpicture}\]
all of whose preconditions are tautologies, so this has the expected behavior:
\[\begin{tikzpicture}[node distance=1em]
  \event{wx0}{\DW{x}{0}}{}
  \event{wy1}{\DW{y}{1}}{right=of wx0}
\end{tikzpicture}\]
There is no dependency between $(\DW x0)$ and $(\DW y1)$,
since $(0=0\land0+1=1)$ is independent of $\aLoc$.

This example demonstrates how preconditions
capture the sequential semantics of memory.
In an execution containing an event with label
$(\aForm \mid \aAct)$, one way the precondition $\aForm$
can be discharged is by an assignment $\aLoc\GETS\aExp$,
which performs a substitution $[\aExp/\aLoc]$.
This is a variant of the Hoare semantics of
assignment \cite{Hoare:1969:ABC:363235.363259}, where if $\aCmd$ has precondition $\aForm$
then $\aLoc\GETS\aExp\SEMI\aCmd$ has precondition
$\aForm[\aExp/\aLoc]$.

\subsection{Concurrent memory accesses}
\label{sec:concurrent-memory}

We now turn to the case of concurrent accesses to memory.
Consider the program %a concurrent version of the program from \S\ref{sec:sequential-memory}:
$(\aLoc\GETS1 \PAR \bLoc\GETS\aLoc+1)$.
In executions of this program, it is possible for the second thread to 
perform a concurrent read of $x$:
\[\begin{tikzpicture}[node distance=1em]
  \event{wx1}{\DW{x}{1}}{}
  \event{rx1}{\DR{x}{1}}{right=2.5em of wx1}
  \event{wy2}{\DW{y}{2}}{right=of rx1}
  \rf{wx1}{rx1}
  \po{rx1}{wy2}
\end{tikzpicture}\]
To see how this execution is modeled, we first
expand out the syntax sugar to get the program
$(\aLoc\GETS1\SEMI\SKIP \PAR \aReg\GETS\aLoc\SEMI \bLoc\GETS\aReg+1\SEMI\SKIP)$.
As before, $\sem{y \GETS r+1\SEMI \SKIP}$ includes:
\[
   (r+1=2) \guard (\DW y2) \prefix \sem{\SKIP}[2/y]
\]
which contains the pomset:
\[\begin{tikzpicture}[node distance=1em]
  \event{wy2}{r+1=2 \mid \DW{y}{2}}{}
\end{tikzpicture}\]
As before, $\sem{r \GETS x\SEMI y \GETS r+1\SEMI \SKIP}$ has two cases.
We are now interested in the concurrent case, which includes:
\[
   (\DR x1) \prefix \sem{y \GETS r+1\SEMI \SKIP}[x/r]
\]
which contains the pomset:
\[\begin{tikzpicture}[node distance=1em]
  \event{rx1}{\DR{x}{1}}{}
  \event{wy2}{\DW{y}{2}}{right=of rx1}
  \po{rx1}{wy2}
\end{tikzpicture}\]
Note that $(\DR x1)$ reads $1$ from $x$, and while
$(x+1=2)[1/x]$ is a tautology,
$(x+1=2)$ is not,
and so there is a dependency
$(\DR x1) < (\DW y2)$.

Now, $\sem{x \GETS 1\SEMI \SKIP}$ includes the pomset:
\[\begin{tikzpicture}[node distance=1em]
  \event{wx1}{\DW{x}{1}}{}
\end{tikzpicture}\]
and so $\sem{x \GETS 1\SEMI \SKIP \PAR r \GETS x\SEMI y \GETS r+1\SEMI \SKIP}$ includes:
\[\begin{tikzpicture}[node distance=1em]
  \event{wx1}{\DW{x}{1}}{}
  \event{rx1}{\DR{x}{1}}{right=2.5em of wx1}
  \event{wy2}{\DW{y}{2}}{right=of rx1}
  \rf{wx1}{rx1}
  \po{rx1}{wy2}
\end{tikzpicture}\]
as expected, including a reads-from dependency
$(\DW x1) < (\DR x1)$.

This example demonstrates how read and write events
capture the concurrent semantics of memory.
In an execution containing an event with label
$(\DR \aLoc\aVal)$, if the execution is
$\aLoc$-closed, then there must be an event
it reads from, for example one labelled
$(\DW \aLoc\aVal)$.

\subsection{Control dependencies}
\label{sec:control-dep}

Conditionals introduce control dependencies, for example consider the program:
\[
  \aReg\GETS\cLoc\SEMI
  \IF(\aReg)\THEN \aLoc\GETS1 \ELSE \bLoc\GETS2 \FI
\]
This includes executions in which the false branch is taken:
\[\begin{tikzpicture}[node distance=1em]
  \event{rz0}{\DR{z}{0}}{}
  \nonevent{wx1}{\DW{x}{1}}{right=of rz0}
  \event{wy2}{\DW{y}{2}}{right=of wx1}
  \po{rz0}{wx1}
  \po[out=30,in=150]{rz0}{wy2}
\end{tikzpicture}\]
and ones where the true branch is taken:
\[\begin{tikzpicture}[node distance=1em]
  \event{rz1}{\DR{z}{1}}{}
  \event{wx1}{\DW{x}{1}}{right=of rz1}
  \nonevent{wy2}{\DW{y}{2}}{right=of wx1}
  \po{rz1}{wx1}
  \po[out=30,in=150]{rz1}{wy2}
\end{tikzpicture}\]
In both cases, we record the actions in the branch that was
not taken. This is a novel feature of this model, and is
intended to capture speculative evaluation. In \S\ref{sec:spectre}
we will show how this model captures Spectre-like information
flow attacks, once the attacker is provided with the ability to
observe such speculations.

To see how these executions are modeled, consider the semantics of
$\sem{x\GETS 1\SEMI\SKIP}$, which contains any pomset of the form:
\[\begin{tikzpicture}[node distance=1em]
  \event{wx1}{\aForm \mid \DW{x}{1}}{}
\end{tikzpicture}\]
in particular it contains:
\[\begin{tikzpicture}[node distance=1em]
  \event{wx1}{r\neq0 \mid \DW{x}{1}}{}
\end{tikzpicture}\]
Similarly $\sem{y\GETS 2\SEMI\SKIP}$ contains:
\[\begin{tikzpicture}[node distance=1em]
  \event{wy2}{r=0 \mid \DW{y}{2}}{}
\end{tikzpicture}\]
and so $\sem{\IF(r)\THEN x\GETS 1\SEMI\SKIP \ELSE y\GETS 2\SEMI\SKIP \FI}$
contains:
\[\begin{tikzpicture}[node distance=1em]
  \event{wx1}{r\neq0 \mid \DW{x}{1}}{}
  \event{wy2}{r=0 \mid \DW{y}{2}}{right=of wx1}
\end{tikzpicture}\]
Now, the semantics of concurrent read performs substitutions, for example:
\[\begin{tikzpicture}[node distance=1em]
  \event{rz0}{\DR{z}{0}}{}
  \event{wx1}{0\neq0 \mid \DW{x}{1}}{right=of rz0}
  \event{wy2}{0=0 \mid \DW{y}{2}}{right=of wx1}
  \po{rz0}{wx1}
  \po[out=25,in=155]{rz0}{wy2}
\end{tikzpicture}\]
which gives the required pomset:
\[\begin{tikzpicture}[node distance=1em]
  \event{rz0}{\DR{z}{0}}{}
  \nonevent{wx1}{\DW{x}{1}}{right=of rz0}
  \event{wy2}{\DW{y}{2}}{right=of wx1}
  \po{rz0}{wx1}
  \po[out=30,in=150]{rz0}{wy2}
\end{tikzpicture}\]
Note that the precondition $r=0$ is dependent on $r$,
and so there is a dependency $(\DR z0) < (\DW y2)$,
modeling the control dependency introduced by the conditional.

\subsection{Control independencies}

In most models of control dependencies, the dependency relation
is syntactic, based on whether the action occurs inside syntactically
inside a conditional. In contrast, the notion in this model is
semantic: if an action can occur on both sides of a conditional,
there is no control dependency. Consider a variant of the example
from \S\ref{sec:control-dep}:
\[
  \aReg\GETS\cLoc\SEMI
  \IF(\aReg)\THEN \aLoc\GETS1 \ELSE \aLoc\GETS1 \FI
\]
This has the expected execution in which the control
dependencies exist:
\[\begin{tikzpicture}[node distance=1em]
  \event{rz0}{\DR{z}{0}}{}
  \nonevent{nwx1}{\DW{x}{1}}{right=of rz0}
  \event{wx1}{\DW{x}{1}}{right=of nwx1}
  \po{rz0}{nwx1}
  \po[out=30,in=150]{rz0}{wx1}
\end{tikzpicture}\]
but it also has an execution in which the two writes
of $1$ to $x$ are merged, resulting in no dependency:
\[\begin{tikzpicture}[node distance=1em]
  \event{rz0}{\DR{z}{0}}{}
  \event{wx1}{\DW{x}{1}}{right=of rz0}
\end{tikzpicture}\]
To see how this arises,
consider the definition of $\sem{\IF(r)\THEN x\GETS1\SEMI\SKIP \ELSE x\GETS1\SEMI\SKIP \FI}$:
\[\begin{array}{rl}
   \aPSS_1 \parallel \aPSS_2 \quad\mbox{where}\quad&
   \aPSS_1 = (r\neq 0) \guard \sem{x\GETS1\SEMI\SKIP} \\&
   \aPSS_2 = (r=0) \guard \sem{x\GETS1\SEMI\SKIP}
\end{array}\]
Now, one pomset in $\aPSS_1$ is:
\[\begin{tikzpicture}[node distance=1em]
  \event{wx1}{r\neq0 \mid \DW{x}{1}}{}
\end{tikzpicture}\]
that is $\aPS_1$ where:
\[
  \Event_1 = \{\aEv\} \quad
  \labelling_1(\aEv) = (r\neq 0, \DW x1)
\]
and similarly, one pomset in $\aPSS_2$ is:
\[\begin{tikzpicture}[node distance=1em]
  \event{wx1}{r=0 \mid \DW{x}{1}}{}
\end{tikzpicture}\]
that is $\aPS_2$ where:
\[
  \Event_2 = \{\aEv\} \quad
  \labelling_2(\aEv) = (r= 0, \DW x1)
\]
Crucially, in the definition of $\aPSS_1 \parallel \aPSS_2$
there is \emph{no} requirement that $\Event_1$ and $\Event_2$ are disjoint,
and in this case they overlap at $\aEv$. As a result, one pomset in
$\aPSS_1\parallel\aPSS_2$ is $\aPS_0$ where:
\[
  \Event_0 = \{\aEv\} \quad
  \labelling_0(\aEv) = (r\neq0 \lor r=0, \DW x1)
\]
that is:
\[\begin{tikzpicture}[node distance=1em]
  \event{wx1}{\DW{x}{1}}{}
\end{tikzpicture}\]
Note that this pomset has no precondition dependent on $r$,
since $(r\neq0 \lor r=0)$ does not depend on $r$, which is why
we end up with an execution without a control dependency:
\[\begin{tikzpicture}[node distance=1em]
  \event{rz0}{\DR{z}{0}}{}
  \event{wx1}{\DW{x}{1}}{right=of rz0}
\end{tikzpicture}\]
This semantics captures compiler optimizations which may, for example,
merge code executed on both branches of a conditional, or hoist
constant assignments out of loops.

We can now see the counterintuitive behavior of conditionals
in the presence of control dependencies.
There are programs such as
\(
  (\IF(\cLoc)\THEN \aLoc\GETS1 \ELSE \aLoc\GETS1 \FI)
\)
with executions in which  $(\DW x1)$ is independent of $(\DR z1)$:
\[\begin{tikzpicture}[node distance=1em]
  \event{rz1}{\DR{z}{1}}{}
  \event{wx1}{\DW{x}{1}}{right=of rz1}
\end{tikzpicture}\]
while programs such as
\(
  (\IF(\cLoc)\THEN \aLoc\GETS1 \ELSE \bLoc\GETS2 \FI)
\)
only have executions in which $(\DW x1)$ is dependent on $(\DR z1)$:
\[\begin{tikzpicture}[node distance=1em]
  \event{rz1}{\DR{z}{1}}{}
  \event{wx1}{\DW{x}{1}}{right=of rz1}
  \nonevent{wy2}{\DW{y}{2}}{right=of wx1}
  \po{rz1}{wx1}
  \po[out=30,in=150]{rz1}{wy2}
\end{tikzpicture}\]
These programs have executions with different dependency relations, depending only
on conditional branches that were \emph{not} taken. In \S\ref{sec:info-flow-attack}
we shall see that this has security implications, since relaxed
memory can observe dependency. The attack is similar to Spectre, so
we shall take a detour to see how Spectre can be modeled in this
setting.

\subsection{Relaxed memory}
\label{sec:relaxed-memory}

In \S\ref{sec:info-flow-attack} we present an information flow attack
on relaxed memory, similar to Spectre in that it relies on speculative
evaluation. Unlike Spectre it does not depend on timing attacks,
but instead is based on the sensitivity of relaxed memory to data
dependencies. % For this reason, we present a simple model of relaxed
% memory, which is strong enough to capture this attack.

Our model includes concurrent memory accesses, which can introduce concurrent
reads-from. 
Since we are allowing events to be partially ordered, this gives a simple
model of relaxed memory.  For example an independent read independent write
(IRIW) example is:
\[\begin{array}{l}
  x\GETS0\SEMI x\GETS x+1
  \PAR
  y\GETS0\SEMI y\GETS y+1
\\{}
  \PAR
  r_1\GETS x\SEMI r_2\GETS y
  \PAR
  s_1\GETS y\SEMI s_2\GETS x
\end{array}\]
which includes the execution:
\[\begin{tikzpicture}[node distance=1em]
  \event{wx0}{\DW{x}{0}}{}
  \event{wx1}{\DW{x}{1}}{right=of wx0}
  \event{wy0}{\DW{y}{0}}{right=2.5em of wx1}
  \event{wy1}{\DW{y}{1}}{right=of wy0}
  \event{ry1}{\DR{y}{1}}{below=4ex of wx0}
  \event{rx0}{\DR{x}{0}}{right=of ry1}
  \event{rx1}{\DR{x}{1}}{right=2.5 em of rx0}
  \event{ry0}{\DR{y}{0}}{right=of rx1}
  \rf{wx1}{rx1}
  \rf{wy0}{ry0}
  \rf[out=210,in=30]{wy1}{ry1}
  \rf{wx0}{rx0}
  \wk{rx0}{wx1}
  \wk{ry0}{wy1}
\end{tikzpicture}\]
This model does not introduce thin-air reads (TAR).
For example the TAR pit
\((
  x\GETS y \PAR y \GETS x
)\)
fails to produce a value for $x$ from thin air
since this produces a cycle in $\le$, as shown on the left below:
\begin{align*}
\begin{tikzpicture}[node distance=1em]
  \event{ry42}{\DR{y}{42}}{}
  \event{wx42}{\DW{x}{42}}{below=of ry42}
  \event{rx42}{\DR{x}{42}}{right=2.5em of ry42}
  \event{wy42}{\DW{y}{42}}{below=of rx42}
  \po{ry42}{wx42}
  \po{rx42}{wy42}
  \rf{wx42}{rx42}
  \rf{wy42}{ry42}
\end{tikzpicture}
&&
\begin{tikzpicture}[node distance=1em]
  \event{ry1}{\DR{y}{1}}{}
  \event{wx1}{\DW{x}{1}}{below=of ry1}
  \event{rx1}{\DR{x}{1}}{right=2.5em of ry1}
  \event{wy1}{\DW{y}{1}}{below=of rx1}
  \po{ry1}{wx1}
  \rf{wx1}{rx1}
  \rf{wy1}{ry1}
\end{tikzpicture}
\end{align*}
This cycle can be broken by removing a dependency. For example
\((
  x\GETS y \PAR r\GETS x\SEMI y \GETS r+1-r
)\)
has the execution on the right above.
% \[\begin{tikzpicture}[node distance=1em]
%   \event{ry1}{\DR{y}{1}}{}
%   \event{wx1}{\DW{x}{1}}{below=of ry1}
%   \event{rx1}{\DR{x}{1}}{right=2.5em of ry1}
%   \event{wy1}{\DW{y}{1}}{below=of rx1}
%   \po{ry1}{wx1}
%   \rf{wx1}{rx1}
%   \rf{wy1}{ry1}
% \end{tikzpicture}\]
Note that $(\DR x1) \not\le (\DW y1)$, so this does not introduce a cycle.

Although it is not the primary focus of this paper, our model may be an
attractive model of relaxed memory.  Many prior models either permit
thin-air executions that our model forbids or forbid desirable executions
that our model permits.
%% In \S\ref{sec:logic}, we develop a logic which allows us to prove that our
%% semantics forbids thin air examples that are permitted by prior speculative
%% models
%% \cite{Manson:2005:JMM:1047659.1040336,DBLP:conf/esop/JagadeesanPR10,DBLP:conf/popl/KangHLVD17}.
% Our model passes all of the causality test cases
% \cite{PughWebsite}.
%% Significantly, this
%% includes test case 9, which is forbidden by \cite{DBLP:conf/lics/JeffreyR16},
%% one of the few models that disallows the thin air example from
%% \S\ref{sec:logic}.  We present this test case in the appendix, where we also
%% discuss the thread inlining examples from
%% \cite{Manson:2005:JMM:1047659.1040336}.

In \refapp{logic}, we present a variant of the TAR-pit
example %from \S\ref{sec:relaxed-memory}
that is allowed under prior speculative semantics
\cite{Manson:2005:JMM:1047659.1040336,DBLP:conf/esop/JagadeesanPR10,DBLP:conf/popl/KangHLVD17}.
We develop a logic that allows us to prove that the problematic execution is
forbidden in our model.  \citet{DBLP:conf/esop/BattyMNPS15} showed that the
thin-air problem has no per-candidate-execution solution for C++.  This
result does not apply to our model, which has a different notion of
dependency.
% as the semantics of a conditional can depend on the semantics
% of both branches.

\citet{PughWebsite} developed a set of twenty {causality test cases} in the
process of revising the Java Memory Model (JMM)
\cite{Manson:2005:JMM:1047659.1040336}.  Using hand calculation, we have
confirmed that our model gives the desired result for all twenty cases,
unrolling loops as necessary.  Our model also gives the desired results for
all of the examples in \citet[\textsection 4]{DBLP:conf/esop/BattyMNPS15} and
all but one in \citet[\textsection 5.3]{SevcikThesis}:
redundant-write-after-read-elimination fails for any
sensible non-coherent semantics.  Our model agrees with the JMM on the
``surprising and controversial behaviors'' of \citet[\textsection
8]{Manson:2005:JMM:1047659.1040336}, and thus fails to validate thread
inlining.

In \refapp{tc}, we discuss three of the causality test cases and the thread
inlining from \cite{Manson:2005:JMM:1047659.1040336}.  In presenting the
examples, we unroll loops, correct typos and simplify the code.  

