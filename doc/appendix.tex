\section{Memory model examples}
\label{sec:appendix}


\subsection{Causality test cases}

FROM THE LICS paper:

In the process of revising the Java Memory Model, \citet{PughWebsite}
developed a set of twenty \emph{causality test cases}.  Using hand
calculation, we tested our semantics against eighteen of these cases (TC14
and TC15 require loops).

% Also by hand calculation, we found that our semantics gives the desired
% results for all examples in \citet[\textsection
% 4]{DBLP:conf/esop/BattyMNPS15} and all but one in \citet[\textsection
% 5.3]{SevcikThesis}: redundant-write-after-read-elimination---this
% counterexample applies to any sensible non-coherent semantics.

A variation on TC1:
\begin{verbatim}
  x:=0; if(x<2){y:=1;}  ||  x:=y;
\end{verbatim}
Compiler can determine that x and y are both non-negative.

Causality test cases 1, 6, 8, 9, 18, 20  (12?) require that the logic make
  assertions about the domain of variables

  A variation on TC9:
Here is an example that \cite{DBLP:conf/lics/JeffreyR16} fails on:
\begin{verbatim}
  x:=0; if(x<2){y:=1;}  ||  x:=y;  || y:=2;
\end{verbatim}
This should allow \texttt{r1==r2==1}.  Hopefully this is allowed by our
semantics...


\subsection{Thread inlining}

One property one could ask of a model of shared memory is thread
inlining: any execution of $\sem{P\SEMI Q}$ is an execution of $\sem{P
  \PAR Q}$. This is \emph{not} a goal of our model, and indeed is not
satisfied, due to the different semantics of concurrent and sequential
memory accesses. We demonstrate this by considering an example from
the Java Memory Model~\cite{Manson:2005:JMM:1047659.1040336}, which shows that Java does not
satisfy thread inlining either.

The lack of thread inlining is related to the different dependency
relations introduced by sequential and concurrent access.
Recall from \S\ref{sec:sequential-memory} that the program
\verb`(x := 0; y := x+1;)` has execution:
\[\begin{tikzpicture}[node distance=1em]
  \event{wx0}{\DW{x}{0}}{}
  \event{wy1}{\DW{y}{1}}{right=of wx0}
\end{tikzpicture}\]
but that \verb`(x := 1; || y := x+1;)` has:
\[\begin{tikzpicture}[node distance=1em]
  \event{wx1}{\DW{x}{1}}{}
  \event{rx1}{\DR{x}{1}}{right=2.5em of wx1}
  \event{wy2}{\DW{y}{2}}{right=of rx1}
  \rf{wx1}{rx1}
  \po{rx1}{wy2}
\end{tikzpicture}\]
That is, in the sequential case there is no dependency from the
write of $x$ to the write of $y$, but in the concurrent case there
is such a dependency.

This can be used to construct a counter-example to thread inlining, based on~\cite[Ex~11]{Manson:2005:JMM:1047659.1040336}:
\begin{verbatim}
  x := 0; if (x == 1) { z := 1; } else { x := 1; } || y := x; || x := y;
\end{verbatim}
This has no execution containing $(\DW z1)$. Any attempt to build such an execution
results in a cycle:
\[\begin{tikzpicture}[node distance=1em]
  \event{rx1a}{\DR{x}{1}}{}
  \event{wz1}{\DW{z}{1}}{right=of rx1a}
  \nonevent{wx1a}{\DW{x}{1}}{right=of wz1}
  \event{rx1b}{\DR{x}{1}}{right=2.5em of wx1a}
  \event{wy1}{\DW{y}{1}}{right=of rx1b}
  \event{ry1}{\DR{y}{1}}{right=2.5em of wy1}
  \event{wx1b}{\DW{x}{1}}{right=of ry1}
  \po{rx1a}{wz1}
  \po[out=25, in=150]{rx1a}{wx1a}
  \po{rx1b}{wy1}
  \po{ry1}{wx1b}
  \rf{wy1}{ry1}
  \rf[out=160, in=30]{wx1b}{rx1a}
  \rf[out=160, in=30]{wx1b}{rx1b}
\end{tikzpicture}\]
Inlining the thread \verb|(y := x)| gives~\cite[Ex~12]{Manson:2005:JMM:1047659.1040336}:
\begin{verbatim}
  x := 0; if (x == 1) { z := 1; } else { x := 1; } y := x; || x := y;
\end{verbatim}
with execution:
\[\begin{tikzpicture}[node distance=1em]
  \event{rx1a}{\DR{x}{1}}{}
  \event{wz1}{\DW{z}{1}}{right=of rx1a}
  \nonevent{wx1a}{\DW{x}{1}}{right=of wz1}
  \event{wy1}{\DW{y}{1}}{right=of wx1a}
  \event{ry1}{\DR{y}{1}}{right=2.5em of wy1}
  \event{wx1b}{\DW{x}{1}}{right=of ry1}
  \po{rx1a}{wz1}
  \po[out=25, in=150]{rx1a}{wx1a}
  \po{ry1}{wx1b}
  \rf{wy1}{ry1}
  \rf[out=160, in=30]{wx1b}{rx1a}
\end{tikzpicture}\]
To see why this execution exists, consider the program fragment:
\begin{verbatim}
  if (x == 1) { z := 1; } else { x := 1; } y := x;
\end{verbatim}
Removing the syntax sugar, this is:
\begin{verbatim}
  r1 := x; if (r1 == 1) {
    z := 1; r2 := x; y := r2; skip
  } else {
    x := 1; r3 := x; y := r3; skip
  }
\end{verbatim}
Now, $\sem{z := 1\SEMI r_2 := x\SEMI y := r_2\SEMI \SKIP}$
includes pomset:
\[\begin{tikzpicture}[node distance=1em]
  \event{wz1}{r_1=1 \mid \DW{z}{1}}{}
  \event{wy1}{r_1=x=1 \mid \DW{y}{1}}{right=of wz1}
\end{tikzpicture}\]
and $\sem{x := 1\SEMI r_3 := x\SEMI y := r_3\SEMI \SKIP}$
includes pomset:
\[\begin{tikzpicture}[node distance=1em]
  \event{wx1a}{r_1\neq 1 \mid \DW{x}{1}}{}
  \event{wy1}{r_1\neq 1 \mid \DW{y}{1}}{right=of wx1a}
\end{tikzpicture}\]
so  $\sem{\IF (r_1 = 1) \THEN z := 1\SEMI r_2 := x\SEMI y := r_2\SEMI \SKIP \ELSE x := 1\SEMI r_3 := x\SEMI y := r_3\SEMI \SKIP \FI}$ includes:
\[\begin{tikzpicture}[node distance=1em]
  \event{wz1}{r_1=1 \mid \DW{z}{1}}{}
  \event{wx1a}{r_1\neq1 \mid \DW{x}{1}}{right=of wz1}
  \event{wy1}{(r_1=x=1) \lor (r_1\neq1) \mid \DW{y}{1}}{right=of wx1a}
\end{tikzpicture}\]
which means $\sem{\IF (r_1 = 1) \THEN z := 1\SEMI r_2 := x\SEMI y := r_2\SEMI \SKIP \ELSE x := 1\SEMI r_3 := x\SEMI y := r_3\SEMI \SKIP \FI}[x/r_1]$ includes:
\[\begin{tikzpicture}[node distance=1em]
  \event{wz1}{x=1 \mid \DW{z}{1}}{}
  \event{wx1a}{x\neq1 \mid \DW{x}{1}}{right=of wz1}
  \event{wy1}{(x=x=1) \lor (x\neq1)) \mid \DW{y}{1}}{right=of wx1a}
\end{tikzpicture}\]
Now $(x=x=1) \lor (x\neq1)$ is a tautology, so this is just:
\[\begin{tikzpicture}[node distance=1em]
  \event{wz1}{x=1 \mid \DW{z}{1}}{}
  \event{wx1a}{x\neq1 \mid \DW{x}{1}}{right=of wz1}
  \event{wy1}{\DW{y}{1}}{right=of wx1a}
\end{tikzpicture}\]
and so $\sem{r_1 \GETS x\SEMI \IF (r_1 = 1) \THEN z := 1\SEMI r_2 := x\SEMI y := r_2\SEMI \SKIP \ELSE x := 1\SEMI r_3 := x\SEMI y := r_3\SEMI \SKIP \FI}$ includes:
\[\begin{tikzpicture}[node distance=1em]
  \event{rx1a}{\DR{x}{1}}{}
  \event{wz1}{1=1 \mid \DW{z}{1}}{right=of rx1a}
  \event{wx1a}{1\neq1 \mid \DW{x}{1}}{right=of wz1}
  \event{wy1}{\DW{y}{1}}{right=of wx1a}
  \po{rx1a}{wz1}
  \po[out=25, in=150]{rx1a}{wx1a}
\end{tikzpicture}\]
which simplifies to:
\[\begin{tikzpicture}[node distance=1em]
  \event{rx1a}{\DR{x}{1}}{}
  \event{wz1}{\DW{z}{1}}{right=of rx1a}
  \nonevent{wx1a}{\DW{x}{1}}{right=of wz1}
  \event{wy1}{\DW{y}{1}}{right=of wx1a}
  \po{rx1a}{wz1}
  \po[out=25, in=150]{rx1a}{wx1a}
\end{tikzpicture}\]
as required. The rest of the example is straightforward, and shows that our semantics
agrees with the JMM in not supporting thread inlining.



% \subsection{Word tearing}

% \todo{Remove this section, since it's not needed for transactions?}

% In \S\ref{sec:transactions}, we shall be considering transactional memory,
% and in \S\ref{sec:transactions} show that we can model a simplified version
% of an information flow attack on transactions. In order to model transactions,
% we need to consider actions that can write many memory locations at once,
% since this is part of the semantics of commitment. To lead up to this, we first
% consider a simpler scenario of many-location writes and reads, which is word
% tearing.

% In word tearing, a program contains a write instruction with data larger
% than the hardware word size, for example copying a byte array, or assigning
% a 64-bit float on a 32-bit architecture. For example, consider the program:
% \begin{verbatim}
%   (x := [0, 0];) || (x := [1, 1];) || (r := x;)
% \end{verbatim}
% This has executions in which the read of $x$ only reads from one of the writes,
% for example:
% \[\begin{tikzpicture}[node distance=1em]
%   \event{wx00}{\DW{x}{[0,0]}}{}
%   \event{wx11}{\DW{x}{[1,1]}}{right=2.5em of wx00}
%   \event{rx00}{\DR{x}{[0,0]}}{right=2.5em of wx11}
%   \rf[out=20, in=160]{wx00}{rx00}
% \end{tikzpicture}\]
% but also has executions in which the read of $x$ reads from both writes,
% for example:
% \[\begin{tikzpicture}[node distance=1em]
%   \event{wx00}{\DW{x}{[0,0]}}{}
%   \event{wx11}{\DW{x}{[1,1]}}{right=2.5em of wx00}
%   \event{rx01}{\DR{x}{[0,1]}}{right=2.5em of wx11}
%   \rfx[out=20, in=160]{wx00}{x[0]}{rx01}
%   \rfx[out=-20, in=-160]{wx11}{x[1]}{rx01}
% \end{tikzpicture}\]
% Word tearing can occur, for example, in Java extended floating point~\cite{jmm},
% LLVM 64-bit instructions on 32-bit hardware~\cite{llvm}, or in
% JavaScript SharedArrayBuffers~\cite{js-sab}.

% \newcommand{\rfControl}[4][]{\draw[rf,#1](#2) .. controls (#3) .. (#4);}
% \[\begin{tikzpicture}[node distance=1em]
%   \event{wx0}{\DW{x}{0}}{}
%   \event{wx1}{\DW{x}{1}}{right=of wx0}
%   \event{wy0}{\DW{y}{0}}{right=2.5em of wx1}
%   \event{wy1}{\DW{y}{1}}{right=of wy0}
%   \event{rx1}{\DR{x}{1}}{right=2.5 em of wy1}
%   \event{ry0}{\DR{y}{0}}{right=of rx1}
%   \event{ry1}{\DR{y}{1}}{right=2.5 em of ry0}
%   \event{rx0}{\DR{x}{0}}{right=of ry1}
%   \rf[out=20,in=160]{wx1}{rx1}
%   \rf[out=20,in=160]{wy0}{ry0}
%   \rf[out=340,in=200]{wy1}{ry1}
%   \coordinate (a) [below=of wy1];
%   \rfControl[out=340,in=200]{wx0}{a}{rx0}
%   \wk{wx0}{wx1}
%   \wk{wy0}{wy1}
%   \po{rx1}{ry0}
%   \po{ry1}{rx0}
% \end{tikzpicture}\]


% Batty section 4:
% \cite[\S4]{DBLP:conf/esop/BattyMNPS15},
% Example LB+ctrldata+ctrl-double (language must allow)
% r1=loadrlx(x) //reads 42
% if (r1 == 42)
%   storerlx(y,r1)

% r2=loadrlx(y) //reads 42
% if (r2 == 42)
%   storerlx (x,42)
% else
% storerlx (x,42)

% a:RRLX x=42 sb,dd,cd
% c:RRLX y=42 sb,cd
%   This is forbidden on hardware if compiled naively, as the architectures respect read-to-write control dependencies, but in practice compilers will collapse con- ditionals like that of the second thread, removing the control dependencies from the read of y to the writes of x and making the code identical to the previous example. As that example is allowed and observable on hardware (and we pre- sume that it would be impractical to outlaw such optimisation for C or C++), the language must also allow this execution. But this execution has a cycle in the union of reads-from and dependency, so we cannot simply exclude all those.
% Then one might hope for some other adaptation of the C/C++11 model, but the following example shows at least that there is no per-candidate-execution solution.
% Example LB+ctrldata+ctrl-single (language can and should forbid)
% r1=loadrlx(x) //reads 42 if (r1 == 42)
% storerlx (y,r1) r2=loadrlx (y) //reads 42 if (r2 == 42)
% a:RRLX x=42 sb,dd,cd
% rf
% b:WRLX y=42
% c:RRLX y=42 sb,cd
% rf
% d:WRLX x=42
% rf rf
% b:WRLX y=42 d:WRLX x=42
%   storerlx (x,42)